\section{Interpretacija instance reducibility v toposu}


\begin{izrek}
  %Let \(X\) be a topological space. Then \(X\) is locally \(T₆\) if and only if the
  %reduction \(\lem ≤_I \lpo\) holds.
  Naj bo \(X\) topološki prostor. Tedaj je \(X\) lokalno \(T₆\) natanko tedaj,
  ko nad \(X\) velja redukcija \(\lem ≤_I \lpo\).
\end{izrek}
\begin{proof}
  \begin{enumerate}
  \item[\((⇒)\)]
    Denimo, da je \(X\) lokalno \(T₆\).
    Potem imamo pokritje s \(T₆\) prostori, t.j. za vsak \(Xᵢ\) iz pokritja in
    vsako zaprto množico \(A ⊆ Xᵢ\), imamo zvezno funkcijo \(f : Xᵢ → ℝ\), tako
    da je \(f_A(x) = 0\) natanko tedaj, ko je \(x ∈ A\). Poleg tega lahko
    zahtevamo, da je ta funkcija nenegativna.

    %Potem imamo, za vsako zaprto množico \(A ⊆ X\), zvezno funkcijo
    %\(f_A : X → ℝ\), tako da je \(f_A(x) = 0\) natanko tedaj, ko je \(x ∈ A\).
    %Poleg tega lahko zahtevamo, da je ta funkcija nenegativna.

    Dokažimo sedaj, da velja redukcija.
    Naj bo \(U' ⊆ X\) poljubna odprta množica in \(p ∈ Ω(U')\).
    Dokazati želimo, da obstaja tako pokritje \(U'\), da na njem lokalno
    obstajajo realna števila, za katera \(\lpo\) pomeni, da (lokalno) velja
    \(p ∨ ¬p\).

    Naj bo \(Uᵢ' ≔ U'∩Xᵢ\) pokritje \(U'\) in \(Aᵢ ≔ \c{p ∨ ¬p} ∩ Uᵢ\),
    in \(xᵢ : Uᵢ' → ℝ\) \(T₆\) separacijska funckija \(Aᵢ\) na \(Uᵢ'\).

    %Izkaže se, da lahko za pokritje vzamemo kar celoten \(U'\), za realno število
    %pa vzemimo \(x ≔ f_A\) funkcija, ki jo dobimo iz \(T₆\) lastnosti.

    Predpostavimo sedaj, da je \(Uᵢ ⊆ Uᵢ'\) in da velja \(Uᵢ ⊩ \lpo{\p{xᵢ}}\).
    % To pomeni, da obstaja pokritje \(\b{Uᵢⱼ} ∈ \cov{Uᵢ}\), tako da za vsak element
    To pomeni, da obstaja pokritje \(\cov{Uᵢ}{j}\), tako da za vsak element
    pokritja velja \(xᵢ\res{Uᵢⱼ} ≤ 0\) ali \(xᵢ\res{Uᵢⱼ} > 0\).

    Ker je \(xᵢ\) nenegativna, je \(xᵢ ≤ 0\) natanko tedaj, ko je \(xᵢ = 0\), torej
    po definiciji zgolj na množici \(Aᵢ\).
    Od tod sledi, da je \(xᵢ\res{Uᵢⱼ} ≤ 0\) zgolj v primeru, ko je \(Uᵢⱼ ⊆ Aᵢ\).
    Vendar pa ima \(Aᵢ\) prazno notranjost, tako da je tak \(Uᵢⱼ\) nujno prazen.

    Ostane le še drugi primer, torej \(xᵢ\res{Uᵢⱼ} > 0\).
    Ker množice \(Uᵢⱼ\) tvorijo pokritje \(Uᵢ\), je potem tudi \(xᵢ > 0\) na
    celotnem \(Uᵢ\). Sledi, da je množica \(Aᵢ∩Uᵢ\) prazna,
    in velja \(Uᵢ ⊩ \lem{\p{p}}\).
  \item[\((⇐)\)]
    V obratno smer pa predpostavimo, da obstaja redukcja iz \(\lem\) na \(\lpo\).

    % To pomeni, da za vsako odprto množico \(U ⊆ X\) in \(p ∈ Ω(U)\) obstaja tako
    % pokritje \(\b{Uᵢ} ∈ \cov{Uᵢ}\) in realna števila \(xᵢ : Uᵢ → ℝ\),
    % da lokalno

    Trditev bomo dokazali v dveh korakih. Najprej, naj bo \(A ⊆ X\) zaprta
    množica s prazno notranjostjo.

    Tedaj na \(U ≔ X\) in \(p = Aᶜ\) uporabimo redukcijo \(\lem ≤_I \lpo\).
    % To pomeni, da dobimo pokritje \(\b{Xᵢ} ∈ \cov{X}\) in realna števila
    To pomeni, da dobimo pokritje \(\cov{X}{i}\) in realna števila
    \(xᵢ : Xᵢ → ℝ\), tako da za vsak \(i\) in \(Uᵢ ⊆ Xᵢ\) velja
    % \(\exists{\b{Uᵢⱼ} ∈ \cov{\p{Uᵢ}}}{∀_j xᵢ\res{Uᵢⱼ} ≤ 0 ∨ xᵢ\res{Uᵢⱼ} > 0} ⇒ Uᵢ ⊩ \lem{\p{p}}\).
    \(\exists{\cov{Uᵢ}j}{∀_j xᵢ\res{Uᵢⱼ} ≤ 0 ∨ xᵢ\res{Uᵢⱼ} > 0} ⇒ Uᵢ ⊩ \lem{\p{p}}\).

    Ker ima \(A\) prazno notranjost, je \(p ∨ ¬p = p\), torej je
    \(Uᵢ ⊩ \lem{\p{p}}\) ekvivalentno \(Uᵢ ⊆ p\).

    Brez škode za splošnost lahko vzamemo \(Uᵢ = Xᵢ\) in
    \(\b{Uᵢⱼ} = \b{Vᵢ,Vᵢ'}\), tako da \(xᵢ\res{Vᵢ} ≤ 0 ∨ xᵢ\res{Vᵢ'} > 0 ⇒ Xᵢ ⊆ p\).

    Dokazati želimo, da je

  \end{enumerate}
\end{proof}


\begin{izrek}
  %Let \(X\) be a topological space. Then \(X\) is locally \(T₆\) if and only if the
  %reduction \(\lem ≤_I \lpo\) holds.
  Naj bo \(X\) topološki prostor. Tedaj je \(X\) lokalno \(T₆\) natanko tedaj,
  ko nad \(X\) velja funkcijska redukcija \(\lem ≤_{FI} \lpo\).
\end{izrek}
\begin{proof}
  \begin{enumerate}
  \item[\((⇒)\)]
    Denimo, da je \(X\) lokalno \(T₆\).
    Potem imamo pokritje s \(T₆\) prostori, t.j. za vsak \(Xᵢ\) iz pokritja in
    vsako zaprto množico \(A ⊆ Xᵢ\), imamo zvezno funkcijo \(f : Xᵢ → ℝ\), tako
    da je \(f_A(x) = 0\) natanko tedaj, ko je \(x ∈ A\). Poleg tega lahko
    zahtevamo, da je ta funkcija nenegativna.

    %Potem imamo, za vsako zaprto množico \(A ⊆ X\), zvezno funkcijo
    %\(f_A : X → ℝ\), tako da je \(f_A(x) = 0\) natanko tedaj, ko je \(x ∈ A\).
    %Poleg tega lahko zahtevamo, da je ta funkcija nenegativna.

    Po aksiomu o izbiri dobimo preslikavo \(Cl(Xᵢ) → \p{Xᵢ → ℝ}\).
    Če jo preuredimo in uporabimo vložitev \(Ω(Xᵢ) \hookrightarrow Cl(Xᵢ)\),
    ki \(p\) slika v \(\c{p ∪ ¬p}\), dobimo preslikavo \(Fᵢ: Xᵢ → \p{Ω(Xᵢ) → ℝ}\).

    Dokažimo sedaj, da velja redukcija.
    Vzamimo množice \(Uᵢ = Xᵢ\) in \(fᵢ = Fᵢ\).

    Naj bo sedaj \(U ⊆ Xᵢ\), \(p ∈ Ω(U)\), \(A ≔ \c{p ∪ ¬p}\), \(x ≔ fᵢ(p)\) in predpostavimo,
    da velja \(Xᵢ ⊩ \lpo{\p{x}}\)

    % To pomeni, da obstaja pokritje \(\b{Uᵢ} ∈ \cov{U}\), tako da za vsak element
    To pomeni, da obstaja pokritje \(\cov{U}i\), tako da za vsak element
    pokritja velja \(x\res{Uᵢ} ≤ 0\) ali \(x\res{Uᵢ} > 0\).

    Ker je \(x\) nenegativen, je \(x ≤ 0\) natanko tedaj, ko je \(x = 0\), torej
    po definiciji zgolj na množici \(A\).
    Od tod sledi, da je \(x\res{Uᵢ} ≤ 0\) zgolj v primeru, ko je \(Uᵢ ⊆ A\).
    Vendar pa ima \(A\) prazno notranjost, tako da je tak \(Uᵢ\) nujno prazen.

    Ostane le še drugi primer, torej \(x\res{Uᵢ} > 0\).
    Ker množice \(Uᵢ\) tvorijo pokritje \(U\), je potem tudi \(x > 0\) na
    celotnem \(U\). Sledi, da je množica \(A∩U\) prazna,
    in velja \(U ⊩ \lem{\p{p}}\).
  \item[\((⇐)\)]
    V obratno smer pa predpostavimo, da obstaja redukcja iz \(\lem\) na \(\lpo\).

    To pomeni, da obstaja pokritje \(Xᵢ \circlearrowright X\)
    in preslikave \(Fᵢ: Xᵢ → \p{Ω(Xᵢ) → ℝ}\), za katere velja
    \[Xᵢ ⊩ \for{p∈Ω}{\lpo{\p{Fᵢ(p)}} ⇒ \lem{\p{p}}}\text.\label{eq:lemlpo}\]

    Trditev bomo dokazali v dveh korakih. Najprej, naj bo \(A ⊆ Xᵢ\) zaprta
    množica s prazno notranjostjo.

    Tedaj na \(U ≔ Xᵢ\) in \(p = Aᶜ\) uporabimo \ref{eq:lemlpo}.
    To pomeni, da dobimo implikacijo \(U⊩\lpo{\p{Fᵢ(p)}} ⇒ U⊩\lem{\p{p}}\).

    Naj bo \(f≔Fᵢ(p)\). Potem, če obstaja pokritje \(Uᵢ \circlearrowright U\),
    tako da za vsak \(i\) velja \(f\res{Uᵢ} ≤ 0\) ali \(f\res{Uᵢ} > 0\),
    potem velja tudi \(U ⊆ p ∪ ¬p = p\), oziroma \(A ⊆ Uᶜ\).

    Če tako pokritje obstaja, ga lahko združimo v dve množici, eno, na kateri je
    \(f\) nepozitiven (torej ker je nenegativen kar \(0\)), in eno, na kateri je pozitiven.
    Naj bosta torej \(U₀ in Uₚ\) ti dve (odprti) množici.

    Vednar je pa \(U₀ = f⁻¹{0}\), torej zaprta.

  \end{enumerate}
\end{proof}

\begin{lema}
  Vsaka preslikava \(f : C → I\) inducira preslikavo \(\floor f : C → C\),
  s predpisom \(\floor f(a) ≔ \max\set{x ∈ C}{x ≤ f(a)}\).
\end{lema}

\begin{dokaz}
  Ker je \(C\) kompakten, je predpis dobro definiram, torej moramo preveriti
  zgolj zveznost. Naj bo \(U=w\) bazična odprta množica.

  Naj bodo \(l ≔ (w-1)\hat{1}\), \(r ≔ (w+1)\hat{1}\), in \(U' ≔ \p{l, r}\).
  Potem je \(U ⊆ \cli{w\hat{0}, w\hat{1}} ⊆ U'\) in \(U = U'∩C\).

  Naj bo \(a ∈ α⁻¹(U)\), torej \(\max\set{x ∈ C}{x ≤ f(a)} ∈ U\).
  Od tod sledi, da je \(f(a) ∈ U'\), torej je \(α⁻¹(U) ⊆ f⁻¹(U')\).

  Če je pa \(a ∈ f⁻¹(U')\), je pa \(l < f(a) < r\).
  Iz \(f(a) < r\) sledi \(α(a) < r\), torej \(α(a) ≤ w\hat{1}\),
  iz \(l < f(a)\) pa sledi, da je \(l ≤ α(a)\).

  Iz tega dvojega lahko sedaj sklepamo, da je \(α⁻¹(U) = f⁻¹(U') ⧵ \{l\}\),
  torej odprta množica, kar pa pomeni, da je \(α\) zvezna.
\end{dokaz}

\begin{posledica}
  V zgornji lemi lahko domeno zamenjamo s poljubno odprto množico.
\end{posledica}

\begin{lema}
  Nad Cantorjevim prostorom je \(\lpo\) ekvivalenten \(\lpoR\).
\end{lema}

\begin{dokaz}
  Vemo, da velja \(\lpo ≤_I \lpoR\).

  Pokažimo sedaj, da velja tudi obrat.
  Naj bo \(U ⊆ C\), \(f : U → ℝ\).
  Brez škode za splošnost, lahko \(f\) popravimo do preslikave \(U → I\).
  Vzamemo trivialno pokritje in naj bo \(α ≔ \floor f\).

  Potem je očitno \(α ≤ f\), torej \(f(x) = 0 ⇒ α(x) = 0\).
  Obratno pa, če je \(α(x) = 0\) pomeni, da imamo za vsak \(a ≠ 0\)
  neenakost \(f(x) < a\), torej \(f(x) = 0\).

  Sledi torej, da je \(\lpo\p{α} = \lpoR\p{f}\).
\end{dokaz}

% \begin{izrek}
%   Nad Cantorjevim prostorom je \(\lpo\) idempotenten.
% \end{izrek}

% \begin{dokaz}
%   Ker je \(\lpoR ≤_I \lpo\) moramo pokazati zgolj \(\lpo×\lpo ≤_I \lpoR\).

%   Naj bodo \(U ⊆ C\), \(α,β : U → C\) in \(Uᵢ = U\) trivialno pokritje.

%   Označimo \(A ≔ \c{\lpo α}\) in \(B ≔ \c{\lpo β}\).
%   Ker je \(C\) metrizabilen obstaja funkcija \(f : U → ℝ\), ki je \(0\) na
%   \(A ∪ B\) in \(> 0\) sicer.

%   Naj bo \(V ⊆ U\) poljuben in predpostavimo, da velja \(V ⊩ \lpoR f\).
%   Ker je notranjost \(A\) in \(B\) prazna, je prazna tudi notranjost \(A ∪ B\),
%   torej velja \(f\res V > 0\).
%   Ker je \(f\) pozitivna natanko na komplementu \(A ∪ B\) je torej \(V\)
%   podmnožica \(\c{A ∪ B} = \lpo α ∩ \lpo β\), kar je točno to, kar smo želeli.
% \end{dokaz}

\begin{izrek}
  Če je prostor \(X\) lokalno \(T₆\), velja \(\lpo×\lpo ≤_I \lpoR\).
\end{izrek}

\begin{dokaz}
  Naj bo \(\cov X i\) \(T₆\) pokritje, \(U ⊆ X\), \(α,β : U → C\)
  in \(Uᵢ = U∩Xᵢ \covsymb U\).

  Označimo \(A ≔ {\lpo\p{α}}ᶜ\) in \(B ≔ {\lpo\p{β}}ᶜ\) (pod \(Uᵢ\)).
  Ker je \(Uᵢ\) \(T₆\) obstaja funkcija \(fᵢ : Uᵢ → ℝ\), ki je \(0\) na
  \(A ∪ B\) in pozitivna sicer.

  Naj bo \(Vᵢ ⊆ Uᵢ\) poljuben in predpostavimo, da velja \(Vᵢ ⊩ \lpoR\p{fᵢ}\).
  Ker sta notranjosti množic \(A\) in \(B\) prazni, je prazna tudi notranjost \(A ∪ B\),
  torej velja \({fᵢ\res{Vᵢ} > 0}\).
  Ker je \(fᵢ\) pozitivna natanko na komplementu \(A ∪ B\) je torej \(Vᵢ\)
  podmnožica \(\c{A ∪ B} = \lpo\p{α} ∩ \lpo\p{β}\), kar je točno to, kar smo želeli.
\end{dokaz}

\begin{posledica}
  Nad Cantorjevim prostorom je \(\lpo\) idempotenten.
\end{posledica}


%%% Local Variables:
%%% mode: latex
%%% TeX-master: "main"
%%% End:
