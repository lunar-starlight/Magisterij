\subsection{Principi izbire}\label{sec:izbire}

Za princip izbire ste verjetno že slišali. Ta pravi, da za vsaki množici \(A\)
in \(B\), in vsako celovito relacijo \(R\) med njima, obstaja funkcija
\(f : A → B\), tako da velja \(\for{a : A}{R(a, f(a))}\).

Ta princip lahko ošibimo na tri načine: lahko omejimo množici \(A\) in \(B\),
lahko pa omejimo relacijo \(R\), tako da jemlje resničnostne vrednosti zgolj iz
določene množice \(Σ ⊆ Ω\).

\begin{definicija}
  \emph{Princip izbire nad \(Σ\)} je shema, ki za vsaka \(A\) in \(B\) pravi,
  da za vsako relacijo \(R : A×B → Σ\) za katero velja
  \(\for{a:A}{\exist{b:B}{R(a,b)}}\), obstaja funkcija izbire \(f : A → B\),
  tako da velja \(\for{a:A}{R(a,f(a))}\). Objektu \(A\) tako pravimo
  \emph{domena}, objektu \(B\) \emph{kodomena}, objektu \(Σ\) pa
  \emph{Sierpinskijev objekt}.
  Pogoju na \(R\) pravimo \emph{celovitost}.
  To označimo z \(\AC_Σ\). Če je \(Σ = Ω\) indeks opustimo in temu pravimo
  \emph{princip izbire}.
\end{definicija}
\begin{definicija}
  Če v principu izbire fiksiramo domeno na nek \(A\), temu pravimo
  \emph{princip izbire nad \(Σ\) iz \(A\)}, in če fiksiramo še kodomeno na nek
  \(B\) temu pravimo \emph{princip izbire nad \(Σ\) iz \(A\) v \(B\)}. Ta potem
  označujemo \(\AC_Σ(A)\) in \(\AC_Σ(A, B)\). Podobno kot zgoraj opuščamo \(Σ\)
  ko je ta enaka \(Ω\).
\end{definicija}

TODO: enolična izbira

Hitro lahko opazimo, da so principi izbire neobčutljivi za izomorfizme.
\begin{trditev}
  Naj bo \(φ : A ≅ A'\) in \(ψ : B ≅ B'\). Potem je \(\AC_Σ(A, B)\) ekvivalenten
  \(\AC_Σ(A', B')\).
\end{trditev}
\begin{dokaz}
  Situacija je očitno simetrična, tako da pokažimo zgolj eno implikacijo.

  Če je \(R : A'×B' → Σ\) celovita, je potem tudi \(R∘\p{φ×ψ} : A×B → Σ\)
  celovita. Potem pa za to relacijo obstaja funkcija izbire, tako da je
  \(\for{a : A}{R∘\p{φ×ψ}{(a, f(a))}}\), kar je pa enako
  \(\for{a:A}{R(φ(a),ψ(f(a)))}\). Ker pa je \(φ\) izomorfizem, lahko
  preindeksiramo kvantifikator in dobimo \(\for{a:A'}{R(a',ψ∘f∘φ⁻¹(a'))}\),
  torej je \(ψ∘f∘φ⁻¹\) funkcija izbire za \(R\).
\end{dokaz}
V teoriji množic to pomeni, da je aksiom izbire določen do kardinalnosti
natančno. To nam tudi utemelji zakaj lahko \(\AC(ℕ)\) pravimo princip
\emph{števne} izbire, saj deluje za poljubno števno (neskončno) množico. Tako si
lahko tudi predstavljamo, kako to shemo imen razširiti na poljubno kardinalnost.
Označimo ga torej \(\CC\). Pomembno vlogo bo igral tudi \emph{princip
  števne disjunktivne izbire} \(\AC(ℕ, 2)\), ki ga označimo z \(\CCv\). 

TODO: vložitve domene in kodomene tud
\begin{trditev}
  Če je \(Σ' ⊆ Σ\), \(\AC_Σ(A, B)\) implicira \(\AC_{Σ'}(A, B)\).
\end{trditev}
\begin{dokaz}
  Vsaka relacija nad \(Σ'\) je tudi relacija nad \(Σ\), tako da če imajo
  relacije nad \(Σ\) funkcije izbire jih imajo tudi relacije nad \(Σ'\).
\end{dokaz}

\begin{trditev}
  Princip končne izbire velja.
\end{trditev}
\begin{dokaz}
  TODO: sintaksa za standardne končne?
  Princip končne izbire je princip izbire, kjer domeno omejimo na končne
  množice. Brez škode za splošnost, naj bo domena kar enaka neki
  standardni končni množici \(\cli n\).

  Potem pa \(\AC(\cli n)\) pravi, da če
  obstajajo \(bₖ:B\), da velja \(R(1,b₁)∧\dots ∧R(n,bₙ)\), obstaja končno
  zaporedje (torej, obstajajo \(bₖ:B\)), da velja \(R(1,b₁)∧\dots ∧R(n,bₙ)\)… To
  sta pa isti stvari, tako da princip res drži.
\end{dokaz}

% \begin{definicija}
%   \emph{Princip \[Σ\]-izbire iz \(A\) v \(B\)} pravi, da za vsako relacijo
%   \(R : A×B → Σ\) za katero velja \(\for{a:A}{\exist{b:B}{R(a,b)}}\), obstaja
%   funkcija izbire \(f : A → B\), tako da velja \(\for{a:A}{R(a,f(a))}\).

%   To bomo označili z \(\AC_Σ(A,B)\). Z \(\AC_Σ(A)\) bomo označevali shemo
%   \(\for{B}{\AC_Σ(A,B)}\), z \(\AC_Σ\) pa shemo \(\for{A}{\AC_Σ(A)}\).
%   Tem pravimo \emph{princip \(Σ\)-izbire iz \(A\)} in \emph{princip \(Σ\)-izbire}.
%   Kadar je \(Σ = Ω\) bomo \(Σ\) v indeksu opuščali, in ustrezno to poimenovali
%   kar samo \emph{princip izbire (iz \(A\) v \(B\))}.

%   Standardno se principu \(\AC(ℕ)\) pravi \emph{princip števne izbire} in
%   označuje \(\CC\).
%   Posebno pomemben bo princip \(\AC(ℕ, 2)\), ki ga bomo označevali \(\CCv\) in
%   mu pravili \emph{princip števne dvojiške(disjunktivne?) izbire}.
% \end{definicija}



% \begin{definicija}
%   \emph{Princip števne dvojiške izbire} pravi, da če velja
%   \(\for{n : ℕ}{P(n, 0) ∨ P(n, 1)}\) (torej, \(P\) je celovita relacija na
%   \(ℕ×2\)) obstaja funkcija izbire \(f : ℕ → 2\), da velja
%   \(\for{n : ℕ}{P(n,f(n))}\).
% \end{definicija}

Poznamo pa tudi še princip odvisne izbire.
\begin{definicija}
  \emph{Princip odvisne izbire nad \(Σ\) za \(A\)} pravi, da za vsako celovito
  relacijo \(R : A×A → Σ\) in \(a₀ : A\) obstaja zaporedje \((aᵢ)ᵢ\) začenši z
  \(a₀\), tako da za vsak \(n : ℕ\) velja \(R(aₙ, aₙ₊₁)\). Tega označimo z
  \(\DC_Σ(A)\). Če princip velja za vse \(A\) ga označimo \(\DC_Σ\).
\end{definicija}

\begin{trditev}
  Če je \(Σ⊆Ω\) zaprt za končne konjunkcije in števne disjunkcije, velja
  implikacija \(\DC_Σ ⇒ \CC_Σ\).
\end{trditev}
\begin{dokaz}
  Naj bo \(R : ℕ×B → Σ\) celovita.
  Definirajmo \(X ≔ \set{b:B}{\exist{n:ℕ}{R(n,b)}}\) in na \(X×X\) relacijo
  \(Q(x,y) ≔ \exist{n:ℕ}{R(n,x)∧R(n+1,y)}\).

  Ta je celovita, tako da lahko na njej uporabimo \(\DC_Σ\), torej dobimo
  zaporedje \(x : ℕ → X\). Ker je \(X⊑B\) je torej to preslikava \(ℕ → B\), ki
  ima želeno lastnost.
\end{dokaz}

V dokazu je pogoj na Sierpinskijevem objektu res pomemben, sicer \(Q\) ni
relacija nad \(Σ\). 
% NOTE: A lahko to naredimo za poljuben \(X\)?
\begin{trditev}
  Če je \(X\) lokalno povezan, in nad \(X\) velja kateri koli princip izbire,
  velja potem tudi v metateoriji.
\end{trditev}
\begin{proof}
  Oglejmo si princip izbire \(\AC_Σ{\p{A, B}}\).
  Potem za vsako celovito relacijo \(R : A×B → Σ\) dobimo znotraj
  \[ X ⊩ \for{a : \c A}{\exist{b : \c B}{R(a, b)}}\text. \]
  Potem pa po notranjem aksiomu izbire sčasoma obstaja funkcija
  \(f : {\c B}^{\c A}\). Ker je prostor lokalo povezan, je to enako kot element
  \(f : \c{B^A}\), kar nam pa da eksterno funkcijo \(f : A → B\), za katero
  potem velja \(\for{a ∈ A}{R(a, f(a)}\).
\end{proof}

TODO: a je to relevantno?
Ta izrek nam pove, da zares v preveč konstruktivni metamatematiki niti ne moremo
pričakovati, da se bo zgodilo kaj zanimivega. Skoraj noben topološki model ne bo
validiral principov izbire, kar pa pomeni, da bodisi ne bomo mogli najti pametne
topološke karakterizacije, bodisi pa ne bomo mogli pokazati, da topološki
prostor s tako lastostjo sploh obstaja.

\begin{opomba}
  Zgoraj smo lokalno povezanost uporabili zgolj za enakost \({\c B}^{\c A} = \c{B^A}\),
  tako da bi zadoščalo predpostaviti zgolj to (torej, za enaka \(A\) in \(B\)
  kot se pojavita v aksiomu izbire). Enaka opomba bo na mestu za vse izreke o
  lokalni povezanosti v tem razdelku, razen kjer bo eksplicitno omenjeno.
\end{opomba}

% TODO: cite whatever talk he gave
Spodnji rezultat je od Alexa Simpsona, dokaz v primeru topoloških modelov je
pa originalen.
\begin{izrek}\label{th:ac-and-conn-is-pgt}
  Nad \(X\) velja \(\AC_Σ(\c A, \c B)\) in \({\c B}^{\c A} = \c{B^A}\) natanko
  tedaj, ko ima vsaka \(A\)-indeksirana množica \(B\)-indeksiranih
  \(Σ\)-pokritij \(U\) skupno pofinitev.
  %Nad \(X\) velja \(\AC_Σ{\p{\c A, \c B}}\) in \({\c B}^{\c A} = \c{B^A}\)
  %natanko tedaj, ko je Grothendieckova topologija \(X\) zaprta za \(A\)-mnoge
  %preseke \(B\)-indeksiranih \(Σ\)-krovov.
\end{izrek}
\begin{proof}
  \begin{itemize}
  \item[\(\p ⇐\)]
    Pokažimo prvo, da velja gornja enakost.
    Na stopnji \(U ∈ 𝒪X\) je \({\c B}^{\c A}\) navzven enak \(U×ΔA → ΔB\),
    objekt \(\c{B^A}\) pa \(U → Δ\p{B^A}\). Torej moramo pokazati, da vsaka
    zvezna funkcija \(f\) porodi zvezno funkcijo \(x ↦ (a ↦ f(x, a))\).

    Naj bo \(X ⊩ P : \c A × \c B → Σ\) celovita relacija.
    To pomeni, da za vsak \(a ∈ A\) obstaja \(B\)-indeksiran \(Σ\)-krov \(Cₐ\),
    tako da za vsak element \(U ∈ Cₐ\) obstaja tak \(b_{a, U} ∈ B(U)\), da velja
    \(U ⊩ P(a, b_{a, U})\).

    Po predpostavki sedaj vemo, da je \(C ≔ ⋂_{a ∈ A} Cₐ\) tudi \(Σ\)-krov.
    Sedaj pa, za vsak \(U ∈ C\) velja, da je vsebovan v vsakem \(Cₐ\), tako da
    za vsak \(a ∈ A\) obstaja \(b_{a, U}\), da velja \(U ⊩ P(a, b_{a, U})\).
    Tedaj pa je preslikava \(a ↦ b_{a, U}\) ravno želena funkcija izbire.
  \item[\(\p ⇒\)]
    Naj bodo \(Cₐ = \{U_{a,b}\}_{b ∈ B}\) \(Σ\)-pokritja.
    Ta definirajo relacijo \(R(a, b) ≔ U_{a,b}\), ki je celovita, saj so \(Cₐ\)
    pokritja. Tedaj po principu izbire znotraj (sčasoma) dobimo funkcijo
    \(f : A → B\), tako da je \(\for{a : A}{U_{a,f(a)}}\).
    Sedaj pa, ker velja enakost iz predpostavke, je \(f\) tudi funkcija
    \(A → B\) zunaj. To pa pomeni, da je \(\e f\) vsebovan v preseku
    \(U_{a,f(a)}\), torej, ker \(\e f\) pokrivajo cel prostor, tvorijo skupno
    pofinitev pokritij \(Cₐ\).

    TODO: nekej o tem da je to \(Σ\)-pokritje? ker \(\e f\) trenutno nimajo
    nobenega pogoja na sebi. (če je to sploh res)
  \end{itemize}
\end{proof}
NOTE: brez meta-ac je leva stran lahko za vse celovite \(R\) velja
\[ \eventually{V⊆U}{\for{a∈A}{\exist{b∈B}{V⊩R(a,b)}}} \]
NOTE: Mogoče tu rabimo \(\g B\) namesto \(B\).

NOTE: interno je to \(\for{a:A}{\globalen{b:B}{R(a,b)}}\)

TODO: Enkrat sem pokazala, da je \(\for{A,B}{\AC(\c A,\c B)}\) ekvivalentno
\(\AC\), zdej pa ne znam. A je to res?

\begin{definicija}\label{def:psp}
  \emph{P-prostori} so prostori, za katere velja, da je števen presek odprtih
  množic odprt.
\end{definicija}

\begin{trditev}\label{th:psp-is-pgt}
  Če je \(X\) P-prostor ima vsaka množica števno mnogo pokritij \(X\) skupno
  pofinitev.
\end{trditev}
\begin{dokaz}
  Naj bodo \(Cₙ = \{U_{n,i}\}ᵢ\) pokritja. Brez škode za splošnost
  predpostavimo, da so vsa indeksirana z nekim \(I\), saj lahko pokrijim dodamo
  prazno množico.

  Potem naj bo \(φ : ℕ → I\) funkcija. Za to lahko tvorimo
  \(U_φ ≔ ⋂_nU_{n,φ(n)}\). To je števen presek odprtih množic, torej je po
  predpostavki odprta množica. Prav tako je pa vsak \(x ∈ X\) vsebovan v nekem
  \(U_{n,i}\) za vsak \(n ∈ ℕ\), torej po principu izbire v metateoriji obstaja
  preslikava \(φ : ℕ → I\), da je \(x ∈ U_φ\). To pomeni, da \(U_φ\) tvorijo
  pokritje, ki je pofinitev vsakega \(Cₙ\).
\end{dokaz}
\begin{posledica}\label{th:psp-has-cc}
  Nad P-prosotri velja \(\CC\).
\end{posledica}
\begin{opomba}
  V dokazu množica \(ℕ\) ni posebej odlikovana, razen da je enake kardinalnosti
  kot tista, za katero je topologija zaprta za preseke, torej izrek lahko
  posplošimo na princip izbire iz poljubne kardinalnosti.

  Splošnem principu izbire torej pripadajo prostori Aleksandrova, kjer je presek
  poljubne družine odprtih množic odprt.
\end{opomba}

Obrat gornje trditve pa ne velja.
\begin{trditev}\label{th:psp-is-not-pgt}
  Nad prostorom \(\cli{0,1}\) s topologijo \(\set{[0,a)}{a ∈ \cli{0,1}}\) ima
  poljubno mnogo pokritij skupno pofinitev a ni P-prostor.
\end{trditev}
\begin{dokaz}
  Množice \(Uₙ ≔ [0,2⁻ⁿ)\) so odprte, a njihov presek je \(\{0\}\), ki ni
  odprta množica, torej prostor ni P-prostor.

  Sedaj pa, če je \(C\) pokritje \(X\) more pokriti \(1\). Ampak edina odprta
  množica, ki pokrije \(1\) je \(\cli{0,1}\), torej je \(\{\cli{0,1}\}\)
  pofinitev vsakega pokritja \(X\).
\end{dokaz}

Znana je tudi delna karakterizacija \(\DC\) [TODO: cite Elephant D4.5.16,
separating fragments].
\begin{definicija}
  Prostor \(X\) je \emph{ultraparakompakten} ali \emph{nič dimenzionalen}, ko
  ima vsako pokritje prostora pofinitev s particijo.
\end{definicija}
\begin{trditev}
  Nad ultraparakompaktnimi prostori velja \(\DC\).
\end{trditev}
\begin{dokaz}
  TODO: dokaz
\end{dokaz}
\begin{opomba}
  Tu verjetno lahko rečemo, da \(\DC_Σ\) pripada ultraparakompaktnost za
  \(Σ\)-pokritja.
\end{opomba}


% \begin{trditev}
%   Če nad \(X\) velja \(\AC_Σ(\c A, \c B)\) za vse \(B\), velja tudi
%   \(\AC_Σ(A, ℱ)\) za vse \(ℒ\)-množice \(ℱ\).
% \end{trditev}
% \begin{dokaz}
%   Naj bo \(F\) podležna množica \(ℱ\) in \(R : A×F → Σ\) celovita. Potem je tudi
%   celovita kot relacija med \(A\) in \(\c F\). Uporabimo princip izbire, da
%   dobimo preslikavo \(f: A → \c{\g F}\), za katero velja \(\for{a : A}{R(a, f(a))}\).

%   Pokažimo sedaj, da zunaj \(f\) definira \(ℒ\)-morfizem \(A ↬ ℱ\).
%   Če je \(b = b'\) in \(b' = f(a)\),
% \end{dokaz}


%\subsection{Števna izbira in P-prostori}

% \begin{lema}
%   Nad P-prostori velja \(\CC\).
% \end{lema}
% \begin{proof}
%   Trditev \(X ⊩ \AC{\p{ℕ, ℱ}}\) pravi, da če je \(P\) celovita relacija med \(ℕ\) in
%   \(ℱ\), potem obstaja funkcija \(f : ℕ → ℱ\), ki je podrelacija \(P\).

%   Naj bo torej \(P\) taka celovita relacija v \(\sh{X}\).
%   Relacija \(P\) je navzven zaporedje globalnih prerezov potenčnega snopa \(ℱ\).
%   To da je celovita pa pomeni, da imamo za vsak \(n ∈ ℕ\) indeksno množico
%   \(Iₙ\) in pokritje \(\{U_{n,i}\}_{i ∈ Iₙ}\), skupaj z lokalnimi prerezi
%   \({f_{n,i} ∈ ℱ{\p{U_{n,i}}}}\), tako da velja \( U_{n,i} ⊩ Pₙ{\p{f_{n,i}}}\).

%   Obstoj funkcije izbire pa pomeni, da mora obstajati pokritje \(\{Vⱼ\}ⱼ\) in
%   zaporedje prerezov \(f_{n,j} ∈ ℱ{\p{Vⱼ}}\), tako da velja
%   \(Vⱼ ⊩ \for{n : ℕ}{Pₙ{\p{f_{n,j}}}}\).

%   Za odvisno funkcijo \(φ : ∏_{n ∈ ℕ} Iₙ\) definirajmo množico
%   \(U_φ ≔ ⋂_{n ∈ ℕ} U_{n,φ(n)}\), ki je števen presek odprtih množic torej po
%   predpostavki odprt. Množice \(U_φ\) pokrijejo prostor, saj je vsak \(x ∈ X\)
%   vsebovan v nekem \(U_{n, i}\) za vse \(n ∈ ℕ\) torej po aksiomu števne izbire
%   (v metateoriji) obstaja tudi funkcija \(φ\), tako da bo \(x ∈ U_{n, φ{n}}\),
%   torej v \(U_φ\).

%   Še več, za vse \(n\) in \(φ\) velja
%   \[ U_φ ⊆ U_{n,φ(n)} ⊩ Pₙ{\p{f_{n,φ(n)}}}\text. \]

%   To pa pomeni, da za obstoj funkcije izbire vzamemo krov \(\{U_φ\}_φ\) in
%   preslikavo \(φ ↦ \p{f_{n, φ(n)} ∈ ℱ{p{U_φ}}}ₙ\), ki zadošča želenemu pogoju,
%   kot smo pokazali zgoraj.
% \end{proof}
% \begin{opomba}
%   Gornji dokaz ključno uporabi askiom števne izbire v metateoriji.
% \end{opomba}


% \begin{lema}\label{th:t1-ccv-is-psp}
%   Če nad \(T₁\) prostorom velja \(\CCv\), je prostor P-prostor.
% \end{lema}
% \begin{proof}
%   Naj torej velja \(X ⊩ \CCv\) in naj bo \(\{Uₙ\} ⊆ 𝒪X\) števna družina ter
%   \(a ∈ ⋂ₙ Uₙ\) poljubna točka. Dokazati želimo, da ta točka leži v notranjosti
%   preseka, kar bo pokazalo, da je odprt. Definirajmo interno relacijo \(R\) med
%   \(ℕ\) in \(2\) s predpisom
%   %\(X ⊩ R(n, b) ⇔ (b = 1 ∧ Uₙ) ∨ (b = 0 ∧ !a)\).
%   \[
%   R(n, b) ⇔
%   \begin{cases}
%     Uₙ &; b = 1\\
%     !a &; b = 0\text.
%   \end{cases}
%   \]
%   Ker so v \(T₁\) prostorih točke zaprte, je \(!a = X⧵\{a\}\).
%   Ker je \(Uₙ∪{!a} = Uₙ∪(X⧵\{a\}) = X\), je ta relacija celovita, torej lahko na
%   njej uporabimo dani aksiom izbire. Tako v okolici \(W\nbd a\) dobimo funkcijo
%   \({W ⊩ f : ℕ → 2}\), za katero velja \(W ⊩ \for{n : ℕ}{R(n, f(n))}\).

%   Ker je pa to geometrijska implikacija, pa velja tudi \(a ⊩ \for{n : ℕ}{R(n, f(n))}\).
%   Vendar pa za noben \(n\) ne velja \(a ⊩{!a} (= R(n, 0))\), torej je \(f = 1\).
%   Potem pa velja \(a ∈ W ⊆ ⋂ₙ ⟦R(n, f(n))⟧ = ⋂ₙ ⟦R(n, 1)⟧ = ⋂ₙ Uₙ\).
% \end{proof}
% % \begin{proof}
% %   Naj je \(Uₙ\) števno pokritje \(U\) in \(a ∈ ⋂ₙ Uₙ\). Dokazujemo, da obstaja
% %   odprta okolica \(a\), ki je vsebovana v preseku.
% %   Konstruirajmo množice \(Vₙ ≔ ⋃_{k ≠ n} Uₖ ⧵ \{a\}\) in pokritja \(Cₙ ≔ ↓{\{Uₙ, Vₙ\}}\).
% %   Po predpostavki je tudi presek \(C ≔ ⋂ₙ Cₙ\) pokritje, torej za nek \(W ∈ C\)
% %   velja \(a ∈ W\). Potem pa za vse \(n\) velja \(W ∈ Cₙ\), torej imamo \(W ⊆ Uₙ\)
% %   ali \(W ⊆ Vₙ\).

% %   Ker je pa \(a ∈ W\) in \(a ∉ Vₙ\), je potem nujno \(W ⊆ Uₙ\), torej dobimo
% %   \(a ∈ W ⊆ ⋂ₙ Uₙ\), kar je natanko kar smo želeli.
% % \end{proof}
% %\begin{opomba}
% %  Gornji dokaz deluje tudi za \(R₀\) prostore, saj lahko \(Vₙ\) definiramo z
% %  zaprtjem točke \(a\), in potrebujemo zgolj, da je to vsebovano v vsaki okolici
% %  točke \(a\), kar je pa natanko ekvivalentno pogoju \(R₀\). Za občutek, \(T₁\)
% %  prostori so natanko \(T₀\) in \(R₀\) prostori.
% %\end{opomba}




% Gornji lemi lahko združimo, da dobimo
% \begin{trditev}
%   Nad \(T₁\) prostori je števna izbira ekvivalentna disjunktivni števni izbiri.
% \end{trditev}


%%% Local Variables:
%%% mode: latex
%%% TeX-master: "main"
%%% End:
