\section{Principi izbire}

% NOTE: A lahko to naredimo za poljuben \(X\)?
\begin{trditev}
  Če je \(X\) lokalno povezan, in nad \(X\) velja kateri koli princip izbire,
  velja potem tudi v metateoriji.
\end{trditev}
\begin{proof}
  Oglejmo si princip izbire \(\AC_Σ{\p{A, B}}\).
  Potem za vsako celovito relacijo \(R : A×B → Σ\) imamo
  \[ X ⊩ \for{a : \c A}{\exist{b : \c B}{R(a, b)}}\text. \]
  Potem pa po internem aksiomu izbire sčasoma obstaja interna funkcija
  \(f : {\c B}^{\c A}\). Ker je prostor lokalo povezan, je to enako kot funkcija
  \(f : \c{B^A}\), kar nam pa da eksterno funkcijo \(f : A → B\), za katero
  potem velja \(\for{a ∈ A}{R(a, f(a)}\).
\end{proof}

Ta izrek nam pove, da zaes v preveč konstruktivni matematiki niti ne moremo
pričakovati, da se bo zgodilo kaj zanimivega. Skoraj noben topološki model ne bo
validiral principov izbire, kar pa pomeni, da bodisi ne bomo mogli najti pametne
topološke karakterizacije, bodisi pa ne bomo mogli pokazati, da topološki
prostor s tako lastostko sploh obstaja.

\begin{opomba}
  Zgoraj smo lokalno povezanost uporabili zgolj za enakost \({\c B}^{\c A} = \c{B^A}\),
  tako da bi zadoščalo predpostaviti zgolj to (torej, za enaka \(A\) in \(B\)
  kot se pojavita v aksiomu izbire). To bo tema tega razdelka.
\end{opomba}

% %TODO: cite whatever talk he gave
% Spodnji rezultat je odkril Alex Simpson, dokaz v primeru topoloških modelov je
% pa originalen.
% \begin{izrek}
%   Nad \(X\) velja \(\AC_Σ{\p{\c A, \c B}}\) in \({\c B}^{\c A} = \c{B^A}\)
%   natanko tedaj, ko je Grothendieckova topologija \(X\) zaprta za \(A\)-mnoge
%   preseke \(B\)-indeksiranih \(Σ\)-krovov.
% \end{izrek}
% \begin{proof}
%   \begin{itemize}
%   \item[\(\p ⇐\)]
%     Pokažimo prvo, da velja gornja enakost.
%     Na stopnji \(U ∈ 𝒪X\) je \({\c B}^{\c A}\) navzven enak \(U×ΔA → ΔB\),
%     objekt \(\c{B^A}\) pa \(U → Δ\p{B^A}\). Torej moramo pokazati, da vsaka
%     zvezna funkcija \(f\) porodi zvezno funkcijo \(x ↦ (a ↦ f(x, a))\).
    
    
%     Naj bo \(X ⊩ P : \c A × \c B → Σ\) celovita relacija.
%     To pomeni, da za vsak \(a ∈ A\) obstaja \(B\)-indeksiran \(Σ\)-krov \(Cₐ\),
%     tako da za vsak element \(U ∈ Cₐ\) obstaja tak \(b_{a, U} ∈ B(U)\), da velja
%     \(U ⊩ P(a, b_{a, U})\).

%     Po predpostavki sedaj vemo, da je \(C ≔ ⋂_{a ∈ A} Cₐ\) tudi \(Σ\)-krov.
%     Sedaj pa, za vsak \(U ∈ C\) velja, da je vsebovan v vsakem \(Cₐ\), tako da
%     za vsak \(a ∈ A\) obstaja \(b_{a, U}\), da velja \(U ⊩ P(a, b_{a, U})\).
%     Tedaj pa je preslikava \(a ↦ b_{a, U}\) ravno želena funkcija izbire.
%   \item[\(\p ⇒\)]
%     Naj bodo \(Cₐ\) \(Σ\)-krovi in \(C ≔ ⋂_{a ∈ A} Cₐ\). Pokazali bomo, da
%     vsebuje \(Σ\)-krov, kar bo pokazalo, da je tudi sam \(Σ\)-krov.
%     Definiramo relacijo med \(\uline A\) in \(\uline Σ\) s predpisom
%     \(\X ⊩ P(α, U) ⇔ \p{U ∈ Cₐ} ∧ U\).

%     To je celovita relacija, saj \(Cₐ\) pokrivajo prostor, torej lahko uporabimo
%     dani aksiom izbire. To pa pomeni, da za vsak \(x ∈ X\), sčasoma na \(U \nbd x\)
%     obstaja \(f : {\ulineΣ}^{\uline A}(U)\), za katerega za vse \(a ∈ A\) velja
%     \(U ⊩ P(a, f(a))\), oziroma \(U ⊩ f(a) ∈ Cₐ\) in \(U ⊩ f(a)\).
%     To pa pomeni, da velja \(x ⊩ f(a)\), torej \(x ∈ f(x, a) ∈ Cₐ\).
%     Ker je \(a\) poljuben je torej \(x\) vsebovan v 
    
    
    
%   \end{itemize}
% \end{proof}

\subsection{Števna izbira in P-prostori}

\begin{definicija}
  \emph{P-prostori} so prostori, za katere velja, da je števen presek odprtih
  množic odprt.
\end{definicija}

% NOTE: can we factor this through GT(X) σ-closed
\begin{lema}\label{th:psp-has-cc}
  Nad P-prostori velja \(\CC\).
\end{lema}
\begin{proof}
  Trditev \(X ⊩ \AC{\p{ℕ, ℱ}}\) pravi, da če je \(P\) celovita relacija med \(ℕ\) in
  \(ℱ\), potem obstaja funkcija \(f : ℕ → ℱ\), ki je podrelacija \(P\).

  Naj bo torej \(P\) taka celovita relacija v \(\sh{X}\).
  Relacija \(P\) je navzven zaporedje globalnih prerezov potenčnega snopa \(ℱ\).
  To da je celovita pa pomeni, da imamo za vsak \(n ∈ ℕ\) indeksno množico
  \(Iₙ\) in pokritje \(\{U_{n,i}\}_{i ∈ Iₙ}\), skupaj z lokalnimi prerezi
  \({f_{n,i} ∈ ℱ{\p{U_{n,i}}}}\), tako da velja \( U_{n,i} ⊩ Pₙ{\p{f_{n,i}}}\).

  Obstoj funkcije izbire pa pomeni, da mora obstajati pokritje \(\{Vⱼ\}ⱼ\) in
  zaporedje prerezov \(f_{n,j} ∈ ℱ{\p{Vⱼ}}\), tako da velja \(Vⱼ ⊩ \for{n : ℕ}{Pₙ{\p{f_{n,j}}}}\).

  Za odvisno funkcijo \(φ : ∏_{n ∈ ℕ} Iₙ\) definirajmo množico
  \(U_φ ≔ ⋂_{n ∈ ℕ} U_{n,φ(n)}\), ki je števen presek odprtih množic torej po
  predpostavki odprt. Množice \(U_φ\) pokrijejo prostor, saj je vsak \(x ∈ X\)
  vsebovan v nekem \(U_{n, i}\) za vse \(n ∈ ℕ\) torej po aksiomu števne izbire
  (v metateoriji) obstaja tudi funkcija \(φ\), tako da bo \(x ∈ U_{n, φ{n}}\),
  torej v \(U_φ\).

  Še več, za vse \(n\) in \(φ\) velja
  \[ U_φ ⊆ U_{n,φ(n)} ⊩ Pₙ{\p{f_{n,φ(n)}}}\text. \]

  To pa pomeni, da za obstoj funkcije izbire vzamemo krov \(\{U_φ\}_φ\) in
  preslikavo \(φ ↦ \p{f_{n, φ(n)} ∈ ℱ{p{U_φ}}}ₙ\), ki zadošča želenemu pogoju,
  kot smo pokazali zgoraj.
\end{proof}
\begin{opomba}
  Gornji dokaz ključno uporabi askiom števne izbire v metateoriji.
\end{opomba}


\begin{lema}\label{th:t1-ccv-is-psp}
  Če nad \(T₁\) prostorom velja \(\CCv\), je prostor P-prostor.
\end{lema}
\begin{proof}
  Naj torej velja \(X ⊩ \CCv\) in naj bo \(\{Uₙ\} ⊆ 𝒪X\) števna družina ter
  \(a ∈ ⋂ₙ Uₙ\) poljubna točka. Dokazati želimo, da ta točka leži v notranjosti
  preseka, kar bo pokazalo, da je odprt. Definirajmo interno relacijo \(R\) med
  \(ℕ\) in \(2\) s predpisom
  %\(X ⊩ R(n, b) ⇔ (b = 1 ∧ Uₙ) ∨ (b = 0 ∧ !a)\).
  \[
  R(n, b) ⇔
  \begin{cases}
    Uₙ &; b = 1\\
    !a &; b = 0\text.
  \end{cases}
  \]
  Ker so v \(T₁\) prostorih točke zaprte, je \(!a = X⧵\{a\}\).
  Ker je \(Uₙ∪{!a} = Uₙ∪(X⧵\{a\}) = X\), je ta relacija celovita, torej lahko na
  njej uporabimo dani aksiom izbire. Tako v okolici \(W\nbd a\) dobimo funkcijo
  \({W ⊩ f : ℕ → 2}\), za katero velja \(W ⊩ \for{n : ℕ}{R(n, f(n))}\).

  Ker je pa to geometrijska implikacija, pa velja tudi \(a ⊩ \for{n : ℕ}{R(n, f(n))}\).
  Vendar pa za noben \(n\) ne velja \(a ⊩{!a} (= R(n, 0))\), torej je \(f = 1\).
  Potem pa velja \(a ∈ W ⊆ ⋂ₙ ⟦R(n, f(n))⟧ = ⋂ₙ ⟦R(n, 1)⟧ = ⋂ₙ Uₙ\).
\end{proof}
% \begin{proof}
%   Naj je \(Uₙ\) števno pokritje \(U\) in \(a ∈ ⋂ₙ Uₙ\). Dokazujemo, da obstaja
%   odprta okolica \(a\), ki je vsebovana v preseku.
%   Konstruirajmo množice \(Vₙ ≔ ⋃_{k ≠ n} Uₖ ⧵ \{a\}\) in pokritja \(Cₙ ≔ ↓{\{Uₙ, Vₙ\}}\).
%   Po predpostavki je tudi presek \(C ≔ ⋂ₙ Cₙ\) pokritje, torej za nek \(W ∈ C\)
%   velja \(a ∈ W\). Potem pa za vse \(n\) velja \(W ∈ Cₙ\), torej imamo \(W ⊆ Uₙ\)
%   ali \(W ⊆ Vₙ\).

%   Ker je pa \(a ∈ W\) in \(a ∉ Vₙ\), je potem nujno \(W ⊆ Uₙ\), torej dobimo
%   \(a ∈ W ⊆ ⋂ₙ Uₙ\), kar je natanko kar smo želeli.
% \end{proof}
%\begin{opomba}
%  Gornji dokaz deluje tudi za \(R₀\) prostore, saj lahko \(Vₙ\) definiramo z
%  zaprtjem točke \(a\), in potrebujemo zgolj, da je to vsebovano v vsaki okolici
%  točke \(a\), kar je pa natanko ekvivalentno pogoju \(R₀\). Za občutek, \(T₁\)
%  prostori so natanko \(T₀\) in \(R₀\) prostori.
%\end{opomba}




Gornji lemi lahko združimo, da dobimo
\begin{trditev}
  Nad \(T₁\) prostori je števna izbira ekvivalentna disjunktivni števni izbiri.
\end{trditev}
\begin{opomba}
  V dokazu množica \(ℕ\) ni posebej odlikovana, razen da je enake kardinalnosti
  kot tista, za katero je topologija zaprta za preseke, torej izrek lahko
  posplošimo na izbiro poljubne kardinalnosti.

  Splošnem aksiomu izbire pripadajo torej prostori Aleksandrova, kjer je presek
  poljubne družine odprtih množic odprt.
\end{opomba}


%%% Local Variables:
%%% mode: latex
%%% TeX-master: "main"
%%% End:
