\section{Principi izbire v topoloških modelih}\label{sec:izbire}

Od tu naprej bomo ponovno delali v klasični metateoriji. Čeprav bi bilo lepo
razviti ekvivalence konstruktivno, se izkaže, da v konstruktivni metateoriji
pogosto ne bomo mogli pokazati, da modeli zadoščajo logičnim principom. To nam
nakazuje spodnji izrek.

\begin{trditev}
  Če je \(X\) lokalno povezan, in nad \(X\) velja kateri koli princip izbire,
  velja potem tudi v metateoriji.
\end{trditev}
\begin{dokaz}
  Oglejmo si princip izbire \(\AC_Σ{\p{A, B}}\).
  Za celovito relacijo \({R : A×B → Σ}\) znotraj dobimo
  \[ X ⊩ \for{a : \c A}{\exist{b : \c B}{R(a, b)}}\text. \]
  Potem pa po notranjem aksiomu izbire sčasoma obstaja morfizem
  \(f : \c A ↬ {\c B}\). Ker je prostor lokalno povezan, je to enako kot element
  \(f : \c{B^A}\), kar pa nam da funkcijo \(f : A → B\) zunaj, za katero
  potem velja \(\for{a ∈ A}{R(a, f(a))}\).
\end{dokaz}
\begin{opomba}
  Zgoraj smo lokalno povezanost uporabili zgolj za enakost \({\c B}^{\c A} = \c{B^A}\),
  tako da bi zadoščalo predpostaviti zgolj to (torej, za enaka \(A\) in \(B\)
  kot se pojavita v aksiomu izbire). Enaka opomba bo na mestu za vse izreke o
  lokalni povezanosti v tem razdelku, razen kjer bo eksplicitno omenjeno.
\end{opomba}

Torej, za lokalno povezane modele brez \(\AC(A,B)\) v metateorji ne bomo morali
najti \emph{nobenega} modela, ki bi ta princip validiral. To pa tudi pomeni, da
ne bomo mogli najti ustrezne topološke lastnosti, ki bi temu principu pripadala.

Seveda pa glede na to, da nas zanimajo topološke lastnosti, in se teorija
topologije večinoma razvija klasično, je to še močnejši razlog, zakaj bi
preferirali klasično metateorijo.

Spodnji rezultat je v malo drugačni obliki pokazal Alex Simpson~\cite{Simpson24},
dokaz v primeru topoloških modelov je pa originalen.
\begin{izrek}\label{th:ac-and-conn-is-pgt}
  Nad \(X\) velja \(\AC(\c A, \c B)\) in \(\c A ↬ {\c B} = \c{B^A}\) natanko
  tedaj, ko ima vsaka \(A\)-indeksirana množica \(B\)-indeksiranih
  pokritij \(U\) skupno pofinitev.
  %Nad \(X\) velja \(\AC_Σ{\p{\c A, \c B}}\) in \({\c B}^{\c A} = \c{B^A}\)
  %natanko tedaj, ko je Grothendieckova topologija \(X\) zaprta za \(A\)-mnoge
  %preseke \(B\)-indeksiranih \(Σ\)-krovov.
\end{izrek}
\begin{dokaz}
  \begin{itemize}
  \item[\(\p ⇐\)]
    Pokažimo najprej, da velja gornja enakost.
    Tip \(\c A ↬ \c B\) bo enak tipu \(\c{A → B}\) natanko tedaj, ko je vsak
    morfizem operacija. To velja, saj je vsaka funkcija množic operacija med
    konstantnimi tipi. Naj bo \(f : \c A ↬ \c B\). Potem
    \(Cₐ ≔ \set{f(a,b)}{b∈B}\) tvorijo pokritje \(\e a = X\). Po predpostavki
    imajo skupno pofinitev \(C\), tako da za vsak \(U ∈ C\) obstaja \(b_{a,U}\),
    tako da velja \(U ⊩ f(a,b_{a,U})\). Potem pa lahko na \(U\) definiramo
    preslikavo \(a ↦ b_{a,U}\), ki je enaka \(f\), torej enakost velja.

    Naj bo sedaj \(R\) celovita relacija med \(\c A\) in \(\c B\).
    To pomeni, da za vsak \(a ∈ A\) obstaja \(B\)-indeksirano pokritje \(Cₐ\),
    tako da za vsak element \(U ∈ Cₐ\) obstaja tak \(b_{a, U} ∈ B\), da velja
    \(U ⊩ R(a, b_{a, U})\).

    Po predpostavki sedaj vemo, da imajo te krovi skupno pofinitev \(C\), torej
    za vsak \(U ∈ C\) in \(a ∈ A\) lahko izmed \(b_{a,Uₐ}\), kjer je \(Uₐ\)
    nadmnožica \(U\) v \(Cₐ\), izberemo \(b_{a, U}'\). Potem pa lahko na \(U\)
    definiramo funkcijo tipa \(A → B\), ki slika \(a\) v \(b_{a,U}'\).
  \item[\(\p ⇒\)]
    Naj bodo \(Cₐ = \{U_{a,b}\}_{b ∈ B}\) pokritja.
    Ta definirajo relacijo \(R(a, b) ≔ U_{a,b}\), ki je celovita, saj so \(Cₐ\)
    pokritja. Tedaj po principu izbire znotraj (sčasoma) dobimo funkcijo
    \(f : A → B\), tako da je \(\for{a : A}{U_{a,f(a)}}\).
    Sedaj pa, ker velja enakost iz predpostavke, je \(f\) tudi funkcija
    \(A → B\) zunaj. To pa pomeni, da je \(\e f\) vsebovan v vsakem od
    \(U_{a,f(a)}\), torej, ker \(\e f\) pokrivajo cel prostor, tvorijo skupno
    pofinitev pokritij \(Cₐ\).
  \end{itemize}
\end{dokaz}
%NOTE: brez meta-ac je leva stran lahko za vse celovite \(R\) velja
%\[ \eventually{V⊆U}{\for{a∈A}{\exist{b∈B}{V⊩R(a,b)}}} \]
%NOTE: Mogoče tu rabimo \(\g B\) namesto \(B\).
%NOTE: interno je to \(\for{a:A}{\globalen{b:B}{R(a,b)}}\)
\begin{posledica}
  Nad \(X\) velja \(\AC(\c A)\) in \(\for{B}{\c A ↬ B = \c{A ↝ B}}\) natanko
  tedaj, ko ima vsaka \(A\)-indeksirana množica pokritij \(U\) skupno pofinitev.
\end{posledica}
To ne zares sledi iz gornjega, a je dokaz enak, tako da ga ne ponovimo.

\begin{opomba}
  Gornje lahko relativiziramo še glede na Sierpinskijev objekt \(Σ\), tako da na
  desni vzamemo pokritja, ki so podmnožice \(Σ\), na desni pa moramo omejiti
  morfizme, tako da bo \(f(a,b) ∈ Σ\), in aksiom izbire sam. Dokaz nato poteka
  enako.
\end{opomba}

Ta lastnost o pofinitvah pa ni zares tipična topološka lastost, saj ponavadi
govorimo o pofinitvi specifičnega tipa \emph{enega} pokritja. Definicija izhaja
iz teorije Grothendieckovih topologij, kjer definiramo \emph{krov} kot navzdol
zaprto pokrtije, nato je pa Grothendieckova topologija na \(U\) enaka množici
krovov \(U\). Potem se lastnost zgoraj glasi ``Grothendieckova topologija na
\(X\) je zaprta za \(A\)-mnoge preseke''. To pa že zgleda bolj podobno čemu
znanemu, namreč prostori, katerih topologije so zaprte za števne preseke imajo
ime.

\begin{definicija}\label{def:psp}
  Prostor je \emph{P-prostor}, ko je števen presek odprtih množic odprt.
\end{definicija}

Ta lastnost prostora potem implicira tisto o Grothendieckovi topologiji.

\begin{trditev}\label{th:psp-is-pgt}
  Če je \(X\) P-prostor, ima vsaka množica števno mnogo pokritij \(X\) skupno
  pofinitev.
\end{trditev}
\begin{dokaz}
  Naj bodo \(Cₙ = \{U_{n,i}\}ᵢ\) pokritja. Brez škode za splošnost
  predpostavimo, da so vsa indeksirana z nekim \(I\), saj lahko pokrijim dodamo
  prazno množico.

  Potem naj bo \(φ : ℕ → I\) funkcija. Za to lahko tvorimo
  \(U_φ ≔ ⋂_nU_{n,φ(n)}\). To je števen presek odprtih množic, torej je po
  predpostavki odprta množica. Prav tako je pa vsak \(x ∈ X\) vsebovan v nekem
  \(U_{n,i}\) za vsak \(n ∈ ℕ\), torej po principu izbire v metateoriji obstaja
  preslikava \(φ : ℕ → I\), da je \(x ∈ U_φ\). To pomeni, da \(U_φ\) tvorijo
  pokritje, ki je pofinitev vsakega \(Cₙ\).
\end{dokaz}
\begin{posledica}\label{th:psp-has-cc}
  Nad P-prosotri velja \(\CC\).
\end{posledica}
Seveda se vse to posploši na poljbne kardinalnosti, a v literaturi razen za
števni primer nimamo imena. Poznamo le še prostore Aleksandrova, ki so natanko
prostori, kjer so \emph{vsi} preseki odprtih množic odprti. Ta nam da idejo, da
potem nad temi prostori velja polnomočen \(\AC\), a žal temu ni tako.

Očitno imajo diskretni prostori to lastnost, in kot smo že povedali, so to
natanko prostori, nad katerimi velja \(\lem*\). Ker pa velja aksiom izbire
implicira princip izključene tretje možnosti (ta izrek je znan kot izrek
Diaconescuja~\cite{Bauer16}), mora vsak prostor, nad katerim velja \(\AC\) biti
diskreten.

Očitno torej ne more veljati \(\AC\) nad \emph{vsemi} prostori Aleksandrova, saj
obstajajo taki, ki niso diskretni. In res, gornje nam da zgolj, da za vsako
množico \(A\) velja \(\AC(\c A)\), ne da velja \(\AC(A)\) za vse \emph{tipe}
\(A\). To je torej pomembna razlika, kar je do neke mere tudi za pričakovati.
Zgoraj smo bistveno uporabljali, da so elementi \(\c A\) definirani na istem
nivoju \(X\), torej so tudi pokritja \(Cₐ\) pokritja \(X\). Za splošen tip \(A\)
bodo pa pokritja \(Cₐ\) pokrivala \(\e a\), torej ne bomo mogli niti govoriti o
skupnih pofinitvah.

TODO: a kej vem pol o AC?

Torej gornje pokaže, da ima P-prostor gornjo lastnost na Grothendieckovi
topologiji. Obrat pa ne velja.
\begin{trditev}\label{th:psp-is-not-pgt}
  Nad prostorom \(\cli{0,1}\) s topologijo \(\set{[0,a)}{a ∈ \cli{0,1}}\) ima
  poljubno mnogo pokritij skupno pofinitev, a ta ni P-prostor.
\end{trditev}
\begin{dokaz}
  Množice \(Uₙ ≔ [0,2⁻ⁿ)\) so odprte, a njihov presek je \(\{0\}\), ki ni
  odprta množica, torej prostor ni P-prostor.

  Sedaj pa, če je \(C\) pokritje \(X\) more pokriti \(1\). Ampak edina okolica
  \(1\) je \(\cli{0,1}\), torej je \(\{\cli{0,1}\}\) pofinitev vsakega pokritja
  \(X\).
\end{dokaz}
To pravzaprav pomeni, da nad tem prostorom velja \(\for{A}{\AC(\c A)}\), 
vseeno ni {P-prostor}. Izkaže pa se, da pod predpostavko \(T₁\) lahko pokažemo
obrat.

\begin{trditev}\label{th:t1-pgt-is-psp}
  Če je prostor \(X\) \(T₁\) in so njegove Grothendieckove topologije zaprte za
  števne preseke, je P-prostor.
\end{trditev}
\begin{dokaz}
  Naj bo \(\{Uₙ\}ₙ\) števna družina odprtih množic in za \(t ∈ ⋂ₙUₙ\) tvorimo
  pokritja \(\{Uₙ, X⧵{\{t\}}\}\). Ta so po predpostavki \(T₁\) odprta pokritja,
  torej imajo skupno pofinitev \(C\). Naj \(U ∈ C\) pokriva točko \(t\). Tedaj
  očitno ne velja \(U ⊆ X⧵{\{t\}}\), torej je \(U ⊆ Uₙ\). Potem je pa \(U\)
  odprta okolica \(t\), ki je v celoti vsebovana v \(⋂ₙUₙ\), torej je ta presek
  odprt.
\end{dokaz}
\begin{opomba}
  Vseeno pa to ne pomeni, da so \(T₁\) P-prostori natanko \(T₁\) prostori, nad
  katerimi velja \(\CC\). Namreč, prostor lahko validira \(\CC\), a ne validira
  enakosti iz~\ref{th:ac-and-conn-is-pgt}, torej ne bo P-prostor.
\end{opomba}

Znana je pa delna karakterizacija \(\DC\)~\cite[lema~D4.5.16]{Johnstone02}\cite[trd.~2.2]{HL16}.
\begin{definicija}
  Prostor \(X\) je \emph{ultraparakompakten}, ko ima vsako pokritje prostora
  pofinitev s particijo.
\end{definicija}
\begin{trditev}
  Nad ultraparakompaktnimi prostori velja \(\DC\).
\end{trditev}
\begin{dokaz}
  Denimo, da smo že konstruirali \(xₙ\). Potem po celovitosti množice
  \(\i{R(xₙ,x)}\) tvorijo pokritje, torej po predpostavki obstaja pofinitev s
  particijo \(\{P_α\}_α\). Potem lahko uporabimo aksiom odvisne izbire v
  metateoriji, da za vsak \(α\) izberemo tak \(x_α\), da je \(P_α ⊆ \i{R(xₙ,x_α)}\).
  Potem pa lahko definiramo \(xₙ\) tako, da je na \(P_α\) enak \(x_α\). Ker
  \(P_α\) tvorijo pokritje, in so paroma disjunktni, to dobro definra term tipa
  \(X\).

  Ker smo začeli z globalnim \(x₀\), to definira globalno zaporedje \((xₙ)ₙ\).
\end{dokaz}
\begin{opomba}
  Gornjo trditev lahko preprosto posplošimo na \(\DC_Σ\), tako da od prostora
  zahtevamo, imajo \(Σ\)-pokrtija pofinitve s particijo.
\end{opomba}


% \begin{trditev}
%   Če nad \(X\) velja \(\AC_Σ(\c A, \c B)\) za vse \(B\), velja tudi
%   \(\AC_Σ(A, ℱ)\) za vse \(ℒ\)-množice \(ℱ\).
% \end{trditev}
% \begin{dokaz}
%   Naj bo \(F\) podležna množica \(ℱ\) in \(R : A×F → Σ\) celovita. Potem je tudi
%   celovita kot relacija med \(A\) in \(\c F\). Uporabimo princip izbire, da
%   dobimo preslikavo \(f: A → \c{\g F}\), za katero velja \(\for{a : A}{R(a, f(a))}\).

%   Pokažimo sedaj, da zunaj \(f\) definira \(ℒ\)-morfizem \(A ↬ ℱ\).
%   Če je \(b = b'\) in \(b' = f(a)\),
% \end{dokaz}


%\subsection{Števna izbira in P-prostori}

% \begin{lema}
%   Nad P-prostori velja \(\CC\).
% \end{lema}
% \begin{dokaz}
%   Trditev \(X ⊩ \AC{\p{ℕ, ℱ}}\) pravi, da če je \(P\) celovita relacija med \(ℕ\) in
%   \(ℱ\), potem obstaja funkcija \(f : ℕ → ℱ\), ki je podrelacija \(P\).

%   Naj bo torej \(P\) taka celovita relacija v \(\sh{X}\).
%   Relacija \(P\) je navzven zaporedje globalnih prerezov potenčnega snopa \(ℱ\).
%   To da je celovita pa pomeni, da imamo za vsak \(n ∈ ℕ\) indeksno množico
%   \(Iₙ\) in pokritje \(\{U_{n,i}\}_{i ∈ Iₙ}\), skupaj z lokalnimi prerezi
%   \({f_{n,i} ∈ ℱ{\p{U_{n,i}}}}\), tako da velja \( U_{n,i} ⊩ Pₙ{\p{f_{n,i}}}\).

%   Obstoj funkcije izbire pa pomeni, da mora obstajati pokritje \(\{Vⱼ\}ⱼ\) in
%   zaporedje prerezov \(f_{n,j} ∈ ℱ{\p{Vⱼ}}\), tako da velja
%   \(Vⱼ ⊩ \for{n : ℕ}{Pₙ{\p{f_{n,j}}}}\).

%   Za odvisno funkcijo \(φ : ∏_{n ∈ ℕ} Iₙ\) definirajmo množico
%   \(U_φ ≔ ⋂_{n ∈ ℕ} U_{n,φ(n)}\), ki je števen presek odprtih množic torej po
%   predpostavki odprt. Množice \(U_φ\) pokrijejo prostor, saj je vsak \(x ∈ X\)
%   vsebovan v nekem \(U_{n, i}\) za vse \(n ∈ ℕ\) torej po aksiomu števne izbire
%   (v metateoriji) obstaja tudi funkcija \(φ\), tako da bo \(x ∈ U_{n, φ{n}}\),
%   torej v \(U_φ\).

%   Še več, za vse \(n\) in \(φ\) velja
%   \[ U_φ ⊆ U_{n,φ(n)} ⊩ Pₙ{\p{f_{n,φ(n)}}}\text. \]

%   To pa pomeni, da za obstoj funkcije izbire vzamemo krov \(\{U_φ\}_φ\) in
%   preslikavo \(φ ↦ \p{f_{n, φ(n)} ∈ ℱ{p{U_φ}}}ₙ\), ki zadošča želenemu pogoju,
%   kot smo pokazali zgoraj.
% \end{dokaz}
% \begin{opomba}
%   Gornji dokaz ključno uporabi askiom števne izbire v metateoriji.
% \end{opomba}


% \begin{lema}\label{th:t1-ccv-is-psp}
%   Če nad \(T₁\) prostorom velja \(\CCv\), je prostor P-prostor.
% \end{lema}
% \begin{dokaz}
%   Naj torej velja \(X ⊩ \CCv\) in naj bo \(\{Uₙ\} ⊆ 𝒪X\) števna družina ter
%   \(a ∈ ⋂ₙ Uₙ\) poljubna točka. Dokazati želimo, da ta točka leži v notranjosti
%   preseka, kar bo pokazalo, da je odprt. Definirajmo interno relacijo \(R\) med
%   \(ℕ\) in \(2\) s predpisom
%   %\(X ⊩ R(n, b) ⇔ (b = 1 ∧ Uₙ) ∨ (b = 0 ∧ !a)\).
%   \[
%   R(n, b) ⇔
%   \begin{cases}
%     Uₙ &; b = 1\\
%     !a &; b = 0\text.
%   \end{cases}
%   \]
%   Ker so v \(T₁\) prostorih točke zaprte, je \(!a = X⧵\{a\}\).
%   Ker je \(Uₙ∪{!a} = Uₙ∪(X⧵\{a\}) = X\), je ta relacija celovita, torej lahko na
%   njej uporabimo dani aksiom izbire. Tako v okolici \(W\nbd a\) dobimo funkcijo
%   \({W ⊩ f : ℕ → 2}\), za katero velja \(W ⊩ \for{n : ℕ}{R(n, f(n))}\).

%   Ker je pa to geometrijska implikacija, pa velja tudi \(a ⊩ \for{n : ℕ}{R(n, f(n))}\).
%   Vendar pa za noben \(n\) ne velja \(a ⊩{!a} (= R(n, 0))\), torej je \(f = 1\).
%   Potem pa velja \(a ∈ W ⊆ ⋂ₙ ⟦R(n, f(n))⟧ = ⋂ₙ ⟦R(n, 1)⟧ = ⋂ₙ Uₙ\).
% \end{dokaz}
% % \begin{dokaz}
% %   Naj je \(Uₙ\) števno pokritje \(U\) in \(a ∈ ⋂ₙ Uₙ\). Dokazujemo, da obstaja
% %   odprta okolica \(a\), ki je vsebovana v preseku.
% %   Konstruirajmo množice \(Vₙ ≔ ⋃_{k ≠ n} Uₖ ⧵ \{a\}\) in pokritja \(Cₙ ≔ ↓{\{Uₙ, Vₙ\}}\).
% %   Po predpostavki je tudi presek \(C ≔ ⋂ₙ Cₙ\) pokritje, torej za nek \(W ∈ C\)
% %   velja \(a ∈ W\). Potem pa za vse \(n\) velja \(W ∈ Cₙ\), torej imamo \(W ⊆ Uₙ\)
% %   ali \(W ⊆ Vₙ\).

% %   Ker je pa \(a ∈ W\) in \(a ∉ Vₙ\), je potem nujno \(W ⊆ Uₙ\), torej dobimo
% %   \(a ∈ W ⊆ ⋂ₙ Uₙ\), kar je natanko kar smo želeli.
% % \end{dokaz}
% %\begin{opomba}
% %  Gornji dokaz deluje tudi za \(R₀\) prostore, saj lahko \(Vₙ\) definiramo z
% %  zaprtjem točke \(a\), in potrebujemo zgolj, da je to vsebovano v vsaki okolici
% %  točke \(a\), kar je pa natanko ekvivalentno pogoju \(R₀\). Za občutek, \(T₁\)
% %  prostori so natanko \(T₀\) in \(R₀\) prostori.
% %\end{opomba}




% Gornji lemi lahko združimo, da dobimo
% \begin{trditev}
%   Nad \(T₁\) prostori je števna izbira ekvivalentna disjunktivni števni izbiri.
% \end{trditev}


%%% Local Variables:
%%% mode: latex
%%% TeX-master: "main"
%%% End:
