\section{Principi izbire v topoloških modelih}\label{sec:izbire}

Od tu naprej bomo ponovno delali v klasični metateoriji. Definirajmo najprej
nekaj izrazov o indeksiranosti in pokritjih.
\begin{definicija}
  \emph{Družina množic} je preslikava \(A : I → \setcat\), iz množice \(I\) v
  razred množic \(\setcat\). Pravimo, da je \(A\) \emph{\(I\)-indeksirana}
  družina. Pokritje je \emph{\(I\)-indeksirano}, ko je \(I\)-indeksirano kot
  družina.
\end{definicija}
\begin{definicija}
  Pokritje \(C\) je \emph{pofinitev} pokritja \(D\), ko za vsak \(U ∈ C\)
  obstaja \(U' ∈ D\), da je \(U ⊆ U'\).
  Če je \(𝒟\) družina pokritij, je \(C\) njihova \emph{skupna pofinitev}, ko je
  pokritje \(C\) pofinitev vsakega pokritja \(D ∈ 𝒟\).
\end{definicija}

\begin{izrek}\label{th:ac-and-conn-is-pgt}
  Nad \(X\) velja \(\AC(\c A, \c B)\) in \(\c A ↬ \c B ≅ \c{B^A}\) natanko
  tedaj, ko ima vsaka \(A\)-indeksirana družina \(B\)-indeksiranih
  pokritij \(X\) skupno pofinitev.
  %Nad \(X\) velja \(\AC_Σ{\p{\c A, \c B}}\) in \({\c B}^{\c A} ≅ \c{B^A}\)
  %natanko tedaj, ko je Grothendieckova topologija \(X\) zaprta za \(A\)-mnoge
  %preseke \(B\)-indeksiranih \(Σ\)-krovov.
\end{izrek}
\begin{dokaz}
  \begin{itemize}
  \item[\(\p ⇐\)]
    Pokažimo najprej, da velja gornja enakost.
    Tip \(\c A ↬ \c B\) bo enak tipu \(\c{A → B}\) natanko tedaj, ko je vsak
    morfizem operacija. To velja, saj je vsaka funkcija množic operacija med
    konstantnimi tipi. Naj bo \(f : \c A ↬ \c B\). Potem
    \(Cₐ ≔ \set{f(a,b)}{b∈B}\) tvorijo pokritje \(\e a = X\). Po predpostavki
    imajo skupno pofinitev \(C\), tako da za vsak \(U ∈ C\) obstaja \(b_{a,U}\),
    tako da velja \(U ⊩ f(a,b_{a,U})\). Potem pa lahko na \(U\) definiramo
    preslikavo \(a ↦ b_{a,U}\), ki je enaka \(f\), torej izomorfizem
    \(\c A ↬ \c B ≅ \c{B^A}\) velja.

    Naj bo sedaj \(R\) celovita relacija med \(\c A\) in \(\c B\).
    To pomeni, da za vsak \(a ∈ A\) obstaja \(B\)-indeksirano pokritje \(Cₐ\),
    tako da za vsak element \(U ∈ Cₐ\) obstaja tak \(b_{a, U} ∈ B\), da velja
    \(U ⊩ R(a, b_{a, U})\).

    Po predpostavki vemo, da imajo te krovi skupno pofinitev \(C\), torej
    za vsak \(U ∈ C\) in \(a ∈ A\) lahko izmed \(b_{a,Uₐ}\), kjer je \(Uₐ\)
    nadmnožica \(U\) v \(Cₐ\), izberemo \(b_{a, U}'\). Potem pa lahko na \(U\)
    definiramo funkcijo tipa \(A → B\), ki slika \(a\) v \(b_{a,U}'\).
  \item[\(\p ⇒\)]
    Naj bodo \(Cₐ = \{U_{a,b}\}_{b ∈ B}\) pokritja.
    Znotraj lahko definiramo relacijo \({R(a, b) ≔ U_{a,b}}\), ki je celovita,
    saj so družine \(Cₐ\) pokritja. Tedaj po principu izbire in predpostavki o
    morfizmih obstaja pokritje \(C\), da na vsakem \(V ∈ C\) obstaja
    \emph{funkcija} \({f : A → B}\), za katero velja
    \(V ⊩ \for{a : \c A}{U_{a,f(a)}}\). To pomeni, da je \(C\) skupna pofinitev
    pokritij \(Cₐ\), kar zaključi dokaz.
  \end{itemize}
\end{dokaz}
%NOTE: brez meta-ac je leva stran lahko za vse celovite \(R\) velja
%\[ \eventually{V⊆U}{\for{a∈A}{\exist{b∈B}{V⊩R(a,b)}}} \]
%NOTE: Mogoče tu rabimo \(\g B\) namesto \(B\).
%NOTE: interno je to \(\for{a:A}{\globalen{b:B}{R(a,b)}}\)
\begin{trditev}\label{th:ac-and-conn-is-pgt-2}
  Nad \(X\) velja \(\AC(\c A)\) in \(\for{B}{\c A ↬ B ≅ \c{A ↝ B}}\) natanko
  tedaj, ko ima vsaka \(A\)-indeksirana množica pokritij skupno pofinitev.
\end{trditev}
Dokaz te trditve poteka enako kot prejšnji, tako da ga ne ponovimo.

Ta rezultat je na primeru \(A = ℕ\) za splošne Grothendieckove topose pokazal
Alex Simpson~\cite{Simpson24}, dokaz v primeru topoloških modelov je pa
originalen.

\begin{opomba}
  Trditvi~\ref{th:ac-and-conn-is-pgt} in \ref{th:ac-and-conn-is-pgt-2} je mogoče
  relativizirati glede na Sierpinskijev objekt \(Σ\), kot smo to zastavili v
  podpoglavju~\ref{sec:logika-izbire}, a nastanejo zapleti pri relativizaciji
  pogoja \(\c A ↬ \c B ≅ \c{B^A}\), tako da tega tu eksplicitno ne navedemo.
\end{opomba}

Do konca tega poglavja označimo lastnost,
"vsaka \(A\)-indeksirana družina pokritij ima skupno pofinitev"
z \((*)\).
Ta ni tipična topološka lastnost, saj ponavadi govorimo o pofinitvi \emph{enega}
pokritja. Definicija izhaja iz teorije Grothendieckovih topologij, kjer
definiramo \emph{krov} kot navzdol zaprto pokritje, nato je pa Grothendieckova
topologija na \(U\) enaka množici krovov \(U\). Potem se lastnost \((*)\) glasi
``Grothendieckova topologija na \(X\) je zaprta za \(A\)-indeksirane preseke''.
To je že bolj podobno čemu znanemu, saj imajo prostori, katerih topologije so
zaprte za števne preseke, ime.

\begin{definicija}\label{def:psp}
  Prostor je \emph{P-prostor}, ko je števen presek odprtih množic odprt.
\end{definicija}

Ta lastnost prostora implicira lastnost \((*)\) za števne indeksne množice.

\begin{trditev}\label{th:psp-is-pgt}
  Če je \(X\) P-prostor, ima vsaka števna družina pokritij \(X\) skupno
  pofinitev.
\end{trditev}
\begin{dokaz}
  Naj bodo \(Cₙ = \{U_{n,i}\}ᵢ\) pokritja \(U\). Brez škode za splošnost
  predpostavimo, da so vsa indeksirana z nekim \(I\), saj lahko pokrijem dodamo
  prazno množico.

  Potem naj bo \(φ : ℕ → I\) funkcija. Za to lahko tvorimo
  \(U_φ ≔ ⋂_nU_{n,φ(n)}\). To je števen presek odprtih množic, torej je po
  predpostavki odprta množica. Prav tako je pa vsak \(x ∈ X\) vsebovan v nekem
  \(U_{n,i}\) za vsak \(n ∈ ℕ\), torej po principu izbire v metateoriji obstaja
  preslikava \(φ : ℕ → I\), da je \(x ∈ U_φ\). To pomeni, da \(U_φ\) tvorijo
  pokritje, ki je pofinitev vsakega \(Cₙ\).
\end{dokaz}
\begin{posledica}\label{th:psp-has-cc}
  Nad P-prostori velja \(\CC\).
\end{posledica}
Seveda se vse to posploši na poljubne kardinalnosti, a v literaturi razen za
števni primer nimamo imena. Poznamo le še prostore Aleksandrova, ki so natanko
prostori, kjer so \emph{vsi} preseki odprtih množic odprti. Ta nam da idejo, da
morda nad temi prostori velja polnomočen \(\AC\), a žal temu ni tako.

Diskretni prostori so očitno tudi P-prostori, in kot smo že povedali, so to
natanko prostori, nad katerimi velja \(\lem*\). Ker aksiom izbire implicira
princip izključene tretje možnosti (ta izrek je znan kot izrek
Diaconescuja~\cite{Diaconescu75}), mora biti vsak prostor, nad katerim velja
\(\AC\), diskreten.

Očitno torej \(\AC\) ne more veljati nad \emph{vsemi} prostori Aleksandrova, saj
niso vsi diskretni. In res, posledica~\ref{th:psp-has-cc}, posplošena na
poljubne kardinalnosti, nam pove zgolj, da za vsako množico \(A\) velja
\(\AC(\c A)\), ne pa da velja \(\AC(A)\) za vse \emph{tipe} \(A\). To je torej
pomembna razlika, kar je do neke mere tudi za pričakovati. Zgoraj smo bistveno
uporabljali, da imajo elementi \(\c A\) razpon enak \(X\), torej so tudi \(Cₐ\)
pokritja \(X\). Za splošen tip \(A\) bodo pokritja \(Cₐ\) pokrivala \(\e a\),
torej ne moremo niti govoriti o skupnih pofinitvah.
Vseeno pa lahko nad diskretnimi prostori pokažemo, da velja \(\AC\).
\begin{trditev}\label{th:disc-has-ac}
  Če je \(X\) diskreten prostor, nad njem velja \(\AC\).
\end{trditev}
\begin{dokaz}
  Naj bosta \(A\) in \(B\) tipa in \(R : A×B → Ω\) celovita relacija. Ker je
  prostor diskreten, je \(\e aᶜ\) odprta množica in je \(\i{\e a ⇒ R(a,b)}\)
  enak \(\e aᶜ ∪ R(a,b)\). Relacija \(Q(a,b) ≔ \e aᶜ∪R(a,b)\) med \(\c A\) in
  \(B\) je potem očitno celovita, in lahko na njej uporabimo \(\AC(\c A)\), ki
  velja zaradi diskretnosti prostora. Imamo torej morfizem \(f : \c A ↬ B\), za
  katerega velja \(\for{x : \c A}{Q(x,f(x))}\). Definirajmo sedaj \(g : A ↬ B\)
  s predpisom \(\p{a,b} ↦ ⋃_{x∈A}\i{x = a}∩f(x,b)\). S previdnim računom lahko
  preverimo, da velja \(\for{x : A}{R(x,g(x))}\), torej je \(g\) funkcija izbire
  za \(R\) in \(\AC\) drži.
\end{dokaz}
Tako lahko združimo trditvi~\ref{th:lem-is-discrete} in~\ref{th:disc-has-ac},
skupaj z izrekom Diaconescuja, da dobimo sledeči izrek:
\begin{izrek}\label{th:ac-is-lem}
  V topoloških modelih je \(\AC\) ekvivalenten \(\lem*\).
\end{izrek}
To je presenetljivo, saj je običajno \(\AC\) bistveno močnejši od izključene
tretje možnosti. To nam nakazuje, da so v topoloških modelih principi odločitve
bolj bistveni od principov izbire.

V trditvi~\ref{th:psp-is-pgt} smo pokazali, da imajo P-prostori lastnost \((*)\).
Obrat ne velja.
\begin{trditev}\label{th:psp-is-not-pgt}
  Nad prostorom \(\cli{0,1}\) s topologijo \(\set{[0,a)}{a ∈ \cli{0,1}} ∪ \cli{0,1}\)
  ima vsaka družina pokritij skupno pofinitev, a ta ni P-prostor.
\end{trditev}
\begin{dokaz}
  Množice \(Uₙ ≔ [0,2⁻ⁿ)\) so odprte, a njihov presek je \(\{0\}\), ki ni
  odprta množica, torej prostor ni P-prostor. Vseeno pa mora vsako pokritje
  \(X\) pokriti \(1\). Ampak edina okolica \(1\) je \(\cli{0,1}\), torej je
  \(\{\cli{0,1}\}\) pofinitev vsakega pokritja \(X\).
\end{dokaz}
To pravzaprav pomeni, da nad tem prostorom velja \(\for{A}{\AC(\c A)}\), 
vseeno ni {P-prostor}. Izkaže pa se, da pod predpostavko \(T₁\) lahko pokažemo
obrat.

\begin{trditev}\label{th:t1-pgt-is-psp}
  Če je prostor \(X\) \(T₁\) in ima vsaka števna družina pokritij \(X\) skupno
  pofinitev, je P-prostor.
\end{trditev}
\begin{dokaz}
  Naj bo \(\{Uₙ\}ₙ\) števna družina odprtih množic in za \(t ∈ ⋂ₙUₙ\) tvorimo
  pokritja \(\{Uₙ, X⧵{\{t\}}\}\). Ta so po predpostavki \(T₁\) odprta pokritja,
  torej imajo skupno pofinitev \(C\). Naj \(U ∈ C\) pokriva točko \(t\). Tedaj
  očitno ne velja \(U ⊆ X⧵{\{t\}}\), torej je \(U ⊆ Uₙ\). Potem je pa \(U\)
  odprta okolica \(t\), ki je v celoti vsebovana v \(⋂ₙUₙ\), torej je ta presek
  odprt.
\end{dokaz}
\begin{opomba}
  Vendar pa ne moremo sklepati, da so \(T₁\) P-prostori natanko tisti \(T₁\)
  prostori, nad katerimi velja \(\CC\). Prostor namreč lahko validira \(\CC\) in
  ne \({ℕ ↬ B ≅ \c{ℕ ↝ B}}\), torej ni P-prostor. Primer takega prostora je
  podprostor \(\{0\}∪\set{2⁻ⁿ}{n ∈ ℕ} ⊆ ℝ\).
\end{opomba}

Znana je tudi delna karakterizacija \(\DC\)~\cite[lema~D4.5.16]{Johnstone02}\cite[trd.~2.2]{HL16}.
\begin{definicija}
  Prostor \(X\) je \emph{ultraparakompakten}, ko ima vsako pokritje prostora
  pofinitev s particijo.
\end{definicija}
\begin{trditev}
  Nad ultraparakompaktnimi prostori velja \(\DC\).
\end{trditev}
\begin{dokaz}
  Denimo, da smo že konstruirali \(xₙ\). Ker je \(R\) celovita, množice
  \(\i{R(xₙ,x)}\) tvorijo pokritje, torej po predpostavki obstaja pofinitev s
  particijo \(\{P_α\}_α\). Potem lahko uporabimo aksiom odvisne izbire v
  metateoriji, da za vsak \(α\) izberemo tak \(x_α\), da je \(P_α ⊆ \i{R(xₙ,x_α)}\).
  Element \(xₙ\) definiramo tako, da je na \(P_α\) enak \(x_α\). Ker množice
  \(P_α\) tvorijo pokritje, in so paroma disjunktne, to dobro definira term tipa
  \(X\). Ker smo začeli z globalnim \(x₀\), to definira globalno zaporedje
  \((xₙ)ₙ\).
\end{dokaz}
\begin{opomba}
  Gornjo trditev lahko preprosto posplošimo na \(\DC_Σ\), tako da od prostora
  zahtevamo, imajo \(Σ\)-pokritja pofinitve s particijo.
\end{opomba}


% \begin{trditev}
%   Če nad \(X\) velja \(\AC_Σ(\c A, \c B)\) za vse \(B\), velja tudi
%   \(\AC_Σ(A, ℱ)\) za vse \(ℒ\)-množice \(ℱ\).
% \end{trditev}
% \begin{dokaz}
%   Naj bo \(F\) podležna množica \(ℱ\) in \(R : A×F → Σ\) celovita. Potem je tudi 
%   celovita kot relacija med \(A\) in \(\c F\). Uporabimo princip izbire, da
%   dobimo preslikavo \(f: A → \c{\g F}\), za katero velja \(\for{a : A}{R(a, f(a))}\).

%   Pokažimo sedaj, da zunaj \(f\) definira \(ℒ\)-morfizem \(A ↬ ℱ\).
%   Če je \(b = b'\) in \(b' = f(a)\),
% \end{dokaz}


%\subsection{Števna izbira in P-prostori}

% \begin{lema}
%   Nad P-prostori velja \(\CC\).
% \end{lema}
% \begin{dokaz}
%   Trditev \(X ⊩ \AC{\p{ℕ, ℱ}}\) pravi, da če je \(P\) celovita relacija med \(ℕ\) in
%   \(ℱ\), potem obstaja funkcija \(f : ℕ → ℱ\), ki je podrelacija \(P\).

%   Naj bo torej \(P\) taka celovita relacija v \(\sh{X}\).
%   Relacija \(P\) je navzven zaporedje globalnih prerezov potenčnega snopa \(ℱ\).
%   To da je celovita pa pomeni, da imamo za vsak \(n ∈ ℕ\) indeksno množico
%   \(Iₙ\) in pokritje \(\{U_{n,i}\}_{i ∈ Iₙ}\), skupaj z lokalnimi prerezi
%   \({f_{n,i} ∈ ℱ{\p{U_{n,i}}}}\), tako da velja \( U_{n,i} ⊩ Pₙ{\p{f_{n,i}}}\).

%   Obstoj funkcije izbire pa pomeni, da mora obstajati pokritje \(\{Vⱼ\}ⱼ\) in
%   zaporedje prerezov \(f_{n,j} ∈ ℱ{\p{Vⱼ}}\), tako da velja
%   \(Vⱼ ⊩ \for{n : ℕ}{Pₙ{\p{f_{n,j}}}}\).

%   Za odvisno funkcijo \(φ : ∏_{n ∈ ℕ} Iₙ\) definirajmo množico
%   \(U_φ ≔ ⋂_{n ∈ ℕ} U_{n,φ(n)}\), ki je števen presek odprtih množic torej po
%   predpostavki odprt. Množice \(U_φ\) pokrijejo prostor, saj je vsak \(x ∈ X\)
%   vsebovan v nekem \(U_{n, i}\) za vse \(n ∈ ℕ\) torej po aksiomu števne izbire
%   (v metateoriji) obstaja tudi funkcija \(φ\), tako da bo \(x ∈ U_{n, φ{n}}\),
%   torej v \(U_φ\).

%   Še več, za vse \(n\) in \(φ\) velja
%   \[ U_φ ⊆ U_{n,φ(n)} ⊩ Pₙ{\p{f_{n,φ(n)}}}\text. \]

%   To pa pomeni, da za obstoj funkcije izbire vzamemo krov \(\{U_φ\}_φ\) in
%   preslikavo \(φ ↦ \p{f_{n, φ(n)} ∈ ℱ{p{U_φ}}}ₙ\), ki zadošča želenemu pogoju,
%   kot smo pokazali zgoraj.
% \end{dokaz}
% \begin{opomba}
%   Gornji dokaz ključno uporabi askiom števne izbire v metateoriji.
% \end{opomba}


% \begin{lema}\label{th:t1-ccv-is-psp}
%   Če nad \(T₁\) prostorom velja \(\CCv\), je prostor P-prostor.
% \end{lema}
% \begin{dokaz}
%   Naj torej velja \(X ⊩ \CCv\) in naj bo \(\{Uₙ\} ⊆ 𝒪X\) števna družina ter
%   \(a ∈ ⋂ₙ Uₙ\) poljubna točka. Dokazati želimo, da ta točka leži v notranjosti
%   preseka, kar bo pokazalo, da je odprt. Definirajmo interno relacijo \(R\) med
%   \(ℕ\) in \(2\) s predpisom
%   %\(X ⊩ R(n, b) ⇔ (b = 1 ∧ Uₙ) ∨ (b = 0 ∧ !a)\).
%   \[
%   R(n, b) ⇔
%   \begin{cases}
%     Uₙ &; b = 1\\
%     !a &; b = 0\text.
%   \end{cases}
%   \]
%   Ker so v \(T₁\) prostorih točke zaprte, je \(!a = X⧵\{a\}\).
%   Ker je \(Uₙ∪{!a} = Uₙ∪(X⧵\{a\}) = X\), je ta relacija celovita, torej lahko na
%   njej uporabimo dani aksiom izbire. Tako v okolici \(W\nbd a\) dobimo funkcijo
%   \({W ⊩ f : ℕ → 2}\), za katero velja \(W ⊩ \for{n : ℕ}{R(n, f(n))}\).

%   Ker je pa to geometrijska implikacija, pa velja tudi \(a ⊩ \for{n : ℕ}{R(n, f(n))}\).
%   Vendar pa za noben \(n\) ne velja \(a ⊩{!a} (= R(n, 0))\), torej je \(f = 1\).
%   Potem pa velja \(a ∈ W ⊆ ⋂ₙ ⟦R(n, f(n))⟧ = ⋂ₙ ⟦R(n, 1)⟧ = ⋂ₙ Uₙ\).
% \end{dokaz}
% % \begin{dokaz}
% %   Naj je \(Uₙ\) števno pokritje \(U\) in \(a ∈ ⋂ₙ Uₙ\). Dokazujemo, da obstaja
% %   odprta okolica \(a\), ki je vsebovana v preseku.
% %   Konstruirajmo množice \(Vₙ ≔ ⋃_{k ≠ n} Uₖ ⧵ \{a\}\) in pokritja \(Cₙ ≔ ↓{\{Uₙ, Vₙ\}}\).
% %   Po predpostavki je tudi presek \(C ≔ ⋂ₙ Cₙ\) pokritje, torej za nek \(W ∈ C\)
% %   velja \(a ∈ W\). Potem pa za vse \(n\) velja \(W ∈ Cₙ\), torej imamo \(W ⊆ Uₙ\)
% %   ali \(W ⊆ Vₙ\).

% %   Ker je pa \(a ∈ W\) in \(a ∉ Vₙ\), je potem nujno \(W ⊆ Uₙ\), torej dobimo
% %   \(a ∈ W ⊆ ⋂ₙ Uₙ\), kar je natanko kar smo želeli.
% % \end{dokaz}
% %\begin{opomba}
% %  Gornji dokaz deluje tudi za \(R₀\) prostore, saj lahko \(Vₙ\) definiramo z
% %  zaprtjem točke \(a\), in potrebujemo zgolj, da je to vsebovano v vsaki okolici
% %  točke \(a\), kar je pa natanko ekvivalentno pogoju \(R₀\). Za občutek, \(T₁\)
% %  prostori so natanko \(T₀\) in \(R₀\) prostori.
% %\end{opomba}




% Gornji lemi lahko združimo, da dobimo
% \begin{trditev}
%   Nad \(T₁\) prostori je števna izbira ekvivalentna disjunktivni števni izbiri.
% \end{trditev}


%%% Local Variables:
%%% mode: latex
%%% TeX-master: "main"
%%% End:
