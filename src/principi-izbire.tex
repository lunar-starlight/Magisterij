\section{Principi izbire}


\subsection{Števna izbira in P-prostori}

\begin{definicija}
  \emph{P-prostori} so prostori, za katere velja, da je števen presek odprtih
  množic odprt.
\end{definicija}

% TODO: factor this through GT(X) σ-closed
\begin{lema}\label{th:psp-has-cc}
  Nad P-prostori velja \(\CC\).
\end{lema}
\begin{proof}
  Trditev \(X ⊩ \AC{\p{ℕ, ℱ}}\) pravi, da če je \(P\) celovita relacija med \(ℕ\) in
  \(ℱ\), potem obstaja funkcija \(f : ℕ → ℱ\), ki je podrelacija \(P\).

  Naj bo torej \(P\) taka celovita relacija v \(\sh{X}\).
  Relacija \(P\) je navzven zaporedje globalnih prerezov potenčnega snopa \(ℱ\).
  To da je celovita pa pomeni, da imamo za vsak \(n ∈ ℕ\) indeksno množico
  \(Iₙ\) in pokritje \(\{U_{n,i}\}_{i ∈ Iₙ}\), skupaj z lokalnimi prerezi
  \({f_{n,i} ∈ ℱ{\p{U_{n,i}}}}\), tako da velja \( U_{n,i} ⊩ Pₙ{\p{f_{n,i}}}\).

  Obstoj funkcije izbire pa pomeni, da mora obstajati pokritje \(\{Vⱼ\}ⱼ\) in
  zaporedje prerezov \(f_{n,j} ∈ ℱ{\p{Vⱼ}}\), tako da velja \(Vⱼ ⊩ \for{n : ℕ}{Pₙ{\p{f_{n,j}}}}\).

  Za odvisno funkcijo \(φ : ∏_{n ∈ ℕ} Iₙ\) definirajmo množico
  \(U_φ ≔ ⋂_{n ∈ ℕ} U_{n,φ(n)}\), ki je števen presek odprtih množic torej po
  predpostavki odprt. Množice \(U_φ\) pokrijejo prostor, saj je vsak \(x ∈ X\)
  vsebovan v nekem \(U_{n, i}\) za vse \(n ∈ ℕ\) torej po aksiomu števne izbire
  (v metateoriji) obstaja tudi funkcija \(φ\), tako da bo \(x ∈ U_{n, φ{n}}\),
  torej v \(U_φ\).

  Še več, za vse \(n\) in \(φ\) velja
  \[ U_φ ⊆ U_{n,φ(n)} ⊩ Pₙ{\p{f_{n,φ(n)}}}\text. \]

  To pa pomeni, da za obstoj funkcije izbire vzamemo krov \(\{U_φ\}_φ\) in
  preslikavo \(φ ↦ \p{f_{n, φ(n)} ∈ ℱ{p{U_φ}}}ₙ\), ki zadošča želenemu pogoju,
  kot smo pokazali zgoraj.
\end{proof}
\begin{opomba}
  Gornji dokaz ključno uporabi askiom števne izbire v metateoriji.
\end{opomba}


\begin{lema}\label{th:t1-ccv-is-psp}
  Če nad \(T₁\) prostorom velja \(\CCv\), je prostor P-prostor.
\end{lema}
\begin{proof}
  Naj torej velja \(X ⊩ \CCv\) in naj bo \(\{Uₙ\} ⊆ 𝒪X\) števna družina ter
  \(a ∈ ⋂ₙ Uₙ\) poljubna točka. Dokazati želimo, da ta točka leži v notranjosti
  preseka, kar bo pokazalo, da je odprt. Definirajmo interno relacijo \(R\) med
  \(ℕ\) in \(2\) s predpisom
  %\(X ⊩ R(n, b) ⇔ (b = 1 ∧ Uₙ) ∨ (b = 0 ∧ !a)\).
  \[
  R(n, b) ⇔
  \begin{cases}
    Uₙ &; b = 1\\
    !a &; b = 0\text.
  \end{cases}
  \]
  Ker so v \(T₁\) prostorih točke zaprte, je \(!a = X⧵\{a\}\).
  % preslikavo
  %\[ Pₐ(x)(n,ν) ≔
  %  \begin{cases}
  %    Uₙ &, ν = 1\\
  %    Vₙ &, ν = 0\text.
  %  \end{cases}
  %\]
  %Ker je \(Pₐ\) konstanten v \(x\) pišemo kar \(Pₐ ≔ Pₐ(x)\).

  %Ker je \(Uₙ ∪ Vₙ = Uₙ ∪ ⋃_{k ≠ n} Uₖ ⧵ \{a\} = ⋃ₙ Uₙ = U\) in po konstrukciji
  %Ker je \(Uₙ ∪ Vₙ = U\) in po konstrukciji veljata \(Uₙ ⊩ P(n, 1)\) in
  %\(Vₙ ⊩ Pₐ(n, 0)\), je \(Pₐ\) na \(U\) celovita, torej lahko na \(Pₐ\) uporabimo
  %dani aksiom izbire. Tedaj dobimo pokritje \(Wᵢ ⊆ U\) in funkcije \(fᵢ : Wᵢ → 2^ℕ\),
  %tako da za vsak \(n ∈ ℕ\) velja \(Wᵢ ⊩ Pₐ(n, fᵢ(n))\).
  Ker je \(Uₙ∪{!a} = Uₙ∪(X⧵\{a\}) = X\), je ta relacija celovita, torej lahko na
  njej uporabimo dani aksiom izbire. Tako v okolici \(W\nbd a\) dobimo funkcijo
  \({W ⊩ f : ℕ → 2}\), za katero velja \(W ⊩ \for{n : ℕ}{R(n, f(n))}\).

  Ker je pa to geometrijska implikacija, pa velja tudi \(a ⊩ \for{n : ℕ}{R(n, f(n))}\).
  Vendar pa za noben \(n\) ne velja \(a ⊩{!a} (= R(n, 0))\), torej je \(f = 1\).
  Potem pa velja \(a ∈ W ⊆ ⋂ₙ ⟦R(n, f(n))⟧ = ⋂ₙ ⟦R(n, 1)⟧ = ⋂ₙ Uₙ\).
  % Ampak tedaj mora biti \(Wᵢ ⊆ ⋂ₙ Pₐ{\p{n, fᵢ(x)(n)}}\) za vse \(x ∈ Wᵢ\).
  % Sedaj pa, ker je \(a ∈ U\) mora biti tudi element nekega \(Wᵢ\), saj je le to
  % pokritje, ampak \(a\) ni element \(Pₐ(n, 0)\) za noben \(n ∈ ℕ\), torej je
  % \(fᵢ\) nujno identična \(1\), in \(a ∈ Wᵢ ⊆ ⋂ₙ Uₙ\). To pokaže, da ima vsak
  % element preseka neko odprto okolico, torej je presek odprt, kar pokaže, da je
  % \(X\) P-prostor.
  % Ampak tedaj mora biti \(W ⊆ ⋂ₙ ⟦n R f(n)⟧\). Ker je \(a ∈ W\) je \(a\)
  % tudi element \(⟦n R f(n)⟧\) za vse \(n\). Sedaj pa, ker \(a\) ni element
  % \(⟦n R 0⟧\) za noben \(n\), je \(f = 1\) in je \(a ∈ W ⊆ ⋂ₙ Uₙ\). To pokaže,
  % da je presek odprt, kar pomeni, da je \(X\) P-prostor.
\end{proof}
% \begin{proof}
%   Naj je \(Uₙ\) števno pokritje \(U\) in \(a ∈ ⋂ₙ Uₙ\). Dokazujemo, da obstaja
%   odprta okolica \(a\), ki je vsebovana v preseku.
%   Konstruirajmo množice \(Vₙ ≔ ⋃_{k ≠ n} Uₖ ⧵ \{a\}\) in pokritja \(Cₙ ≔ ↓{\{Uₙ, Vₙ\}}\).
%   Po predpostavki je tudi presek \(C ≔ ⋂ₙ Cₙ\) pokritje, torej za nek \(W ∈ C\)
%   velja \(a ∈ W\). Potem pa za vse \(n\) velja \(W ∈ Cₙ\), torej imamo \(W ⊆ Uₙ\)
%   ali \(W ⊆ Vₙ\).

%   Ker je pa \(a ∈ W\) in \(a ∉ Vₙ\), je potem nujno \(W ⊆ Uₙ\), torej dobimo
%   \(a ∈ W ⊆ ⋂ₙ Uₙ\), kar je natanko kar smo želeli.
% \end{proof}
%\begin{opomba}
%  Gornji dokaz deluje tudi za \(R₀\) prostore, saj lahko \(Vₙ\) definiramo z
%  zaprtjem točke \(a\), in potrebujemo zgolj, da je to vsebovano v vsaki okolici
%  točke \(a\), kar je pa natanko ekvivalentno pogoju \(R₀\). Za občutek, \(T₁\)
%  prostori so natanko \(T₀\) in \(R₀\) prostori.
%\end{opomba}




Gornji lemi lahko združimo, da dobimo
\begin{trditev}
  Nad \(T₁\) prostori je števna izbira ekvivalentna disjunktivni števni izbiri.
\end{trditev}
\begin{opomba}
  V dokazu množica \(ℕ\) ni posebej odlikovana, razen da je enake kardinalnosti
  kot tista, za katero je topologija zaprta za preseke, torej izrek lahko
  posplošimo na izbiro poljubne kardinalnosti.

  Splošnem aksiomu izbire pripadajo torej prostori Aleksandrova, kjer je presek
  poljubne družine odprtih množic odprt.
\end{opomba}


%%% Local Variables:
%%% mode: latex
%%% TeX-master: "main"
%%% End:
