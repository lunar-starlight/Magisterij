\section{Principi izbire}

\subsection{\(\CC\) in \(\CCv\) nad \(T₁\) prostori}

% TODO: think if this just fits better in previous section anyways
\begin{lema}\label{th:psp-has-cc}
  Nad P-prostori velja \(\CC\).
\end{lema}
\begin{proof}
  Trditev \(X ⊩ \AC{\p{ℕ, ℱ}}\) pravi, da če je \(P\) celovita relacija med \(ℕ\) in
  \(ℱ\), potem obstaja funkcija \(f : ℕ → ℱ\), ki je podrelacija \(P\).

  Naj bo torej \(P\) taka celovita relacija v \(\sh{X}\).
  Relacija \(P\) je navzven zaporedje globalnih prerezov potenčnega snopa \(ℱ\).
  To da je celovita pa pomeni, da imamo za vsak \(n ∈ ℕ\) indeksno množico
  \(Iₙ\) in pokritje \(\{U_{n,i}\}_{i ∈ Iₙ}\), skupaj z lokalnimi prerezi
  \({f_{n,i} ∈ ℱ{\p{U_{n,i}}}}\), tako da velja \( U_{n,i} ⊩ Pₙ{\p{f_{n,i}}}\).

  Obstoj funkcije izbire pa pomeni, da mora obstajati pokritje \(\{Vⱼ\}ⱼ\) in
  zaporedje prerezov \(f_{n,j} ∈ ℱ{\p{Vⱼ}}\), tako da velja \(Vⱼ ⊩ \for{n : ℕ}{Pₙ{\p{f_{n,j}}}}\).

  Za odvisno funkcijo \(φ : ∏_{n ∈ ℕ} Iₙ\) definirajmo množico
  \(U_φ ≔ ⋂_{n ∈ ℕ} U_{n,φ(n)}\), ki je števen presek odprtih množic torej po
  predpostavki odprt. Množice \(U_φ\) pokrijejo prostor, saj je vsak \(x ∈ X\)
  vsebovan v nekem \(U_{n, i}\) za vse \(n ∈ ℕ\) torej po aksiomu števne izbire
  (v metateoriji) obstaja tudi funkcija \(φ\), tako da bo \(x ∈ U_{n, φ{n}}\),
  torej v \(U_φ\).

  Še več, za vse \(n\) in \(φ\) velja
  \[ U_φ ⊆ U_{n,φ(n)} ⊩ Pₙ{\p{f_{n,φ(n)}}}\text. \]

  To pa pomeni, da za obstoj funkcije izbire vzamemo krov \(\{U_φ\}_φ\) in
  preslikavo \(φ ↦ \p{f_{n, φ(n)} ∈ ℱ{p{U_φ}}}ₙ\), ki zadošča želenemu pogoju,
  kot smo pokazali zgoraj.
\end{proof}
\begin{opomba}
  Gornji dokaz ključno uporabi askiom števne izbire v metateoriji.
\end{opomba}

Gornjo lemo lahko združimo z Lemo~\ref{th:t1-ccv-is-psp}, da dobimo
\begin{trditev}
  Nad \(R₀\) prostori je števna izbira ekvivalentna disjunktivni števni izbiri.
\end{trditev}
\begin{opomba}
  V dokazu množica \(ℕ\) ni posebej odlikovana, razen da je enake kardinalnosti
  kot tista, za katero je topologija zaprta za preseke, torej izrek lahko
  posplošimo na izbiro poljubne kardinalnosti.

  Splošnem aksiomu izbire pripadajo torej prostori Aleksandrova, kjer je presek
  poljubne družine odprtih množic odprt.
\end{opomba}


%%% Local Variables:
%%% mode: latex
%%% TeX-master: "main"
%%% End:
