\section{Dedekindova in Cauchyjeva realna števila v topoloških modelih}


Najprej si oglejmo klasičen dokaz ekvivalence Dedekindovih in Cauchyjevih
realnih števil.
\begin{izrek}[Klasični]
  Pokazati je zgolj potrebno, da ima vsak Dedekindov rez pripadajoče Cauchyjevo
  zaporedje.
  Naj bo \(\p{L, U}\) obojestranski dedekindov rez. Cauchyjevo zaporedje lahko
  podamo kot zaporedje hitro padajočih racionalnih intervalov.

  Naj bosta \(p₀ ∈ L\) in \(q₀ ∈ U\) racionalni števili.
  Potem pa na \(n\)-tem koraku definiramo
  \[ a ≔ \frac{2pₙ + qₙ}{3}\text,\quad b ≔ \frac{pₙ + 2qₙ}{3}\text{, in}\quad
     \p{pₙ₊₁, qₙ₊₁} ≔ \begin{cases}
       \p{a, qₙ} ;& a ∈ L\\
       \p{pₙ, b} ;& b ∈ U\text.
     \end{cases}
  \]
  Te intervali hitro konvergirajo proti \(\p{L, U}\), torej je to želeno
  Cauchyjevo število.
\end{izrek}

Gornji izrek naredi števno mnogo odločitev, ko se odločamo, če velja \(a ∈ L\)
ali \(b ∈ U\) (oziroma ali velja \(a < x\) ali \(x < b\)), torej dokaz ni
konstruktiven.
Znana sta dva nekonstruktivna principa, ki sta zadostna za gornji odkaz in sta
šibkejša od izključene tretje možnosti. To sta \(\alpo*\) ter \(\CCv\).
Če velja \(\alpo*\) je potem \(a < x\) odločljivo, torej lahko vnaprej popravimo
drugi primer na \(x < b ∧ ¬\p{a < x}\).

Če pa imamo na voljo \(\CCv\) pa preprosto lahko naredimo števno mnogo
odločitev. Zares potrebujemo tu zgolj \(\CCv_{Σ_ℝ}\), kar je
pa tudi posledica \(\alpo*\).

Čeprav je iz tega očitno, da niti \(\alpo*\) niti \(\CCv\) nista potrebna za
\(\Rd = \Rc\), jih vseeno poskusimo strogo ločiti.

\subsection{\(\Rd{} = \Rc\)}

\begin{konstrukcija}
  Nad \(\Ncof\) velja \(\Rd = \Rc\), a ne velja princip števne
  disjunktivne izbire.
\end{konstrukcija}
\begin{dokaz}
  Prostor naravnih števil s kokončno topologijo ima lastnost, da je vsaka
  funkcija \(ℕ → ℝ\) konstantna. To velja tudi za (neprazne) odprte podmnožice,
  ker so števno neskončne s kokončno topologijo, kar pa pomeni, da se, v toposu
  snopov nad \(ℕ\) s to topologijo, Dedekindova in Cauchyjeva realna števila
  ujemajo.

  Naj bo \(R(n, b) ≔ ℕ⧵\{2n+b\}\). Ta relacija je celovita, saj lahko za vsak
  \(n : ℕ\) \(ℕ\) pokrijemo z \(R(n,0)\) in \(R(n,1)\). Pokažimo, da za to
  relacijo ne obstaja funkcija izbire.

  Denimo, da je \(f : ℕ → 2\) njena funkcija izbire, torej da velja
  \(\for{n:ℕ}{R(n,f(n))}\). Če bi bila \(f\) konstantno \(b\), bi potem moralo
  veljati \(ℕ ⊩ R(n, b)\) za vse \(n\), kar pa ni res. To pomeni, da sta
  \(\i{0 = f(n)}\) in \(\i{1 = f(n)}\) obe neprazni, torej, ker sta odprti,
  imata neprazen presek (ki je tudi neskončna odprta množica). Potem pa na tej
  množici velja \(0 = 1\), kar pa očitno ne drži.\contradiction

  Sledi, da funkcija izbire za ta \(R\) ne more obstajati, torej \(\CCv\) ne
  drži.
\end{dokaz}
\begin{dokaz}
  Naj bodo \(Cₙ ≔ \{ℕ⧵\{2n\}, ℕ⧵\{2n+1\}\}\) pokritja \(\Ncof\) in \(C\) njihova
  skupna pofinitev. Potem mora vsak \(U ∈ C\) biti podmnožica enega od elementov
  vsakega od \(Cₙ\). To pa pomeni, da ima \(U\) neskončen komplement, torej je
  prazna množica. Sledi, da \(C\) pokrije zgolj prazno množico, torej \(\CCv\)
  ne drži.
\end{dokaz}

Primer podan zgoraj pa vseeno zadošča principu \(\alpo*\), ki konstruktivno
implicira ujemanje Cantorjevih in Dedekindovih realnih
števil~\cite{Birchfield24}.

TODO: restructure?
Izkaže se, da za lokalno povezane prostore to tudi pričakujemo.
\begin{izrek}
  Če je \(X\) lokalno povezan in velja \(X ⊩ \Rd = \Rc\), velja \(X ⊩ \alpo*\).
\end{izrek}
Ta izrek je zares kar posledica slednjega izreka.
\begin{izrek}
  Če je \(X\) lokalno povezan, velja \(X ⊩ \lpo*\).
\end{izrek}
\begin{dokaz}
  Naj bo \(α : 2^ℕ\). Ker je \(X\) lokalno povezan, je množica \(2^ℕ\) kar
  množica preslikav iz \(ℕ\) v \(2\). To pa pomeni, da je lokalno \(α = f\), za
  nek \(f : ℕ → 2\). Zunaj pa imamo \(\lpo*\), tako da lahko odločimo \(f = 0 ∨
  f \apart 0\), kar pa pomeni, da to lahko odločimo tudi za \(α\), torej nad
  \(X\) velja \(\lpo*\).
\end{dokaz}
\begin{dokaz}[Dokaz posledice]
  Ker je \(\lpo*\) ekvivalenten \(\alpo*\) za Cauchyjeva realna števila,
  avtomatsko sledi, da če velja \(\Rd = \Rc\), je potem \(\lpo*\) ekvivalenten
  \(\alpo*\). 
\end{dokaz}
\begin{opomba}
  Zares namesto lokalne povezanosti zadošča \(X ⊩ \Rc = \c ℝ\). To je zato, ker
  je \(\alpo*\) za Cauchyjeva realna števila ekvivalenten \(\lpo*\).

  V splošnem bi se izrek torej lahko glasil \(X ⊩ R = \c ℝ ⇒ \alpo*_R\), kjer je
  \(R\) nek objekt realnih števil (Dedekindova, Cauchyjeva, Escardo-Simpsonova,
  MacNeillova, itd⹁).
\end{opomba}
% \begin{proof}
%   Ker je \(X\) lokalno povezan so Cauchyjeva realna števila natanko lokalno
%   konstante preslikave v \(ℝ\). Denimo torej, da so vse preslikave v \(ℝ\)
%   lokalno konstantne in naj bo \(U ⊩ x : \Rd\).

%   Tedaj po predpostavki vemo, da je \(x\) lokalno konstanten, kar pomeni, da je
%   lokalno enak eksternemu realnemu številu. Ampak za ta pa znamo odločiti
%   \(\alpo*\), torej velja \(\eventually{V ⊆ U}{x ≤ 0 ∨ x > 0}\), kar smo tudi
%   želeli dokazati.
% \end{proof}

To pomeni, da če želimo ločiti \(\Rd = \Rc\) in \(\alpo*\) potrebujemo prostor,
ki ni lokalno povezan. Avtorica meni, da je Fortov prostor na števno neskončni
množici dober kandidat, a ji ni uspelo preveriti detajlov. Vseeno, se je pa
zadostno prepričala, da niti \(\alpo*\) niti \(\CCv\) nad tem prostorom ne
držita. Še več, ker je ta prostor \(T₆\) validira \(AKS\), torej ne velja niti
\(\CCv\) za ``realne'' resničnostne vrednosti. 

\subsection{Analitična Kripkejeva shema in lokalno \(T₆\) prostori}

\begin{definicija}[Kripkejeva shema]
  Kripkejeva shema je sledeči princip
  \[ \for{p : Ω}{\exist{α : 2^ℕ}{p ⇔ a \apart 0}}\text. \]
\end{definicija}
\begin{definicija}[Analitična Kripkejeva shema]
  Analitična Kripkejeva shema je sledeči princip
  \[ \for{p : Ω}{\exist{x : ℝ}{p ⇔ x \apart 0}}\text. \]
\end{definicija}

\begin{trditev}\label{th:lT6-have-AKS}
  Nad lokalno \(T₆\) prostori velja analitična Kripkejeva shema.
\end{trditev}
\begin{proof}
  Brez škode za splošnost lahko predpostavimo, da je prostor \(X\) \(T₆\).
  Naj bo \(U ⊆ X\). Tedaj obstaja \(f : X → ℝ\), ki je \(0\) natanko na
  komplementu \(U\). To pa pomeni, da je natanko na \(U\) različen od \(0\)
  kar je pa točno to, kar zahtevamo za analitično Kripkejevo shemo.
\end{proof}
\begin{opomba}
  V dokazu smo zares pokazali \emph{globalen} obstoj za element \(x : ℝ\). To
  nam da slutiti, da je (lokalno) \(T₆\) lastnost močnejša od \(\aks*\). In res
  se izkaže, da je temu tako.
  TODO: a se zmislim primer?
\end{opomba}

Vseeno pa velja obrat, če analitično Kripkejevo shemo malo ojačamo. Specifično,
če predpostavimo obstoj \emph{funkcije izibre} za shemo \(\aks*\).
\begin{trditev}
  Če velja \(X ⊩ \exist{f : Ω → ℝ}{\for{p : Ω}{p ⇔ f(p) \apart 0}}\), je \(X\)
  lokalno \(T₆\).
\end{trditev}
\begin{dokaz}
  razponi \(f\) pokrijejo \(X\)

  za \(f\) velja da za vse \(U\) pod \(\e f\) velja \(U = \i{f(U) \apart 0}\).

  pika TODO: napiši zadevo normalno.
\end{dokaz}
\begin{opomba}
  Ponovno smo pokazali malo več kot zgolj lokalno \(T₆\) lastnost. Dobili smo
  \emph{funkcije izbire} za vsako od \(T₆\) komponent. Če v metateoriji
  predpostavimo princip izbire, potem je to ekvivalentno tej močnejši verziji
  \(\aks*\).
\end{opomba}

% TODO: res rabim prevod za to
Fortov prostor zgoraj je \(T₆\), tako da v njemu ne velja niti šibkejša oblika
\(\CCv\), ki je omejena na ``realne'' resničnostne vrednosti.
Res, če še enkrat podrobno pogledamo dokaz enakosti, se potrebujemo števno
mnogokrat odločiti za neenakosti z \emph{istim} realnim številom, \(\CCv\) pa
govori o števno mnogo različnih realnih številih.
Avtorica sumi, da je \(\Rd = \Rc\) ekvivalentno temu:
\begin{quotation}
  Naj bo \(p\) naraščajoče in \(q\) padajoče zaporedje, in naj bo \(\sup p ≤ \inf q\)
  (tu morda \(=\)?), in naj bo \(x : ℝ\). Potem velja \(\for{n : ℕ}{pₙ < x ∨ x < qₙ}\),
\end{quotation}
kar je pa zelo šibek princip števne odločitve.


%%% Local Variables:
%%% mode: latex
%%% TeX-master: "main"
%%% End:
