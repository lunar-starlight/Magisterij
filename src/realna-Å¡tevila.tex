\section{Dedekindova in Cauchyjeva realna števila v topoloških modelih}

\subsection{\(\Rd = \Rc\) in \(\ccv\)}

\begin{definicija}
  \emph{P-prostori} so prostori, za katere velja, da je števen presek odprtih
  množic odprt.
\end{definicija}


\begin{lema}\label{th:t1-ccv-is-psp}
  Če nad \(T₁\) prostorom velja \(\CCv\), je P-prostor.
\end{lema}
\begin{proof}
  Naj torej velja \(X ⊩ \CCv\), in naj bo \(Uₙ\) števno pokritje \(U\).
  Za vsak \(a ∈ ⋂ₙUₙ\) tvorimo množice \(Vₙ ≔ ⋃_{k ≠ n} Uₖ ⧵ \{a\}\) in
  preslikavo
  \[ Pₐ(x)(n,ν) ≔
    \begin{cases}
      Uₙ &, ν = 1\\
      Vₙ &, ν = 0\text.
    \end{cases}
  \]
  Ker je \(Pₐ\) konstanten v \(x\) pišemo kar \(Pₐ = Pₐ(x)\).

  Ker je \(Uₙ ∪ Vₙ = Uₙ ∪ ⋃_{k ≠ n} Uₖ ⧵ \{a\} = ⋃ₙ Uₙ = U\) in po konstrukciji
  veljata \(Uₙ ⊩ P(n, 1)\) in \(Vₙ ⊩ Pₐ(n, 0)\), lahko na \(Pₐ\) uporabimo dani
  aksiom izbire. Tedaj dobimo pokritje \(Wᵢ ⊆ U\) in funkcije \(fᵢ : Wᵢ → 2^ℕ\),
  tako da za vsak \(n ∈ ℕ\) velja \(Wᵢ ⊩ Pₐ{\p{n, fᵢ(n)}}\).

  Ampak tedaj mora biti \(Wᵢ ⊆ ⋂ₙ Pₐ{\p{n, fᵢ(x)(n)}}\) za vse \(x ∈ Wᵢ\).
  Sedaj pa, ker je \(a ∈ U\) mora biti tudi element nekega \(Wᵢ\), saj je le to
  pokritje, ampak \(a\) ni element \(Pₐ(n, 0)\) za noben \(n ∈ ℕ\), torej je
  \(fᵢ\) nujno identična \(1\), in \(a ∈ Wᵢ ⊆ ⋂ₙ Uₙ\). To pokaže, da ima vsak
  element preseka neko odprto okolico, torej je presek odprt, kar pokaže, da je
  \(X\) P-prostor.
\end{proof}
\begin{opomba}
  Gornji dokaz deluje tudi za \(R₀\) prostore, saj lahko \(Vₙ\) definiramo z
  zaprtjem točke \(a\), in potrebujemo zgolj, da je to vsebovano v vsaki okolici
  točke \(a\), kar je pa natanko ekvivalentno pogoju \(R₀\). Za občutek, \(T₁\)
  prostori so natanko \(T₀\) in \(R₀\) prostori.
\end{opomba}

\begin{izrek}
  Aksiom disjunktivne števne izbire \(\CCv\) ni potreben pogoj za ujemanje
  Cantorjevih in Dedekindovih realnih števil.
\end{izrek}
\begin{dokaz}
  Prostor naravnih števil s kokončno topologijo ima lastnost, da je vsaka
  funkcija \(ℕ → ℝ\) konstantna. To velja tudi za (neprazne) odprte podmnožice,
  ker so števno neskončne s kokončno topologijo, kar pa pomeni, da se v toposu
  snopov nad \(ℕ\) s to topologijo realna števila ujemajo. Vendar je presek
  množic \(\{0, n, n+1, …\} \) enak \(\{0\}\), ki pa ni odprta, torej ta prostor
  ni P-prostor. Sledi, da \({ℕ^{\text{cof}}~\not⊩ \mathrm{AC}(ℕ, 2)}\), kar
  dokaže izrek.
\end{dokaz}


\subsection{\(\Rd = \Rc\) in \(\alpo\)}
\subsection{Analitična Kripkejeva shema in lokalno \(T₆\) prostori}


%%% Local Variables:
%%% mode: latex
%%% TeX-master: "main"
%%% End:
