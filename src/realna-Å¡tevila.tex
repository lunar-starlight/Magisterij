\section{Dedekindova in Cauchyjeva realna števila v topoloških modelih}


Najprej si oglejmo klasičen dokaz ekvivalence Dedekindovih in Cauchyjevih
realnih števil.
\begin{izrek}[Klasični]
  Pokazati je zgolj potrebno, da ima vsak Dedekindov rez pripadajoče Cauchyjevo
  zaporedje.
  Naj bo \(\p{L, U}\) obojestranski dedekindov rez. Cauchyjevo zaporedje lahko
  podamo kot zaporedje hitro padajočih racionalnih intervalov.

  Naj bosta \(p₀ ∈ L\) in \(q₀ ∈ U\) racionalni števili.
  Potem pa na \(n\)-tem koraku definiramo
  \[ a ≔ \frac{2pₙ + qₙ}{3}\text,\quad b ≔ \frac{pₙ + 2qₙ}{3}\text{, in}\quad
     \p{pₙ₊₁, qₙ₊₁} ≔ \begin{cases}
       \p{a, qₙ} ;& a ∈ L\\
       \p{pₙ, b} ;& b ∈ U\text.
     \end{cases}
  \]
  Te intervali hitro konvergirajo proti \(\p{L, U}\), torej je to želeno
  Cauchyjevo število.
\end{izrek}

Gornji izrek naredi števno mnogo odločitev, ko se odločamo, če velja \(a ∈ L\)
ali \(b ∈ U\) (oziroma ali velja \(a < x\) ali \(x < b\)), torej dokaz ni
konstruktiven.
Znana sta dva nekonstruktivna principa, ki sta zadostna za gornji odkaz in sta
šibkejša od izključene tretje možnosti. To sta \(\alpo*\) ter \(\CCv\).
Če velja \(\alpo*\) je potem \(a < x\) odločljivo, torej lahko vnaprej popravimo
drugi primer na \(x < b ∧ ¬\p{a < x}\).

Če pa imamo na voljo \(\CCv\) pa preprosto lahko naredimo števno mnogo
% TODO: prevod tega, nujno
odločitev. Zares potrebujemo tu zgolj \[\CCv\] za ``realne'' odločitve, kar je
pa tudi posledica \(\alpo*\).

Čeprav je iz tega očitno, da niti \(\alpo*\) niti \(\CCv\) nista potrebna za
\(\Rd\ = \Rc\), jih vseeno poskusimo strogo ločiti.

\subsection{\(\Rd{} = \Rc\) in \(\CCv\)}


\begin{konstrukcija}
  Aksiom disjunktivne števne izbire \(\CCv\) ni potreben pogoj za ujemanje
  Cantorjevih in Dedekindovih realnih števil.
\end{konstrukcija}
\begin{dokaz}
  Prostor naravnih števil s kokončno topologijo ima lastnost, da je vsaka
  funkcija \(ℕ → ℝ\) konstantna. To velja tudi za (neprazne) odprte podmnožice,
  ker so števno neskončne s kokončno topologijo, kar pa pomeni, da se, v toposu
  snopov nad \(ℕ\) s to topologijo, Dedekindova in Cauchyjeva realna števila
  ujemajo. Vendar pa je presek množic \(\{0, n, n+1, …\} \) enak \(\{0\}\), ki
  ni odprta, torej ta prostor ni P-prostor. Ker je ta prostor \(T₁\),
  po~\ref{th:t1-ccv-is-psp} sledi, da ne validira \(\CCv\).
\end{dokaz}


\subsection{\(\Rd{} = \Rc\) in \alpo*}

Primer podan zgoraj pa vseeno zadošča principu \(\alpo*\), ki konstruktivno
implicira ujemanje Cantorjevih in Dedekindovih realnih
števil~\cite{Birchfield24}.

% TODO: a je to X l. pov ⇒ LPO? Ja, should be
Izkaže se, da za lokalno povezane prostore to tudi pričakujemo.
\begin{izrek}
  Če je \(X\) lokalno povezan in velja \(X ⊩ \Rd\ = \Rc\), potem velja \(X ⊩ \alpo*\).
\end{izrek}
\begin{proof}
  Ker je \(X\) lokalno povezan so Cauchyjeva realna števila natanko lokalno
  konstante preslikave v \(ℝ\). Denimo torej, da so vse preslikave v \(ℝ\)
  lokalno konstantne in naj bo \(U ⊩ x : \Rd\).

  Tedaj po predpostavki vemo, da je \(x\) lokalno konstanten, kar pomeni, da je
  lokalno enak eksternemu realnemu številu. Ampak za ta pa znamo odločiti
  \(\alpo*\), torej velja \(\eventually{V ⊆ U}{x ≤ 0 ∨ x > 0}\), kar smo tudi
  želeli dokazati.
\end{proof}

To pomeni, da če želimo ločiti \(\Rd = \Rc\) in \(\alpo*\) potrebujemo prostor,
ki ni lokalno povezan. Avtorica meni, da je Fortov prostor na števno neskončni
množici dober kandidat, a ji ni uspelo preveriti detajlov. Vseeno, se je pa
zadostno prepričala, da niti \(\alpo*\) niti \(\CCv\) nad tem prostorom ne
držita. Še več, ker je ta prostor \(T₆\) validira \(AKS\), torej ne velja niti
\(\CCv\) za ``realne'' resničnostne vrednosti. 

\subsection{Analitična Kripkejeva shema in lokalno \(T₆\) prostori}

\begin{definicija}[Kripkejeva shema]
  Kripkejeva shema je sledeči princip
  \[ \for{p : Ω}{\exist{α : 2^ℕ}{p ⇔ a \apart 0}}\text. \]
\end{definicija}
\begin{definicija}[Analitična Kripkejeva shema]
  Analitična Kripkejeva shema je sledeči princip
  \[ \for{p : Ω}{\exist{x : ℝ}{p ⇔ x \apart 0}}\text. \]
\end{definicija}

\begin{trditev}\label{th:lT6-have-AKS}
  Nad lokalno \(T₆\) prostori velja analitična Kripkejeva shema.
\end{trditev}
\begin{proof}
  Brez škode za splošnost lahko predpostavimo, da je prostor \(X\) \(T₆\).
  Naj bo \(U ⊆ X\). Tedaj obstaja \(f : X → ℝ\), ki je \(0\) natanko na
  komplementu \(U\). To pa pomeni, da je natanko na \(U\) različen od \(0\)
  kar je pa točno to, kar zahtevamo za analitično Kripkejevo shemo.
\end{proof}

% TODO: res rabim prevod za to
Fortov prostor zgoraj je \(T₆\), tako da v njemu ne velja niti šibkejša oblika
\(\CCv\), ki je omejena na ``realne'' resničnostne vrednosti.
Res, če še enkrat podrobno pogledamo dokaz enakosti, se potrebujemo števno
mnogokrat odločiti za neenakosti z \emph{istim} realnim številom, \(\CCv\) pa
govori o števno mnogo različnih realnih številih.
Avtorica sumi, da je \(\Rd = \Rc\) ekvivalentno temu:
\begin{quotation}
  Naj bo \(p\) naraščajoče in \(q\) padajoče zaporedje, in naj bo \(\sup p ≤ \inf q\)
  (tu morda \(=\)?), in naj bo \(x : ℝ\). Potem velja \(\for{n : ℕ}{pₙ < x ∨ x < qₙ}\),
\end{quotation}
kar je pa zelo šibek princip števne odločitve.


%%% Local Variables:
%%% mode: latex
%%% TeX-master: "main"
%%% End:
