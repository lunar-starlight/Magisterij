\section{Dedekindova in Cauchyjeva realna števila v topoloških modelih}

\subsection{\(\Rd = \Rc\) in \(\ccv\)}

% TODO: update theorems
\begin{lema}
  P-prostori so natanko prostori, v katerih velja \(\mathrm{AC}(ℕ,2)\).
\end{lema}
\begin{dokaz}
  Trditev \(X ⊩ \mathrm{AC}(ℕ, 2)\) se razpiše v
  \begin{gather*}
    \for{U ⊆ X, P : U → 𝒪X^{ℕ×2}}
    {U ⊩ \for{n ∈ ℕ}{\exist{ν ∈ 2}{P(n, ν)}}\\ ⇒ U ⊩ \exist{f : ℕ → 2}{\for{n ∈ ℕ}{P(n, f(n))}}}\text.
  \end{gather*}
  % ⇒ \exist{\cov{U}{i}, fᵢ : Uᵢ → 2^ℕ}{\for{n ∈ ℕ, i}{Uᵢ ⊩ P(n,
  % fᵢ(n))}} } \]
   
  Naj bo najprej \(X\) P-prostor, in naj bodo \(U\) in \(P\) kot zgoraj.
  Tedaj predpostavka pravi, da za vsak \(ℕ\) obstaja pokritje \(Uₙ₀, Uₙ₁ ⊆ U\),
  tako da velja \(Uₙᵢ ⊩ P(n, i)\).

  Iz teh parov lahko tvorimo množice \(U_α ≔ ⋂_{n ∈ ℕ}U_{n,α(n)}\), \(a ∈ 2^ℕ\),
  ki pa so števni preseki odprtih množic, torej pod predpostavko tudi sami
  odprti. Poleg tega pa tvorijo pokritje množice \(U\), in še več, velja
  \(U_α ⊩ P(n, α(n))\), kar je pa natanko to, kar moramo pokazati.

  V obratno smer naj velja \(X ⊩ \mathrm{AC}(ℕ,2)\), in naj bo \(Uₙ\) števno
  pokritje \(U\).
  Za vsak \(a ∈ ⋂ₙUₙ\) tvorimo množice \(Vₙ ≔ ⋃_{k ≠ n} Uₖ ⧵ \{a\}\) in
  preslikavo
  \[ P(x)(n,ν) ≔
    \begin{cases}
      Uₙ &, ν = 1\\
      Vₙ &, ν = 0\text.
    \end{cases}
  \]
  Ker je \(Uₙ ∪ Vₙ = Uₙ ∪ ⋃_{k ≠ n} Uₖ ⧵ \{a\} = ⋃ₙ Uₙ = U\) in po konstrukciji
  veljata \(Uₙ ⊩ P(n, 1)\) in \(Vₙ ⊩ P(n, 0)\), lahko na \(P\) uporabimo dani
  aksiom izbire. Tedaj dobimo pokritje \(Wᵢ ⊆ U\) in funkcije \(fᵢ : Wᵢ → 2^ℕ\),
  tako da za vsak \(n ∈ ℕ\) velja \(Wᵢ ⊩ P(n, fᵢ(n))\).

  Ampak tedaj mora biti \(Wᵢ ⊆ ⋂ₙ P(x)(n, fᵢ(x)(n))\) za vse \(x ∈ Wᵢ\).
  Sedaj pa, ker je \(a ∈ U\) mora biti tudi element nekega \(Wᵢ\), saj je
  pokritje, ampak \(a ∉ P(x)(n, 0)\) za noben \(x ∈ U\), torej je \(fᵢ\) nujno
  identična \(1\), in \(a ∈ Wᵢ ⊆ ⋂ₙ Uₙ\). To pokaže, da ima vsak element preseka
  neko odprto okolico, torej je presek odprt, kar pokaže, da je \(X\) P-prostor.
\end{dokaz}

\begin{izrek}
  Aksiom izbire \(\mathrm{AC}(ℕ,2)\) ni potreben pogoj za ujemanje Cantorjevih
  in Dedekindovih realnih števil.
\end{izrek}
\begin{dokaz}
  Prostor naravnih števil s kokončno topologijo ima lastnost, da je vsaka
  funkcija \(ℕ → ℝ\) konstantna. To velja tudi za odprte podmnožice, ker so
  števno neskončne s kokončno topologijo, kar pa pomeni, da se v toposu snopov
  nad \(ℕ\) s to topologijo realna števila ujemajo. Vendar je presek množic
  \(\{0, n, n+1, …\} \) enak \(\{0\}\), ki pa ni odprta, torej ta prostor ni
  P-prostor. Sledi, da \(ℕ~\not⊩ \mathrm{AC}(ℕ, 2)\), kar dokaže izrek.
\end{dokaz}


\subsection{\(\Rd = \Rc\) in \(\alpo\)}
\subsection{Analitična Kripkejeva shema in lokalno \(T₆\) prostori}


%%% Local Variables:
%%% mode: latex
%%% TeX-master: "main"
%%% End:
