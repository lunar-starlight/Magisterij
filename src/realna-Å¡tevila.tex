%\section{Realna števila v topoloških modelih}


\subsection{\(\Rc{} = \c ℝ\)}\label{sec:reals-Rc=R}

Poglejmo si tu še enkrat trditev~\ref{th:lpov-lpo}, ki pravi, da nad lokalno
povezanimi prostori velja \(\lpo*\). Zares pa potrebujemo zgolj \(\Rc = \c ℝ\),
saj smo potrebovali le, da so elementi \(\Rc\) lokalno konstantni. Izkaže se, da
velja tudi obrat.

\begin{trditev}
  Če nad \(X\) velja \(\lpo*\), velja \(\Rc = \c ℝ\).
\end{trditev}
\begin{dokaz}
  Naj bo \(x : \Rc\). To je zunaj preslikava \(\hat x : \e x → ℝ\), saj je vsako
  Cauchyjevo realno število tudi Dedekindovo.

  Ker velja \(\lpo*\), imamo za vsak \(a ∈ ℝ\) \(\lpo(x-a)\), torej
  \(x = a ∨ x \apart a\). Med drugim to pomeni, da so množice
  \(\set{t∈\e x}{\hat x(t) = a}\) odprte, torej je \(\hat x\) lokalno konstantna
  preslikava.
\end{dokaz}
\begin{opomba}
  Tu smo uporabili \(\abs ℝ\) mnogo instanc \(\lpo*\).
\end{opomba}

Skupaj s trditvijo~\ref{th:lpov-lpo} lahko zatrdimo ekvivalenco.
\begin{izrek}\label{th:Rc=R-is-lpo}
  Nad \(X\) velja \(\lpo*\) natanko tedaj, ko je \(\Rc = \c ℝ\).
\end{izrek}
Tako lahko namesto \(\Rc = \c ℝ\) od tu naprej pišemo kar \(\lpo*\)

Izkaže pa se, da začetna formulacija trditve~\ref{th:lpov-lpo} ni tako napačna.
Nad \(2\)-števnimi prostori je namreč princip \(\lpo*\) ekvivalenten lokalni
povezanosti~\cite[1026]{Johnstone02}.

% Karakterizirajmo najprej \(\lpo*\) brez realnih števil.
\begin{lema}
  Nad \(X\) velja \(\lpo*\) natanko tedaj, ko za vsak \(U\) in števno družino
  \(\{Vₙ\}ₙ\) odprtih množic, ki so relativno zaprte v \(U\), je njihov presek
  odprt.
\end{lema}
\begin{dokaz}
  Označimo \(V ≔ ⋂ₙVₙ\) in definirajmo relacijo
  \[ α(n, b) ≔ \i{b=1∧(U⧵Vₙ) ∨ b=0∧Vₙ} \text. \]
  Ker so \(Vₙ\) odprte in zaprte, je ta celovita in enolična, torej definira
  \(α : 2^ℕ\) z razponom \(U\). Uporabimo sedaj \(\lpo*\), in dobimo, da je
  \(U = \i{α = 0} ∪ \i{α\apart 0}\). Vrednost \(α\apart 0\) je natanko unija
  vseh \(U⧵Vₙ\), torej natanko \(U⧵V\). Sledi, da je \(\i{α=0} = V\), torej je
  presek odprt.

  Obratno, naj bo \(α : 2^ℕ\) in definirajmo \(Uₙ ≔ \i{α(n) = 0}\) in
  \(Vₙ ≔ \i{α(n) = 1}\). Velja \(\e α = Uₙ ∪ Vₙ\), in \(Uₙ ⊥ Vₙ\), torej so
  \(Uₙ\) in \(Vₙ\) relativno zaprte v \(\e α\).

  Sledi, da je presek \(Uₙ\) odprt, torej je odprta tudi množica \(t ∈ \e α\),
  kjer je \(α = 0\). To pa pomeni, da velja \(α = 0 ∨ α \apart 0\), torej velja
  \(\lpo*\).
\end{dokaz}

\begin{trditev}
  Če nad \(2\)-števnim prostorom velja \(\lpo*\), je lokalno povezan.
\end{trditev}
Dokaz za vajo prepuščen bralki (avtorica ne zna izpeljat dokaza,
\cite[1026]{Johnstone02} pravi zgolj, da to drži in za podrobnosti dokaza bralce
preusmeri na članek v francoščini, ki prav tako ne vsebuje dokaza tega dejstva).


\subsection{\(\Rd{} = \Rc\)}\label{sec:reals-Rd=Rc}

Najprej si oglejmo klasičen dokaz ekvivalence Dedekindovih in Cauchyjevih
realnih števil.
\begin{dokaz}[Klasični dokaz]
  Pokazati je zgolj potrebno, da ima vsak Dedekindov rez pripadajoče Cauchyjevo
  zaporedje.
  Naj bo \(\p{L, U}\) obojestranski Dedekindov rez. Cauchyjevo zaporedje lahko
  podamo kot zaporedje hitro padajočih racionalnih intervalov.

  Naj bosta \(p₀ ∈ L\) in \(q₀ ∈ U\) racionalni števili.
  Potem pa na \(n\)-tem koraku definiramo
  \[ a ≔ \frac{2pₙ + qₙ}{3}\text,\quad b ≔ \frac{pₙ + 2qₙ}{3}\text{, in}\quad
     \p{pₙ₊₁, qₙ₊₁} ≔ \begin{cases}
       \p{a, qₙ} ;& a ∈ L\\
       \p{pₙ, b} ;& b ∈ U\text.
     \end{cases}
  \]
  Te intervali hitro konvergirajo proti \(\p{L, U}\), torej je to želeno
  Cauchyjevo število.
\end{dokaz}

Gornji izrek naredi števno mnogo odločitev, ko se odločamo, če velja \(a ∈ L\)
ali \(b ∈ U\) (oziroma ali velja \(a < x\) ali \(x < b\)), torej dokaz ni
konstruktiven.
Znana sta dva nekonstruktivna principa, ki sta zadostna za gornji dokaz in sta
šibkejša od izključene tretje možnosti. To sta \(\alpo*\) ter \(\CCv\).
Če velja \(\alpo*\), je \(a < x\) odločljivo, torej lahko vnaprej popravimo
drugi primer na \(x < b ∧ ¬\p{a < x}\).
Če pa imamo na voljo \(\CCv\) preprosto lahko naredimo števno mnogo odločitev.
Zares potrebujemo tu zgolj \(\CCv_{Σ_ℝ}\), ki je tudi posledica \(\alpo*\).

Čeprav nam gornje namiguje, da niti \(\alpo*\) niti \(\CCv\) nista potrebna za
\(\Rd = \Rc\), jih vseeno želimo strogo ločiti.

\begin{trditev}
  Nad \(\Ncof\) velja \(\Rd = \Rc\), a ne velja princip števne
  disjunktivne izbire.
\end{trditev}
\begin{dokaz}
  Po lemi~\ref{th:Ncof-str-conn} so vse zvezne funkcije \(U → ℝ\) za neprazne
  \(U ∈ 𝒪\Ncof\) konstantne, torej se Dedekindova in Cauchyjeva realna števila
  ujemajo.

  Naj bo \(R(n, b) ≔ ℕ⧵\{2n+b\}\). Ta relacija je celovita, saj lahko za vsak
  \(n : ℕ\) \(ℕ\) pokrijemo z \(R(n,0)\) in \(R(n,1)\). Pokažimo, da za to
  relacijo ne obstaja funkcija izbire.

  Denimo, da je \(f : ℕ ↬ 2\) njena funkcija izbire, torej da velja
  \(\for{n:ℕ}{R(n,f(n))}\). Če bi bila \(f\) konstantno \(b\), bi veljalo
  \(\Ncof ⊩ R(n, b)\) za vse \(n\), kar pa ni res. To pomeni, da sta množici 
  \(f(n,0)\) in \(f(n,1)\) obe neprazni, torej je njun presek neprazna odprta
  množica. Na tej množici po enoličnosti \(f\) velja \(0 = 1\), kar pa očitno ne
  drži, tako da \(f\) ni funkcija izbire za \(R\) in \(\CCv\) ne drži.
\end{dokaz}
% \begin{dokaz}
%   Naj bodo \(Cₙ ≔ \{ℕ⧵\{2n\}, ℕ⧵\{2n+1\}\}\) pokritja \(\Ncof\) in \(C\) njihova
%   skupna pofinitev. Potem mora vsak \(U ∈ C\) biti podmnožica enega od elementov
%   vsakega od \(Cₙ\). To pa pomeni, da ima \(U\) neskončen komplement, torej je
%   prazna množica. Sledi, da \(C\) pokrije zgolj prazno množico, torej \(\CCv\)
%   ne drži.
% \end{dokaz}

Prostor \(\Ncof\) pa vseeno zadošča principu \(\alpo*\), ki konstruktivno
implicira ujemanje Cauchyjevih in Dedekindovih realnih
števil~\cite{Birchfield24}.

Izkaže se, da za lokalno povezane prostore to tudi pričakujemo.
\begin{trditev}
  Če je \(X\) lokalno povezan, velja \(X ⊩ \Rd = \Rc ⇒ \alpo*\).
\end{trditev}
Ta izrek je zares kar posledica izreka~\ref{th:lpov-lpo}, saj je \(\lpo*\)
natanko \(\alpo*\) za Cauchyjeva realna števila.
\begin{opomba}
  Spet namesto lokalne povezanosti zadošča \(X ⊩ \Rc = \c ℝ\).
  V splošnem bi se izrek torej lahko glasil \(X ⊩ R = \c ℝ ⇒ \alpo*_R\), kjer je
  \(R\) nek objekt realnih števil (Dedekindova, Cauchyjeva, Escardo-Simpsonova,
  MacNeilleova, itd.), kar smo pa za primer Dedekindovih realnih števil zares
  pokazali že v izreku~\ref{th:alpo-is-zerosets-open}.
\end{opomba}

To pomeni, da če želimo ločiti \(\Rd = \Rc\) in \(\alpo*\) potrebujemo prostor,
ki ni lokalno povezan. Seveda, \(\Ncof + (ω+1)\) deluje. Na prvi komponenti ne
velja \(\CCv\), na drugi pa \(\alpo*\). Kljub temu velja \(\CCv ∨ \alpo*\),
torej klasični dokaz enakosti \(\Rd = \Rc\) še vedno deluje.

\emph{Modificiran Fortov prostor na \(ℝ\)}, je prostor \(F ≔ ℝ∪\{∞₀,∞₁\}\), za
katerega so vse točke \(ℝ\) izolirane, odprte okolice \(∞ᵢ\) so pa kokončne
podmnožice \(F\), ki to točko vsebujejo.

Nad \(F\) principa \(\alpo*\) in \(\CCv\) ne veljata. Prvi spodleti zaradi
funkcije, ki ima na naravnih številih vrednost \(2⁻ⁿ\) in je enaka \(0\) sicer,
drugi pa zaradi relacije \({R(n,i) = ℝ⧵\{2n+i\}∪\{∞ᵢ\}}\). Vseeno pa velja
\(F ⊩ \CCv_{Σ_{\Rd}}\). Res, ključne so zvezne funkcije iz \(F\), saj te vedno
slikajo obe točki v neskončnosti v isto vrednost, torej se celovite relacije z
vrednostmi v \(Σ_{\Rd}\) obnašajo ravno tako kot v navadnem Fortovem prostoru na
\(ℝ\), tam pa velja odvisna izbira.

Več od tega avtorici ni znano, saj je zelo težko preverjati, če nad prostorom
velja \(\Rd = \Rc\), kajti nimamo nobene karakterizacije Cauchyjevih realnih
števil v topoloških modelih. Avtorica sumi, da nad prebito Knaster-Kuratowskijevo
pahljačo ali Erdősevim prostorom velja \(\Rd = \Rc\), a ne velja
\(\CCv_{Σ_{\Rd}}\), a tega ni uspela preveriti.

% Kar je pa tudi za pričakovati (namreč, da je \(\Rd = \Rc\) šibkejši), saj če
% podrobno pogledamo klasičen dokaz enakosti, se potrebujemo števno mnogokrat
% odločiti za neenakosti z \emph{istim} realnim številom, \(\CCv\) pa govori o
% števno mnogo različnih realnih številih. Avtorica sumi, da je \(\Rd = \Rc\)
% ekvivalentno temu:
% \begin{quotation}
%   Naj bo \(p\) naraščajoče in \(q\) padajoče zaporedje, in naj bo \(\sup p ≤ \inf q\)
%   (tu morda \(=\)?), in naj bo \(x : ℝ\). Potem velja \(\for{n : ℕ}{pₙ < x ∨ x < qₙ}\),
% \end{quotation}
% kar je pa zelo šibek princip števne odločitve.

\subsection{\(\Rm{} = \Rd\)}\label{sec:reals-Rm=Rd}

% https://gist.github.com/andrejbauer/689b17b10a4e80ea409d03ec030c98b3
Andrej Bauer je 2023 za prvoaprilsko šalo objavil, kar zgleda kot konstruktiven
dokaz \(\wlem*\). V njem začne z ``znanimi dejstvi'' o MacNeilleovih realnih
številih, zraven pa podtakne še lociranost. Vemo že, da so MacNeilleova
realna števila locirana natanko tedaj, ko se ujemajo z Dedekindovimi realnimi
števili.
Skratka, podal je konstruktiven dokaz implikacije \(\Rm = \Rd ⇒ \wlem*\), ki ga
lahko v topoloških modelih še poenostavimo:
\begin{trditev}
  Nad vsakim \(X\) velja \(\Rm = \Rd ⇒ \wlem*\).
\end{trditev}
\begin{dokaz}
  Naj bo \(U ∈ 𝒪X\), \(\uline f ≔ χ_{¬¬U}\), in \(\bar f ≔ χ_{\cl U}\). Ti
  preslikavi sta tesni, torej definirata MacNeilleovo realno število. Ta so po
  predpostavki enaka Dedekindovim, torej sta preslikavi \(\uline f\) in
  \(\bar f\) enaki, kar pomeni, da sta množici \(¬¬U\) in \(\cl U\) enaki.
  Sledi, da je \(X = \cl U ∪ ¬U = ¬¬U ∪ ¬U\), torej nad \(X\) velja \(\wlem*\).
\end{dokaz}
Izkaže pa se, da velja tudi obrat. Inspiracija za to dejstvo je prišla iz
predmeta Banachove mreže, kjer se obravnava sledeči izrek:
\begin{izrek}\label{th:ban-mr}
  Če je \(X\) ekstremalno nepovezan \(T_{3.5}\) prostor, je \(𝒞(X,ℝ)\) polna mreža.
\end{izrek}
To pa zgleda zelo sumljivo! Če se spomnimo, to da je \(X\) ekstremalno povezan
ravno pomeni, da velja \(\wlem*\). Prav tako je \(𝒞(X,ℝ)\) znotraj množica
globalnih Dedekindovih realnih števil. Konstruktivno ta niso polna, so pa
MacNeilleova realna števila, tako da bo supremum Dedekindovih realnih v
MacNeilleovih realnih številih obstajal, le da bo to MacNeilleovo realno število.
Potem pa lahko polnost Dedekindovih realnih števil izrazimo kot ``vsako
MacNeilleovo realno število je Dedekindovo''.

Vredno je še omeniti, da je ekstremalna nepovezanost dedna lastnost na odprte
podmnožice, tako da lahko konsekvent napišemo tudi kot ``\(𝒞(U,ℝ)\) je polna
mreža za vse \(U\)''. To je pa že bistveno bližje temu, da bi rekli ``realna
števila so polna'' v interni logiki. Ampak, preveriti moramo še, da se notranja
in zunanja polnost ujemata. 

Izkaže se, da se ne ujemata, polnost znotraj pravi, da ima vsaka
naseljena \emph{odsekoma} omejena množica supremum, medtem ko polnost zunaj
zahteva globalno omejenost. Ampak to ni problem, saj je vsaka omejena množica
očitno tudi odsekoma omejena, torej je notranja trditev močnejša. Ostane le še
vprašanje, ali je \(T_{3.5}\) predpostavka potrebna. Izkaže se da ni, torej
izrek~\ref{th:ban-mr} sledi iz naslednje trditve:
\begin{trditev}
  Nad vsakim \(X\) velja \(\wlem* ⇒ \Rm = \Rd\).
\end{trditev}
\begin{dokaz}
  Spomnimo se, da za MacNeilleova realna števila \(x : \Rm\) velja
  \[ ¬(x < q) ⇒ \for{s<q}{s < x}\text. \]
  Naj bosta \(p < q\) racionalni števili in definirajmo \(r ≔ \frac{2p+q}3\) in
  \(s ≔ \frac{p+2q}3\), tako da velja \(p<r<s<q\).
  Uporabimo \(\wlem*\) na \(x < r\) in \(s < x\) in ločimo
  na štiri primere, od katerih je en trivialen, druge tri pa zajamemo s
  spodnjima dvema implikacijama:
  \begin{itemize}
  \item Če ne velja \(x < r\), potem velja \(p < x\).
  \item Če ne velja \(s < x\), potem velja \(x < q\).
  \end{itemize}
  V vseh primerih dobimo \(p < x ∨ x < q\), torej je \(x\) lociran.
\end{dokaz}
Skicirajmo še topološki dokaz.
\begin{dokaz}[Topološki dokaz]
  Ker je \(X\) ekstremalno nepovezan sledi, da je ekstremalno nepovezana tudi
  vsaka njegova odprta podmnožica. Naj bosta \(\uline f\) in \(\bar f\) par
  tesnih preslikav. Potem obstaja zvezna preslikava \(f\), za katero velja
  \(\uline f ≤ f ≤ \bar f\). Ker sta pa polzvezni preslikavi tesni, je
  \(f ≤ \uline f\) in \(\bar f ≤ f\), torej so preslikave enake in zvezne.
\end{dokaz}

\begin{izrek}\label{th:Rm=Rd-wlem}
  MacNeilleova realna števila se ujemajo z Dedekindovimi natanko tedaj, ko velja
  princip šibke izključene tretje možnosti.
\end{izrek}

V literaturi se izrek~\ref{th:Rm=Rd-wlem} pojavi v
recimo~\cite[trd.~D4.7.11]{Johnstone02}, sem pa to tudi formalizirala v
dokazovalnem pomočniku Agda~\cite{BS25}.

Kaj pa, če bi zahtevali šibkejši pogoj, da ima vsaka števna omejena množica
supremum? V literaturi~\cite[vaja~3N.5]{GJ60} najdemo izrek, ki za \(T_{3.5}\)
prostore pravi, da to velja natanko nad realno nepovezanimi prostori. Po zgledu
izreka~\ref{th:ban-mr} bi upali, da je tudi tu predpostavka \(T_{3.5}\)
nepotrebna, ampak avtorici ni uspelo najti dokaza te trditve, tako da tega ne
more zatrditi.


%%% Local Variables:
%%% mode: latex
%%% TeX-master: "main"
%%% End:
