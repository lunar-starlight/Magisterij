\section{Realna števila in principi odločitve v topoloških modelih}
\label{sec:reals}
\label{sec:odločitve}

\subsection{Objekti realnih števil}

V~\ref{sec:modeli-heyting} smo videli dva ``topološka'' objekta realnih števil,
objekt \(ℛ\) iz primera~\ref{ex:reals} in objekt \(\c ℝ\), definiran
v~\ref{def:constant-hvs}. Vprašanje je, kako sta dva objekta povezana z
internimi konstrukcijami realnih števil definiranih v~\ref{sec:logika-reals}.

Objekt \(ℛ\) je vedno enak objektu Dedekindovih realnih števil.
\begin{trditev}\label{th:Rd-maps}
  Nad \(X\) je objekt Dedekindovih realnih števil tip \(ℛ\) iz
  primera~\ref{ex:reals}.
  % Nad \(X\) je objekt Dedekindovih realnih števil tip
  % \(\set{f : U → ℝ}{U ∈ 𝒪X\text{, }f\text{ zvezna}}\), z enakostjo definirano
  % kot \(\i{f = g} ≔ \int{\set{t ∈ X}{f(t) = g(t)}}\).
\end{trditev}
\begin{dokaz}
  Naj bo \(x = \p{L, U}\) Dedekindov rez v interni logiki.
  Sedaj lahko zunaj za \(t ∈ \e x\) definiramo
  \begin{align*}
    L(t) &≔ \set{q : ℚ}{t ∈ \i{q ∈ L}}\text{ in}\\
    U(t) &≔ \set{r : ℚ}{t ∈ \i{r ∈ U}}\text.
  \end{align*}
  Ta tvorita Dedekindov rez zunaj, tako da nam \(t ↦ \p{L(t), U(t)}\) definira
  preslikavo \(\hat x : \e x → ℝ\).
  Pokazati ostane, da je ta preslikava zvezna.
  Naj bo \(\p{a,b}\) racionalen interval v \(ℝ\). Praslika tega intervala z
  \(\hat x\) je množica \(\set{t ∈ \e x}{a < \hat x(t) < b}\).
  Ta pogoj je ekvivalenten \(t ∈ \i{a ∈ L ∧ b ∈ U}\), torej je praslika kar
  enaka \(\i{a∈L∧b∈U}\), torej je odprta in je \(\hat x\) zvezna.

  Obratno, če je \(x : \e x → ℝ\) zvezna preslikava, lahko tvorimo preslikavi
  \begin{align*}
    L(t) &≔ \set{q : ℚ}{q < x(t)}\text{ in}\\
    U(t) &≔ \set{r : ℚ}{x(t) < r}\text.
  \end{align*}
  Ti znotraj tvorita Dedekindov rez \(\p{L, U}\).
\end{dokaz}

Med drugim to pomeni, da so Cauchyjeva realna števila podmnožica zveznih
preslikav v \(ℝ\). Ampak katerih? Oglejmo si najprej tip \(\c ℝ\).

\begin{definicija}
  Naj bo \(U ∈ 𝒪X\). Preslikava \(f : U → ℝ\) je \emph{lokalno konstantna}, ko
  velja \(\eventually{V ⊆ U}{f{\res V}\text{ je konstantna}}\).
\end{definicija}
Očitno torej vsaka lokalno konstantna preslikava določa Dedekindovo realno
število. Povemo pa lahko malo več.

\begin{trditev}\label{th:R-into-Rc}
  Vsaka lokalno konstantna preslikava \(f : U → ℝ\) določa Cauchyjevo realno
  število.
\end{trditev}
\begin{dokaz}
  Naj bo najprej \(f : U → ℝ\) konstantno \(x ∈ ℝ\). Zunaj je \(x\)
  predstavljeno s Cauchyjevim zaporedjem \((xₙ)ₙ\). Sedaj lahko tvorimo
  konstantne preslikave \(\hat xₙ : U → ℚ\), ki znotraj tvorijo Cauchyjevo
  zaporedje, torej določajo Cauchyjevo realno število \(x\). Po
  trditvi~\ref{th:Rd-maps}, in ker je vsako Cauchyjevo realno število tudi
  Dedekindovo, je to predstavljeno s preslikavo \(f\).

  Če je \(f : U → ℝ\) lokalno konstantna, po definiciji obstaja pokritje \(C\),
  tako da je na vsakem elementu pokritja \(f\) konstantna. Po prvi točki torej
  na vsakem \(V ∈ C\) obstaja Cauchyjevo realno število, ki je določeno s
  preslikavo \(f{\res V}\). Potem po lokalnosti veljavnosti~\ref{th:valid-local}
  obstaja Cauchyjevo realno število na celotnem \(U\), ki je določeno s
  preslikavo \(f\).
\end{dokaz}
V določenih primerih pa velja tudi obrat.
\begin{trditev}\label{th:Rc-maps}
  Naj bo prostor \(X\) lokalno povezan. Tedaj je nad \(X\) objekt Cauchyjevih
  realnih števil tip \(\c ℝ\).
\end{trditev}
\begin{dokaz}
  Pokazati moramo zgolj, da je vsako Cauchyjevo realno število zunaj lokalno
  konstantna preslikava.

  Nad \(t ∈ \e x\) dobimo Cauchyjevo zaporedje \((xₙ(t))ₙ\). Ker je \(X\)
  lokalno povezan, ima pokritje iz povezanih množic, tako da brez škode za
  splošnost predpostavimo, da je \(\e x\) povezan. Sledi, da so \(xₙ\)
  konstantne preslikave, torej je tudi preslikava \(t ↦ (xₙ(t))ₙ\) konstantna.
\end{dokaz}

Trditvi~\ref{th:Rd-maps} in~\ref{th:Rc-maps} nakazujeta, da vsebovanosti
\(\c ℝ ⊆ \Rc ⊆ \Rd\) v splošnem niso obrnljive.

\begin{trditev}
  Nad \(ℝ\) sta objekta Dedekindovih in Cauchyjevih realnih števil različna.
\end{trditev}
\begin{dokaz}
  Identiteta na \(ℝ\) je Dedekindovo realno število, ki pa ni lokalno
  konstantno, torej ni Cauchyjevo.
\end{dokaz}

\begin{trditev}
  Nad \(2^ℕ\) sta objekta Cauchyjevih in lokalno konstantnih realnih števil
  različna.
\end{trditev}
\begin{dokaz}
  Definirajmo \(x : 2^ℕ → ℝ\) kot \(x(α) ≔ \sum_{n=0}^∞ \frac{2α(n)}{3ⁿ}\). Ta
  preslikava je vložitev Cantorjeve množice v \(ℝ\), in je znotraj Cauchyjevo
  realno število, saj je limita racionalnih števil. Ampak ni lokalno konstantno,
  saj je \({x⁻¹(0) = \{0\}}\), ki ni odprta množica.
\end{dokaz}

Kaj pa MacNeillova realna števila? Teh je več kot Dedekindovih, tako da če jih
želimo karakterizirati kot nekakšne funkcije iz odprtih podmnožic \(X\), bodo te
morale biti nezvezne. Izkaže se, da so odgovor polzvezne
funkcije~\cite[posl.~D4.7.5]{Johnstone02}. Seveda, če je funkcija polzvezna
navzdol in navzgor, je zvezna. Ampak, lahko vzamemo par navzdol in navzgor
polzveznih preslikav \(\uline f\) in \(\bar f\), ki sta si nekako ``čim
bližje''. To pomeni, da želimo, da je \(\uline f\) največja navzdol polzvezna
preslikava manjša ali enaka \(\bar f\), in obratno.

\begin{lema}\label{th:onesided-cuts-are-scts}
  V topoloških modelih so dolnji (gornji) enostranski rezi predstavljeni z
  navzdol (navzgor) polzveznimi preslikavami.
\end{lema}
\begin{dokaz}
  Naj bo \(L\) dolnji enostranski rez. Kot za Dedekindove reze tvorimo
  preslikavo \(t ↦ \set{q:ℚ}{t ∈ \i{q∈L}}\). To je preslikava \(x : \e L → ℝ\),
  saj so klasično enostranski rezi ekvivalentni dvostranskim.

  Oglejmo si sedaj \(x⁻¹\p{p,∞}\). To je natanko \(\set{t∈\e L}{p < x(t)}\). Po
  definiciji je ta pogoj ekvivalenten \(t ∈ \i{p ∈ L}\), torej je praslika
  odprta.

  Obratno, naj bo \(x : U → ℝ\) navzdol polzvezna. Potem lahko definiramo
  preslikavo \(L(t) ≔ \set{p ∈ ℚ}{p < x(t)}\). Ta znotraj tvori dolnji
  enostranski rez, saj so morfizmi v \(Ω\) natanko operacije v \(Ω\).
\end{dokaz}
Ker pa interno nimamo gornjega reza, ne bomo znali izraziti pogoja \(x(t) < q\)
na lep način, tako da v splošnem ta preslikava ni zvezna.

\begin{trditev}\label{th:Rm-maps}
  Objekt MacNeilleovih realnih števil je natanko tip parov polzveznih
  funkcij \(\p{\uline f, \bar f}\) tipa \(U → ℝ\) za \(U ∈ 𝒪X\), za kateri velja
  pogoj \emph{tesnosti}, ki se glasi:
  \[
    \uline f(t) = \liminf_{y → t}\bar   f(y)\qquad\text{ in}\qquad
    \bar   f(t) = \limsup_{y → t}\uline f(y)\text,
  \]
  z enakostjo definirano kot
  \[
    \i{\p{\uline f, \bar f} = \p{\uline g, \bar g}} ≔
    \int\set{t ∈ U}{\uline f(t) = \uline g(t) ∧ \bar f(t) = \bar g(t)}\text.
  \]
\end{trditev}
Dokaz te trditve ni kompliciran, ampak se zanaša na veliko analize, tako da ga
tu ne ponovimo. Bralka si ga lahko ogleda v~\cite[D4.7.5]{Johnstone02}.
% \begin{dokaz}
%   Po gornji lemi lahko za MacNeillovo realno število \(x = \p{L,U}\)
%   konstruiramo navzdol in navzgor polzvezni preslikavi \(\uline f\) in
%   \(\bar f\). Pokazati nam ostane, da sta preslikavi tesni. Pokažimo zgolj prvi
%   pogoj, saj sta si simetrična.

%   Limes inferior definira navzdol polzvezno preslikavo, namreč supremum vseh
%   navzgor polzveznih preslikav manjših ali enakih \(\bar f\).
%   To pomeni, da moramo pokazati, da je vsaka navzdol polzvezna preslikava manjša
%   ali enaka \(\uline f\).

%   Vemo, da so navzdol polzvezne preslikave natanko enostranski rezi v interni
%   logiki \(X\), tako da naj bo \(L'\) tak rez. Pogoj, da je preslikava manjša al
%   enaka \(\bar f\) pa pravi, da je \(L'\) vsebovan v komplementu \(U\).

%   Sedaj pa naj bo \(q ∈ L'\). Ker je \(L'\) navzgor odprt obstaja nek \(s > q\),
%   ki je še vedno v \(L'\), torej je \(s ∈ Uᶜ\). To pa pomeni, da je
%   \(q ∈ \int{\p{Uᶜ}} = L\), in je \(L' ⊆ L\). To pa očitno pomeni, da je
%   \(f ≤ \uline f\) in je \(\uline f = \liminf_{y → t}\bar f(y)\).

%   Obratno, če je pa \(\p{\uline f, \bar f}\) tak par. Potem sta navdol oziroma
%   navzgor polzvezni, torej definirata enostranska reza \(L\) in \(U\).

%   Očitno je \(L ⊆ \int{\p{Uᶜ}}\). Ampak \(\int{\p{Uᶜ}}\) je tudi enostranski
%   rez, torej definira navzdol polzvezno preslikavo, ki je manjša od \(\bar f\),
%   torej je tudi \(\int{\p{Uᶜ}} ⊆ L\).
% \end{dokaz}

Spet vidimo, da sta v splošnem objekta Dedekindovih in MacNeilleovih realnih
števil očitno različna.
\begin{lema}
  MacNeilleovo realno število \(x : \Rm\) je Dedekindovo natanko tedaj, ko sta
  pripadajoči tesni preslikavi \(\uline x\) in \(\bar x\) enaki.
\end{lema}
\begin{dokaz}
  Če sta preslikavi \(\uline x\) in \(\bar x\) enaki, je ta preslikava zvezni, torej po
  trditvi~\ref{th:Rd-maps} določa Dedekindovo realno število.

  Obratno, vsako Dedekindovo realno število je zunaj zvezna preslikava
  \(f : \e x → ℝ\) in par \(\p{f,f}\) je očitno tesen.
\end{dokaz}

\begin{trditev}
  Nad \(ℝ\) ne velja \(\Rm = \Rd\).
\end{trditev}
\begin{dokaz}
  Preslikavi \(χ_{(0,∞)}\) in \(χ_{[0,∞)}\) sta tesni, torej določata
  MacNeilleovo realno število, a se v \(0\) ne ujemata, tako da ne določata
  Dedekindovega realnega števila.
\end{dokaz}

Ostane nam še množica \(2^ℕ\).
\begin{definicija}
  \emph{Cantorjev prostor} je prostor \(ℕ → 2\) s kompaktno-odprto topologijo.
\end{definicija}
Cantorjev prostor se ponavadi predstavi kot podmnožico \(ℝ\) z vložitvijo
\[ (bₙ)ₙ ↦ ∑_{n}\frac{2bₙ}{3ⁿ}\text. \]
\begin{trditev}
  Tip \(2^ℕ\) je natanko snop lokalnih prerezov za Cantorjev prostor. To so
  torej zvezne preslikave \(U → 2^ℕ\).
\end{trditev}


%\section{Principi odločitve v topoloških modelih}\label{sec:odločitve}

V podrazdelku~\ref{sec:modeli-logika-odprtih} smo karakterizirali \(\lem*\) in
\(\wlem*\) z logiko odprtih množic. V tej logiki igra množica \(𝒪X\) vlogo
``tipa resničnostnih vrednosti'', v topoloških modelih smo pa v
primeru~\ref{ex:omega} tip resničnostnih vrednosti definirali kot tip \(Ω\).
Izkaže se, da se interpretacije formul kvantificiranih po \(Ω\) ne spremenijo,
če \(Ω\) zamenjamo z \(\c{𝒪X}\), saj so vsi elementi \(Ω\) globalni, formuli pa
ne vsebujeta enakosti. Tako trditvi~\ref{th:lem-is-discrete}
in~\ref{th:wlem-is-ext-disc} veljata kot navedeni tudi v topoloških modelih.
\begin{retrditev}{th:lem-is-discrete}
  Nad topološkim prostorom velja princip izključene tretje možnosti natanko
  tedaj, ko je prostor diskreten.
\end{retrditev}
\begin{retrditev}{th:wlem-is-ext-disc}
  Nad topološkim prostorom velja DeMorganov zakon natanko tedaj, ko je prostor
  ekstremalno nepovezan.
\end{retrditev}
Ker so dokazi enaki, jih tu ne ponovimo.

S pomočjo tipov pa lahko povemo tudi kaj o principih števne odločitve.

\begin{trditev}\label{th:lpov-lpo}
  Nad lokalno povezanimi prostori velja \(\lpo*\).
\end{trditev}
\begin{dokaz}
  Naj bo \(α : 2^ℕ\). Po trditvi~\ref{th:lpov-exponentiable} je \(α\) lokalno
  funkcija \(ℕ → 2\). Ker imamo zunaj \(\lpo*\), lahko odločimo
  \(α = 0 ∨ α \apart 0\) tam, kar pa pomeni, da to lahko odločimo tudi znotraj.
\end{dokaz}
To je zgolj delna karakterizacija, tako da se k tej še vrnemo.

\begin{izrek}\label{th:alpo-is-zerosets-open}
  Nad \(X\) velja \(\alpo*\) natanko tedaj, ko je vsaka ničelna množica odprta.
\end{izrek}
\begin{dokaz}
  Če je vsaka ničelna množica odprta je \(\i{x = 0}\) enak ničelni množici,
  torej skupaj z \(\i{x\apart 0}\) pokrijeta prostor.

  Obratno, če velja \(\alpo*\), morata \(\i{x=0}\) in \(\i{x\apart 0}\) pokriti
  cel prostor. Ker je ničelna množica disjunktna množici \(\i{x\apart 0}\), je
  enaka \(\i{x=0}\), torej je odprta.
\end{dokaz}

Sedaj lahko pokažemo implikacije iz podrazdelka~\ref{sec:logika-odločitve} v
splošnem niso obrnljive.
\begin{trditev}
  Nad Cantorjevim prostorom ne velja \(\lpo*\).
\end{trditev}
\begin{dokaz}
  Identiteta \(α : 2^ℕ → 2^ℕ\) je znotraj element tipa \(2^ℕ\) in zanjo je
  \(\i{\lpo(α)} = 2^ℕ⧵\{0\}\), torej \(\lpo*\) ne velja povsod.
\end{dokaz}

\begin{trditev}
  Nad \(ℝ\) velja \(\lpo*\) in ne velja \(\alpo*\).
\end{trditev}
\begin{dokaz}
  Ker je prostor \(ℝ\) lokalno povezan, po trditvi~\ref{th:lpov-lpo} nad njim
  velja \(\lpo*\). Poglejmo si sedaj identiteto \(\id : ℝ → ℝ\). Ta je znotraj
  Dedekindovo realno število in zanjo je \(\i{\alpo(\id)} = ℝ⧵\{0\}\), torej
  \(\alpo*\) ne velja nad \(ℝ\).
\end{dokaz}

Ker prostor \(\Ncof\) naravnih števil s kokončno topologijo ni diskreten, nad
njim ne velja \(\lem*\). To lahko izkoristimo, za ločevanje \(\lem*\) od
\(\alpo*\) in \(\wlem*\) v naslednjih trditvah. Ta prostor sem odkrila s pomočjo
matematične podatkovne baze~\cite{pibase}, in ga bomo še srečali.

\begin{trditev}
  Nad \(\Ncof\) velja \(\alpo*\) in ne velja \(\lem*\).
\end{trditev}
\begin{dokaz}
  Naj bo \(f : U → ℝ\) zvezna preslikava in \(x, x' ∈ \im f\). Potem sta
  \(f⁻¹(x)\) in \(f⁻¹(x')\) neprazni odprti množici, torej imata neprazen
  presek. Sledi, da je \(x = x'\), torej je funkcija \(f\) konstantna. Ker lahko
  potem v metateoriji odločimo \(\alpo(f(t))\) (za nek \(t ∈ U\)), lahko
  odločimo tudi \(\alpo(f)\) znotraj.
\end{dokaz}

\begin{trditev}
  Nad \(\Ncof\) velja \(\wlem*\) in ne velja \(\lem*\).
\end{trditev}
\begin{dokaz}
  Naj bo \(U\) odprta množica. Njena zunanjost (torej negacija) je pa bodisi cel
  prostor, bodisi prazna. Za te odprte množice pa velja, da so komplementirane,
  torej \(\wlem*\) drži.
\end{dokaz}

% Obrate teh implikacij lahko torej jemljemo kot nekonstruktivni principi, a o teh
% ni veliko znano. Vemo le, da očitno \(\Rd = \Rc\) implicira \(\lpo* ⇒ \alpo*\),
% saj je \(\lpo*\) ekvivalenten \(\alpo*\) za Cauchyjeva realna števila, tako da
% je to zelo šibek princip. Kasneje si bomo ogledali kdaj obrati veljajo na nek
% strožji in bolj strukturiran način, ki ima povezavo s teorijo izračunljivosti,
% in ta upamo, da nam da močnejše principe, ki bodo potem tudi topološko bolj
% zanimivi.

Oglejmo si še \(\awlpo*\). Ta pravi, da za vsak \(x : ℝ\) velja \(x≤0 ∨¬(x≤0)\).
Ker je \(x≤0\) natanko \(¬(x>0)\) je \(\awlpo*\) ekvivalenten \(\wlem*_{Σ_{\Rd}}\).
Če si pogledamo definicijo ekstremalno nepovezanih prostorov, in v njej zamenjamo
odprte množice z realnimi, dobimo sledečo lastnost.
\begin{definicija}
  Prostor je \emph{realno nepovezan}, ko je za vsako funkcijo \(f : X → ℝ\)
  množica \(\cl\set{t∈X}{f(t)>0}\) odprta.
\end{definicija}

\begin{trditev}\label{th:awlpo-is-basically-disconnected}
  Nad \(X\) velja \(\awlpo*\) natanko tedaj, ko je vsaka odprta podmnožica \(X\)
  realno nepovezana.
\end{trditev}
Dokaz trditve je enak kot~\ref{th:wlem-is-ext-disc}, tako da ga ne ponovimo.
Velja pa zanimiva posledica, namreč, da nad realno nepovezanimi prostori
obstaja funkcija "predznak" v naslednjem smislu.
\begin{trditev}
  Če nad \(X\) velja \(\awlpo*\), potem za vsak \(x : ℝ\) obstaja
  \(u : \e x → ℝ\), tako da je \(u\) nad \(\e x\) obrnljiv, in velja
  \(\e x ⊩ x = u\abs x\).
\end{trditev}
\begin{dokaz}
  Brez škode za splošnost naj bo \(\e x = X\).

  Princip \(\awlpo*\) potem pravi, da je množica \(U ≔ \cl{\i{x>0}}\) odprta,
  torej tvori particijo in lahko definiramo
  \[ u ≔
    \begin{cases}
       1&; U\\
      -1&; Uᶜ\text.
    \end{cases}
  \]
  %\(U ⊩ u = 1\) in \(Uᶜ ⊩ u = -1\).
  Ta \(u\) je obrnljiv, saj je \(u⋅u = 1\). Prav tako velja želena enačba, saj
  je \(U\) vsebovan v \(x ≥ 0\), in velja \(f ≤ 0\) na \(Uᶜ\).
\end{dokaz}

Če želimo zares definirati predznak, moramo uporabiti \(\alpo\). Ta
pravi, da je ničelna množica \(x\) odprta, torej lahko na njej definiramo
\[ u ≔
  \begin{cases}
     1&; x < 0\\
     0&; x = 0\\
    -1&; x > 0\text.
  \end{cases}
\]

\begin{definicija}
  Prostor \(X\) je \emph{skoraj P-prostor}, ko je za vsak \(f : X → ℝ\) množica
  \(\i{f > 0}\) regularna~\cite{Levy77}.
\end{definicija}
Seveda je vsak P-prostor tudi skoraj P-prostor.
Res, naj bodo \(Zᵢ ≔ f⁻¹[2⁻ⁱ,∞)\). Njihova unija \(Z = \i{f > 0}\) je zaprta,
saj je \(X\) P-prostor. Potem je pa \(\int{\p{\cl{Z}}} = \int Z = Z\).

\begin{trditev}\label{th:amp-is-almost-psp}
  Nad \(X\) velja \(\amp*\) natanko tedaj, ko je vsaka odprta podmnožica \(X\)
  skoraj P-prostor.
\end{trditev}
\begin{dokaz}
  Princip pravi, da za vsak tak \(f\) velja
  \(\int{\p{\cl{\i{f > 0}}}} ⊆ \i{f > 0}\). Ker obratna enakost očitno velja, je
  to ravno definicija regularnosti odprte množice.
\end{dokaz}
Analogno velja tudi \(\mp*\) natanko tedaj, ko je vsaka semiodločljiva odprta
množica regularna.
\begin{opomba}
  V~\cite[2.1]{Levy77} piše, da je vsaka odprta podmnožica skoraj P-prostora
  ``očitno'' skoraj P-prostor. Jaz tega ne vidim, mogoče je res samo za
  \(T_{3.5}\) prostore, na katere se omeji članek.
\end{opomba}

Konstruktivno velja \(\awlpo*∧\amp* ⇔ \alpo*\).
V trditvah~\ref{th:awlpo-is-basically-disconnected},~\ref{th:amp-is-almost-psp},
in~\ref{th:alpo-is-zerosets-open}, smo karakterizirali vse prostore, ki so v
igri v tej ekvivalenci, tako da bi morala veljati ekvivalenca med njimi.
\begin{izrek}
  Ničelne množice \(f : X → ℝ\) so odprte natanko tedaj, ko je \(X\) skoraj
  P-prostor in realno nepovezan.
\end{izrek}
\begin{dokaz}
  Če je vsaka množica \(\i{f > 0}\) zaprta, je tudi regularna (vse odprto zaprte
  množice so regularne). Prav tako je njeno zaprtje odprto.

  Obratno, če je \(\i{f > 0} = \int{\p{\cl{\i{f>0}}}}\), in je zaprtje
  \(\i{f>0}\) odprto, je potem \(\i{f>0} = \cl{\i{f>0}}\), torej je zaprta.
\end{dokaz}
Avtorici ni znano, če se ta izrek pojavi kje v literaturi, predvsem ker je
večina literature o teh prostorih objavljene pod predpostavko \(T_{3.5}\).
Pod predpostavko \(T_{3.5}\) torej najdemo izrek
"P-prostori so natanko realno nepovezani skoraj P-prostori" v~\cite{Levy77}
in~\cite[4J(3)]{GJ60}.
To porodi dve zanimivi vprašanji. Prvič, kako nujna je predpostavka \(T_{3.5}\)
za razvoj teorije kolobarjev realnih funkcij, in drugič, kaj lastnost
\(T_{3.5}\) pomeni v interni logiki topološkega modela. Avtorica žal nima
odgovora na nobeno od teh vprašanj, saj je prvo preobsežno za to delo, za
drugega pa ni našla pravega navdiha.


%%% Local Variables:
%%% mode: latex
%%% TeX-master: "main"
%%% End:

\input{realna-števila.tex}


%%% Local Variables:
%%% mode: latex
%%% TeX-master: "main"
%%% End:

