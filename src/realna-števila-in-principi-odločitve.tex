\section{Realna števila in principi odločitve v topoloških modelih}
\label{sec:reals}
\label{sec:odločitve}

\subsection{Objekti realnih števil}

V~\ref{sec:modeli-heyting} smo videli dva ``topološka'' objekta realnih števil,
objekt \(ℛ\) iz primera~\ref{ex:reals} in objekt \(\c ℝ\), definiran
v~\ref{def:constant-hvs}. Vprašanje je, kako sta dva objekta povezana z
internimi konstrukcijami realnih števil definiranih v~\ref{sec:logika-reals}.

Objekt \(ℛ\) je vedno enak objektu Dedekindovih realnih števil.
\begin{trditev}\label{th:Rd-maps}
  Nad \(X\) je objekt Dedekindovih realnih števil tip \(ℛ\) iz
  primera~\ref{ex:reals}.
  % Nad \(X\) je objekt Dedekindovih realnih števil tip
  % \(\set{f : U → ℝ}{U ∈ 𝒪X\text{, }f\text{ zvezna}}\), z enakostjo definirano
  % kot \(\i{f = g} ≔ \int{\set{t ∈ X}{f(t) = g(t)}}\).
\end{trditev}
\begin{dokaz}
  Naj bo \(x = \p{L, U}\) Dedekindov rez v interni logiki.
  Sedaj lahko zunaj za \(t ∈ \e x\) definiramo
  \begin{align*}
    L(t) &≔ \set{q : ℚ}{t ∈ \i{q ∈ L}}\text{ in}\\
    U(t) &≔ \set{r : ℚ}{t ∈ \i{r ∈ U}}\text.
  \end{align*}
  Ta tvorita Dedekindov rez zunaj, tako da nam \(t ↦ \p{L(t), U(t)}\) definira
  preslikavo \(\hat x : \e x → ℝ\).
  Pokazati ostane, da je ta preslikava zvezna.
  Naj bo \(\p{a,b}\) racionalen interval v \(ℝ\). Praslika tega intervala z
  \(\hat x\) je množica \(\set{t ∈ \e x}{a < \hat x(t) < b}\).
  Ta pogoj je ekvivalenten \(t ∈ \i{a ∈ L ∧ b ∈ U}\), torej je praslika kar
  enaka \(\i{a∈L∧b∈U}\), torej je odprta in je \(\hat x\) zvezna.

  Obratno, če je \(x : \e x → ℝ\) zvezna preslikava, lahko tvorimo preslikavi
  \begin{align*}
    L(t) &≔ \set{q : ℚ}{q < x(t)}\text{ in}\\
    U(t) &≔ \set{r : ℚ}{x(t) < r}\text.
  \end{align*}
  Ti znotraj tvorita Dedekindov rez \(\p{L, U}\).
\end{dokaz}

Med drugim to pomeni, da so Cauchyjeva realna števila podmnožica zveznih
preslikav v \(ℝ\). Ampak katerih? Oglejmo si najprej tip \(\c ℝ\).

\begin{definicija}
  Naj bo \(U ∈ 𝒪X\). Preslikava \(f : U → ℝ\) je \emph{lokalno konstantna}, ko
  velja \(\eventually{V ⊆ U}{f{\res V}\text{ je konstantna}}\).
\end{definicija}
Očitno torej vsaka lokalno konstantna preslikava določa Dedekindovo realno
število. Povemo pa lahko malo več.

\begin{trditev}\label{th:R-into-Rc}
  Vsaka lokalno konstantna preslikava \(f : U → ℝ\) določa Cauchyjevo realno
  število.
\end{trditev}
\begin{dokaz}
  Naj bo najprej \(f : U → ℝ\) konstantno \(x ∈ ℝ\). Zunaj je \(x\)
  predstavljeno s Cauchyjevim zaporedjem \((xₙ)ₙ\). Sedaj lahko tvorimo
  konstantne preslikave \({\mathrm{c}_{xₙ} : U → ℚ}\), ki znotraj tvorijo
  Cauchyjevo zaporedje, torej določajo Cauchyjevo realno število \(x\).

  Če je \(f : U → ℝ\) lokalno konstantna, po definiciji obstaja pokritje \(C\),
  tako da je na vsakem elementu pokritja \(f\) konstantna. Po prvi točki torej
  na vsakem \(V ∈ C\) obstaja Cauchyjevo realno število, ki je določeno s
  preslikavo \(f{\res V}\). Potem po lokalnosti veljavnosti~\ref{th:valid-local}
  obstaja Cauchyjevo realno število na celotnem \(U\), ki je določeno s
  preslikavo \(f\).
\end{dokaz}
V določenih primerih pa velja tudi obrat.
\begin{trditev}\label{th:Rc-maps}
  Naj bo prostor \(X\) lokalno povezan. Tedaj je nad \(X\) objekt Cauchyjevih
  realnih števil tip \(\c ℝ\).
\end{trditev}
\begin{dokaz}
  Pokazati moramo zgolj, da je vsako Cauchyjevo realno število zunaj lokalno
  konstantna preslikava.

  Nad \(t ∈ \e x\) dobimo Cauchyjevo zaporedje \((xₙ(t))ₙ\). Ker je \(X\)
  lokalno povezan, ima pokritje iz povezanih množic, tako da brez škode za
  splošnost predpostavimo, da je \(\e x\) povezan. Sledi, da so \(xₙ\)
  konstantne preslikave, torej je tudi preslikava \(t ↦ (xₙ(t))ₙ\) konstantna.
\end{dokaz}

Trditvi~\ref{th:Rd-maps} in~\ref{th:Rc-maps} nakazujeta, da vsebovanosti
\(\c ℝ ⊆ \Rc ⊆ \Rd\) v splošnem niso obrnljive.

\begin{trditev}
  Nad \(ℝ\) sta objekta Dedekindovih in Cauchyjevih realnih števil različna.
\end{trditev}
\begin{dokaz}
  Identiteta na \(ℝ\) je Dedekindovo realno število, ki pa ni lokalno
  konstantno, torej ni Cauchyjevo.
\end{dokaz}

\begin{trditev}
  Nad \(2^ℕ\) sta objekta Cauchyjevih in lokalno konstantnih realnih števil
  različna.
\end{trditev}
\begin{dokaz}
  Definirajmo \(x : 2^ℕ → ℝ\) kot \(x(α) ≔ \sum_{n=0}^∞ \frac{2α(n)}{3ⁿ}\). Ta
  preslikava je vložitev Cantorjeve množice v \(ℝ\), in je znotraj Cauchyjevo
  realno število, saj je limita racionalnih števil. Ampak ni lokalno konstantno,
  saj je \({x⁻¹(0) = \{0\}}\), ki ni odprta množica.
\end{dokaz}

Kaj pa MacNeillova realna števila? Teh je več kot Dedekindovih, tako da če jih
želimo karakterizirati kot nekakšne funkcije iz odprtih podmnožic \(X\), bodo te
morale biti nezvezne. Izkaže se, da so odgovor polzvezne
funkcije~\cite[posl.~D4.7.5]{Johnstone02}. Seveda, če je funkcija polzvezna
navzdol in navzgor, je zvezna. Ampak, lahko vzamemo par navzdol in navzgor
polzveznih preslikav \(\uline f\) in \(\bar f\), ki sta si nekako ``čim
bližje''. To pomeni, da želimo, da je \(\uline f\) največja navzdol polzvezna
preslikava manjša ali enaka \(\bar f\), in obratno.

\begin{lema}\label{th:onesided-cuts-are-scts}
  V topoloških modelih so dolnji (gornji) enostranski rezi predstavljeni z
  navzdol (navzgor) polzveznimi preslikavami.
\end{lema}
\begin{dokaz}
  Naj bo \(L\) dolnji enostranski rez. Kot za Dedekindove reze tvorimo
  preslikavo \(t ↦ \set{q:ℚ}{t ∈ \i{q∈L}}\). To je preslikava \(x : \e L → ℝ\),
  saj so klasično enostranski rezi ekvivalentni dvostranskim.

  Oglejmo si sedaj \(x⁻¹\p{p,∞}\). To je natanko \(\set{t∈\e L}{p < x(t)}\). Po
  definiciji je ta pogoj ekvivalenten \(t ∈ \i{p ∈ L}\), torej je praslika
  odprta.

  Obratno, naj bo \(x : U → ℝ\) navzdol polzvezna. Potem lahko definiramo
  preslikavo \(L(t) ≔ \set{p ∈ ℚ}{p < x(t)}\). Ta znotraj tvori dolnji
  enostranski rez, saj so morfizmi v \(Ω\) natanko operacije v \(Ω\).
\end{dokaz}
Ker pa interno nimamo gornjega reza, ne bomo znali izraziti pogoja \(x(t) < q\)
na lep način, tako da v splošnem ta preslikava ni zvezna.

\begin{trditev}\label{th:Rm-maps}
  Objekt MacNeilleovih realnih števil je natanko tip parov polzveznih
  funkcij \(\p{\uline f, \bar f}\) tipa \(U → ℝ\) za \(U ∈ 𝒪X\), za kateri velja
  pogoj \emph{tesnosti}, ki se glasi:
  \[
    \uline f(t) = \liminf_{y → t}\bar   f(y)\qquad\text{ in}\qquad
    \bar   f(t) = \limsup_{y → t}\uline f(y)\text,
  \]
  z enakostjo definirano kot
  \[
    \i{\p{\uline f, \bar f} = \p{\uline g, \bar g}} ≔
    \int\set{t ∈ U}{\uline f(t) = \uline g(t) ∧ \bar f(t) = \bar g(t)}\text.
  \]
\end{trditev}
Dokaz te trditve ni kompliciran, ampak se zanaša na veliko analize, tako da ga
tu ne ponovimo. Bralka si ga lahko ogleda v~\cite[D4.7.5]{Johnstone02}.
% \begin{dokaz}
%   Po gornji lemi lahko za MacNeillovo realno število \(x = \p{L,U}\)
%   konstruiramo navzdol in navzgor polzvezni preslikavi \(\uline f\) in
%   \(\bar f\). Pokazati nam ostane, da sta preslikavi tesni. Pokažimo zgolj prvi
%   pogoj, saj sta si simetrična.

%   Limes inferior definira navzdol polzvezno preslikavo, namreč supremum vseh
%   navzgor polzveznih preslikav manjših ali enakih \(\bar f\).
%   To pomeni, da moramo pokazati, da je vsaka navzdol polzvezna preslikava manjša
%   ali enaka \(\uline f\).

%   Vemo, da so navzdol polzvezne preslikave natanko enostranski rezi v interni
%   logiki \(X\), tako da naj bo \(L'\) tak rez. Pogoj, da je preslikava manjša al
%   enaka \(\bar f\) pa pravi, da je \(L'\) vsebovan v komplementu \(U\).

%   Sedaj pa naj bo \(q ∈ L'\). Ker je \(L'\) navzgor odprt obstaja nek \(s > q\),
%   ki je še vedno v \(L'\), torej je \(s ∈ Uᶜ\). To pa pomeni, da je
%   \(q ∈ \int{\p{Uᶜ}} = L\), in je \(L' ⊆ L\). To pa očitno pomeni, da je
%   \(f ≤ \uline f\) in je \(\uline f = \liminf_{y → t}\bar f(y)\).

%   Obratno, če je pa \(\p{\uline f, \bar f}\) tak par. Potem sta navdol oziroma
%   navzgor polzvezni, torej definirata enostranska reza \(L\) in \(U\).

%   Očitno je \(L ⊆ \int{\p{Uᶜ}}\). Ampak \(\int{\p{Uᶜ}}\) je tudi enostranski
%   rez, torej definira navzdol polzvezno preslikavo, ki je manjša od \(\bar f\),
%   torej je tudi \(\int{\p{Uᶜ}} ⊆ L\).
% \end{dokaz}

Spet vidimo, da sta v splošnem objekta Dedekindovih in MacNeilleovih realnih
števil očitno različna.
\begin{lema}
  MacNeilleovo realno število \(x : \Rm\) je Dedekindovo natanko tedaj, ko sta
  pripadajoči tesni preslikavi \(\uline x\) in \(\bar x\) enaki.
\end{lema}
\begin{dokaz}
  Če sta preslikavi \(\uline x\) in \(\bar x\) enaki, je ta preslikava zvezni, torej po
  trditvi~\ref{th:Rd-maps} določa Dedekindovo realno število. Obratno, vsako
  Dedekindovo realno število je zunaj zvezna preslikava \(f : \e x → ℝ\) in par
  \(\p{f,f}\) je očitno tesen.
\end{dokaz}

\begin{trditev}
  Nad \(ℝ\) ne velja \(\Rm = \Rd\).
\end{trditev}
\begin{dokaz}
  Preslikavi \(χ_{(0,∞)}\) in \(χ_{[0,∞)}\) sta tesni, torej določata
  MacNeilleovo realno število, a se v \(0\) ne ujemata, tako da ne določata
  Dedekindovega realnega števila.
\end{dokaz}

Ostane nam še množica \(2^ℕ\).
\begin{definicija}
  \emph{Cantorjev prostor} je prostor preslikav \(ℕ → 2\) s kompaktno-odprto
  topologijo.
\end{definicija}
Cantorjev prostor se ponavadi predstavi kot podmnožico \(ℝ\) z vložitvijo
\[ (bₙ)ₙ ↦ ∑_{n}\frac{2bₙ}{3ⁿ}\text. \]
\begin{trditev}
  Tip \(2^ℕ\) je natanko snop lokalnih prerezov za Cantorjev prostor. To so
  torej zvezne preslikave \(U → 2^ℕ\).
\end{trditev}


\section{Principi odločitve}\label{sec:odločitve}

Spomnimo se najprej karakterizacij \(\lem*\) in \(\wlem*\) iz
podrazdelka~\ref{sec:modeli-logika-odprtih}.

TODO: renumber
\begin{trditev}\label{th:lem-is-partition-second}
  Nad topološkim prostorom velja princip izključene tretje možnosti natanko
  tedaj, ko je prostor particijski.
\end{trditev}
\begin{trditev}\label{th:wlem-is-ext-disc-second}
  Nad topološkim prostorom velja DeMorganov zakon natanko tedaj, ko je prostor
  ekstremalno nepovezan.
\end{trditev}

Ker so naši prostori \(T₀\), so particijski prostori natanko diskretni. Res, če
so točke zaprte, in je vsaka zaprta množica odprta, so točke odprte, torej je
prostor diskreten.


Pokažimo sedaj, da implikacije iz~\ref{sec:logika-odločitve} niso obrnljive.

\begin{trditev}
  Nad \(2^ℕ\) ne velja \(\lpo*\).
\end{trditev}
\begin{dokaz}
  Naj bo \(α : 2^ℕ → 2^ℕ\) identiteta. Resničnostna vrednost \(\lpo(α)\) je
  \(2^ℕ⧵\{0\}\), saj je resničnostna vrednost \(α = 0\) enaka \(∅\).
\end{dokaz}

\begin{trditev}
  Nad \(ℝ\) velja \(\lpo*\) in ne velja \(\alpo*\).
\end{trditev}
\begin{dokaz}
  Naj bo \(x : ℝ\) zunaj identiteta. Potem je \(\i{\alpo(x)} = ℝ⧵\{0\}\).

  Preostanek tega dokaza izpeljemo malo kasneje, v izreku~\ref{th:lpov-lpo}
\end{dokaz}

\begin{trditev}
  Nad \(\Ncof\) velja \(\alpo*\) in ne velja \(\lem*\).
\end{trditev}
\begin{dokaz}
  Prostor očitno ni diskreten. Naj bo \(x : ℝ\), torej preslikava \(\e x → ℝ\).
  Naj bo \(t, t' ∈ \im x\). Potem sta \(x⁻¹(t)\) in \(x⁻¹(t')\) neprazni odprti
  množica. To pa pomeni, da sta obe kokončni, torej imata neprazen presek, od
  koder sledi, da sta si enaka. Torej je vsako notranje realno število
  konstantno. Sedaj pa lahko za to realno število \(\alpo(x)\) odločimo zunaj.
\end{dokaz}

\begin{trditev}
  Nad \(\Ncof\) velja \(\wlem*\) in ne velja \(\lem*\).
\end{trditev}
\begin{dokaz}
  Prostor še vedno ni diskreten. Naj bo sedaj \(U\) odprta množica.
  Njena zunanjost (torej negacija) je pa bodisi cel prostor, bodisi prazna. Za
  te odprte množice pa velja da so komplementirane, torej \(\wlem*\) drži.
\end{dokaz}

Obrate teh implikacij lahko torej jemljemo kot nekonstruktivni principi, a o teh
ni veliko znano. Vemo le, da očitno \(\Rd = \Rc\) implicira \(\alpo* ⇔ \lpo*\),
saj je \(\lpo*\) ekvivalenten \(\alpo*\) za Cauchyjeva realna števila, tako da
je to zelo šibek princip. Kasneje si bomo ogledali kdaj obrati veljajo na nek
strožji in bolj strukturiran način, ki ima povezavo s teorijo izračunljivosti,
in ta upamo, da nam da močnejše principe, ki bodo potem tudi topološko bolj
zanimivi.

TODO: premakni zgoraj?

Vemo pa vsaj nekaj.
\begin{izrek}\label{th:lpov-lpo}
  Če je \(X\) lokalno povezan, velja \(X ⊩ \lpo*\).
\end{izrek}
\begin{dokaz}
  Naj bo \(α : 2^ℕ\). Ker je \(X\) lokalno povezan, je množica \(2^ℕ\) kar
  množica preslikav iz \(ℕ\) v \(2\). To pa pomeni, da je lokalno \(α = f\), za
  nek \(f : ℕ → 2\). Zunaj pa imamo \(\lpo*\), tako da lahko odločimo \(f = 0 ∨
  f \apart 0\), kar pa pomeni, da to lahko odločimo tudi za \(α\), torej nad
  \(X\) velja \(\lpo*\).
\end{dokaz}
\begin{opomba}
  Ker je \(\lpo*\) natanko \(\alpo*\) za Cauchyjeva realna števila (TODO: daj to
  nekam na samo, zdej sem že petič napisala), in potrebujemo le, da so elementi
  lokalno konstantni, zadošča predpostaviti \(\Rc = \c ℝ\). To pomeni, da je to
  kar močen princip, saj je zadosten za \(\lpo*\).

  TODO: a je ekvivalenca? smz da, ker \(x : \Rc\) je zvezna \(\e x → ℝ\), če
  mamo \(\lpo*\) je \(\i{x = a} = \i{x \apart a}ᶜ\), tko da je \(x\) lokalno
  konstantna.
\end{opomba}

TODO: Elephant, D4.7: It can be shown that, for a locale \(X\), \(\Rc\) coincides
with \(\c ℝ\) iff for every open \(U\), the lattice of (relatively) clopen
sublocales of \(U\) is closed under countable unions. (For a second countable
locale this is equivalent to local connectedness)


%%% Local Variables:
%%% mode: latex
%%% TeX-master: "main"
%%% End:

\section{Realna števila v topoloških modelih}\label{sec:reals}

\begin{trditev}\label{th:Rd-maps}
  Nad \(X\) je objekt Dedekindovih realnih števil \(𝒪X\)-množica iz
  primera~\ref{ex:reals}.
  % Nad \(X\) je objekt Dedekindovih realnih števil \(𝒪X\)-množica
  % \(\set{f : U → ℝ}{U ∈ 𝒪X\text{, }f\text{ zvezna}}\), z enakostjo definirano
  % kot \(\i{f = g} ≔ \int{\set{t ∈ X}{f(t) = g(t)}}\).
\end{trditev}
\begin{dokaz}
  Naj bo \(x = \p{L, U}\) dedekindov rez v interni logiki.
  Sedaj lahko zunaj za \(t ∈ \e x\) definiramo
  \[ Lₜ ≔ \set{q : ℚ}{t ∈ \i{q ∈ L}}\text{ in} \]
  \[ Uₜ ≔ \set{r : ℚ}{t ∈ \i{r ∈ U}}\text. \]
  Ta tvorita Dedekindov rez, tako da nam \(t ↦ \p{Lₜ, Uₜ}\) definira preslikavo
  iz \(\e x → ℝ\).
  Pokazati ostane, da je ta preslikava (označimo jo \(\hat x\)) zvezna.
  Naj bo \(\p{a,b}\) racionalen interval v \(ℝ\). Praslika tega intervala s
  \(\hat x\) je množica \(\set{t ∈ \e x}{a < \hat x(t) < b}\).
  Ta pogoj je pa ekvivalenten \(t ∈ \i{a ∈ L ∧ b ∈ U}\), torej je praslika kar
  enaka \(\i{a∈L∧b∈U}\). Ta je pa odprta, torej je \(\hat x\) zvezna.

  Obratno, če je \(x : \e x → ℝ\) zvezna preslikava, lahko tvorimo preslikavi
  \[ L(t) ≔ \set{q : ℚ}{q < x(t)}\text{ in} \]
  \[ U(t) ≔ \set{r : ℚ}{x(t) < r}\text. \]
  Ti znotraj tvorita Dedekindov rez \(\p{L, U}\).
\end{dokaz}

\begin{trditev}\label{th:Rc-maps}
  Naj bo prostor \(X\) lokalno povezan. Tedaj je nad \(X\) objekt Cauchyjevih
  realnih števil \(𝒪X\)-množica \(\c ℝ\).
\end{trditev}
\begin{dokaz}
  Pokazati je zares treba, da je objekt Cauchyjevih realnih števil \(\g{\c ℝ}\),
  torej da so lokalno konstantne preslikave v \(ℝ\).

  Podobno kot zgoraj nad \(t ∈ \e x\) dobimo Cauchyjevo zaporedje \((xₙ(t))ₙ\),
  torej preslikavo \(\e x → ℝ\). Ker je \(X\) lokalno povezan ima pokritje iz
  povezanih množic, tako da brez škode za splošnost predpostavimo, da je
  \(\e x\) povezan. Ker pa je povezan, pa vemo, da so vsi \(xᵢ\) konstantne
  preslikave, torej morajo biti vsi \(xₙ(t)\) enaki (za vse \(t ∈ \e x\)), tako
  da je \(t ↦ (xₙ(t))ₙ\) konstantna.

  Obratno pa, če je \(f : U → ℝ\) konstantna preslikava, recimo konstantno
  \({x ∈ ℝ}\), ima potem pripadajoče Cauchyjevo zaporedje \((xₙ)ₙ\). Sedaj pa
  lahko tvorimo konstantne preslikave \(xₙ : U → ℚ\), ki torej znotraj tvorijo
  Cauchyjevo zaporedje.
\end{dokaz}

\begin{trditev}
  Nad \(ℝ\) sta objekta Dedekindovih in Cauchyjevih realnih števil različna.
\end{trditev}
\begin{dokaz}
  Identiteta na \(ℝ\) je Dedekindovo realno število, ki pa ni lokalno
  konstantno, torej ni Cauchyjevo.
\end{dokaz}

TODO: cite elephant
\begin{trditev}\label{real:Rm-maps}
  Objekt MacNeillovih realnih števil je natanko \(ℒ\)-množica parov funkcij
  \(\p{\uline f, \bar f}\) tipa \(U → ℝ\) za \(U ∈ 𝒪X\), za katere velja
  \begin{align*}
    \bar   f(t) &=
    \limsup_{y → t}\uline f(y)\text{ in}\\
    %\inf\set{\sup \uline f(V)}{V\nbd t}\text{ in}\\
    \uline f(t) &=
    \liminf_{y → t}\bar   f(y)\text.
    % \sup\set{\inf \bar   f(V)}{V\nbd t}\text.
  \end{align*}
\end{trditev}
\begin{dokaz}
  TODO: reword all this. kaj je s temi liminfi? a to dela sploh?

  Če je \(x = \p{L,U}\) Dedekind-MacNeilleov rez, lahko za \(L\) in \(U\)
  konstruiramo funkciji \(\uline f\) in \(\bar f\) kot pri Dedekindovih rezih,
  s tem da slikamo v enostranske reze zunaj. Klasično so te ekvivalentni
  Dedekindovim rezom, tako da dobimo preslikavi \(\e x → ℝ\).

  Limes inferior definira navzdol polzvezno preslikavo, namreč supremum vseh
  navzgor polzveznih preslikav manjših ali enakih \(\bar f\).
  To pomeni, da moramo pokazati, da je vsaka navzdol polzvezna preslikava manjša
  ali enaka \(\uline f\).

  Vemo, da so navzdol polzvezne preslikave natanko enostranski rezi v interni
  logiki \(X\), tako da naj bo \(L'\) tak rez. Pogoj, da je preslikava manjša al
  enaka \(\bar f\) pa pravi, da je \(L'\) vsebovan v komplementu \(U\).

  Sedaj pa naj bo \(q ∈ L'\). Ker je \(L'\) navzgor odprt obstaja nek \(s > q\),
  ki je še vedno v \(L'\), torej je \(s ∈ Uᶜ\). To pa pomeni, da je
  \(q ∈ \int{\p{Uᶜ}} = L\), in je \(L' ⊆ L\). To pa očitno pomeni, da je
  \(f ≤ \uline f\) in je \(\uline f = \liminf_{y → t}\bar f(y)\).

  % Naj bo \(t ∈ \e x\). Ker je \(L = \int{Uᶜ}\), je
  % \(\uline f(t) = \int{\bar f(t)ᶜ} = \int{\set{r:ℚ}{t∈\i{r∈U}}ᶜ}\).
  % Komplement notranje množice je pa kar \(\set{r:ℚ}{t∉\i{r∈U}}\), čigar
  % notranjost je \(\set{r:ℚ}{t∈\i{r∉U}}\).

  % % i'm oopid, to ni treba…
  % Pokažimo zgolj, da je \(\uline f\) navzdol polzvenzna, saj je dokaz
  % simetričen. Naj bo \(a ∈ ℝ\). Potem je
  % \(\uline f⁻¹(a,∞) = \set{t ∈ \e x}{f(t) > a}\).
  % Po definiciji je \(\uline f(t) = \set{q : ℚ}{t ∈ \i{q < x}}\), in ta je večji
  % od \(a\) ko je \(a\) element \(f(t)\), torej ko velja \(t ∈ \i{a < x}\). Sledi,
  % da je \(\uline f⁻¹(a,∞) = \i{a < x}\), torej je navzdol polzvezna.

  Obratno, če je pa \(\p{\uline f, \bar f}\) tak par, preslikavi definirata
  enostranska reza \(L\) in \(U\). Očitno je \(L ⊆ \int{\p{Uᶜ}}\). Ampak
  \(\int{\p{Uᶜ}}\) je tudi enostranski rez, torej definira navzdol polzvezno
  preslikavo, ki je manjša od \(\bar f\), torej je tudi \(\int{\p{Uᶜ}} ⊆ L\).
\end{dokaz}

\begin{trditev}
  Nad \(\p{3,\{∅, \{1\}, \{2\}, \{1,2\}, \{0,1,2\}\}}\) ne velja \(\Rm = \Rd\).
\end{trditev}
\begin{dokaz}
  Naj bo \(x ≔ \sup\set{x : \Rm}{x = 0 ∨ x = 1∧\{1\}}\).
  To je po lemi~\ref{th:Rm-sup} MacNeillovo realno število, saj je poseljeno z
  \(0\) in omejeno z \(1\).

  Poglejmo sedaj, če velja \(x < 1 ∨ x > 0\). Če velja \(x < 1\), je potem
  \(x = 0\), torej \(\{1\}\) ne drži. Sledi, da je \(\i{x < 1} = \{2\}\).
  Podobno lahko sklepamo, da je \(\i{x > 1} = \{1\}\). To pa pomeni, da \(x\) ni
  locirano pri \(0\), torej ni Dedekindovo.
\end{dokaz}
\begin{dokaz}[Alternativni dokaz]
  Definirajmo MacNeillovo realno število s preslikavama
  \begin{align*}
    \uline f(0) = 0 && \bar f(0) = 1\\
    \uline f(1) = 1 && \bar f(1) = 1\\
    \uline f(2) = 0 && \bar f(2) = 0
  \end{align*}

  Ti očitno zadoščata pogojem zgoraj, in se na \(0\) ne ujemata, tako da ne
  definirata Dedekindovega realnega števila.
\end{dokaz}


\subsection{\(\Rd{} = \Rc\)}\label{sec:reals-Rd=Rc}

Najprej si oglejmo klasičen dokaz ekvivalence Dedekindovih in Cauchyjevih
realnih števil.
\begin{izrek}[Klasični]
  Pokazati je zgolj potrebno, da ima vsak Dedekindov rez pripadajoče Cauchyjevo
  zaporedje.
  Naj bo \(\p{L, U}\) obojestranski dedekindov rez. Cauchyjevo zaporedje lahko
  podamo kot zaporedje hitro padajočih racionalnih intervalov.

  Naj bosta \(p₀ ∈ L\) in \(q₀ ∈ U\) racionalni števili.
  Potem pa na \(n\)-tem koraku definiramo
  \[ a ≔ \frac{2pₙ + qₙ}{3}\text,\quad b ≔ \frac{pₙ + 2qₙ}{3}\text{, in}\quad
     \p{pₙ₊₁, qₙ₊₁} ≔ \begin{cases}
       \p{a, qₙ} ;& a ∈ L\\
       \p{pₙ, b} ;& b ∈ U\text.
     \end{cases}
  \]
  Te intervali hitro konvergirajo proti \(\p{L, U}\), torej je to želeno
  Cauchyjevo število.
\end{izrek}

Gornji izrek naredi števno mnogo odločitev, ko se odločamo, če velja \(a ∈ L\)
ali \(b ∈ U\) (oziroma ali velja \(a < x\) ali \(x < b\)), torej dokaz ni
konstruktiven.
Znana sta dva nekonstruktivna principa, ki sta zadostna za gornji odkaz in sta
šibkejša od izključene tretje možnosti. To sta \(\alpo*\) ter \(\CCv\).
Če velja \(\alpo*\) je potem \(a < x\) odločljivo, torej lahko vnaprej popravimo
drugi primer na \(x < b ∧ ¬\p{a < x}\).

Če pa imamo na voljo \(\CCv\) pa preprosto lahko naredimo števno mnogo
odločitev. Zares potrebujemo tu zgolj \(\CCv_{Σ_ℝ}\), kar je
pa tudi posledica \(\alpo*\).

Čeprav je iz tega očitno, da niti \(\alpo*\) niti \(\CCv\) nista potrebna za
\(\Rd = \Rc\), jih vseeno želimo strogo ločiti.

\begin{konstrukcija}
  Nad \(\Ncof\) velja \(\Rd = \Rc\), a ne velja princip števne
  disjunktivne izbire.
\end{konstrukcija}
\begin{dokaz}
  Prostor naravnih števil s kokončno topologijo ima lastnost, da je vsaka
  funkcija \(ℕ → ℝ\) konstantna. To velja tudi za (neprazne) odprte podmnožice,
  ker so števno neskončne s kokončno topologijo, kar pa pomeni, da se, v toposu
  snopov nad \(ℕ\) s to topologijo, Dedekindova in Cauchyjeva realna števila
  ujemajo.

  Naj bo \(R(n, b) ≔ ℕ⧵\{2n+b\}\). Ta relacija je celovita, saj lahko za vsak
  \(n : ℕ\) \(ℕ\) pokrijemo z \(R(n,0)\) in \(R(n,1)\). Pokažimo, da za to
  relacijo ne obstaja funkcija izbire.

  Denimo, da je \(f : ℕ → 2\) njena funkcija izbire, torej da velja
  \(\for{n:ℕ}{R(n,f(n))}\). Če bi bila \(f\) konstantno \(b\), bi potem moralo
  veljati \(ℕ ⊩ R(n, b)\) za vse \(n\), kar pa ni res. To pomeni, da sta
  \(\i{0 = f(n)}\) in \(\i{1 = f(n)}\) obe neprazni, torej, ker sta odprti,
  imata neprazen presek (ki je tudi neskončna odprta množica). Potem pa na tej
  množici velja \(0 = 1\), kar pa očitno ne drži.\contradiction

  Sledi, da funkcija izbire za ta \(R\) ne more obstajati, torej \(\CCv\) ne
  drži.
\end{dokaz}
\begin{dokaz}
  Naj bodo \(Cₙ ≔ \{ℕ⧵\{2n\}, ℕ⧵\{2n+1\}\}\) pokritja \(\Ncof\) in \(C\) njihova
  skupna pofinitev. Potem mora vsak \(U ∈ C\) biti podmnožica enega od elementov
  vsakega od \(Cₙ\). To pa pomeni, da ima \(U\) neskončen komplement, torej je
  prazna množica. Sledi, da \(C\) pokrije zgolj prazno množico, torej \(\CCv\)
  ne drži.
\end{dokaz}

Primer podan zgoraj pa vseeno zadošča principu \(\alpo*\), ki konstruktivno
implicira ujemanje Cantorjevih in Dedekindovih realnih
števil~\cite{Birchfield24}.

TODO: restructure?
Izkaže se, da za lokalno povezane prostore to tudi pričakujemo.
\begin{izrek}
  Če je \(X\) lokalno povezan in velja \(X ⊩ \Rd = \Rc\), velja \(X ⊩ \alpo*\).
\end{izrek}
Ta izrek je zares kar posledica izreka~\ref{th:lpov-lpo}, saj je \(\lpo*\)
natanko \(\alpo*\) za Cauchyjeva realna števila.
\begin{opomba}
  Zares namesto lokalne povezanosti zadošča \(X ⊩ \Rc = \c ℝ\). To je zato, ker
  je \(\alpo*\) za Cauchyjeva realna števila ekvivalenten \(\lpo*\).
  TODO: a velja \(\lpo* ⇒ \Rc = \c ℝ\)?

  V splošnem bi se izrek torej lahko glasil \(X ⊩ R = \c ℝ ⇒ \alpo*_R\), kjer je
  \(R\) nek objekt realnih števil (Dedekindova, Cauchyjeva, Escardo-Simpsonova,
  MacNeillova, itd.).
\end{opomba}

TODO: ta prostor ne dela, oziroma loči ta dva, ampak \(\CC\) vela
To pomeni, da če želimo ločiti \(\Rd = \Rc\) in \(\alpo*\) potrebujemo prostor,
ki ni lokalno povezan. Avtorica meni, da je Fortov prostor na števno neskončni
množici dober kandidat, a ji ni uspelo preveriti detajlov. Vseeno, se je pa
zadostno prepričala, da niti \(\alpo*\) niti \(\CCv\) nad tem prostorom ne
držita. Še več, ker je ta prostor \(T₆\) validira \(\aks*\), torej ne velja niti
\(\CCv\) za ``realne'' resničnostne vrednosti. 


\subsection{\(\Rm{} = \Rd\)}\label{sec:reals-Rm=Rd}

% https://gist.github.com/andrejbauer/689b17b10a4e80ea409d03ec030c98b3
Andrej Bauer je 2023 za prvoaprilsko šalo objavil, kar zgleda kot konstruktiven
dokaz \(\wlem*\). V njem začne z ``znanimi dejstvi'' o MacNeilleovih realnih
številih, zraven pa podtakne še lociranost. Vemo že, da je vsako MacNeillovo
realno število locirano natanko tedaj, ko se ujemajo z Dedekindovimi realnimi
števili.
Skratka, pokazal je sledečo trditev.
\begin{trditev}
  Če velja \(\Rm = \Rd\) velja \(\wlem*\).
\end{trditev}
\begin{dokaz}
  Dokaz je povsem konstruktiven, ampak ga lahko za topološke modele malo
  poenostavimo.

  Naj bo \(U ∈ 𝒪X\). Definirajmo preslikavi \(\uline f\) in \(\bar f\) tako, da
  sta na \(U\) obe \(1\), na \(¬U\) obe \(0\), izven tega naj je pa \(\uline f\)
  enaka \(0\), \(\bar f\) pa enaka \(1\).

  Na \(¬U\) torej velja \(x ≤ 0\), na \(U\) pa \(x ≥ 1\).

  To zadošča pogojema~\ref{real:Rm-maps}, torej definira MacNeillovo realno
  število.
  Sedaj pa uporabimo predpostavko, da je locirano, torej \(x < 1 ∨ x > 0\).
  Vrednost \(x < 1\) je \(¬U\), saj je \(\bar f < 1\) natanko na tej množici,
  vrednost \(x > 0\) je pa vsebovana v \(¬¬U\), saj na \(¬U\) velja \(x ≤ 0\),
  torej \(\i{x>0}\) ne more sekati \(¬U\). 
\end{dokaz}

Izkaže pa se, da velja tudi obrat! Inspiracija za to dejstvo, je prišla iz
predmeta ``Banachove mreže'', kjer se obravnava slednji izrek:
\begin{izrek}
  Če je \(X\) ekstremalno nepovezan je \(𝒞(X,ℝ)\) polna mreža.
\end{izrek}

To pa izgleda zelo sumljivo! Če se spomnimo, to da je \(X\) ekstremalno povezan
ravno pomeni, da velja \(\wlem*\). Prav tako je \(𝒞(X,ℝ)\) znotraj množica
globalnih Dedekindovih realnih števil. Konstruktivno ta niso polna, so pa
MacNeillova realna števila, tako da bo supremum Dedekindovih realnih v
MacNeillovih realnih številih obstajal, le da bo to MacNeillovo realno število.
Potem pa lahko polnost Dedekindovih realnih števil izrazimo kot ``vsako
MacNeillovo realno število je Dedekindovo''.

Vredno je še omeniti, da je ekstremalna nepovezanost dedna lastnost na odprte
podmnožice, tako da lahko konsekvent napišemo tudi kot ``\(𝒞(U,ℝ)\) je polna
mreža za vse \(U\)''. To je pa že bistveno bližje temu, da bi rekli ``realna
števila so polna'' v interni logiki. Ampak, preveriti moramo še, da je polnost
znotraj enaka stvar kot polnost zunaj. 

Izkaže se, da nista zares ista stvar, polnost znotraj pravi da ima vsaka
poseljena \emph{odsekoma} omejena množica supremum, medtem ko polnost zunaj
zahteva globalno omejenost. Ampak to ni problem, saj je vsaka omejena množica
očitno tudi odsekoma omejena, torej je notranja trditev močnejša.

TODO: a dobim tapravi obstoj tu?

Sledi, da lahko gornji izrek sledi iz interpretacije spodnje trditve v interni
logiki.
\begin{trditev}
  Velja implikacija \(\wlem* ⇒ \Rm = \Rd\).
\end{trditev}
\begin{dokaz}
  Za \(x : \Rm\) velja \(¬(x < q) ⇒ \for{s<q}{s < x}\).

  Uporabimo \(\wlem*\) na \(x < 1\) in \(2 < x\).
  Ker hkrati oba očitno ne moreta veljati, lahko primer, ko sta oba \(¬¬\)
  veljavna zanemarimo.
  \begin{itemize}
  \item Če velja \(¬(x < 1)\) potem gotovo velja \(0 < x\).
  \item Če velja \(¬(2 < x)\) potem gotovo velja \(x < 3\).
  \end{itemize}
  V vseh primerih torej dobimo \(0 < x ∨ x < 3\), torej je \(x\) lociran.
\end{dokaz}
\begin{dokaz}[Topološki dokaz]
  TODO: a v topoloških rečemo, da če je \(X\) ekstremalno nepovezan morta bit
  \(\uline f\) in \(\bar f\) zvezni, in a to pomen da sta enaki?

  NOTE: Najdla sm da pomen da je vmes med njima en zvezen… nevem zakaj to pomeni
  da sta enaka… Aaha, sem ugotovila, ker sta ostra dobimo tud da sta enaka.
\end{dokaz}

Gornji dokaz sem tudi formalizirala v dokazovalnem pomočniku Agda, kar nam da
formaliziran, popolnoma konstruktiven dokaz sledečega izreka.
\begin{izrek}\label{th:Rm=Rd-wlem}
  MacNeillova realna števila se ujemajo z Dedekindovimi natanko tedaj, ko velja
  šibka izključena tretja možnost.
\end{izrek}
TODO: objavi kodo in jo citiraj tu.

TODO: elephant D4.7.11 je to


%%% Local Variables:
%%% mode: latex
%%% TeX-master: "main"
%%% End:



%%% Local Variables:
%%% mode: latex
%%% TeX-master: "main"
%%% End:

