\section{Uvod}

Cilj našega dela je najti povezavo med topologijo in logiko. Specifično želimo
vsakemu prostoru na nek način dodeliti logični sistem, katerega lastnosti bodo
potem odražale (upamo da znane) lastnosti topološkega prostora.
Prva opazka, ki jo lahko naredimo je, da lahko gledamo ``logiko odprtih množic'',
kjer smatramo odprte množice kot trditve oziroma resničnostne vrednosti.
O tem si mislimo, kot da se resnica spreminja po prostoru. Na primer poglejmo si
trditev ``funkcija \(\id\) je pozitivna''. To seveda ni res, je pa res če jo
zožimo na pozitivna realna števila. Tako lahko tej trditvi priredimo
resničnostno vrednost \(\p{0, ∞}\). Tako torej presek igra vlogo konjunkcije,
unija disjunkcije, izrazimo lahko pa tudi implikacijo (čeprav jo v topologiji
redko vidimo).

Tako smo prostoru priredili \emph{propozicijsko logiko}. Že s tem lahko izrazimo
veliko, recimo karakteriziramo lahko particijske topologije, ekstremalno
% TODO: prevodi za `extremally' in `basically' disconnected
nepovezane prostore, in še kakšne.

Ampak izkaže se, da lahko našo logiko obogatimo še do logike višjega reda,
kar nam bo dalo več možnosti za karakterizacije topoloških lastnosti.
Recimo, da bi želeli dodati naravna števila v logiko prostora evklidskih realnih
števil. Tako kot zgoraj, bi želeli, da se lahko ``vrednost'' naravnega števila
spreminja po prostoru. To je pa natanko (zvezna) funkcija \(ℝ → ℕ\). Zveznost
zahtevamo, da se ta naravna števila dovolj lepo obnašajo, konec koncev pa to
pomeni, da je ta funkcija konstantna. Zanimivo potem postane, da naravna števila
niso nujno definirana povsod: res, obstajajo funkcije v naravna števila tudi iz
odprtih podmnožic \(ℝ\), ta pa niso nujno povezana, tako da tudi te funkcije
niso nujno konstantne.

Taki logiki bomo potem pravili \emph{topološki model}. Zares smo do tega prišli
v napačnem vrstnem redu, saj imamo ponavadi najprej neko logiko, in potem
najdemo matematični objekt, ki se mu reče \emph{model}, ki tej logiki zadošča.
Tako je bolj korektno, da rečemo, da je cilj naloge pokazati, da v logiki velja
določen neklasičen princip natanko tedaj, ko za topološke modele te logike velja
neka topološka lastnost. Tako dobimo recimo izrek, da izključena tretja možnost
velja natanko v diskretnih prostorih.



% V konstruktivni matematiki poznamo več tako imenovanih \quot{nekonstruktivnih}
% oziroma \quot{klasičnih} principov, najbolj znana sta recimo princip izključene
% tretje možnosti in aksiom izbire, obravnavamo pa lahko tudi šibkejše verzije
% obeh. V dvajsetem stoletju se je razvilo več različnih programov konstruktivne
% misli, na primer Brouwerjev intuicionizem, Bishopov konstruktivizem, Ruski
% konstruktivizem, in drugi. Ruski konstruktivisti so recimo priznavali princip
% Markova, ki pravi, da če zaporedje ničel in enk ni konstantno nič, mora potem
% nujno biti na nekem indeksu enka, pa čeprav ta princip ne velja
% \quot{konstruktivno}.

% Ruskemu konstruktivizmu (in ostalim programom) bi torej lahko pravili
% interpretacija ali celo \emph{implementacija} konstruktivne logike. Temu se bolj
% tehnično reče \emph{model}, se pa tudi uporablja fraza \emph{matematični svet},
% saj je model logike natanko svet, v katerem lahko delamo matematiko z uporabo te
% logike.

% Matematične svetove pa lahko pogrupiramo v neke večje kategorije, ki pa bodo
% matematikom malo bolj znani. Nekateri so motivirani s teorijo množic, drugi z
% izračunljivostjo, torej na primer s Turingovimi stroji (sem recimo spada Ruski
% konstruktivizem), tretji z algebro, četrti pa s topologijo. Ker raziskujemo
% povezavo med logiko in topologijo se v tem delu osredotočimo na topološke
% modele. Natančno to pomeni, da za vsak topološki prostor pogledamo model logike,
% ki pripada temu prostoru. Ideja sedaj je, da \quot{klasične} principe izrazimo v
% jeziku topološkega modela, to interpretiramo kot izjavo o topološkem prosoru
% samem, in tako dobimo slovar med \quot{klasičnimi} principi in topološkimi
% lastnostmi, kot na primer kompaktnost, separabilnost, \(Tₙ\) lastnostmi, in
% podobno.

Močno se zgledujemo po delu Inga Blechschmidta~\cite{Blechschmidt17}, ki zgradi
slovar med algebraično geometrijo in komutativno algebro v (Zariskijevem)
modelu.

To delo predpostavlja znanje teorije kategorij, se pa bo avtorica potrudila
navesti točne citate na vse naprednejše kategorične definicije in izreke. Znanje
teorije modelov ali teorije toposov ni nujno, je pa za globje razumevanje
celostne slike verjetno potrebno. Zaželjeno je osnovno poznavanje konstruktivne
matematike, za kar priporočamo~\cite{Bauer16, Greenleaf20, Bishop85}.

\subsection{Notacija}

Skozi delo uporabljamo več nestandardnih oznak, ki so navedene in razložene v
spodnji tabeli.

% TODO: alignment
\begin{tabularx}{0.9\linewidth}{lX}
  % \(よ\) & vložitev Yonede\\
  \(f{\res U}\) & zožitev \(f\) na \(U\)\\
  \(\Rd\) & realna števila konstruirana z Dedekindovimi rezi\\
  \(\Rc\) & realna števila konstruirana kot kvocijen Cauchyjevih zaporedij\\
  %\(\lem\) & princip izključene tretje možnosti, \\
  % \(\cov{U}{i}\) & pokritje \(\{Uᵢ\}_{i∈I}\) monžice \(U\), kjer je
  %                  \(I\) neka indeksna množica\\
  \(\eventually{V ⊆ U}{P(V)}\) & obstaja pokritje množice \(U\), in za vsak
                                 element pokritja \(V\) velja \(P(V)\)\\
  \(\lightning\) & protislovje\\
  \(𝒪X\) & odprte množice prostora \(X\)\\
  \(𝒜X\) & zaprte množice prostora \(X\)\\
  \(\) & \\
  \(\) & \\
  \(\) &
\end{tabularx}


%%% Local Variables:
%%% mode: latex
%%% TeX-master: "main"
%%% End:
