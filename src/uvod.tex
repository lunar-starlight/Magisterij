\section{Uvod}

V konstruktivni matematiki poznamo več tako imenovanih \quot{nekonstruktivnih}
oziroma \quot{klasičnih} principov, najbolj znan recimo princip izključene
tretje možnosti in aksiom izbire, obravnavamo pa lahko tudi šibkejše verzije
obeh. Ruski konstruktivisti so recimo priznavali princip Markova, ki pravi, da
če zaporedje ničel in enk ni konstantno nič, mora potem nujno biti na nekem
indeksu enka, pa čeprav ta princip ne velja \quot{konstruktivno}.

Ruskemu konstruktivizmu bi torej lahko rekli kot ena interpretacija ali celo
\emph{implementacija} konstruktivne logike. Temu se bolj tehnično reče
\emph{model}, se pa tudi uporablja fraza \emph{matematični svet}, saj je model
logike natanko svet, v katerem lahko delamo matematika z uporabo te logike.

Matematične svetove pa lahko pogrupiramo v neke večje kategorije, ki pa bodo
matematikom malo bolj znani. Nekateri so motivirani s teorijo množic, drugi z
izračunljivostjo, torej na primer s Turingovimi stroji (sem recimo spada Ruski
konstruktivizem), tretji z algebro, četrti pa s topologijo. V tem delu se
osredotočimo na topološke modele. Natančno to pomeni, da za vsak topološki
pogledamo model logike, ki pripada temu prostoru. Ideja sedaj je, da
\quot{klasične} principe izrazimo v jeziku topološkega modela, to interpretiramo
kot izjavo o topološkem prosoru samem, in tako dobimo slovar med
\quot{klasičnimi} principi in topološkimi lastnostmi, kot na primer kompaktnost,
separabilnost, \(Tₙ\) lastnostmi, in podobno.

% TODO: check citations and only include the relevant ones, probably just the
% thesis and follow up paper
Močno se zgledujemo po delu Inga Blechschmidta~\cite{Blechschmidt17,
  Blechschmidt18, Blechschmidt20, Blechschmidt22}, ki zgradi slovar med
algebraično geometrijo in komutativno algebro v (Zariskijevem) modelu.

To delo predpostavlja znanje teorije kategorij, se pa bo avtorica potrudila
navesti točne citate na vse naprednejše kategorične definicije in izreke. Znanje
teorije modelov ali teorije toposov ni nujno, je pa za globje razumevanje
celostne slike verjetno potrebno. Zaželjena je osnovno poznavanje konstruktivne
matematike, za kar priporočamo~\cite{Bauer16, Greenleaf20, Bishop85}.

\subsection{Notacija}

Skozi delo uporabljamo več nestandardnih oznak, ki so navedene in razložene v
spodnji tabeli.

\begin{tabularx}{0.9\linewidth}{cc}
  \(よ\) & vložitev Yonede\\
  \(f{\res U}\) & zožitev \(f\) na \(U\)\\
  \(\Rd\) & realna števila konstruirana z Dedekindovimi rezi\\
  \(\Rc\) & realna števila konstruirana kot kvocijen Cauchyjevih zaporedij\\
  %\(\lem\) & princip izključene tretje možnosti, \\
  \(\cov{U}{i}\) & pokritje \(\{Uᵢ\}_{i∈I}\) monžice \(U\), kjer je
                   \(I\) neka indeksna množica\\
  \(\lightning\) & protislovje\\
  \(𝒪X\) & odprte množice prostora \(X\)\\
  \(𝒜X\) & zaprte množice prostora \(X\)\\
  \(\) & \\
  \(\) & \\
  \(\) &
\end{tabularx}


%%% Local Variables:
%%% mode: latex
%%% TeX-master: "main"
%%% End:
