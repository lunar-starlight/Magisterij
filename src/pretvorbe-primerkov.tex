\section{Interpretacija pretvorbe primerkov v topoloških modelih}

\subsection{\texorpdfstring{\(\lem* ≤ \alpo*\)}{LEM ≤ ALPO} in idempotenca \texopdfstring{\(\alpo*\)}{ALPO} nad lokalno \(T₆\) prostori}

\begin{trditev}
  Nad vsakim okolišem \(X\) je \(\lem\) idempotenten.
\end{trditev}
\begin{dokaz}
  Naj bosta \(U₀,U₁ ⊆ X\). Potem definiramo \(V ≔ (U₀∪¬U₀)∩(U₁∪¬U₁)\),
  za katerega očitno velja \(X ⊩ \lem\p{V} ⇒ \lem²\p{U₀, U₁}\).
\end{dokaz}

\begin{opomba}
  Ta trditev velja že konstruktivno, ampak je dokaz zgoraj elegantnejši.
\end{opomba}

\begin{izrek}
  Naj bo \(X\) lokalno \(T₆\). Potem nad \(X\) velja redukcija \(\lem* ≤ \alpo*\).
\end{izrek}
\begin{proof}
  Denimo, da je \(X\) lokalno \(T₆\).
  To pomeni, da imamo pokritje \(X\) s \(T₆\) prostori \(Xᵢ\).
  Označimo funkcijo, ki jo dobimo iz pogoja \(T₆\) za zaprto množico \(A ⊆ Xᵢ\)
  z \(N_A : Xᵢ → ℝ\). Zanjo velja, da je \(N_A(x) = 0\) natanko tedaj, ko je
  \(x ∈ A\), brez škode za splošnost pa lahko tudi predpostavimo, da je
  nenegativna.
  Če želimo pretvoriti \(\lem*\) na \(\alpo*\), mora za vse
  \(U ⊆ X\) veljati \(\eventually{x : ℝ}{⟦x ≤ 0⟧ ∪ ⟦x > 0⟧ ⊆ U ∪ ¬U}\).
  Ker želimo uporabiti predpostavko, moramo podati take zaprte množice, da velja
  \[ \eventually{A : 𝒜}{⟦N_A ≤ 0⟧ ∪ ⟦N_A > 0⟧ ⊆ U ∪ ¬U}\text. \label{eq:alpo_le_lem}\]

  Za pokritje vzamemo kar množice \(Xᵢ\), za zaprte množice pa \(A ≔ \c{\p{U ∪ ¬U}}\).
  Za take množice velja, da je \(⟦N_{A} ≤ 0⟧ = \int Aᵢ\) in \(⟦N_{A} > 0⟧ = U ∪ ¬U\).
  Ker je množica \(U ∪ ¬U\) gosta, je \(\int A = ∅\), tako da~\ref{eq:alpo_le_lem} očitno velja.
\end{proof}
\begin{posledica}
  Če je prostor \(X\) lokalno \(T₆\), velja \(\alpo*×\alpo* ≤ \alpo*\).
\end{posledica}
\begin{posledica}
  Nad lokalno \(T₆\) prostorom je \(\alpo*\) idempotenten.
\end{posledica}


% \begin{izrek}
%   %Let \(X\) be a topological space. Then \(X\) is locally \(T₆\) if and only if the
%   %reduction \(\lem ≤_I \lpo\) holds.
%   Naj bo \(X\) topološki prostor. Tedaj je \(X\) lokalno \(T₆\) natanko tedaj,
%   ko nad \(X\) velja funkcijska redukcija \(\lem ≤_{FI} \lpo\).
% \end{izrek}
% \begin{proof}
%   \begin{enumerate}
%   \item[\((⇒)\)]
%     Denimo, da je \(X\) lokalno \(T₆\).
%     Potem imamo pokritje s \(T₆\) prostori, t.j. za vsak \(Xᵢ\) iz pokritja in
%     vsako zaprto množico \(A ⊆ Xᵢ\), imamo zvezno funkcijo \(f : Xᵢ → ℝ\), tako
%     da je \(f_A(x) = 0\) natanko tedaj, ko je \(x ∈ A\). Poleg tega lahko
%     zahtevamo, da je ta funkcija nenegativna.

%     %Potem imamo, za vsako zaprto množico \(A ⊆ X\), zvezno funkcijo
%     %\(f_A : X → ℝ\), tako da je \(f_A(x) = 0\) natanko tedaj, ko je \(x ∈ A\).
%     %Poleg tega lahko zahtevamo, da je ta funkcija nenegativna.

%     Po aksiomu o izbiri dobimo preslikavo \(Cl(Xᵢ) → \p{Xᵢ → ℝ}\).
%     Če jo preuredimo in uporabimo vložitev \(Ω(Xᵢ) \hookrightarrow Cl(Xᵢ)\),
%     ki \(p\) slika v \(\c{p ∪ ¬p}\), dobimo preslikavo \(Fᵢ: Xᵢ → \p{Ω(Xᵢ) → ℝ}\).

%     Dokažimo sedaj, da velja redukcija.
%     Vzamimo množice \(Uᵢ = Xᵢ\) in \(fᵢ = Fᵢ\).

%     Naj bo sedaj \(U ⊆ Xᵢ\), \(p ∈ Ω(U)\), \(A ≔ \c{p ∪ ¬p}\), \(x ≔ fᵢ(p)\) in predpostavimo,
%     da velja \(Xᵢ ⊩ \lpo{\p{x}}\)

%     % To pomeni, da obstaja pokritje \(\b{Uᵢ} ∈ \cov{U}\), tako da za vsak element
%     To pomeni, da obstaja pokritje \(\cov{U}i\), tako da za vsak element
%     pokritja velja \(x\res{Uᵢ} ≤ 0\) ali \(x\res{Uᵢ} > 0\).

%     Ker je \(x\) nenegativen, je \(x ≤ 0\) natanko tedaj, ko je \(x = 0\), torej
%     po definiciji zgolj na množici \(A\).
%     Od tod sledi, da je \(x\res{Uᵢ} ≤ 0\) zgolj v primeru, ko je \(Uᵢ ⊆ A\).
%     Vendar pa ima \(A\) prazno notranjost, tako da je tak \(Uᵢ\) nujno prazen.

%     Ostane le še drugi primer, torej \(x\res{Uᵢ} > 0\).
%     Ker množice \(Uᵢ\) tvorijo pokritje \(U\), je potem tudi \(x > 0\) na
%     celotnem \(U\). Sledi, da je množica \(A∩U\) prazna,
%     in velja \(U ⊩ \lem{\p{p}}\).
%   \item[\((⇐)\)]
%     V obratno smer pa predpostavimo, da obstaja redukcja iz \(\lem\) na \(\lpo\).

%     To pomeni, da obstaja pokritje \(Xᵢ \circlearrowright X\)
%     in preslikave \(Fᵢ: Xᵢ → \p{Ω(Xᵢ) → ℝ}\), za katere velja
%     \[Xᵢ ⊩ \for{p∈Ω}{\lpo{\p{Fᵢ(p)}} ⇒ \lem{\p{p}}}\text.\label{eq:lemlpo}\]

%     Trditev bomo dokazali v dveh korakih. Najprej, naj bo \(A ⊆ Xᵢ\) zaprta
%     množica s prazno notranjostjo.

%     Tedaj na \(U ≔ Xᵢ\) in \(p = Aᶜ\) uporabimo \ref{eq:lemlpo}.
%     To pomeni, da dobimo implikacijo \(U⊩\lpo{\p{Fᵢ(p)}} ⇒ U⊩\lem{\p{p}}\).

%     Naj bo \(f≔Fᵢ(p)\). Potem, če obstaja pokritje \(Uᵢ \circlearrowright U\),
%     tako da za vsak \(i\) velja \(f\res{Uᵢ} ≤ 0\) ali \(f\res{Uᵢ} > 0\),
%     potem velja tudi \(U ⊆ p ∪ ¬p = p\), oziroma \(A ⊆ Uᶜ\).

%     Če tako pokritje obstaja, ga lahko združimo v dve množici, eno, na kateri je
%     \(f\) nepozitiven (torej ker je nenegativen kar \(0\)), in eno, na kateri je pozitiven.
%     Naj bosta torej \(U₀ in Uₚ\) ti dve (odprti) množici.

%     Vednar je pa \(U₀ = f⁻¹{0}\), torej zaprta.

%   \end{enumerate}
% \end{proof}

\subsection{Idempotenca \texorpdfstring{\(\lpo*\)}{LPO} nad Cantorjevim prostorm}

\begin{trditev}
  Nad vsakim okolišem \(X\) velja \(\lpo* ≤ \alpo*\).
\end{trditev}
\begin{dokaz}
  Naj bo \(U ⊆ X\), \(α : U → C\). Definirajmo \(x ≔ φ∘α\), kjer
  \(φ : C → ℝ\) slika
  \[ ∑_{i ∈ ℕ₊}\frac{2aᵢ}{3ⁱ} ∈ C\text{, } aᵢ ∈ 2
     \text{, v } ∑_{i ∈ ℕ₊}\frac{aᵢ}{2ⁱ}\text.\]

  Potem je \(φ(r) = 0 ⇔ r = 0\) in velja \(\lpo{\p{α}} ⇔ \alpo{\p{x}}\).
\end{dokaz}

% TODO: Ask Davorin if this is true. Alternatively, find structural lifting
% property Andrej wanted
\begin{lema}
  Vsaka preslikava \(f : C → I\) inducira preslikavo \(\floor f : C → C\),
  s predpisom \(\floor f(a) ≔ \sup\set{x ∈ C}{x < f(a)}\).
\end{lema}
\begin{dokaz}
  % dokaži da maksimum obstaja (da je zaprta).
  Ker je \(C\) kompakten, je predpis dobro definiran, torej moramo preveriti
  zgolj zveznost. Naj bo \(U=w\) bazična odprta množica.

  %Naj bodo \(l ≔ (w-1)\hat{1}\), \(r ≔ (w+1)\hat{0}\), in \(U' ≔ \p{l, r}\).
  Naj bodo \(l ≔ w\hat{0}\), \(r ≔ (w+1)\hat{0}\), in \(U' ≔ \p{l, r}\).
  Potem je \(U ⊆ \cli{w\hat{0}, w\hat{1}} ⊆ U' ∪ \{l\}\).
  Upamo sedaj, da je \(\floor f⁻¹(U) = f⁻¹(U')\).

  Naj bo \(a ∈ f⁻¹(U')\), torej \(l < f(a) < r\).
  Iz \(f(a) < r\) sledi \(\floor f(a) < r\), torej \(\floor f(a) ≤ w\hat{1}\),
  iz \(l < f(a)\) pa sledi, da je \(l ≤ \floor f(a)\).

  Če pa je \(a ∈ \floor f⁻¹(U)\), imamo \(\sup\set{x ∈ C}{x < f(a)} ∈ U\).
  Iz \(w\hat{0} ≤ α(a)\) sledi \(w\hat{0} ≤ f(a)\),
  iz \(α(a) ≤ w\hat{1}\) pa sledi, da je \(f(a) ≤ w\hat{1}\).

  % Če pa je \(a ∈ f⁻¹(U')\), pa je \(l < f(a) < r\).
  % Iz \(f(a) < r\) sledi \(\floor f(a) < r\), torej \(\floor f(a) ≤ w\hat{1}\),
  % iz \(l < f(a)\) pa sledi, da je \(l ≤ \floor f(a)\).

  % l here is wrong, i should fix U' (maybe i can)
  % Iz tega dvojega lahko sedaj sklepamo, da je \(\floor f⁻¹(U) = f⁻¹(U') ⧵ f⁻¹[0,l]\),
  Iz tega dvojega lahko sedaj sklepamo, da je
  \(f⁻¹(U') ⊆ \floor f⁻¹(U) ⊆ f⁻¹(U') ∪ f⁻¹\{w\hat{0}\}\).
  Denimo, da je \(\floor f(a) = w\hat0\).
  Potem obstaja zaporedje \(xᵢ ↗ w\hat0\), za katero velja \(xᵢ < f(a)\)
  Vendar je pa množica \(\left[ w\hat0, 1 \right]\) odprta, torej mora vsebovati
  skoraj vsa števila \(xᵢ\). Ker velja \(xᵢ ≤ w\hat0\), je torej
  \(xᵢ = w\hat{0}\) skoraj povsod, in velja \(f(a) > w\hat0\).

  Sledi, da je \(\floor f⁻¹(U) ⊆ f⁻¹(U')\), torej sta ti dve množici enaki,
  in je funkcija \(\floor f\) zvezna.
\end{dokaz}
\begin{posledica}
  V zgornji lemi lahko domeno zamenjamo s poljubno odprto množico.
\end{posledica}

\begin{lema}
  Nad Cantorjevim prostorom je \(\lpo*\) ekvivalenten \(\alpo*\).
\end{lema}
\begin{dokaz}
  Vemo, da velja \(\lpo* ≤ \alpo*\).

  Pokažimo sedaj, da velja tudi obrat.
  Naj bo \(U ⊆ C\), \(f : U → ℝ\).
  Brez škode za splošnost, lahko \(f\) popravimo do preslikave \(U → I\).
  Vzamemo trivialno pokritje in naj bo \(α ≔ \floor f\).

  Potem je očitno \(α ≤ f\), torej \(f(x) = 0 ⇒ α(x) = 0\).
  Obratno pa, če je \(α(x) = 0\) pomeni, da imamo za vsak \(a ≠ 0\)
  neenakost \(f(x) < a\), torej \(f(x) = 0\).

  Sledi torej, da je \(\lpo\p{α} = \alpo\p{f}\).
\end{dokaz}
\begin{posledica}
  Nad Cantorjevim prostorom je \(\lpo*\) idempotenten.
\end{posledica}




%%% Local Variables:
%%% mode: latex
%%% TeX-master: "main"
%%% End:
