\section{Interpretacija pretvorbe primerkov v topoloških modelih}

\subsection{\(\lem ≤ \alpo\) in idempotenca \(\alpo\) nad lokalno \(T₆\) prostori}

\begin{trditev}
  Nad vsakim okolišem \(X\) je \(\lem\) idempotenten.
\end{trditev}
\begin{dokaz}
  Naj bosta \(U₀,U₁ ⊆ X\). Potem definiramo \(V ≔ (U₀∪¬U₀)∩(U₁∪¬U₁)\),
  za katerega očitno velja \(X ⊩ \lem\p{V} ⇒ \lem²\p{U₀, U₁}\).
\end{dokaz}

\begin{opomba}
  Ta trditev velja že konstruktivno, ampak je dokaz zgoraj elegantnejši.
\end{opomba}

\begin{izrek}
  Naj bo \(X\) lokalno \(T₆\). Potem nad \(X\) velja redukcija \(\lem ≤_I \lpoR\).
\end{izrek}
\begin{proof}
  Denimo, da je \(X\) lokalno \(T₆\).
  Potem imamo pokritje s \(T₆\) prostori, t.j. za vsak \(Xᵢ\) iz pokritja in
  vsako zaprto množico \(A ⊆ Xᵢ\), imamo zvezno funkcijo \(f : Xᵢ → ℝ\), tako
  da je \(f_A(x) = 0\) natanko tedaj, ko je \(x ∈ A\). Poleg tega lahko
  zahtevamo, da je ta funkcija nenegativna.

  Dokažimo sedaj, da velja redukcija.
  Naj bo \(U' ⊆ X\) poljubna odprta množica in \(p ∈ Ω(U')\).
  Dokazati želimo, da obstaja tako pokritje \(U'\), da na njem lokalno
  obstajajo realna števila, za katera \(\lpoR\) pomeni, da (lokalno) velja
  \(p ∨ ¬p\).

  Naj bo \(Uᵢ' ≔ U'∩Xᵢ\) pokritje \(U'\) in \(Aᵢ ≔ \c{p ∨ ¬p} ∩ Uᵢ\),
  in \(xᵢ : Uᵢ' → ℝ\) \(T₆\) separacijska funckija \(Aᵢ\) na \(Uᵢ'\).

  Predpostavimo sedaj, da je \(Uᵢ ⊆ Uᵢ'\) in da velja \(Uᵢ ⊩ \lpoR{\p{xᵢ}}\).
  % To pomeni, da obstaja pokritje \(\b{Uᵢⱼ} ∈ \cov{Uᵢ}\), tako da za vsak element
  To pomeni, da obstaja pokritje \(\cov[i]{U}{j}\), tako da za vsak element
  pokritja velja \(xᵢ\res{Uᵢⱼ} ≤ 0\) ali \(xᵢ\res{Uᵢⱼ} > 0\).

  Ker je \(xᵢ\) nenegativna, je \(xᵢ ≤ 0\) natanko tedaj, ko je \(xᵢ = 0\), torej
  po definiciji zgolj na množici \(Aᵢ\).
  Od tod sledi, da je \(xᵢ\res{Uᵢⱼ} ≤ 0\) zgolj v primeru, ko je \(Uᵢⱼ ⊆ Aᵢ\).
  Vendar pa ima \(Aᵢ\) prazno notranjost, tako da je tak \(Uᵢⱼ\) nujno prazen.

  Ostane le še drugi primer, torej \(xᵢ\res{Uᵢⱼ} > 0\).
  Ker množice \(Uᵢⱼ\) tvorijo pokritje \(Uᵢ\), je potem tudi \(xᵢ > 0\) na
  celotnem \(Uᵢ\). Sledi, da je množica \(Aᵢ∩Uᵢ\) prazna,
  in velja \(Uᵢ ⊩ \lem{\p{p}}\).
\end{proof}

\begin{posledica}
  Če je prostor \(X\) lokalno \(T₆\), velja \(\lpoR×\lpoR ≤_I \lpoR\).
\end{posledica}

% \begin{dokaz}
%   Naj bo \(\cov X i\) \(T₆\) pokritje, \(U ⊆ X\), \(x,y : U → ℝ\)
%   in \(Uᵢ = U∩Xᵢ \covsymb U\).

%   Označimo \(A ≔ {\lpoR\p{x}}ᶜ\) in \(B ≔ {\lpoR\p{y}}ᶜ\) (pod \(Uᵢ\)).
%   Ker je \(Uᵢ\) \(T₆\) obstaja funkcija \(fᵢ : Uᵢ → ℝ\), ki je \(0\) na
%   \(A ∪ B\) in pozitivna sicer.

%   Naj bo \(Vᵢ ⊆ Uᵢ\) poljuben in predpostavimo, da velja \(Vᵢ ⊩ \lpoR\p{fᵢ}\).
%   Ker sta notranjosti množic \(A\) in \(B\) prazni, je prazna tudi notranjost \(A ∪ B\),
%   torej velja \({fᵢ\res{Vᵢ} > 0}\).
%   Ker je \(fᵢ\) pozitivna natanko na komplementu \(A ∪ B\) je torej \(Vᵢ\)
%   podmnožica \(\c{A ∪ B} = \lpoR\p{x} ∩ \lpoR\p{y}\), kar je točno to, kar smo želeli.
% \end{dokaz}

\begin{posledica}
  Nad lokalno \(T₆\) prostorom je \(\lpoR\) idempotenten.
\end{posledica}

% \begin{izrek}
%   %Let \(X\) be a topological space. Then \(X\) is locally \(T₆\) if and only if the
%   %reduction \(\lem ≤_I \lpo\) holds.
%   Naj bo \(X\) topološki prostor. Tedaj je \(X\) lokalno \(T₆\) natanko tedaj,
%   ko nad \(X\) velja redukcija \(\lem ≤_I \lpo\).
% \end{izrek}
% \begin{proof}
%   \begin{enumerate}
%   \item[\((⇒)\)]
%     Denimo, da je \(X\) lokalno \(T₆\).
%     Potem imamo pokritje s \(T₆\) prostori, t.j. za vsak \(Xᵢ\) iz pokritja in
%     vsako zaprto množico \(A ⊆ Xᵢ\), imamo zvezno funkcijo \(f : Xᵢ → ℝ\), tako
%     da je \(f_A(x) = 0\) natanko tedaj, ko je \(x ∈ A\). Poleg tega lahko
%     zahtevamo, da je ta funkcija nenegativna.

%     %Potem imamo, za vsako zaprto množico \(A ⊆ X\), zvezno funkcijo
%     %\(f_A : X → ℝ\), tako da je \(f_A(x) = 0\) natanko tedaj, ko je \(x ∈ A\).
%     %Poleg tega lahko zahtevamo, da je ta funkcija nenegativna.

%     Dokažimo sedaj, da velja redukcija.
%     Naj bo \(U' ⊆ X\) poljubna odprta množica in \(p ∈ Ω(U')\).
%     Dokazati želimo, da obstaja tako pokritje \(U'\), da na njem lokalno
%     obstajajo realna števila, za katera \(\lpo\) pomeni, da (lokalno) velja
%     \(p ∨ ¬p\).

%     Naj bo \(Uᵢ' ≔ U'∩Xᵢ\) pokritje \(U'\) in \(Aᵢ ≔ \c{p ∨ ¬p} ∩ Uᵢ\),
%     in \(xᵢ : Uᵢ' → ℝ\) \(T₆\) separacijska funckija \(Aᵢ\) na \(Uᵢ'\).

%     %Izkaže se, da lahko za pokritje vzamemo kar celoten \(U'\), za realno število
%     %pa vzemimo \(x ≔ f_A\) funkcija, ki jo dobimo iz \(T₆\) lastnosti.

%     Predpostavimo sedaj, da je \(Uᵢ ⊆ Uᵢ'\) in da velja \(Uᵢ ⊩ \lpo{\p{xᵢ}}\).
%     % To pomeni, da obstaja pokritje \(\b{Uᵢⱼ} ∈ \cov{Uᵢ}\), tako da za vsak element
%     To pomeni, da obstaja pokritje \(\cov[i]{U}{j}\), tako da za vsak element
%     pokritja velja \(xᵢ\res{Uᵢⱼ} ≤ 0\) ali \(xᵢ\res{Uᵢⱼ} > 0\).

%     Ker je \(xᵢ\) nenegativna, je \(xᵢ ≤ 0\) natanko tedaj, ko je \(xᵢ = 0\), torej
%     po definiciji zgolj na množici \(Aᵢ\).
%     Od tod sledi, da je \(xᵢ\res{Uᵢⱼ} ≤ 0\) zgolj v primeru, ko je \(Uᵢⱼ ⊆ Aᵢ\).
%     Vendar pa ima \(Aᵢ\) prazno notranjost, tako da je tak \(Uᵢⱼ\) nujno prazen.

%     Ostane le še drugi primer, torej \(xᵢ\res{Uᵢⱼ} > 0\).
%     Ker množice \(Uᵢⱼ\) tvorijo pokritje \(Uᵢ\), je potem tudi \(xᵢ > 0\) na
%     celotnem \(Uᵢ\). Sledi, da je množica \(Aᵢ∩Uᵢ\) prazna,
%     in velja \(Uᵢ ⊩ \lem{\p{p}}\).
%   \item[\((⇐)\)]
%     V obratno smer pa predpostavimo, da obstaja redukcja iz \(\lem\) na \(\lpo\).

%     % To pomeni, da za vsako odprto množico \(U ⊆ X\) in \(p ∈ Ω(U)\) obstaja tako
%     % pokritje \(\b{Uᵢ} ∈ \cov{Uᵢ}\) in realna števila \(xᵢ : Uᵢ → ℝ\),
%     % da lokalno

%     Trditev bomo dokazali v dveh korakih. Najprej, naj bo \(A ⊆ X\) zaprta
%     množica s prazno notranjostjo.

%     Tedaj na \(U ≔ X\) in \(p = Aᶜ\) uporabimo redukcijo \(\lem ≤_I \lpo\).
%     % To pomeni, da dobimo pokritje \(\b{Xᵢ} ∈ \cov{X}\) in realna števila
%     To pomeni, da dobimo pokritje \(\cov{X}{i}\) in realna števila
%     \(xᵢ : Xᵢ → ℝ\), tako da za vsak \(i\) in \(Uᵢ ⊆ Xᵢ\) velja
%     % \(\exists{\b{Uᵢⱼ} ∈ \cov{\p{Uᵢ}}}{∀_j xᵢ\res{Uᵢⱼ} ≤ 0 ∨ xᵢ\res{Uᵢⱼ} > 0} ⇒ Uᵢ ⊩ \lem{\p{p}}\).
%     \(\exists{\cov[i]{U}{j}}{∀_j xᵢ\res{Uᵢⱼ} ≤ 0 ∨ xᵢ\res{Uᵢⱼ} > 0} ⇒ Uᵢ ⊩ \lem{\p{p}}\).

%     Ker ima \(A\) prazno notranjost, je \(p ∨ ¬p = p\), torej je
%     \(Uᵢ ⊩ \lem{\p{p}}\) ekvivalentno \(Uᵢ ⊆ p\).

%     Brez škode za splošnost lahko vzamemo \(Uᵢ = Xᵢ\) in
%     \(\b{Uᵢⱼ} = \b{Vᵢ,Vᵢ'}\), tako da \(xᵢ\res{Vᵢ} ≤ 0 ∨ xᵢ\res{Vᵢ'} > 0 ⇒ Xᵢ ⊆ p\).

%     Dokazati želimo, da je

%   \end{enumerate}
% \end{proof}


% \begin{izrek}
%   %Let \(X\) be a topological space. Then \(X\) is locally \(T₆\) if and only if the
%   %reduction \(\lem ≤_I \lpo\) holds.
%   Naj bo \(X\) topološki prostor. Tedaj je \(X\) lokalno \(T₆\) natanko tedaj,
%   ko nad \(X\) velja funkcijska redukcija \(\lem ≤_{FI} \lpo\).
% \end{izrek}
% \begin{proof}
%   \begin{enumerate}
%   \item[\((⇒)\)]
%     Denimo, da je \(X\) lokalno \(T₆\).
%     Potem imamo pokritje s \(T₆\) prostori, t.j. za vsak \(Xᵢ\) iz pokritja in
%     vsako zaprto množico \(A ⊆ Xᵢ\), imamo zvezno funkcijo \(f : Xᵢ → ℝ\), tako
%     da je \(f_A(x) = 0\) natanko tedaj, ko je \(x ∈ A\). Poleg tega lahko
%     zahtevamo, da je ta funkcija nenegativna.

%     %Potem imamo, za vsako zaprto množico \(A ⊆ X\), zvezno funkcijo
%     %\(f_A : X → ℝ\), tako da je \(f_A(x) = 0\) natanko tedaj, ko je \(x ∈ A\).
%     %Poleg tega lahko zahtevamo, da je ta funkcija nenegativna.

%     Po aksiomu o izbiri dobimo preslikavo \(Cl(Xᵢ) → \p{Xᵢ → ℝ}\).
%     Če jo preuredimo in uporabimo vložitev \(Ω(Xᵢ) \hookrightarrow Cl(Xᵢ)\),
%     ki \(p\) slika v \(\c{p ∪ ¬p}\), dobimo preslikavo \(Fᵢ: Xᵢ → \p{Ω(Xᵢ) → ℝ}\).

%     Dokažimo sedaj, da velja redukcija.
%     Vzamimo množice \(Uᵢ = Xᵢ\) in \(fᵢ = Fᵢ\).

%     Naj bo sedaj \(U ⊆ Xᵢ\), \(p ∈ Ω(U)\), \(A ≔ \c{p ∪ ¬p}\), \(x ≔ fᵢ(p)\) in predpostavimo,
%     da velja \(Xᵢ ⊩ \lpo{\p{x}}\)

%     % To pomeni, da obstaja pokritje \(\b{Uᵢ} ∈ \cov{U}\), tako da za vsak element
%     To pomeni, da obstaja pokritje \(\cov{U}i\), tako da za vsak element
%     pokritja velja \(x\res{Uᵢ} ≤ 0\) ali \(x\res{Uᵢ} > 0\).

%     Ker je \(x\) nenegativen, je \(x ≤ 0\) natanko tedaj, ko je \(x = 0\), torej
%     po definiciji zgolj na množici \(A\).
%     Od tod sledi, da je \(x\res{Uᵢ} ≤ 0\) zgolj v primeru, ko je \(Uᵢ ⊆ A\).
%     Vendar pa ima \(A\) prazno notranjost, tako da je tak \(Uᵢ\) nujno prazen.

%     Ostane le še drugi primer, torej \(x\res{Uᵢ} > 0\).
%     Ker množice \(Uᵢ\) tvorijo pokritje \(U\), je potem tudi \(x > 0\) na
%     celotnem \(U\). Sledi, da je množica \(A∩U\) prazna,
%     in velja \(U ⊩ \lem{\p{p}}\).
%   \item[\((⇐)\)]
%     V obratno smer pa predpostavimo, da obstaja redukcja iz \(\lem\) na \(\lpo\).

%     To pomeni, da obstaja pokritje \(Xᵢ \circlearrowright X\)
%     in preslikave \(Fᵢ: Xᵢ → \p{Ω(Xᵢ) → ℝ}\), za katere velja
%     \[Xᵢ ⊩ \for{p∈Ω}{\lpo{\p{Fᵢ(p)}} ⇒ \lem{\p{p}}}\text.\label{eq:lemlpo}\]

%     Trditev bomo dokazali v dveh korakih. Najprej, naj bo \(A ⊆ Xᵢ\) zaprta
%     množica s prazno notranjostjo.

%     Tedaj na \(U ≔ Xᵢ\) in \(p = Aᶜ\) uporabimo \ref{eq:lemlpo}.
%     To pomeni, da dobimo implikacijo \(U⊩\lpo{\p{Fᵢ(p)}} ⇒ U⊩\lem{\p{p}}\).

%     Naj bo \(f≔Fᵢ(p)\). Potem, če obstaja pokritje \(Uᵢ \circlearrowright U\),
%     tako da za vsak \(i\) velja \(f\res{Uᵢ} ≤ 0\) ali \(f\res{Uᵢ} > 0\),
%     potem velja tudi \(U ⊆ p ∪ ¬p = p\), oziroma \(A ⊆ Uᶜ\).

%     Če tako pokritje obstaja, ga lahko združimo v dve množici, eno, na kateri je
%     \(f\) nepozitiven (torej ker je nenegativen kar \(0\)), in eno, na kateri je pozitiven.
%     Naj bosta torej \(U₀ in Uₚ\) ti dve (odprti) množici.

%     Vednar je pa \(U₀ = f⁻¹{0}\), torej zaprta.

%   \end{enumerate}
% \end{proof}

\subsection{Idempotenca \(\lpo\) nad Cantorjevim prostorm}

\begin{trditev}
  Nad vsakim okolišem \(X\) velja \(\lpo ≤ \lpoR\).
\end{trditev}
\begin{dokaz}
  Naj bo \(U ⊆ X\), \(α : U → C\). Definirajmo \(x ≔ φ∘α\), kjer
  \(φ : C → ℝ\) slika
  \[ ∑_{i ∈ ℕ₊}\frac{2aᵢ}{3ⁱ} ∈ C\text{, } aᵢ ∈ 2
     \text{, v } ∑_{i ∈ ℕ₊}\frac{aᵢ}{2ⁱ}\text.\]

  Potem je \(φ(r) = 0 ⇔ r = 0\) in velja \(\lpo{\p{α}} ⇔ \lpoR{\p{x}}\).
\end{dokaz}

\begin{lema}
  Vsaka preslikava \(f : C → I\) inducira preslikavo \(\floor f : C → C\),
  s predpisom \(\floor f(a) ≔ \sup\set{x ∈ C}{x < f(a)}\).
\end{lema}
\begin{dokaz}
  % dokaži da maksimum obstaja (da je zaprta).
  Ker je \(C\) kompakten, je predpis dobro definiran, torej moramo preveriti
  zgolj zveznost. Naj bo \(U=w\) bazična odprta množica.

  %Naj bodo \(l ≔ (w-1)\hat{1}\), \(r ≔ (w+1)\hat{0}\), in \(U' ≔ \p{l, r}\).
  Naj bodo \(l ≔ w\hat{0}\), \(r ≔ (w+1)\hat{0}\), in \(U' ≔ \p{l, r}\).
  Potem je \(U ⊆ \cli{w\hat{0}, w\hat{1}} ⊆ U' ∪ \{l\}\).
  Upamo sedaj, da je \(\floor f⁻¹(U) = f⁻¹(U')\).

  Naj bo \(a ∈ f⁻¹(U')\), torej \(l < f(a) < r\).
  Iz \(f(a) < r\) sledi \(\floor f(a) < r\), torej \(\floor f(a) ≤ w\hat{1}\),
  iz \(l < f(a)\) pa sledi, da je \(l ≤ \floor f(a)\).

  Če pa je \(a ∈ \floor f⁻¹(U)\), imamo \(\sup\set{x ∈ C}{x < f(a)} ∈ U\).
  Iz \(w\hat{0} ≤ α(a)\) sledi \(w\hat{0} ≤ f(a)\),
  iz \(α(a) ≤ w\hat{1}\) pa sledi, da je \(f(a) ≤ w\hat{1}\).

  % Če pa je \(a ∈ f⁻¹(U')\), pa je \(l < f(a) < r\).
  % Iz \(f(a) < r\) sledi \(\floor f(a) < r\), torej \(\floor f(a) ≤ w\hat{1}\),
  % iz \(l < f(a)\) pa sledi, da je \(l ≤ \floor f(a)\).

  % l here is wrong, i should fix U' (maybe i can)
  % Iz tega dvojega lahko sedaj sklepamo, da je \(\floor f⁻¹(U) = f⁻¹(U') ⧵ f⁻¹[0,l]\),
  Iz tega dvojega lahko sedaj sklepamo, da je
  \(f⁻¹(U') ⊆ \floor f⁻¹(U) ⊆ f⁻¹(U') ∪ f⁻¹\{w\hat{0}\}\).
  Denimo, da je \(\floor f(a) = w\hat0\).
  Potem obstaja zaporedje \(xᵢ ↗ w\hat0\), za katero velja \(xᵢ < f(a)\)
  Vendar je pa množica \(\left[ w\hat0, 1 \right]\) odprta, torej mora vsebovati
  skoraj vsa števila \(xᵢ\). Ker velja \(xᵢ ≤ w\hat0\), je torej
  \(xᵢ = w\hat{0}\) skoraj povsod, in velja \(f(a) > w\hat0\).

  Sledi, da je \(\floor f⁻¹(U) ⊆ f⁻¹(U')\), torej sta ti dve množici enaki,
  in je funkcija \(\floor f\) zvezna.
\end{dokaz}

\begin{posledica}
  V zgornji lemi lahko domeno zamenjamo s poljubno odprto množico.
\end{posledica}

\begin{lema}
  Nad Cantorjevim prostorom je \(\lpo\) ekvivalenten \(\lpoR\).
\end{lema}
\begin{dokaz}
  Vemo, da velja \(\lpo ≤_I \lpoR\).

  Pokažimo sedaj, da velja tudi obrat.
  Naj bo \(U ⊆ C\), \(f : U → ℝ\).
  Brez škode za splošnost, lahko \(f\) popravimo do preslikave \(U → I\).
  Vzamemo trivialno pokritje in naj bo \(α ≔ \floor f\).

  Potem je očitno \(α ≤ f\), torej \(f(x) = 0 ⇒ α(x) = 0\).
  Obratno pa, če je \(α(x) = 0\) pomeni, da imamo za vsak \(a ≠ 0\)
  neenakost \(f(x) < a\), torej \(f(x) = 0\).

  Sledi torej, da je \(\lpo\p{α} = \lpoR\p{f}\).
\end{dokaz}

\begin{posledica}
  Nad Cantorjevim prostorom je \(\lpo\) idempotenten.
\end{posledica}




%%% Local Variables:
%%% mode: latex
%%% TeX-master: "main"
%%% End:
