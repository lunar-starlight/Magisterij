\section{Interpretacija pretvorbe primerkov v topoloških modelih}

\subsection{Pretvorba primerkov}

\subsection{Idempotenca \alpo* nad lokalno \(T₆\) prostori}

\begin{trditev}
  Nad vsakim okolišem \(X\) je \(\lem*\) idempotenten.
\end{trditev}
\begin{dokaz}
  Naj bosta \(p\) in \(q\) resničnostni vrednosti.
  Potem definiramo \(r ≔ \lem²{\p{p, q}}\).
  Ker \(\lem ⁿ\) slika v goste resničnostne vrednosti, je \(\lem{\p r} = r\), tako da
  je \[\lem{\p r} = r = \lem²{\p{p, q}}\text.\qedhere\]
\end{dokaz}

Spomnimo se na Trditev~\ref{th:lT6-have-AKS}:
% TODO: set number
\begin{trditev}\label{th:lT6-have-AKS}
  Nad lokalno \(T₆\) prostori velja analitična Kripkejeva shema.
\end{trditev}

% TODO: Faktoriziraj čez AKS ⇒ LEM ≤ ALPO
\begin{izrek}
  Naj bo \(X\) lokalno \(T₆\). Potem nad \(X\) velja redukcija \(\lem* ≤ \alpo*\).
\end{izrek}
\begin{proof}
  % TODO: reword kot 'ALPO je LEM za Ω_ℝ in AKS je Ω_ℝ = Ω'?
  Ker vemo, da nad \(X\) velja analitična Kripkejeva shema, zadošča
  konstruktivno pokazati, da velja \(\lem* ≤ \alpo*\).
  Po predpostavki, je vsaka resničnostna vrednost \(p\) oblike \(x > 0\) za nek
  \(x : ℝ\). Tedaj pa velja
  \[ \lem{\p p} = p ∨ ¬ p = x > 0 ∨ ¬\p{x > 0} = \alpo{\p x}\text, \]
  torej za vsak \(p : Ω\) obstaja \(x : ℝ\), da je \(\alpo{\p x} ⇒ \lem{\p p}\).
  % Ker dokazujemo resnico v topološkem modelu brez škode za splošnost
  % predpostavimo, da je prostor \(T₆\).
  % To pomeni da za vsako odprto množico \(U\) obstaja nenegativna funkcija
  % \(P_U : X → ℝ\), tako da velja \(P_U > 0 ⇔ U\).
  % Če želimo pretvoriti \(\lem*\) na \(\alpo*\), mora za vse
  % \(U ⊆ X\) veljati \(\eventually{x : ℝ}{x ≤ 0 ∨ x > 0 ⇒ U ∪ ¬U}\).
  % Uporabimo predpostavko na \(U ∪ ¬U\), torej nam ostane pokazati, da
  % velja \({P_U ≤ 0 ∨ P_U > 0 ⇒ P_U > 0}\).
  % Ker je \(⟦P_U ≤ 0⟧ = ¬{\p{U ∪ ¬ U}} = ∅\), velja \(¬\p{x ≤ 0}\), kar dokaže trditev.
\end{proof}
\begin{posledica}
  Če je prostor \(X\) lokalno \(T₆\), velja \(\alpo*×\alpo* ≤ \alpo*\).
\end{posledica}
\begin{posledica}
  Nad lokalno \(T₆\) prostorom je \(\alpo*\) idempotenten.
\end{posledica}


\subsection{Idempotenca \lpo* nad Cantorjevim prostorm}

\begin{trditev}
  Nad vsakim okolišem \(X\) velja \(\lpo* ≤ \alpo*\).
\end{trditev}
\begin{dokaz}
  Naj bo \(φ : 2^ℕ → ℝ\) definirana s predpisom
  \[α ↦ \lim_{n→∞}2^{-\min\set{k ∈ ℕ}{αₖ = 1 ∨ k = n}}\text.\]
  Denimo, da je \(α ∈ 2^ℕ\) in uporabimo \(\alpo*\) na \(φ(α)\).

  Če velja \(φ(α) ≤ 0\) vemo, da \(α\) ne more imeti enice na nobenem mestu.

  Alternativno, če je \(φ(α) > 0\), pa je tudi \(φ(α) > 2^{-n}\) za nek \(n\).
  Sledi, da mora imeti \(α\) na enem izmed prvih \(n\) mest enico.
\end{dokaz}

% Joint with prof. Bauer
\begin{lema}
  Za vsako funkcijo \(f : C → ℝ\) obstaja funkcija \(\hat f : C → C\), za katero
  velja \(C ⊩ \hat f \apart 0 ⇔ f > 0\).
\end{lema}
\begin{proof}
  Naj bo \(U := ⟦f > 0⟧\). Potem obstaja števna particija \(U = ⋃ₖVₖ\) z
  bazičnimi odprtimi. Ker so te kompaktne, za vsak \(k\) velja
  \(\exist{i ∈ ℕ}{f{\res{Vₖ}} > 2⁻ⁱ}\). Po aksiomu izbire torej obstaja funkcija
  izbire \(m : ℕ → ℕ\), da velja \(f{\res{Vₖ}} > 2⁻ᵐ⁽ᵏ⁾\). Potem pa definiramo
  \[ C ⊩ \hat fᵢ ≔
    \begin{cases}
      1;& \exist{k : ℕ}{Vₖ ∧ i = m(k)}\\
      0;& \text{sicer.}
    \end{cases}\]
  Potem pa velja
  \[ \hat f \apart 0 ⇔ \exist{i : ℕ}{\exist{k : ℕ}{Vₖ ∧ i = m(k)}} ⇔ \exist{k : ℕ}{Vₖ} ⇔ U\text, \]
  torej je \(C ⊩ \hat f \apart 0 ⇔ f > 0\).
\end{proof}
\begin{posledica}
  V zgornji lemi lahko domeno zamenjamo s poljubno odprto množico.
\end{posledica}

\begin{lema}
  Nad Cantorjevim prostorom je \(\lpo*\) ekvivalenten \(\alpo*\).
\end{lema}
\begin{dokaz}
  Vemo že, da velja \(\lpo* ≤ \alpo*\) tako da moramo pokazati zgolj obratno redukcijo.
  Naj bo \(U ⊩ x : ℝ\). Potem iz gornje leme dobimo \(U ⊩ \hat x : C\), za
  katerega velja \(\hat x \apart 0 ⇔ x > 0\).
  Kontrapizitivna oblika te ekvivalence pa pravi ravno, da je \(\hat x = 0 ⇔ x ≤ 0\),
  torej res velja \(\lpo{\p{\hat f}} ⇔ \alpo{\p{f}}\).
\end{dokaz}
\begin{posledica}
  Nad Cantorjevim prostorom je \(\lpo*\) idempotenten.
\end{posledica}
\begin{opomba}
  V tem podrazdelku bi lahko namesto Cantorjevega prostora zares vzeli
  katerikoli kompakten nič dimenzionalen prostor (z drugimi besedami, Stoneov prostor).
\end{opomba}



%%% Local Variables:
%%% mode: latex
%%% TeX-master: "main"
%%% End:
