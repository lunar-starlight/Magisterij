\section{Slovar}

\begin{table}[h]
  \centering
  \begin{tabularx}{1.0\linewidth}{X|X}
    Diskreten               & Izključena tretja možnost\\
    \hline
    Ekstremalno nepovezan   & DeMorganov zakon\\
                            & \(≡\) Vsaka neprazna omejena množica realnih funkcij ima supremum\\
                            & \(≡\) \(\Rm = \Rd\)\\
    \hline
    Nivojnice realnih funkcij so odprte & Analitični števen princip odločitve\\
    \(≡\) \(T_{3.5}\) P-prostor   &\\
    \hline
    Realno nepovezan        & \(\awlpo*\)\\
                            & \(≡\) Vsaka števna neprazna omejena množica realnih funkcij ima supremum\\
    \hline
    %F-prostor               & \(\allpo*\)\\
    Skoraj P-prostor        & Analitični princip Markova\\
    \hline
    \((*)\)                 & \(\AC(\c A)\) in \(\for{B}{\c A ↬ B ≅ \c{A ↝ B}}\)\\
    \hline
    Lokalno \(T₆\)          & Funkcija izbire za analitično Kripkejevo shemo\\
                            & Implicira idempotenco \(\alpo*\)\\
    \hline
    Stoneov                 & Implicira idempotenco \(\lpo*\)\\
    \hline
    Lokalno povezan         & Implicira \(\lpo*\), obrat velja za \(2\)-števne prostore\\
                            & Implicira \(\Rc = \c ℝ\), obrat velja za \(2\)-števne prostore\\
    \hline
    P-prostor               & Implicira števno izbiro in \(\alpo*\)\\
    \hline
    Aleksandrov             & Implicira aksiom izbire iz konstantnih tipov\\
    \hline
    Ultraparakompakten      & Implicira odvisno izbiro
    % \(1\)-števen \(T_{3.5}\) & Implicira \(\awmp*\)\\
  \end{tabularx}
  \caption[Slovar]{Slovar topoloških lastnosti in logičnih principov}
  \label{tab:top-logic-dict}
\end{table}


\cleardoublepage
\section{Slovar kratic}

Za lažjo navigacijo tu navedimo pomene in sklice na definicije posameznih
kratic.

\begin{table}[h]
  \centering
  \begin{tabularx}{1.0\linewidth}{p{0.3\linewidth}X}
    \(\Rd\) & Dedekindova realna števila~\ref{def:Rd}\\
    \(\Rc\) & Cauchyjeva realna števila~\ref{def:Rc}\\
    \(\Rm\) & MacNeilleova realna števila~\ref{def:Rm}\\
    \(\AC_Σ(A,B)\) & Izbira nad \(Σ\) iz \(A\) v \(B\)~\ref{pr:ac}\\
    \(\CC\) & Števna izbira\\
    \(\CCv\) & Števna disjunktivna izbira\\
    \(\DC\) & Odvisna izbira~\ref{pr:dc}\\
    \(\lem*\) & Izključena tretja možnost~\ref{pr:lem}\\
    \(\wlem*\) & Šibka izključena tretja možnost~\ref{pr:wlem}\\
    \(\lpo*\) & Števna odločitev~\ref{pr:lpo}\\
    \(\alpo*_ℝ\) & Analitična števna odločitev za \(ℝ\)~\ref{pr:alpo}\\
    \(\alpo*\) & \(\alpo*_{\Rd}\)~\ref{pr:alpo}\\
    \(\awlpo*_ℝ\) & Analitična šibka števna odločitev za \(ℝ\)~\ref{pr:alpo}\\ 
    \(\awlpo*\) & \(\awlpo*_{\Rd}\)~\ref{pr:alpo}\\
    \(φ{\res Σ}\) & \(\for{p∈Σ}{φ(p)}\)~\ref{pr:res}\\
    \(\ks*(Σ)\) & Kripkejeva shema za \(Σ ⊆ Ω\)~\ref{pr:ks}\\
    \(\ks*\) & \(\ks*(Σ⁰₁)\)~\ref{pr:ks}\\
    \(\aks*_ℝ\) & Analitična Kripkejeva shema za \(ℝ\)~\ref{pr:ks}\\
    \(\aks*\) & \(\aks*_{\Rd}\)~\ref{pr:ks}\\
    \(\mp*\) & Princip Markova~\ref{pr:mp}
  \end{tabularx}
  \caption{Slovar kratic logičnih principov}
  \label{tab:log-principi}
\end{table}



%%% Local Variables:
%%% mode: latex
%%% TeX-master: "main"
%%% End:
