\section{Slovar}

\begin{table}[h]
  \centering
  \begin{tabularx}{1.0\linewidth}{C|C}
    Topološke lastnosti     & Logični principi\\
    \hline
    Diskreten               & Izključena tretja možnost\\
    Ekstremalno nepovezan   & DeMorganov zakon\\
                            & Vsaka neprazna omejena množica realnih funkcij ima supremum\\
                            & \(\Rm = \Rd\)\\
    Nivojnice realnih funkcij so odprte & Analitični števen princip odločitve\\
    \(T_{3.5}\) P-prostor   &\\
    Basically disconnected  & \(\awlpo*\)\\
                            & Vsaka števna neprazna omejena množica realnih funkcij ima supremum\\
    %F-prostor               & \(\allpo*\)\\
    Skoraj P-prostor        & Analitični princip Markova\\
    \(\p{A,B,Σ}\) pofinitve & \(\AC_Σ(\c A, \c B)\) in \({{\c B}^{\c A} = \c{B^A}}\)\\
    Lokalno \(T₆\)          & Funkcija izbire za analitično Kripkejevo shemo\\
                            & Implicira idempotenco \(\alpo*\)\\
    Stoneov                 & Implicira idempotenco \(\lpo*\)\\
    Lokalno povezan         & Implicira \(\lpo*\), obrat velja za \(2\)-števne prostore\\
                            & Implicira \(\Rc = \c ℝ\), obrat velja za \(2\)-števne prostore\\
    P-prostor               & Implicira števno izbiro in \(\alpo*\)\\
    Aleksandrov             & Implicira aksiom izbire iz konstantnih tipov\\
    Ultraparakompakten      & Implicira odvisno izbiro
    % \(1\)-števen \(T_{3.5}\) & Implicira \(\awmp*\)\\
  \end{tabularx}
  \caption[Slovar]{Slovar topoloških lastnosti in logičnih principov}
  \label{tab:top-logic-dict}
\end{table}

NOTE: pri basically disconnected rabim dednost na odprte. Isto za F-prostor.

%%% Local Variables:
%%% mode: latex
%%% TeX-master: "main"
%%% End:
