\section{Slovar}

\begin{table}[h]
  \centering
  \begin{tabularx}{1.0\linewidth}{X|X}
    Diskreten               & Izključena tretja možnost\\
    \hline
    Ekstremalno nepovezan   & DeMorganov zakon\\
                            & \(≡\) Vsaka neprazna omejena množica realnih funkcij ima supremum\\
                            & \(≡\) \(\Rm = \Rd\)\\
    \hline
    Nivojnice realnih funkcij so odprte & Analitični števen princip odločitve\\
    \(≡\) \(T_{3.5}\) P-prostor   &\\
    \hline
    Realno nepovezan        & \(\awlpo*\)\\
                            & \(≡\) Vsaka števna neprazna omejena množica realnih funkcij ima supremum\\
    \hline
    %F-prostor               & \(\allpo*\)\\
    Skoraj P-prostor        & Analitični princip Markova\\
    \hline
    \((*)\)                 & \(\AC(\c A, \c B)\) in \({\c A ↬ {\c B} ≅ \c{B^A}}\)\\
    \hline
    Lokalno \(T₆\)          & Funkcija izbire za analitično Kripkejevo shemo\\
                            & Implicira idempotenco \(\alpo*\)\\
    \hline
    Stoneov                 & Implicira idempotenco \(\lpo*\)\\
    \hline
    Lokalno povezan         & Implicira \(\lpo*\), obrat velja za \(2\)-števne prostore\\
                            & Implicira \(\Rc = \c ℝ\), obrat velja za \(2\)-števne prostore\\
    \hline
    P-prostor               & Implicira števno izbiro in \(\alpo*\)\\
    \hline
    Aleksandrov             & Implicira aksiom izbire iz konstantnih tipov\\
    \hline
    Ultraparakompakten      & Implicira odvisno izbiro
    % \(1\)-števen \(T_{3.5}\) & Implicira \(\awmp*\)\\
  \end{tabularx}
  \caption[Slovar]{Slovar topoloških lastnosti in logičnih principov}
  \label{tab:top-logic-dict}
\end{table}


%%% Local Variables:
%%% mode: latex
%%% TeX-master: "main"
%%% End:
