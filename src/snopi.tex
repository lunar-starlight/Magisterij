\section{Snopi nad okolišem}

Naj \(X, Y, …\) označujejo okoliše.



\begin{definicija}
  \emph{Predsnop nad okolišem \(X\)} je funktor \(F : \op{\O{\p{X}}} → \setcat\).
  Elementom množic \(F(U)\) pravimo \emph{prerezi (pri U)}.
\end{definicija}

\begin{opomba}
  Predsnope lahko definiramo tudi nad poljubno kategorijo \(\cat C\).
  V tem primeru pišemo \(\psh{\cat C} ≔ \op{\cat C} → \setcat\), za predsnope nad okolišem
  \(X\) pa raje spustimo \(\mc{O}\) in pišemo kar \(\psh{X}\).
\end{opomba}

Preden lahko definiramo snope, moramo nekaj povedati o družinah prerezov.

\begin{definicija}
  Prereza \(s ∈ F(U)\) in \(t ∈ F(V)\) sta \emph{skladna}, ko velja \(s\res{V} = t\res{U}\).
  Družina prerezov \(sᵢ ∈ F(Uᵢ)\) je \emph{skladna}, ko so njeni elementi paroma
  skladni. Prerez je \emph{skladen} z družino, ko je paroma skladen z vsemi
  njenimi elementi.
\end{definicija}

\begin{definicija}
  \emph{Snop nad okolišem} je predsnop, za katerega velja:
  \begin{enumerate}
  \item Za vsako skladno družino prerezov \(sᵢ ∈ F(Uᵢ)\) obstaja enoličen prerez
    \({s ∈ F\p{⋃ᵢUᵢ}}\), ki je skladen z družino.
  \item Za vsako pokritje \(Uᵢ \circlearrowright U\) in skladno družino prerezov \(sᵢ ∈ F(Uᵢ)\) obstaja enoličen prerez
    \(s ∈ F(U)\), ki je skladen z družino.
  \item Za vsako skladno družino prerezov \(sᵢ\) na \(Uᵢ\) obstaja enoličen prerez
    \(s\) na \(⋃ᵢUᵢ\), ki je skladen z družino.
  % \item Za vsako skladno družino prerezov \(sᵢ : Uᵢ\) obstaja enoličen prerez
  %   \(s : ⋃ᵢUᵢ\), ki je skladen z družino.
  \item Za vsako skladno družino prerezov \(Uᵢ ⊩ sᵢ\) obstaja enoličen prerez
    \({⋃ᵢUᵢ ⊩ s}\), ki je skladen z družino.
  \item Za vsako skladno družino prerezov \(Uᵢ ⊩ sᵢ ∈ F\) obstaja enoličen prerez
    \({⋃ᵢUᵢ ⊩ s ∈ F}\), ki je skladen z družino.
  % \item Za vsak par prerezov \(s, t ∈ F(U)\), je \(s = t\) natanko tedaj, ko za
  %   pokritje \(Uᵢ \circlearrowright U\) velja,
  \end{enumerate}
\end{definicija}

\begin{definicija} %TODO: popravi ta vsaj unija biznis
  \emph{Amalgamacija} družine prerezov je vsak prerez, ki je definiran vsaj na
  uniji družine in je z njo skladen.
\end{definicija}

\begin{opomba}
  Če družina ni skladna očitno nima amalgamacije.
\end{opomba}
\begin{slogan}
  Vsaka skladna družina prerezov v snopu ima enolično amalgamacijo.
\end{slogan}
\begin{slogan}
  Predsnop je snop, kadar ima vsaka skladna družina prerezov enolično amalgamacijo.
\end{slogan}

\begin{definicija}
  Preslikava med (pred)snopoma \(F\) in \(G\) je naravna transformacija med
  njima kot funktorjema.
\end{definicija}

\begin{definicija}
  Kategoriji vseh snopov na \(X\), skupaj z morfizmi med njimi pravimo \(\sh X\).
\end{definicija}


\subsection{Prezentacije snopov}

\begin{definicija}
  \emph{\(ℒ\)-vrednotena množica} oziroma \emph{\(ℒ\)-množica} je množica \(A\),
  skupaj s preslikavo, ki jo pišemo \(⟦- = -⟧ : A×A → ℒ\), tako da velja
  \begin{align*}
    ⟦ a = b ⟧ ≤ ⟦ b = a ⟧\\
    ⟦ a = b ⟧ ∧ ⟦ b = c ⟧ ≤ ⟦ a = c ⟧
  \end{align*}
  Če iz konteksta ni razvidno kateri množici pripada preslikava \(⟦- = -⟧\),
  jo pišemo z indeksom.
\end{definicija}

Če na okvir \(ℒ\) mislimo kot množico resničnostnih vrednosti, lahko rečemo, da
je to simetrična in tranzitivna relacija na množici \(A\).
% TODO: jezik ℒ-množic?

\begin{definicija}
  \emph{Morfizem med \(ℒ\)-množicama} \(A\) in \(B\) je preslikava \(⟦- = f(-)⟧ : B×A → ℒ\),
  za katero velja
  \begin{align}% TODO: change
    &⟦b = b'⟧ ∧ ⟦b' = f(a)⟧ ≤ ⟦b = f(a)⟧    \tag{M1}\label{M1}\\
    &⟦b = f(a)⟧ ∧ ⟦a = a'⟧ ≤ ⟦b = f(a')⟧    \tag{M2}\label{M2}\\
    &⟦b = f(a)⟧ ∧ ⟦b' = f(a)⟧ ≤ ⟦b = b'⟧    \tag{M3}\label{M3}\\
    &⟦a = a⟧ = ⋁_{b ∈ B} ⟦b = f(a)⟧         \tag{M4}\label{M4}
  \end{align}
  Na prvi dve pravili lahko gledamo kot na skladnost z relacijama na \(A\) in
  \(B\), na drugi dve pa kot na enoličnost in celovitost, zato pišemo morfizme
  tudi v bolj funkcijskem zapisu kot \(f : A ↬ B\).

  Če imamo morfizma \(A \oset{f}{↬} B \oset{g}{↬} C\),
  je njun kompozitum \(gf : A ↬ C\) definiran kot
  \[ ⟦c = gf(a)⟧ ≔ ⋁_{b ∈ B} ⟦c = g(b)⟧∧⟦b = f(a)⟧\text. \]
\end{definicija}

Skupaj z identiteto \(⟦- = -⟧_A\), tvorijo \(ℒ\)-množice kategorijo, ki jo označimo \(\cat{Set}(ℒ)\).

\begin{lema}
  Naj bosta \(f, g : A ↬ B\) morfizma.
  Potem sta enaka natanko tedaj, ko velja
  \[ \for{a ∈ A, b ∈ B}{ ⟦b = f(a)⟧ ≤ ⟦b = g(a)⟧ }.\]
\end{lema}
\begin{dokaz}
  V eno smer je trditev očitna, tako da predpostavimo,
  da za vsaka \(a\) in \(b\) velja \(⟦b = f(a)⟧ ≤ ⟦b = g(a)⟧\).
  Dokazati moramo torej, da velja tudi druga neenakost.
  \begin{align*}
    ⟦b = g(a)⟧
    &= ⟦b = g(a)⟧ ∧ ⟦a = a⟧ = ⟦b = g(a)⟧ ∧ \bigvee_{b' ∈ B} ⟦b' = f(a)⟧\\
    &= \bigvee_{b' ∈ B} ⟦b = g(a)⟧ ∧ ⟦b' = f(a)⟧ ∧ ⟦b' = g(a)⟧\\
    &≤ \bigvee_{b' ∈ B} ⟦b = b'⟧ ∧ ⟦b' = f(a)⟧\\
    &= ⟦b = 1_Bf(a)⟧ = ⟦b = f(a)⟧.\qedhere
  \end{align*}
\end{dokaz}

Če na zgornjo formulo gledamo kot neko trditev v jeziku množic in funkcij, bi se
brala kot "če je \(f(a) = b\) je potem tudi \(g(a) = b\)", kar pa ponavadi res
pomeni, da sta funkciji \(f\) in \(g\) enaki.

Na podoben način bi si želeli karakterizirati epimorfizme in monomorfizme, v
kategoriji \(\cat{Set}(ℒ)\), in res jih lahko karakteriziramo kot surjekcije in
injekcije v tem jeziku:

\begin{trditev}
  Morfizem \(f : A ↬ B\) je epimorfizem, če velja
  \[ \for{b ∈ B}{ ⟦b = b⟧ = ⋁_{a ∈ A} ⟦b = f(a)⟧ }\text, \]
  in monomorfizem, če velja
  \[ \for{a,a' ∈ A, b ∈ B}{ ⟦b = f(a)⟧ ∧ ⟦b = f(a')⟧ ≤ ⟦a = a'⟧}\text. \]
\end{trditev}
\begin{dokaz}
  % TODO: dokaz
  long and arduous
\end{dokaz}

\subsubsection{Karakterizacija podobjektov}

Z uporabo gornje karakterizacije monomorfizmov, lahko karakteriziramo tudi podobjekte.

\begin{definicija}
  \(ℒ\)-podmnožica \(A\) je preslikava \( ⟦- ∈ S⟧ : A → ℒ \), za katero velja
  \begin{align*}
    &⟦a ∈ S⟧ ≤ ⟦a = a⟧\\
    &⟦a = a'⟧ ∧ ⟦a' ∈ S⟧ ≤ ⟦a ∈ S⟧
  \end{align*}
  t.~j. je skladna z relacijo na \(A\).
\end{definicija}

\begin{trditev}
  Obstaja ekvivalenca med delnima ureditvima \(ℒ\)-podmnožic \(A\) in podobjektov \(A\).
\end{trditev}
\begin{proof}
  Naj bo \(m : S ↬ A\) monomorfizem \(ℒ\)-množic.
  Potem definiramo preslikavo \( ⟦- ∈ S⟧ : A → ℒ \) s predpisom
  \(a ↦ ⋁_{s ∈ S} ⟦a = m(s)⟧\).

  % TODO: complete proof
\end{proof}

\subsubsection{Polne \texorpdfstring{\(ℒ\)}{ℒ}-množice}

Naj bo \(F\) snop nad \(ℒ\). Potem lahko za \(m ∈ F(U)\) definiramo enojec
\(⟦a ∈ ⟨m⟩⟧ ≔ ⟦a = m⟧\), kjer je \(a ∈ F(V), V ∈ ℒ\).
Na enojcih lahko definiramo tudi \(⟦⟨m⟩ = ⟨n⟩⟧ ≔ ⟦m = n⟧\).
Snop \(F\) lahko sedaj rekonstruiramo iz njegovih enojcev.
Podobno bomo storili za \(ℒ\)-množice.

\begin{definicija}
  \emph{Enojec} na \(ℒ\)-množici \(A\) je \(ℒ\)-podmnožica \(S\), za katero velja
  \[ \for{a,b ∈ S}{⟦a ∈ S⟧ ∧ ⟦b ∈ S⟧ ≤ ⟦a = b⟧}\text. \]
\end{definicija}

\begin{opomba}
  Če razpišemo definicijo \(ℒ\)-podmnožic, dobimo da so enojci preslikave, ki poleg
  gornjemu pogoju zadoščajo zgolj še drugi pogoj \(ℒ\)-podmnožic, torej
  \[ \for{a,b ∈ A}{⟦a = b⟧ ∧ ⟦b ∈ S⟧ ≤ ⟦a ∈ S⟧}\text. \]
\end{opomba}

\begin{lema}
  Za vsak \(m ∈ A\) lahko tvorimo enojec \(σₘ : A → ℒ\), s predpisom
  \(a ↦ ⟦a = m⟧\). \(ℒ\)-množico enojcev \(A\) z relacijo
  \(⟦ρ = τ⟧ ≔ ⋁_{a ∈ A} ρ(a)∧τ(a)\) označimo \(σ(A)\).
\end{lema}
\begin{dokaz}
  Preslikava \(σₘ\) je res enojec, saj velja
  \begin{align*}
    σₘ(a)∧σₘ(b) &= ⟦a = m⟧∧⟦m = b⟧ ≤ ⟦a = b⟧\text{ in}\\
    ⟦a = b⟧∧σₘ(b) &= ⟦a = b⟧∧⟦b = m⟧ ≤ ⟦a = m⟧ = σₘ(a)\text.
  \end{align*}

  Relacija je očitno simetrična, tako da si poglejmo le tranzitivnost:
  \begin{align*}
    ⟦ρ = τ⟧∧⟦τ = θ⟧
    &= \p{⋁_{a ∈ A} ρ(a)∧τ(a)} ∧ \p{⋁_{a ∈ A} τ(a)∧θ(a)}\\
    &= ⋁_{a,a' ∈ A} ρ(a)∧τ(a)∧τ(a')∧θ(a')\\
    &≤ ⋁_{a,a' ∈ A} ρ(a)∧⟦a = a'⟧∧θ(a')\\
    &≤ ⋁_{a ∈ A} ρ(a)∧θ(a) = ⟦ρ = θ⟧\text,
  \end{align*}
  kjer v drugi enakosti uporabimo, da je \(ℒ\) okvir.
\end{dokaz}

Snopi so natanko določeni s svojimi enojci, tako da si poglejmo \(ℒ\)-množice,
ki so prav tako določene s svojimi enojci.

\begin{lema}
  Za \(m,n ∈ A\) in \(τ ∈ σ(A)\) veljajo naslednje enakosti:
  \begin{enumerate}
  \item \(σ_τ(σₘ) = ⟦σₘ = τ⟧ = τ(m)\)
  \item \(⟦σₘ = σₙ⟧ = ⟦m = n⟧\)
  \item \(σₘ(n) = σₙ(m)\)
  \item \(⟦m = n⟧ = ⋁_{τ ∈ σ(A)} τ(m)∧τ(n)\)
  \end{enumerate}
\end{lema}
\begin{dokaz}
  Prvi enačaj velja po definiciji, tako da razpišemo zgolj drugega:
  \begin{equation*}
    ⟦σₘ = τ⟧ = ⋁_{a ∈ A} σₘ(a)∧τ(a) = ⋁_{a ∈ A} ⟦m = a⟧∧τ(a) = τ(m)
  \end{equation*}
  Enakost v drugi točki sledi iz \(⟦σₘ = σₙ⟧ = σₙ(m) = ⟦m = n⟧\),
  tretja točka pa sledi iz druge po simetrčnosti relacije.

  Oglejmo si še zadnjo točko. Neenakost \(≥\) očitno sledi iz aksiomov enojcev,
  tako da moramo preveriti zgolj drugo smer:
  \begin{equation*}
    ⟦m = n⟧ = ⟦m = n⟧∧⟦n = m⟧ = σₘ(n)∧σₙ(m) ≤ ⋁_{τ ∈ σ(A)} τ(m)∧τ(n)\qedhere
  \end{equation*}
\end{dokaz}

\begin{definicija}
  \(ℒ\)-množica \(A\) je \emph{polna}, kadar je preslikava \(m ↦ σₘ\) bijekcija.
  Polno podkategorijo polnih \(ℒ\)-množic označimo s \(\cat{CSet}(ℒ)\).
\end{definicija}

\begin{lema}
  \(ℒ\)-množica \(σ(A)\) je polna.
\end{lema}
\begin{dokaz}
  Pokazati moramo, da je preslikava \(τ ↦ σ_τ\) bijekcija.
  % TODO: reword
  Injektivnost:
  Denimo, da velja \(σ_τ = σ_ρ\). Potem velja
  \begin{equation*}
    τ(a) = σ_τ(σₐ) = σ_ρ(σₐ) = ρ(a)\text,
  \end{equation*}
  torej je preslikava injektivna.

  Naj bo sedaj \(Σ ∈ σσ(A)\). Definirajmo \(τ(a) ≔ Σ(σₐ)\).
  To je enojec, saj velja
  \begin{align*}
    τ(a)∧τ(b) &= Σ(σₐ)∧Σ(σ_b) ≤ ⟦σₐ = σ_b⟧ = ⟦a = b⟧\text{ in}\\
    ⟦a = b⟧∧τ(b) &= ⟦a = b⟧∧Σ(σ_b) = ⟦σₐ = σ_b⟧∧Σ(σ_b) ≤ Σ(σₐ) = τ(a)\text.
  \end{align*}

  Pokažimo še, da je \(Σ = σ_τ\).
  \begin{align*}
    σ_τ(ρ) &= ⟦τ = ρ⟧ = ⋁_{a ∈ A} τ(a)∧ρ(a) = ⋁_{a ∈ A} Σ(σₐ)∧⟦ρ = σₐ⟧ ≤ Σ(ρ)\\
    Σ(ρ)
    &= ⟦ρ = ρ⟧∧Σ(ρ) = ⋁_{a ∈ A} ρ(a)∧ρ(a)∧Σ(ρ)\\
    &= ⋁_{a ∈ A} ρ(a)∧⟦ρ = σₐ⟧∧Σ(ρ) ≤ ⋁_{a ∈ A} ρ(a)∧Σ(σₐ)\\
    &= ⋁_{a ∈ A} ρ(a)∧τ(a) = ⟦ρ = τ⟧ = σ_τ(ρ)
  \end{align*}
  Sledi, da za vsak enojec v \(σσ(A)\) obstaja enojec v \(σ(A)\), ki se vanj
  slika, tako da je preslikava \(σ\) res bijekcija, in je \(σ(A)\) polna.
\end{dokaz}

\begin{trditev}
  Naj bo \(f : A ↬ B \) in \(B\) polna \(ℒ\)-množica. Tedaj obstaja preslikava
  \(φ : A → B\), za katero velja \(⟦b = f(a)⟧ = ⟦b = φ(a)⟧\).
  Poleg tega velja tudi \(⟦φ(a) = φ(a')⟧ ≥ ⟦a = a'⟧\), in enakost drži kadar je \(a = a'\).
\end{trditev}
\begin{dokaz}
  Za vsak \(a ∈ A\) je \(b ↦ ⟦b = f(a)⟧\) enojec, torej po polnosti \(B\)
  natanko določa en element \(B\), ki ga označimo \(φ(a)\). To definira
  preslikavo \(φ\).
\end{dokaz}
\begin{posledica}
  Kategorija polnih \(ℒ\)-množic je ekvivalentna kategoriji, katere
  \catdef
    {so polne \(ℒ\)-množice in}
    {\(A → B\) so funkcije \(f : A → B\), ki zadoščajo sledečima pogojema:
      \begin{itemize}
      \item \(⟦a = a'⟧ ≤ ⟦f(a) = f(a')⟧\), in
      \item \(⟦a = a⟧ = ⟦f(a) = f(a)⟧\).
      \end{itemize}}
    %\newline
    %\begin{tabular}{l}
    %  \(⟦a = a'⟧ ≤ ⟦f(a) = f(a')⟧\)\\
    %  \(⟦a = a⟧ = ⟦f(a) = f(a)⟧\)
    %\end{tabular}}
\end{posledica}

\begin{izrek}\label{th:sigmaiso}
  \(ℒ\)-množica \(σ(A)\) je izomorfna \(A\).
\end{izrek}
\begin{dokaz}
  Definirajmo preslikavi \(⟦τ = f(a)⟧ = τ(a)\) in \(⟦a = g(τ)⟧ = τ(a)\).
  Očitno sta morfizma skladna z relacijama, tako da moramo preveriti zgolj
  enoličnost in celovitost. Enoličnost sledi iz neenakosti
  \(τ(a)∧ρ(a) ≤ ⟦τ = ρ⟧\) in \({τ(a)∧τ(b) ≤ ⟦a = b⟧}\).
  Celovitost sledi iz enakosti
  \[ ⟦a = a⟧ = ⟦σₐ = σₐ⟧ = ⋁_{τ ∈ σ(A)} ⟦σₐ = τ⟧ = ⋁_{τ ∈ σ(A)} τ(a)\text{ in} \]
  \[ ⟦τ = τ⟧ = ⋁_{a ∈ A} τ(a)\text.\]

  Preverimo, da sta oba kompozituma enaka identiteti:
  \begin{align*}
    ⟦τ = fg(ρ)⟧ &= ⋁_{a ∈ A} ⟦τ = f(a)⟧∧⟦a = g(ρ)⟧ = ⋁_{a ∈ A} τ(a)∧ρ(a) = ⟦τ = ρ⟧\\
    ⟦a = gf(b)⟧ &= ⋁_{τ ∈ σ(A)} ⟦a = g(τ)⟧∧⟦τ = f(b)⟧ = ⋁_{τ ∈ σ(A)} τ(a)∧τ(b) = ⟦a = b⟧\qedhere
  \end{align*}
\end{dokaz}
\begin{posledica}
  Kategorija \(ℒ\)-množic je ekvivalentna kategoriji polnih \(ℒ\)-podmnožic.
\end{posledica}

\begin{lema}
  Polne \(ℒ\)-množice imajo operator zožitve, t.~j. za vsak \(m ∈ A\) in
  \(U ∈ ℒ\) obstaja \(m\res U ∈ A\), tako da velja \(⟦a = m\res U⟧ = ⟦a = m⟧∧U\).
  V posebnem torej tudi velja \(‖m\res U‖ = ‖m‖∧U\) in \(m\res U \res V = m\res{U∧V}\).
\end{lema}
\begin{dokaz}
  Tvorimo kar \(τ(a) = ⟦a = m⟧∧U\). To je enojec, saj
  \begin{align*}
    τ(a)∧τ(b) &= ⟦a = m⟧∧U∧⟦b = m⟧∧U ≤ ⟦a = b⟧∧U\text{ in}\\
    ⟦a = b⟧∧τ(b) &= ⟦a = b⟧∧⟦b = m⟧∧U ≤ ⟦a = m⟧∧U = τ(a)\text.
  \end{align*}
  Ker je \(A\) polna, ta ustreza nekemu elementu, ki mu lahko rečemo
  \(m\res U\), zanj pa očitno velja željena enakost
  \begin{align*}
    ⟦a = m\res U⟧ = ⟦σₐ = τ⟧ = τ(a) = ⟦a = m⟧∧U\text.\qedhere
  \end{align*}
\end{dokaz}
\begin{posledica}
  Za \(m ∈ A\) in \(U ∈ ℒ\) velja \(σₘ\res U(n) = σₘ(n)∧U\).
\end{posledica}
\begin{posledica}
  Za \(τ ∈ σ(A)\) in \(U ∈ ℒ\) velja \(σ_τ\res U = σ_{τ\res U}\).
\end{posledica}

\begin{lema}
  Polne \(ℒ\)-množice imajo lepljenje skladnih družin, t.~j. za vsako družino
  \(mᵢ ∈ A\), za katero velja \(mᵢ\res{Uⱼ} = mⱼ\res{Uᵢ}\), kjer je \(Uᵢ ≔ ‖mᵢ‖\),
  obstaja tak \(m ∈ A\), da je \(m\res{Uᵢ} = mᵢ\).
  %\(mᵢ ∈ A\) za katero velja \(⟦mᵢ = mⱼ⟧ = ‖mᵢ‖∧‖mⱼ‖\) obstaja nek \(m ∈ A\),
  %tako da je \(⟦m = mᵢ⟧ = ‖mᵢ‖\) in \(‖m‖ = ⋁ᵢ‖mᵢ‖\).
\end{lema}
\begin{dokaz}
  Naj bodo \(mᵢ\) taki in tvorimo preslikavo \(τ(a) = ⋁ᵢ⟦a = mᵢ⟧\).
  Iz skladnosti družine sledi, da velja
  \[ ⟦a = mᵢ⟧∧Uⱼ = ⟦a = mᵢ\res{Uⱼ}⟧ = ⟦a = mⱼ\res{Uᵢ}⟧ = ⟦a = mⱼ⟧∧Uᵢ\text. \]
  
  Preslikava \(τ\) je enojec, saj velja
  \begin{align*}
    ⟦a = b⟧∧τ(b)
    &= ⋁ᵢ⟦a = b⟧∧⟦b = mᵢ⟧ ≤ ⋁ᵢ⟦a = mᵢ⟧ = τ(a)\text{ in}\\
    τ(a)∧τ(b)
    &= ⋁ᵢⱼ⟦a = mᵢ⟧∧⟦b = mⱼ⟧\\
    &= ⋁ᵢⱼ⟦a = mᵢ⟧∧Uᵢ∧⟦b = mⱼ⟧∧Uⱼ\\
    &= ⋁ᵢⱼ⟦a = mᵢ⟧∧Uⱼ∧⟦b = mⱼ⟧∧Uᵢ\\
    &= ⋁ᵢ⟦a = mᵢ⟧∧⟦b = mᵢ⟧ ≤ ⟦a = b⟧\text,
  \end{align*}
  torej ustreza nekemu \(m ∈ A\). Po zgornji lemi velja
  \[ ⟦mᵢ = m\res{Uᵢ}⟧ = ⟦mᵢ = m⟧∧Uᵢ = σₘ(mᵢ)∧Uᵢ = ⋁ⱼ⟦mᵢ = mⱼ⟧∧Uᵢ = Uᵢ\text, \]
  torej se ujemata na celotnem razponu in sta enaka.

  Razpon \(m\) je \(‖m‖ = σₘ(m) = ⋁ᵢ⟦m = mᵢ⟧ = ⋁ᵢUᵢ\), kar mora veljati za vsako
  amalgamacijo družine \(mᵢ\), tako da je \(m\) tudi enoličen.
\end{dokaz}

\begin{konstrukcija}
  Naj bo \(A\) \(ℒ\)-množica. Potem ji lahko priredimo snop
  \[ Θ(A)(U) = \set{τ ∈ σ(A)}{‖τ‖ = U}\text. \]
  Morfizmu \(f : A ↬ B\) \(ℒ\)-množic priredimo preslikavo
  \[ Θ(f)_U(τ)(b) ≔ ρ(b) = ⋁_{a ∈ A}⟦b = f(a)⟧∧τ(a)\text. \]
  To je res enojec na \(B\), saj velja
  \begin{align*}
    ⟦b' = b⟧∧ρ(b)
    &= ⋁_{a ∈ A}⟦b' = b⟧∧⟦b = f(a)⟧ ≤ ⋁_{a ∈ A}⟦b' = f(a)⟧ = ρ(b')\text{ in}\\
    ρ(b)∧ρ(b')
    &= ⋁_{a,a' ∈ A} ⟦b = f(a)⟧∧τ(a)∧τ(a')∧⟦b' = f(a')⟧\\
    &≤ ⋁_{a,a' ∈ A} ⟦b = f(a)⟧∧⟦a = a'⟧∧⟦b' = f(a')⟧\\
    &≤ ⋁_{a' ∈ A} ⟦b = f(a')⟧∧⟦b' = f(a')⟧ ≤ ⟦b = b'⟧\text.
  \end{align*}
  To definira funktor med kategorijama \(\cat{Set}(ℒ)\) in \(\cat{Sh}(ℒ)\).
\end{konstrukcija}

\begin{konstrukcija}
  Naj bo \(P\) predsnop. Potem mu lahko priredimo \(ℒ\)-množico
  \(Γ(P) ≔ ∑_{U ∈ ℒ} P(U)\) z relacijo
  \[ ⟦\p{U, f} = \p{V, g}⟧ ≔ ⋁\set{W ∈ ℒ}{W≤U∧V \text{ in } f\res W = g\res W}\text. \]
  % TODO: dokaži polnost
  Morfizmu predsnopov \(α : P ⇒ Q\) priredimo preslikavo \(Γ(α)(U, f) ≔ α_U(f)\).
  To definira funktor med kategorijama \(\cat{PSh}(ℒ)\) in \(\cat{Set}(ℒ)\).
\end{konstrukcija}

\begin{trditev}
  Funktor \(Γ\) je levi adjunkt funktorju \(Θ\).
\end{trditev}
\begin{dokaz}
  % Naj bo \(φ : Θ(A) → P\) morfizem predsnopov. Definirati želimo morfizem
  % \mbox{\(ℒ\)-množic} \(A ↬ Γ(P)\).
  % Elementoma \(a ∈ A\) z \(‖a‖ = U\) in \(\p{V, f} ∈ Γ(P)\) lahko priredimo
  % \(⟦f = φ_U(σₐ)⟧\).
  % To je očitno morfizem \(ℒ\)-množic.
  
  % Obratno, naj bo \(φ : A ↬ Γ(P)\) morfizem \(ℒ\)-množic.
  % Tedaj elementu \(τ ∈ Θ(A)(U)\) priredimo element \(\).
  % To spet očitno definira morfizem snopov, tako da sta funktorja \(Θ\) in \(Γ\)
  % res adjungirana in definirata ekvivalenco željenih kategorij.
  % % TODO: correct this detail
  % Naj bo \(φ : P ⇒ Θ(A)\). Potem \(⟦a = f\p{U, p}⟧ ≔ φ_U(p)(a)\) definira
  % morfizem \(ℒ\)-množic \(Γ(P) ↬ A\), saj velja \(φ_U(p)(a) = ⟦σₐ = φ_U(p)⟧\).
 
  % Obratno, naj bo \(f : Γ(P) ↬ A\). Morfizem predsnopov \(φ : P ⇒ Θ(A)\)
  % konstruiramo kot \(φ_U(p) ≔ ⟦- = f\p{U, p}⟧\), ki je enojec na \(A\).
  % Razpon tega enojca je pa \(⋁_{a ∈ A} ⟦a = f\p{U, p}⟧ = ‖\p{U,p}‖ = U\).

  % Ti prireditvi sta si očitno inverzni, tako da definirata adjunkcijo \(Γ ⊣ Θ\).
  Koenota adjunkcije je izomorfizem iz izreka \ref{th:sigmaiso},
  enota pa slika \(p ∈ P(U)\) v \(σ_{(U,p)}\). To je naravna
  transformacija, saj \(σ\) \quot{komutira} z zožitvami.
\end{dokaz}

\begin{izrek}
  Kategorija polnih \(ℒ\)-množic je ekvivalentna kategoriji snopov nad \(ℒ\).
\end{izrek}
\begin{dokaz}
  Če funktorja \(Γ\) in \(Θ\) zožimo na dani kategoriji, sta si spet
  adjungirana, tako da moramo zgolj pokazati, da sta enota in koenota
  izomorfizma.

  Koenota je že izomorfizem po \ref{th:sigmaiso}, tako da moramo preveriti zgolj
  enoto.

  Naj bo \(F\) snop. Potem \(η_U : F(U) → ΘΓ(F)(U)\) slika \(f ↦ σ_{(U,f)}\).
  Ker je \(F\) snop je \(ℒ\)-množica \(Γ(F)\) že polna, tako da so elementi
  \(ΘΓ(F)(U)\) že natanko elementi \(Γ(F)\) z razponom \(U\), ki so pa natanko
  elementi \(F(U)\), torej je tudi enota izomorfizem.
\end{dokaz}

%%% Local Variables:
%%% TeX-master: "main"
%%% End:
