\section{Snopi nad okolišem}

Naj \(X, Y, …\) označujejo okoliše.



\begin{definicija}
  \emph{Predsnop nad okolišem \(X\)} je funktor \(F : \op{\O{\p{X}}} → \setcat\).
  Elementom množic \(F(U)\) pravimo \emph{prerezi (pri U)}.
\end{definicija}

\begin{opomba}
  Predsnope lahko definiramo tudi nad poljubno kategorijo \(\cat C\).
  V tem primeru pišemo \(\psh{\cat C} ≔ \op{\cat C} → \setcat\), za predsnope nad okolišem
  \(X\) pa raje spustimo \(\mc{O}\) in pišemo kar \(\psh{X}\).
\end{opomba}

Preden lahko definiramo snope, moramo nekaj povedati o družinah prerezov.

\begin{definicija}
  Prereza \(s ∈ F(U)\) in \(t ∈ F(V)\) sta \emph{skladna}, ko velja \(s\res{V} = t\res{U}\).
  Družina prerezov \(sᵢ ∈ F(Uᵢ)\) je \emph{skladna}, ko so njeni elementi paroma
  skladni. Prerez je \emph{skladen} z družino, ko je paroma skladen z vsemi
  njenimi elementi.
\end{definicija}

\begin{definicija}
  \emph{Snop nad okolišem} je predsnop, za katerega velja:
  \begin{enumerate}
  \item Za vsako skladno družino prerezov \(sᵢ ∈ F(Uᵢ)\) obstaja enoličen prerez
    \({s ∈ F\p{⋃ᵢUᵢ}}\), ki je skladen z družino.
  \item Za vsako pokritje \(Uᵢ \circlearrowright U\) in skladno družino prerezov \(sᵢ ∈ F(Uᵢ)\) obstaja enoličen prerez
    \(s ∈ F(U)\), ki je skladen z družino.
  \item Za vsako skladno družino prerezov \(sᵢ\) na \(Uᵢ\) obstaja enoličen prerez
    \(s\) na \(⋃ᵢUᵢ\), ki je skladen z družino.
  % \item Za vsako skladno družino prerezov \(sᵢ : Uᵢ\) obstaja enoličen prerez
  %   \(s : ⋃ᵢUᵢ\), ki je skladen z družino.
  \item Za vsako skladno družino prerezov \(Uᵢ ⊩ sᵢ\) obstaja enoličen prerez
    \({⋃ᵢUᵢ ⊩ s}\), ki je skladen z družino.
  \item Za vsako skladno družino prerezov \(Uᵢ ⊩ sᵢ ∈ F\) obstaja enoličen prerez
    \({⋃ᵢUᵢ ⊩ s ∈ F}\), ki je skladen z družino.
  % \item Za vsak par prerezov \(s, t ∈ F(U)\), je \(s = t\) natanko tedaj, ko za
  %   pokritje \(Uᵢ \circlearrowright U\) velja,
  \end{enumerate}
\end{definicija}

\begin{definicija} %TODO: popravi ta vsaj unija biznis
  \emph{Amalgamacija} družine prerezov je vsak prerez, ki je definiran vsaj na
  uniji družine in je z njo skladen.
\end{definicija}

\begin{opomba}
  Če družina ni skladna očitno nima amalgamacije.
\end{opomba}
\begin{slogan}
  Vsaka skladna družina prerezov v snopu ima enolično amalgamacijo.
\end{slogan}
\begin{slogan}
  Predsnop je snop, kadar ima vsaka skladna družina prerezov enolično amalgamacijo.
\end{slogan}

\begin{definicija}
  Preslikava med (pred)snopoma \(F\) in \(G\) je naravna transformacija med
  njima kot funktorjema.
\end{definicija}

\begin{definicija}
  Kategoriji vseh snopov na \(X\), skupaj z morfizmi med njimi pravimo \(\sh X\).
\end{definicija}



%%% Local Variables:
%%% TeX-master: "main"
%%% End:
