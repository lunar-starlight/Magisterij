% LTeX: enabled=false

\usepackage[normalem]{ulem}

\newpf{protislovje}{\lightning}
\newenvironment{slogan}[1][Slogan]{
  \noindent\textbf{#1:}
  \begin{quote}
}{
  \end{quote}
}

\renewcommand{\hat}{\widehat}
\renewcommand{\tilde}{\widetilde}
\let\oldbar\bar
\renewcommand{\bar}{\overline}
\newcommand{\contradiction}{\,\,\lightning}

\DeclarePairedDelimiterX{\parens}[1]{(}{)}{#1}
\DeclarePairedDelimiterX{\absolute}[1]{\lvert}{\rvert}{#1}
\def\p{\parens*}
\def\abs{\absolute*}
\newcommand{\cli}[1]{\left[ {#1} \right]}
\newcommand{\floor}[1]{\left\lfloor {#1} \right\rfloor}
\newcommand{\set}[2]{\left\{ #1 \mid #2 \right\}}
\newcommand{\apart}{\ensuremath{\mathrel{\#}}}
\newcommand{\nbd}{\oset{\text{\clap{\tiny nbd}}}{∋}}
\newcommand{\quot}[2]{{#1}/_{\!#2}}
\newcommand{\mb}[1]{\mathbf{#1}}
\newcommand{\bb}[1]{\mathbb{#1}}
\newcommand{\mf}[1]{\mathfrak{#1}}
\renewcommand{\mc}[1]{\mathcal{#1}}
\newcommand{\cat}[1]{\mathbf{#1}}
\newcommand{\opcat}[1]{\left(\mathbf{#1}\right)^{op}}
%\newcommand{\op}[1]{{#1}^{op}}
\newcommand{\op}{\oset{\text{op}}{⊆}}
\newcommand{\res}[1]{↾_{\hspace{-0.15em}#1}}
% \newcommand{\res}[2]{#1{↾_{\hspace{-0.15em}#2}}}
\newcommand{\sh}[1]{\textrm{Sh}{\left( #1 \right)}}
%\newcommand{\psh}[1]{\textrm{PSh}{\left( #1 \right)}}
\newcommand{\psh}[1]{\hat{#1}}
\DeclareMathOperator{\im}{im}
\DeclareMathOperator{\coker}{coker}
\DeclareMathOperator{\coim}{coim}
\DeclareMathOperator{\id}{id}
\DeclareMathOperator{\codim}{codim}
\DeclareMathOperator{\cl}{cl}
\DeclareMathOperator{\ext}{ext}
\DeclareMathOperator{\dom}{dom}
\DeclareMathOperator{\ev}{ev}
\DeclareMathOperator{\dm}{DM}
%\DeclareMathOperator{\cov}{Cov}
\AtBeginDocument{
  %\def\c#1{\left( {#1} \right)^c}
  %\def\c#1{{#1}^c}
  \def\c{\uline}
  \newcommand{\g}[1]{\left\langle {#1} \right\rangle}
  \renewcommand{\b}[1]{\left\{ {#1} \right\}}
  \renewcommand{\i}[1]{\left⟦ {#1} \right⟧}
  \newcommand{\e}[1]{\left‖ {#1} \right‖}
  \newcommand{\s}[1]{\left\{ {#1} \right\}}
  \renewcommand{\O}[1]{\mathcal{O}{#1}}
  \renewcommand{\int}{\textrm{int}}
}

\usepackage{graphicx}
\newcommand{\covsymb}{\raisebox{\depth}{\rotatebox[origin=c]{180}{\circlearrowright}}\,}
%\newcommand{\cov}[2]{#1_{#2} \covsymb #1}
\newcommand{\cov}[3][]{#2_{#1#3} \covsymb #2_{#1}}

\newcommand{\setcat}{\cat{Set}}

\usepackage{xstring}
\newcommand{\mat}[1]{\begin{matrix} #1 \end{matrix}}
\newcommand{\pmat}[1]{\left(\mat{#1}\right)}
\newcommand{\bmat}[1]{\left[\mat{#1}\right]}

\makeatletter
\newcommand{\oset}[3][0ex]{%
  \mathrel{\mathop{#3}\limits^{
    \vbox to#1{\kern-2\ex@
    \hbox{$\scriptstyle#2$}\vss}}}}

\newcommand{\listintertext}{\@ifstar\listintertext@\listintertext@@}
\newcommand{\listintertext@}[1]{% \listintertext*{#1}
  \hspace*{-\@totalleftmargin}#1}
\newcommand{\listintertext@@}[1]{% \listintertext{#1}
  \hspace{-\leftmargin}#1}
\makeatother

\newcommand{\defquantifier}[2]{%
  \expandafter\undef\csname #1\endcsname%
  \expandafter\newcommand\csname #1\endcsname[2]{{#2 ##1.}\;##2}%
}
\defquantifier{for}{\forall}
\defquantifier{exist}{\exists}
\defquantifier{unique}{\exists!}
%\defquantifier{globalen}{\exists ᵍ}
\defquantifier{exact}{ι}
\defquantifier{eventually}{\nabla}
\AtBeginDocument{
\defquantifier{sigma}{\Sigma}
\defquantifier{pi}{\Pi}
}
\renewcommand{\check}{ \(\checkmark\)}

\newcommand{\germ}[2]{\textrm{germ}_{#2}#1}
%\usepackage{xparse}
\newcommand{\forces}[2]{#1\;⊩\;#2}
\makeatletter
\NewDocumentCommand{\@defprinciple}{mmm}{%
  \ExpandArgs{c}\NewDocumentCommand{#1}{s}{%
    \IfBooleanTF##1%
    {\textnormal{\sffamily #2}}%
    {\textnormal{\sffamily #3}}%
  }%
  \AtEndPreamble{%
    \pdfstringdefDisableCommands{%
      \expandafter\def\csname #1\endcsname*{#2}%
    }%
  }%
}
\newcommand{\defprinciple}[1]{\@defprinciple{#1}{\MakeUppercase{#1}}{\MakeLowercase{#1}}}
\makeatother
\newcommand{\principle}[1]{\textnormal{\sffamily #1}} % TODO: rename
\defprinciple{lem}
\defprinciple{wlem}
\defprinciple{lpo}
\defprinciple{wlpo}
\AtBeginDocument{  
\undef\mp
\defprinciple{mp}
}
\defprinciple{ks}
\defprinciple{alpo}
\defprinciple{awlpo}
%\defprinciple{allpo}
\defprinciple{amp}
\defprinciple{aks}
\newcommand{\AC}{\principle{AC}}
\newcommand{\IAC}{\principle{IAC}}
\newcommand{\CC}{\principle{CC}}
\newcommand{\CCv}{\principle{CC}^∨}
\newcommand{\DC}{\principle{DC}}
\def\Rm{ℝ_{\text{M}}}
\def\Rd{ℝ_{\text{D}}}
\def\Rc{ℝ_{\text{C}}}
\def\Ncof{ℕ^{\text{cof}}}

\usepackage{tabularx}
\newcolumntype{C}{>{\centering\arraybackslash}X}
\newcommand{\catdef}[3][]{
  \newline
  #1
  \begin{tabularx}{0.9\linewidth}{l X}
  \textbf{\textup{objekti}}  & {#2}\\
  \textbf{\textup{morfizmi}} & {#3}
  \end{tabularx}
}

%\newcommand{\quot}[1]{``#1''}

\newdef{konstrukcija}
\newdef{oznaka}

\makeatletter
\theoremstyle{plain}
\newtheorem*{@retrditev*}{Izrek~\ref{\@ref}}
\newenvironment{retrditev}[2][]{%
  \def\@ref{#2}%
  \begin{@retrditev*}[#1]}
  {\end{@retrditev*}}
\theoremstyle{definition}
\newtheorem*{@redefinicija*}{Definicija~\ref{\@ref}}
\newenvironment{redefinicija}[2][]{%
  \def\@ref{#2}%
  \begin{@redefinicija*}[#1]}
  {\end{@redefinicija*}}
\makeatother

\renewcommand{\ln}{\mathcal{\mathrm{Ln}}}

\newcommand{\angl}[1]{}
%\newcommand{\angl}[1]{(angl.~\textenglish{#1})}

%%% Local Variables:
%%% TeX-master: "main"
%%% End:

