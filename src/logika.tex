\section{Nekonstruktivni principi}\label{sec:logika}

\subsection{Realna števila}\label{sec:logika-reals}

Spomnimo se nekaj definicij realnih števil.
\begin{definicija}
  Podmnožica \(L ⊆ ℚ\) je
  \begin{itemize}
  \item \emph{naseljena}, ko obstaja element \(L\),
  \item \emph{doljna}, ko za vsak \(q ∈ L\) in \(p < q\) velja \(p ∈ L\),
  \item \emph{navzgor odprta}, ko za vsak \(p ∈ L\) obstaja \(q > p\), da je \(q ∈ L\).
  \end{itemize}
  Pojme \emph{gornja} in \emph{navzdol odprta} definiramo simetrično.
\end{definicija}

\begin{definicija}[Dedekindova realna števila]
  Par \(\p{L, U} ∈ 𝒫(ℚ)×𝒫(ℚ)\) je \emph{Dedekindov rez}, ko velja
  \begin{enumerate}
    \item \(L\) je naseljen
    \item \(U\) je naseljen
    \item \(L\) je doljna
    \item \(U\) je gornja
    \item \(L\) je navzgor odprt
    \item \(U\) je navzdol odprt
    \item \(0 ∈ L ∨ 1 ∈ U\)
    \item za vsaka \(p\), \(q : ℚ\) velja \(p ∈ L ∧ q ∈ U ⇒ p < q\)
  \end{enumerate}
  Množica Dedekindovih rezov tvori \emph{Dedekindova realna števila}, ki jih
  označimo z \(\Rd\).
  Za \(x : \Rd\) naj \(Lₓ\) in \(Uₓ\) označujeta podmnožici \(ℚ\), ki tvorita
  dedekindov rez \(x\).

  Za racionalno število \(p\) definiramo \(p < x\) kot \(p ∈ Lₓ\) in \(x < p\)
  kot \(p ∈ Uₓ\). Relacijo \(x < y\) za Dedekindovi realni števili potem
  definiamo kot ``obstaja racionalno število \(p\), da je \(x < p < y\)''.
  Z rezi to izrazimo kot \(U_x \between L_y\).

  Relacijo \(\p{L,U} ≤ \p{L',U'}\) definiramo kot \(L ⊆ L' ∧ U' ⊆ U\).
  Pogoja sta ekvivalentna, tako da zadošča pokazati zgolj eno vsebovanost.
\end{definicija}
\begin{opomba}
  Sedmemu pogoju zgoraj pravimo \emph{lociranost}. Izkaže se, da sta števili
  \(0\) in \(1\) arbitrarni, saj je pogoj ekvivalenten pogoju
  ``za vse \(p < q\) je \(p ∈ L\) ali \(q ∈ U\)''.
  Res, relacija \(<\) je invariantna na afine preslikave realne premice, tako da
  lahko vsak interval \(\p{p,q}\) preslikamo na \(\p{0,1}\) in nazaj.
\end{opomba}

\begin{definicija}[Cauchyjeva realna števila]
  Zaporedje racionalnih števil \((xₙ)ₙ\) je \emph{Cauchyjevo}, ko za vsaka
  \(i,j : ℕ\) velja \(|xᵢ - xⱼ| ≤ 2⁻ⁱ+2⁻ʲ\).

  Dve Cauchyjevi zaporedji predstavljata isto realno število,
  ko za vse \(i : ℕ\) velja \(|xᵢ - yᵢ| ≤ 2⁻ⁱ⁺¹\).

  Množica Cauchyjevih zaporedij kvocientno z gornjo relacijo tvori
  \emph{Cauchyjeva realna števila}, ki jih označimo z \(\Rc\).

  Relacijo \(x < y\) definiramo kot \(\exist{n:ℕ}{xₙ + 2⁻ⁿ < yₙ - 2⁻ⁿ}\),
  relacijo \(x ≤ y\) pa enako kot zgoraj le da \(<\) spremenimo v \(≤\).
\end{definicija}
\begin{opomba}
  Takim zaporedjim ponavadi pravimo hitra Cauchyjeva zaporedja.
  Izkaže se, da so ekvivalentna moduliranim Cauchyjevim
  zaporedjim~\cite{nlab-cauchy-real}, a je definicija tistih bolj komplicirana,
  tako da raje uporabimo to.

  Paziti je pa treba, da klasična \(ε\)-\(δ\) definicija Cauchyjevih zaporedij
  \emph{ni} ekvivalentna gornji. Tem se potem pravi bodisi klasična Cauchyjeva
  zaporedja ali pa Cantorjeva zaporedja~\cite{nlab-cauchy-real}.
  Ker pa v našem delu teh ne potrebujemo, bomo objekt hitrih Cauchyjevih realnih
  števil imenovali kar samo Cauchyjeva.
\end{opomba}

\begin{lema}\label{th:real-order-lemma}
  Relaciji \(<\) in \(≤\) sta delni ureditvi v Dedekindovih in Cauchyjevih
  realnih številih. Poleg tega velja \(x < y ≤ z ⇒ x < z\) in
  \(x ≤ y < z ⇒ x < z\).
\end{lema}

\begin{trditev}
  Vsako Cauchyjevo realno število je tudi Dedekindovo.
\end{trditev}
\begin{dokaz}
  Vzemimo \(L ≔ \set{q : ℚ}{q < x}\) in \(U ≔ \set{r : ℚ}{x < r}\).
  Očitno ta zadoščata prvim šestim pogojem zgoraj, in prav tako zadoščata
  zadnjemu pogoju, tako da se osredotočimo zgolj na zadnji pogoj.

  Pokazati moramo, da za vsako Cauchyjevo realno število \(x\) velja
  \(x > 0 ∨ x < 3\).

  Za racionalno število \(x₀\) velja \(x₀ > 1 ∨ x₀ < 2\), saj imajo racionalna
  števila odločljivo neenakost. Sedaj pa vemo, da za vse \(i : ℕ\) velja
  \(|x₀ - xᵢ| ≤ 1\), torej, če je \(x₀ > 1\), bodo vsi \(xᵢ > 0\), po drugi strani
  pa, če je \(x₀ < 2\), bodo vsi \(xᵢ < 3\), kar pa zaključi dokaz.
\end{dokaz}

Obrat pa ne velja konstruktivno. To pomeni, da lahko na ``\(\Rd = \Rc\)''
gledamo kot nekakšen nekonstruktiven princip. Iz konstruktivne matematike pa
poznamo tudi druge konstrukcije realnih števil, na primer MacNeillovo
konstrukcijo~\cite{nlab-macneille-reals}\cite[pogl.~D4.7]{Johnstone02}, ki nam
lahko da nove principe kot ta zgoraj. Obstajajo še druge konstrukcije, a o njih
ni veliko znano, tako da jih v tem delu ne obravnavamo.

\begin{definicija}[MacNeillova realna števila]
  Par \(\p{L, U} ∈ 𝒫(ℚ)×𝒫(ℚ)\) je \emph{Dedekind-MacNeilleov rez}, ko velja
  \begin{enumerate}
    \item \(L\) je naseljen
    \item \(U\) je naseljen
    \item \(L\) je doljna
    \item \(U\) je gornja
    \item \(L\) je navzgor odprt
    \item \(U\) je navzdol odprt
    \item \(L = \int{\p{Uᶜ}}\)
    \item \(U = \int{\p{Lᶜ}}\)
  \end{enumerate}
  Množica Dedekind-MacNeilleovih rezov tvori \emph{MacNeillova realna števila},
  ki jih označimo z \(\Rm\).

  Ureditev na MacNeillovih realnih številih definiramo enako kot na Dedekindovih
  realnih številih.
\end{definicija}
\begin{opomba}
  Zadnji pogoj je ekvivalenten trditvi \(q ∈ U ⇔ \exist{p<q}{p ∉ L}\)
  (in simetrično velja tudi za predzadnji pogoj).
\end{opomba}

Lema~\ref{th:real-order-lemma} velja tudi za ureditve na MacNeillovih realnih številih.

\begin{trditev}
  Vsako Dedekindovo realno število je tudi MacNeillovo.
\end{trditev}
\begin{dokaz}
  Točke \(1\) skozi \(6\) so enake, tako da je treba pokazati zgolj zadnji dve
  lastnosti. Ker sta simetrični, pokažimo zgolj predzadnjo.

  Množica \(U\) je odprta, in je podmnožica \(Lᶜ\), torej je tudi podmnožica
  njene notranjosti.
  Obratno pa, če je \(p < q\) in \(p ∉ L\), po lociranorsti vemo, da je \(q ∈ U\).
\end{dokaz}

V obratno smer lahko pokažemo zgolj, da sta \(L\) in \(U\) ločena.
\begin{lema}
  Za MacNeillovo realno število \(\p{L,U}\) velja \(a∈L∧b∈U⇒a<b\).
\end{lema}
\begin{dokaz}
  Naj bo \(p∈L\) in \(q∈U\). Potem je \(q ∈ \int{\p{Lᶜ}} ⊆ Lᶜ\), torej ni v
  \(L\). Ker je \(L\) doljna je torej \(q\) zgornja meja, in je večji od \(p\).
\end{dokaz}

Ostalo bi torej pokazati zgolj lociranost vsakega MacNeillovega realnega
števila, a to žal ni mogoče.
Temu principu bomo torej pravili ``\(\Rm = \Rd\)''.

MacNeillova realna števila so zanimiva, saj imajo lepo lastnost, ki pravi, da
ima vsaka naseljena omejena množica MacNeillovih realnih števil supremum.
\begin{lema}\label{th:Rm-sup}
  MacNeillova realna števila so polna.
\end{lema}
\begin{dokaz}
  Naj bo \(S ⊂ \Rm\) naseljena in omejena z \(a : \Rm\).

  Definirajmo \(L ≔ ⋃\set{Lₓ}{x ∈ S}\) in \(U ≔ \int(Lᶜ)\).
  Lastnosti \((1)-(7)\) zanju očitno veljajo.

  Potem je pa \(\int(Uᶜ) = \int(\cl L)\). Ampak \(L\) je navzdol zaprta, torej
  je notranjost njenega zaprtja kar \(L\).
\end{dokaz}

Tako princip \(\Rm = \Rd\) pravi, da ima vsaka naseljena omejena množica
\emph{Dedekindovih} realnih števil supremum!

Poglejmo si bolj podrobno ureditve \(<\) in \(≤\). Sedaj smo definirali vsako na
vsaki od množic realnih števil, vemo pa, da velja \(\Rc ⊆ \Rd ⊆ \Rm\). Izkaže
se, da so si relacije povezane.

\begin{lema}
  Za Cauchyjevi realni števili \(x\) in \(y\) velja \(x <_{\Rc} y ⇔ x <_{\Rd} y\)
  in \(x ≤_{\Rc} y ⇔ x ≤_{\Rd} y\).
  Za Dedekindovi realni števili \(x\) in \(y\) velja \(x <_{\Rd} y ⇔ x <_{\Rm} y\)
  in \(x ≤_{\Rd} y ⇔ x ≤_{\Rm} y\).
\end{lema}

\begin{lema}\label{th:dedekind-real-≤-is-¬>}
  Za Dedekindovi realni števili \(x\) in \(y\) je \(x ≤ y ⇔ ¬(y < x)\).
\end{lema}
\begin{dokaz}
  Očitno \(x ≤ y < x\) vodi v protislovje, torej je smer \(⇒\) dokazana. Naj
  sedaj velja \(¬(y < x)\), torej za noben \(p : ℚ\) ne velja \(y < p < x\).
  To pa pomeni, da je \(U_y⊥L_x\). Če je sedaj \(p ∈ L_x\), mora po odprtosti
  obstajati \(q ∈ L_x\), ki je večji od \(p\). Potem pa po lociranosti \(y\)
  velja \(p ∈ L_y ∨ q ∈ U_y\). Ampak \(q\) ne more biti v \(L_x\) in \(U_y\),
  tako da je \(p ∈ L_y\).
\end{dokaz}

A to žal ne velja za ureditvi na MacNeilleovih realnih številih, kljub temu, da
se na \(\Rd\) ujemajo. Vseeno pa velja implikacija v desno, saj ta sledi iz
leme~\ref{th:real-order-lemma}. Vseeno pa velja sledeče:
\begin{lema}
  Za (poljubni) realni števili \(x\) in \(y\) velja \(x < y ∨ x=y ⇒ x ≤ y\).
\end{lema}
Gornjo lemo je preprosto preveriti. V klasični logiki tu velja tudi obrat
implikacije, a tega ne moremo pokazati konstruktivno.


\subsection{Principi izbire}\label{sec:logika-izbire}

Za princip izbire ste verjetno že slišali. Ta pravi, da za vsaki množici \(A\)
in \(B\), in vsako celovito relacijo \(R\) med njima, obstaja funkcija
\(f : A → B\), tako da velja \(\for{a : A}{R(a, f(a))}\).

Ta princip lahko ošibimo na tri načine: lahko omejimo množici \(A\) in \(B\),
lahko pa omejimo relacijo \(R\), tako da jemlje resničnostne vrednosti zgolj iz
določene množice \(Σ ⊆ Ω\).

\begin{definicija}
  \emph{Princip izbire nad \(Σ\)} je shema, ki za vsaka \(A\) in \(B\) pravi,
  da za vsako relacijo \(R : A×B → Σ\) za katero velja
  \(\for{a:A}{\exist{b:B}{R(a,b)}}\), obstaja funkcija izbire \(f : A → B\),
  tako da velja \(\for{a:A}{R(a,f(a))}\). Objektu \(A\) tako pravimo
  \emph{domena}, objektu \(B\) \emph{kodomena}, objektu \(Σ\) pa
  \emph{Sierpinskijev objekt}.
  Pogoju na \(R\) pravimo \emph{celovitost}.
  To označimo z \(\AC_Σ\). Če je \(Σ = Ω\), indeks opustimo in temu pravimo
  \emph{princip izbire}.
\end{definicija}
\begin{definicija}
  Če v principu izbire fiksiramo domeno na nek \(A\), temu pravimo
  \emph{princip izbire nad \(Σ\) iz \(A\)}, in če fiksiramo še kodomeno na nek
  \(B\) temu pravimo \emph{princip izbire nad \(Σ\) iz \(A\) v \(B\)}. Ta potem
  označujemo \(\AC_Σ(A)\) in \(\AC_Σ(A, B)\). Podobno kot zgoraj opuščamo \(Σ\)
  ko je ta enaka \(Ω\).
\end{definicija}

Hitro lahko opazimo, da so principi izbire neobčutljivi za izomorfizme.
\begin{trditev}
  Naj bo \(φ : A ≅ A'\) in \(ψ : B ≅ B'\). Potem je \(\AC_Σ(A, B)\) ekvivalenten
  \(\AC_Σ(A', B')\).
\end{trditev}
\begin{dokaz}
  Situacija je očitno simetrična, tako da pokažimo zgolj eno implikacijo.
  Če je \({R : A'×B' → Σ}\) celovita, je potem tudi \({R∘\p{φ×ψ} : A×B → Σ}\)
  celovita. Za to relacijo po predpostavki obstaja funkcija izbire, tako da je
  \(\for{x : A}{R∘\p{φ×ψ}{(x, f(x))}}\), kar je pa enako
  \(\for{x:A}{R(φ(x),ψ(f(x)))}\). Ker je \(φ\) izomorfizem, lahko
  kvantifikator preindeksiramo in dobimo \(\for{x:A'}{R(x,ψ∘f∘φ⁻¹(x))}\),
  torej je \(ψ∘f∘φ⁻¹\) funkcija izbire za \(R\).
\end{dokaz}
V teoriji množic to pomeni, da je aksiom izbire določen do kardinalnosti
natančno. To nam tudi utemelji zakaj lahko \(\AC(ℕ)\) pravimo princip
\emph{števne} izbire, saj deluje za poljubno števno (neskončno) množico. Tako si
lahko tudi predstavljamo, kako to shemo imen razširiti na poljubno kardinalnost.
Označimo ga torej \(\CC\). Pomembno vlogo bo igral tudi \emph{princip
  števne disjunktivne izbire} \(\AC(ℕ, 2)\), ki ga označimo z \(\CCv\).

Seveda pa v internem jeziku nismo definirali, kaj ``kardinalnost'' sploh je,
tako da bo vsaka omemba kardinalnosti v tem delu bila v smislu zunanjega sveta.

\begin{trditev}
  Če je \(Σ' ⊆ Σ\), \(\AC_Σ(A, B)\) implicira \(\AC_{Σ'}(A, B)\).
\end{trditev}
\begin{dokaz}
  Vsaka relacija nad \(Σ'\) je tudi relacija nad \(Σ\), tako da če imajo
  relacije nad \(Σ\) funkcije izbire jih imajo tudi relacije nad \(Σ'\).
\end{dokaz}

\begin{trditev}
  Princip končne izbire velja.
\end{trditev}
\begin{dokaz}
  Princip končne izbire je princip izbire, kjer domeno omejimo na končne
  množice. Brez škode za splošnost, naj bo domena kar enaka neki
  standardni končni množici \(\cli n\).

  Potem pa \(\AC(\cli n)\) pravi, da če
  obstajajo \(bₖ:B\), da velja \(R(1,b₁)∧\dots ∧R(n,bₙ)\), obstaja končno
  zaporedje (torej, obstajajo \(bₖ:B\)), da velja \(R(1,b₁)∧\dots ∧R(n,bₙ)\)… To
  sta pa isti stvari, tako da princip res drži.
\end{dokaz}

% \begin{definicija}
%   \emph{Princip \[Σ\]-izbire iz \(A\) v \(B\)} pravi, da za vsako relacijo
%   \(R : A×B → Σ\) za katero velja \(\for{a:A}{\exist{b:B}{R(a,b)}}\), obstaja
%   funkcija izbire \(f : A → B\), tako da velja \(\for{a:A}{R(a,f(a))}\).

%   To bomo označili z \(\AC_Σ(A,B)\). Z \(\AC_Σ(A)\) bomo označevali shemo
%   \(\for{B}{\AC_Σ(A,B)}\), z \(\AC_Σ\) pa shemo \(\for{A}{\AC_Σ(A)}\).
%   Tem pravimo \emph{princip \(Σ\)-izbire iz \(A\)} in \emph{princip \(Σ\)-izbire}.
%   Kadar je \(Σ = Ω\) bomo \(Σ\) v indeksu opuščali, in ustrezno to poimenovali
%   kar samo \emph{princip izbire (iz \(A\) v \(B\))}.

%   Standardno se principu \(\AC(ℕ)\) pravi \emph{princip števne izbire} in
%   označuje \(\CC\).
%   Posebno pomemben bo princip \(\AC(ℕ, 2)\), ki ga bomo označevali \(\CCv\) in
%   mu pravili \emph{princip števne dvojiške(disjunktivne?) izbire}.
% \end{definicija}



% \begin{definicija}
%   \emph{Princip števne dvojiške izbire} pravi, da če velja
%   \(\for{n : ℕ}{P(n, 0) ∨ P(n, 1)}\) (torej, \(P\) je celovita relacija na
%   \(ℕ×2\)) obstaja funkcija izbire \(f : ℕ → 2\), da velja
%   \(\for{n : ℕ}{P(n,f(n))}\).
% \end{definicija}

Poznamo pa tudi še princip odvisne izbire.
\begin{definicija}
  \emph{Princip odvisne izbire nad \(Σ\) za \(A\)} pravi, da za vsako celovito
  relacijo \(R : A×A → Σ\) in \(a₀ : A\) obstaja zaporedje \((aᵢ)ᵢ\) začenši z
  \(a₀\), tako da za vsak \(n : ℕ\) velja \(R(aₙ, aₙ₊₁)\). Tega označimo z
  \(\DC_Σ(A)\). Če princip velja za vse \(A\) ga označimo \(\DC_Σ\).
\end{definicija}

\begin{trditev}
  Če je \(Σ⊆Ω\) zaprt za končne konjunkcije in števne disjunkcije, velja
  implikacija \(\DC_Σ ⇒ \CC_Σ\).
\end{trditev}
\begin{dokaz}
  Naj bo \(R : ℕ×B → Σ\) celovita.
  Definirajmo \(X ≔ \set{b:B}{\exist{n:ℕ}{R(n,b)}}\) in na \(X×X\) relacijo
  \(Q(x,y) ≔ \exist{n:ℕ}{R(n,x)∧R(n+1,y)}\).

  Ta je celovita, tako da lahko na njej uporabimo \(\DC_Σ\), torej dobimo
  zaporedje \(x : ℕ → X\). Ker je \(X⊑B\) je torej to preslikava \(ℕ → B\), ki
  ima želeno lastnost.
\end{dokaz}

V dokazu je pogoj na Sierpinskijevem objektu res pomemben, sicer \(Q\) ni
relacija nad \(Σ\).

Poznamo tudi princip enolične izbire. Ta pravi, da vsaka funkcijska relacija
določa (enolično) funkcijo. Ampak v topoloških modelih smo funkcije definirali
natanko kot funkcijske relacije, tako da v naših modelih ta princip vedno velja,
tako da ga ne bomo posebej obravnavali.


\subsection{Principi odločitve}\label{sec:logika-odločitve}

\begin{definicija}[Izključena tretja možnost]\label{pr:lem}
  \emph{Princip izključene tretje možnosti} pravi, da za vsako resničnostno
  vrednost \(p\) velja \(p∨¬p\). Formulo \(p∨¬p\) označimo \(\lem(p)\), formulo
  \(\for{p:Ω}{p∨¬p}\) pa z \(\lem*\).
\end{definicija}

\begin{definicija}[Šibka izključena tretja možnost]\label{pr:wlem}
  \emph{Princip šibke izključene tretje možnosti} pravi, da za vsako
  resničnostno vrednost \(p\) velja \(¬p∨¬¬p\). Formulo \(¬p∨¬¬p\) označimo
  \(\wlem(p)\), formulo \(\for{p:Ω}{¬p∨¬¬p}\) pa z \(\wlem*\).
\end{definicija}
\begin{trditev}
  Velja implikacija \(\lem* ⇒ \wlem*\).
\end{trditev}
\begin{dokaz}
  Formula \(\wlem(p)\) je natanko \(\lem(¬p)\), tako da trditev očitno velja.
\end{dokaz}

\begin{definicija}\label{pr:lpo}
  \emph{Princip števne odločitve} pravi, da za vsako števno zaporedje ničel in enic
  lahko odločimo, ali je celo nič, ali pa obstaja mesto, na katerem je enica.
  Formulo \(α = 0∨α\apart 0\) označimo \(\lpo(α)\), formulo
  \(\for{α : 2^ℕ}{\lpo(α)}\) pa z \(\lpo*\).
\end{definicija}

%TODO: tega verjetno ne rabim?
% \begin{definicija}\label{pr:wlpo}
%   \emph{Princip šibke števne odločitve} pravi, da za vsako števno zaporedje
%   ničel in enic lahko odločimo, ali je celo nič, ali ni.
%   Formulo \(α = 0∨α ≠ 0\) označimo \(\wlpo(α)\), formulo
%   \(\for{a:2^ℕ}{\wlpo(α)}\) pa z \(\wlpo*\)
% \end{definicija}

\begin{definicija}\label{pr:alpo}
  Naj bo \(ℝ\) nek objekt realnih števil (bodisi Dedekindova, Cauchyjeva, itd.).
  \emph{Analitični princip števne odločitve za \(ℝ\)} pravi, da za vsako realno
  število lahko odločimo, ali je pozitivno ali nenegativno. Formulo
  \(x > 0 ∨ x ≤ 0\) označimo \(\alpo_ℝ(x)\), formulo \(\for{x : ℝ}{\alpo_ℝ(x)}\) pa
  z \(\alpo*_ℝ\).

  Če je \(ℝ = \Rd\), indeks opustimo in pravimo le \emph{analitični princip
    števne odločitve}.

  Podobno definiramo \emph{analitični princip šibke števne odločitve za \(ℝ\)},
  ki ga označimo \(\awlpo*_ℝ\), s formulo \(\awlpo_ℝ(x) = x≤0 ∨ ¬(x≤0)\).
\end{definicija}

\begin{trditev}\label{th:alpo-equiv}
  Ekvivalentno lahko definiramo \(\alpo*_ℝ\) kot \(\for{x:ℝ}{x = 0 ∨ x \apart 0}\) in
  \(\awlpo*_ℝ\) kot \(\for{x:ℝ}{x = 0 ∨ ¬(x=0)}\).
\end{trditev}
\begin{dokaz}
  Če uporabimo \(\alpo*_ℝ\) na \(x\) in \(-x\) dobimo želeno formulo. Obratno pa
  \({x = 0 ∨ x < 0}\) implicira \(x ≤ 0\). Podobno pokažemo tudi drugi del.
\end{dokaz}
Tako bomo v dokazih prosto menjali med ekvivalentnima pogojema.


\begin{trditev}\label{th:alpoc-is-lpo}\label{th:implications}
  Velja veriga implikacij \(\lem* ⇒ \alpo* ⇒ \alpo*_{\Rc} ⇔ \lpo*\).
\end{trditev}
\begin{dokaz}
  Ker je \(¬(x > 0)\) natanko \(x ≤ 0\), je \(\alpo*\) očitno posledica
  \(\lem*\). Prav tako je vsako Cauchyjevo realno število tudi Dedekindovo, tako
  da druga implikacija tudi velja.

  Za zadnjo ekvivalenco pa potrebujemo malo dela. Naj bo \(α\) zaporedje, za
  katerega odločamo, ali ima na kakem mestu enico. Definirajmo realno število
  \[ x ≔ \lim_{n → ∞}2^{-\min\set{k : ℕ}{α(k) = 1 ∨ k = n}}\text. \]

  To je po definiciji Cauchyjevo realno število. Uporabimo sedaj predpostavko
  \(\alpo(x)\). Če velja \(x ≤ 0\), mora zaporedje limitirati proti \(0\), kar
  pa pomeni, da \(α(k) = 1\) ni nikoli zadoščeno, torej je \(α = 0\).
  Po drugi strani, če je \(x > 0\) pa obstaja nek \(k : ℕ\), tako da je
  \(x > 2⁻ᵏ\). To pa pomeni, da ima \(α\) enico na enem izmed prvih \(k\)
  mestih, kar pomeni, da želeni indeks obstaja.

  Preostanek dokaza izvira iz~\cite{Gro-Tsen24}.
  Obratno, naj bo \(x = (xᵢ)ᵢ\) Cauchyjevo zaporedje.
  Definiramo lahko zaporedje
  \[ β(k,n) ≔
    \begin{cases}
      1 &; \for  {m≥n}{xₘ ≤ 2⁻ᵏ}\\
      0 &; \exist{m≥n}{xₘ > 2⁻ⁿ}\text.
    \end{cases} \]
  Vse te odločitve lahko naredimo, saj velja \(\lpo*\) in je neenakost med
  racionalnimi števili odločljiva. Sedaj uporabimo \(\lpo*\) na \(n↦β(k,n)\).

  Podobno definiramo
  \[ α(k) ≔
    \begin{cases}
      1 &; \for  {n:ℕ}{β(k,n) = 0}\\
      0 &; \exist{n:ℕ}{β(k,n) = 1}\text.
    \end{cases} \]
  Nazadnje uporabimo \(\lpo*\) na \(α\). Če je \(α = 0\), imamo za vsak
  \(k\) nek \(n\), da je za vsak \(m≥n\) \(xₘ ≤ 2⁻ᵏ\). To pa pomeni, da je
  \(x = 0\). Po drugi strani pa, če je \(α(k) = 1\) za nek \(k\) pa velja, da za
  vsak \(n\) obstaja \(m≥n\), da je \(xₘ > 2⁻ᵏ\), torej je \(x > 2⁻ᵏ\) in je
  \(x \apart 0\).
\end{dokaz}

To pomeni, da je analitični princip števne odločitve res smiselen zgolj za
Dedekindova realna števila. Od tu naprej prosto menjamo med \(\alpo*_{\Rc}\) in
\(\lpo*\) po potrebi, brez posebne omembe. Prav tako bomo raje pisali \(\lpo*\)
in \(\lpo\), kjer ni dvoumno.

\subsection{Ostali principi}\label{sec:logika-ostalo}

Seveda pa se principi ne delijo popolnoma zgolj na principe izbire in odločitve.
Nekaj takih imamo zgoraj o realnih številih, nekaj jih bomo pa še definirali.

\begin{definicija}\label{pr:mp}
  \emph{Princip Markova} je \(\for{α:2^ℕ}{¬(α=0) ⇒ α \apart 0}\). Kot ponavadi
  označimo z \(\mp(α)\) notranjo formulo in z \(\mp*\) kvantificiran izraz.

  Podobno kot v~\ref{pr:alpo} definiramo analitične različice principa.
\end{definicija}
Ta je zožitev principa eliminacije dvojne negacije, \(\for{p:Ω}{¬¬p ⇒ p}\), na
Sierpinskijev objekt \(Σ₀¹\). Za tega vemo, da je ekvivalenten izključeni tretji
možnosti, a je \(\mp*\) vseeno šibkejši od \(\lpo*\).
\begin{dokaz}
  Če velja \(α = 0 ∨ α \apart 0\), in ne velja \(α=0\), potem velja \(α \apart 0\).
\end{dokaz}

\subsubsection{Redukcija instanc}

Oglejmo si najprej razne Sierpinskijeve objekte, ki smo jih do sedaj definirali.
\begin{align*}
  Ω   &= \set{p}{p : Ω}\\
  Σ_ℝ &= \set{x > 0}{x : ℝ}\\
  Σ₀¹ &= \set{α \apart 0}{α : 2^ℕ}\\
  Δ   &= \set{p : Ω}{p ∨ ¬p}\\
  R   &= \set{p : Ω}{¬¬p ⇒ p}
\end{align*}
Tu nam \(Δ\) predstavlja Sierpinskijev objekt \emph{odločljivih} resničnostnih
vrednosti, \(R\) pa \emph{regularnih}.

\begin{trditev}
  Naj bo \(ℝ\) objekt realnih števil. Potem je Sierpinskijev objekt \(Σ_ℝ\) enak
  \(\set{x \apart 0}{x : ℝ}\).
\end{trditev}
\begin{dokaz}
  TODO: a je to res, in če je, a je pol velja \ref{th:alpo-equiv} kot redukcija?

  Če je \(x : ℝ\), je potem \(x \apart 0 ⇔ \abs x > 0\).
  Obratno pa, je \(x > 0 ⇔ \max\{x,0\} \apart 0\).
\end{dokaz}
Prav tako bomo potem v dokazih prosto menjavali med \(x > 0\) in \(x \apart 0\).

\begin{definicija}
  Vse logične principe, kvantificirane po \(Ω\), lahko razširimo tako, da domeno
  kvantifikacije omejimo na nek Sierpinskijev objekt. Specifično je torej
  \(\lem*{\res Σ} ≔ \for{p∈Σ}{\lem(p)}\), itd.
\end{definicija}

\begin{trditev}
  Princip \(\lem*{\res Δ}\) velja.
\end{trditev}

Izkaže se, da lahko principe odločitve izrazimo kot princip izključene tretje
možnosti, omejene na te Sierpinskijeve objekte.
\begin{trditev}
  Imamo \(\alpo* = \lem*{\res{Σ_{\Rd}}}\), \(\alpo*_{\Rc} = \lem*{\res{Σ_{\Rc}}}\), in
  \(\lpo* = \lem*{\res{Σ₀¹}}\).
\end{trditev}
\begin{dokaz}
  Ker je \(¬(x > 0) ⇔ x ≤ 0\), in \(¬(α\apart 0) ⇔ α = 0\), enakosti sledijo.
\end{dokaz}
Podobno lahko šibke verzije gornjih principov dobimo kot zožitve \(\wlem*\) na
vsakega od teh objektov. To je tudi razlog, zakaj tej skupini principov pravimo
``principi odločitve''.

Ampak mi smo pa tudi pokazali, da je \(\alpo*_{\Rc} ⇔ \lpo*\). Ali to pomeni, da
je \(Σ_{\Rc} = Σ₀¹\)? Namreč, v dokazu~\ref{th:alpoc-is-lpo} skonsturiramo taka
\(x : \Rc\) in \(α : 2^ℕ\), da je \(x > 0\) natanko tedaj, ko je \(α \apart 0\).

A vendar se izkaže, da velja zgolj \(Σ₀¹ ⊆ Σ_{\Rc}\). Če pogledamo dokaz, v to
smer brez predpostavk skonstruiramo \(x : \Rc\), da velja gornje, ampak v
obratno smer pa bistveno uporabimo več (celo neskončno) \emph{instanc} \(\lpo*\)
v konstrukciji zaporedja \(α\).

Izkaže se, da lahko matematično natančno opišemo gornji pojav~\cite{Bauer22}.
Ideja pride iz teorije izračunljivosti, specifično iz Weihrauchovih redukcij.
Tam skrbno pazimo, kolikokrat se uporabi posamezne predpostavke.

\begin{definicija}
  Predikat \(φ ⊑ A\) je \emph{reducibilen} na predikat \(ψ ⊑ B\), ko velja
  \[ \for{x:A}{\exist{y:B}{ψ(y) ⇒ φ(x)}}\text. \label{eq:inst-red} \]
  To označimo z \(\p{φ, A} ≤ \p{ψ, B}\), ali kar z \(φ ≤ ψ\), ko so domene
  predikatov razvidne iz konteksta.
\end{definicija}

Reducibilnost nam torej pove, da lahko vsako instanco \(\for{x:A}{φ(x)}\)
(torej, vsak \(φ(a)\)) dokažemo tako, da poiščemo nek \(b:B\), da bo \(ψ(b)\)
dovolj močen, da lahko pokaže \(φ(a)\). Pomembno tu je, da lahko poiščemo samo
en \(b\), torej lahko uporabimo samo eno instanco \(\for{y:b}{ψ(b)}\).

Tako vsi principi odločitve in izbire postanejo instance, saj so vsi univerzalno
kvantificirani. Prav tako, če pogledamo dokaz iz gornjega razdelka vidimo, da
dobimo redukcije \(\lpo* ≤ \alpo*_{\Rc} ≤ \alpo* ≤ \lem*\) in \(\wlem* ≤ \lem*\).

Redukcije \(\p{φ, A} ≤ \p{φ, B}\), ko je \(A ⊆ B\) so očitne, tako da poglejmo
le ta zanimivo redukcijo \(\lpo* ≤ \alpo*_{\Rc}\).

\begin{trditev}
  Velja redukcija \(\lpo* ≤ \alpo*_{\Rc}\).
\end{trditev}
\begin{dokaz}
  Naj bo \(α:2^ℕ\). Definirajmo realno število
  \[ x ≔ \lim_{n → ∞}2^{-\min\set{k : ℕ}{α(k) = 1 ∨ k = n}}\text. \]

  Kot smo zgoraj pokazali velja \(x \apart 0 ⇔ α \apart 0\), torej redukcija
  velja.
\end{dokaz}

Reducibilnost nam torej definira refleksivno in tranzitivno relacijo. To lahko
dopolnimo do ekvivalenčne relacije \(≡\), ki ji pravimo \emph{ekvivalenca instanc}.
Ekvivalenčnim razredom po tej relaciji pa pravimo \emph{stopnje}.
Izkaže se, da ima ta ureditev zelo lepo strukturo. Več o njej si lakho pogledate
v~\cite{Bauer22}, saj za naše potrebe ni relevantna.

Pomembne so pa operacije na stopnjah. Definiramo lahko dva seštevanja in dva
množenja, kar se izkaže, da ima povezave z linearno logiko, a tud definirajmo
samo navadno množenje.

\begin{definicija}
  \emph{Produkt \(φ⊑A\) in \(ψ⊑B\)} je predikat \(φ×ψ⊑A×B\) definiran po točkah.
  Na očiten način lahko definiramo tudi končne potence predikata \(φ\).
\end{definicija}

\begin{definicija}
  Pravimo, da je \(φ⊑A\) \emph{idempotenten}, ko velja \(φ²≤φ\).
\end{definicija}
Idempotentnost pomeni, da lahko odločitev \(φ\) za dve vrednosti iz \(A\)
odločimo zgolj z eno ``poizvedbo'' \(φ\), na nekem drugem \(a\).

Poglejmo si recimo \(\lem*\). Zgleda, kot da se moramo odločiti med štirimi
alternativami, \(p∧q\), \(p∧¬q\), \(¬p∧q\), in \(¬p∧¬q\), medtem ko nam ena
instanca \(\lem*\) da zgolj dve možnosti. Zgleda nemogoče, ampak se izkaže, da
to lahko pokažemo.

\begin{lema}
  Za vse \(p:Ω\) velja \(¬¬\lem(p)\).
\end{lema}
\begin{dokaz}
  Formula \(¬\lem(p)\) je natanko formula principa neprotislovja za \(¬p\), ki
  ni veljaven.
\end{dokaz}

\begin{trditev}
  Princip \(\lem*\) je idempotenten.
\end{trditev}
\begin{dokaz}
  Naj bosta \(p\) in \(q\) resničnostni vrednosti.
  Potem definiramo \(r ≔ \lem²{\p{p, q}}\).
  Ker \(\lem ⁿ\) slika v goste resničnostne vrednosti, je \(\lem{\p r} = r\), tako da
  je \[\lem{\p r} = r = \lem²{\p{p, q}}\text.\qedhere\]
\end{dokaz}
Reducibilnost izhaja iz teorije izračunljivosti. Tam se idempotenca ne pojavi
pogosto, saj tam pomeni, da lahko dve izvedbi algoritma izvedemo že z eno
izvedbo, na posebej izbranem vhodu. Recimo \(\lpo*\) ni idempotenten.
Hipotetični algoritem, ki bi odločal \(\lpo(α)\), najprej reče, da je odločil
\(α=0\). Nato bere zaporedje, in ko naleti na enico, potuje skozi čas in
spremeni svoj odgovor. Če iščemo enico v dveh zaporedjih, mora za vsakega od
njih potovati skozi čas, tako da gotovo ne moremo tega izračunati z eno izvedbo
algoritma. Se pa izkaže, da v topoloških modelih temu ni nujno tako, a k temu se
vrnemo kasneje.

V~\ref{th:alpoc-is-lpo} smo pokazali, da so principi, ki govorijo o \(Σ₀¹\) in
\(Σ_{\Rc}\) ekvivalentni. Ampak vseeno pa ta Sierpinskijeva objekta nista nujno
enaka. Na kratko se lahko o tem prepričamo tako, da rečemo, da je dokaz tega
desjtva zahteval neskončno instanc \(Σ₀¹\), torej so elementi \(Σ_{\Rc}\)
ekvivalentni števnim konjunkcijam elementov \(Σ₀¹\). Velja torej naslednja
redukcija.
\begin{trditev}
  Velja redukcija \(\alpo*_{\Rc} ≤ \lpo*^ℕ\).
\end{trditev}
\begin{dokaz}
  To smo zares pokazali že zgoraj. Uporabimo \(ω⋅ω + ω + 1\) instanc \(\lpo*\),
  kar je števno mnogo, torej je števno mnogo instanc \(\lpo*\) dovolj za gornjo
  redukcijo.
\end{dokaz}


\subsubsection{Kripkejeve sheme}

Oglejmo si sedaj malce strožjo reducibilnost, namreč, ko v~\ref{eq:inst-red} zamenjamo
\(⇒\) z \(⇔\). Specifično nas to zanima, saj si želimo ogledati principe
\(Σ = Ω\), za razne Sierpinskijeve objekte, kajti to velja natanko tedaj, ko
velja \(\for{p:Ω}{\exist{s∈Σ}{s ⇔ p}}\). Kot vidimo, je to enaka formula
kot~\ref{eq:inst-red}, le da smo \(⇒\) zamenjali z \(⇔\).

\begin{definicija}
  \emph{Kripkejeva shema za \(Σ ⊆ Ω\)} pravi, da je \(Σ = Ω\). Formulo
  \(\for{p : Ω}{\exist{s : Σ}{s = p}}\) bomo označili z \(\ks*(Σ)\).

  Navadna \emph{kripkejeva shema} je kripkejeva shema za
  \(Σ₀¹ ≔ \set{α \apart 0}{α : 2^ℕ} ⊆ Ω\).

  \emph{Analitična kripkejeva shema za \(ℝ\)} je kripkejeva shema za \(Σ_ℝ\),
  kjer je \(ℝ\) nek objekt realnih števil. Označili jo bomo z \(\aks*_ℝ\)

  Posebej bomo \(\ks*(Σ_{\Rd})\) pravili \emph{analitična kripkejeva shema}.
\end{definicija}

\begin{trditev}
  Kripkejeva shema za \(Δ\) je natanko \(\lem*\).
\end{trditev}

\begin{trditev}\label{th:aks-impl-lem≤alpo}
  Če velja \(\ks*(Σ)\), velja \(\p{\lem*,Ω} ≤ \p{\lem*,Σ}\). V posebnem velja
  \(\lem* ≤ \alpo*\), če velja \(\aks*\).
\end{trditev}
\begin{dokaz}
  Naj bo \(p:Ω\). Po predpostavki je \(Σ = Ω\), torej je \(p∈Σ\). Potem pa
  lahko na \(p\) uporabimo \(\p{\lem*,Σ}\) in \(\lem(p)\) velja.
\end{dokaz}
\begin{posledica}
  Če velja \(\ks*(Σ)\), je \(\p{\lem*,Σ}\) idempotenten.
\end{posledica}


\subsubsection{Princip markova}

Kot na začetku tega podrazdelka pa lahko zožimo še Kripkejevo shemo.
\begin{trditev}
  Kripkejeva shema \(Σ₀¹ = R\) je natanko \(\mp*\). Podobno velja za
  analitične različice principa.
\end{trditev}

%%% Local Variables:
%%% mode: latex
%%% TeX-master: "main"
%%% End:
