\section{Nekonstruktivni principi}

\subsection{Realna števila}

Spomnimo se nekaj definicij realnih števil.
\begin{definicija}[Dedekindova realna števila]
  Par \(\p{L, U} ∈ 𝒫(ℚ)×𝒫(ℚ)\) je \emph{Dedekindov rez}, ko velja
  \begin{align}
    L \text{ je poseljen}\\
    U \text{ je poseljen}\\
    L \text{ je navzdol zaprt}\\
    U \text{ je navzgor zaprt}\\
    L \text{ je navzgor odprt}\\
    U \text{ je navzdol odprt}\\
    \text{za } a < b \text{ je } a ∈ L ∨ b ∈ U\\
    \text{za } a ∈ L ∧ b ∈ U \text{ je } a < b
  \end{align}
  Množica dedekindovih rezov tvori \emph{Dedekindova realna števila}, ki jih
  označimo z \(\Rd\)
\end{definicija}
\begin{definicija}[Cauchyjeva realna števila]
  Zaporedje racionalnih števil \(\p{qₙ}ₙ\) je \emph{Cauchyjevo}, ko velja
  \[ \for{n : ℕ}{\exist{N : ℕ}{\for{k, l : ℕ}{k ≥ N ∧ l ≥ N ⇒ |qₖ - qₗ| < 2⁻ⁿ}}}\text. \]
  TODO: modulusi
\end{definicija}

\begin{trditev}
  Vsako Cauchyjevo realno število je tudi Dedekindovo.
\end{trditev}
\begin{dokaz}
  Vzemimo \(L ≔ \set{q : ℚ}{q < x}\) in \(U ≔ \set{s : ℚ}{x < s}\).

  Očitno ta zadoščata prvim šestim pogojem zgoraj, tako da se osredotočimo zgolj
  na zadnja dva pogoja. Naj bo \(a < b\). Potem je \(x - a\) Cauchyjevo.
\end{dokaz}

\begin{izrek}
  Vsako Dedekindovo realno število je tudi Cauchyjevo.
\end{izrek}
\begin{dokaz}[Dokaz (klasični)]
  Pokazati je zgolj potrebno, da ima vsak Dedekindov rez pripadajoče Cauchyjevo
  zaporedje. Naj bo \(\p{L, U}\) obojestranski dedekindov rez. Cauchyjevo
  zaporedje lahko podamo kot zaporedje hitro padajočih racionalnih intervalov.

  Naj bosta \(p₀ ∈ L\) in \(q₀ ∈ U\) racionalni števili. Potem pa na \(n\)-tem
  koraku definiramo
  \[ a ≔ \frac{2pₙ + qₙ}{3}\text,\quad b ≔ \frac{pₙ + 2qₙ}{3}\text{, in}\quad
    \p{pₙ₊₁, qₙ₊₁} ≔ \begin{cases}
      \p{a, qₙ} ;& a ∈ L\\
      \p{pₙ, b} ;& b ∈ U\text.
    \end{cases}
  \]
  Te intervali hitro konvergirajo proti \(\p{L, U}\), torej je to želeno
  Cauchyjevo število.
\end{dokaz}

\subsection{Principi odločitve}

\begin{definicija}[Izključena tretja možnost]\label{pr:lem}
  \emph{Princip izključene tretje možnosti} pravi, da za vsako resničnostno vrednost
  \(p\) velja \(p∨¬p\). Formulo \(p∨¬p\) označimo \(\lem(p)\), formulo
  \(\for{p:Ω}{p∨¬p}\) pa z \(\lem*\).
\end{definicija}

\begin{definicija}
  \emph{Princip števne odločitve} pravi, da za vsako števno zaporedje ničel in enic
  lahko odločimo, ali je celo nič, ali pa obstaja mesto, na katerem je enica.
  Formulo \(α = 0∨α\apart 0\) označimo \(\lpo(α)\), formulo
  \(\for{α : 2^ℕ}{\lpo(α)}\) pa z \(\lpo*\).
\end{definicija}

\begin{definicija}
  \emph{Analitični princip števne odločitve} pravi, da za vsako realno število
  lahko odločimo, ali je pozitivno ali nenegativno. Formulo \(x > 0 ∨ x ≤ 0\)
  označimo \(\alpo(x)\), formulo \(\for{x : ℝ}{\alpo(x)}\) pa z \(\alpo*\).
\end{definicija}

\begin{trditev}
  Velja veriga implikacij \(\lem* ⇒ \alpo* ⇒ \lpo*\).
\end{trditev}
\begin{dokaz}
  Ker je \(¬(x > 0)\) natanko \(x ≤ 0\), je \(\alpo*\) očitno posledica
  \(\lem*\).

  Za drugo implikacijo pa potrebujemo malo dela. Naj bo \(α\) zaporedje, za
  katerega odločamo, ali ima na kakem mestu enico. Definirajmo realno število
  \[ x ≔ \lim_{n → ∞}2^{-\min\set{k : ℕ}{α(k) = 1 ∨ k ≥ n}}\text. \]

  To je po definiciji Cauchyjevo realno število. Uporabimo sedaj predpostavko
  \(\alpo(x)\). Če velja \(x ≤ 0\), mora zaporedje limitirati proti \(0\), kar
  pa pomeni, da \(α(k) = 1\) ni nikoli zadoščeno, torej je \(α = 0\).
  Po drugi strani, če je \(x > 0\) pa obstaja nek \(k : ℕ\), tako da je
  \(x > 2⁻ᵏ\). To pa pomeni, da ima \(α\) enico na enem izmed prvih \(k\)
  mestih, kar konstruktivno pomeni, da želeni indeks obstaja.
\end{dokaz}

\subsection{Principi izbire}

Za princip izbire ste verjetno že slišali. Ta pravi, da za vsaki množici \(A\)
in \(B\), in vsako celovito relacijo \(R\) med njima, obstaja funkcija
\(f : A → B\), tako da velja \(\for{a : A}{R(a, f(a))}\).

Ta princip lahko ošibimo na tri načine: lahko omejimo množici \(A\) in \(B\),
lahko pa omejimo relacijo \(R\), tako da jemlje resničnostne vrednosti zgolj iz
določene množice \(Σ ⊆ Ω\).

\begin{definicija}
  \emph{Princip izbire nad \(Σ\)} je shema, ki za vsaka \(A\) in \(B\) pravi,
  da za vsako relacijo \(R : A×B → Σ\) za katero velja
  \(\for{a:A}{\exist{b:B}{R(a,b)}}\), obstaja funkcija izbire \(f : A → B\),
  tako da velja \(\for{a:A}{R(a,f(a))}\). Objektu \(A\) tako pravimo
  \emph{domena}, objektu \(B\) \emph{kodomena}, objektu \(Σ\) pa TODO: kaj?.
  Pogoju na \(R\) pravimo \emph{celovitost}.

  To označimo z \(\AC_Σ\). Če je \(Σ = Ω\) indeks opustimo in temu pravimo
  \emph{princip izbire}.
\end{definicija}
\begin{definicija}
  Če v principu izbire fiksiramo domeno na nek \(A\), temu pravimo
  \emph{princip izbire nad \(Σ\) iz \(A\)}, in če fiksiramo še kodomeno na nek
  \(B\) temu pravimo \emph{princip izbire nad \(Σ\) iz \(A\) v \(B\)}. Ta potem
  označujemo \(\AC_Σ(A)\) in \(\AC_Σ(A, B)\). Podobno kot zgoraj opuščamo \(Σ\)
  ko je ta enaka \(Ω\).
\end{definicija}
Hitro lahko opazimo, da so principi izbire neobčutljivi za izomorfizme.
\begin{trditev}
  Naj bo \(φ : A ≅ A'\) in \(ψ : B ≅ B'\). Potem je \(\AC_Σ(A, B)\) ekvivalenten
  \(\AC_Σ(A', B')\).
\end{trditev}
\begin{dokaz}
  Situacija je očitno simetrična, tako da pokažimo zgolj eno implikacijo.

  Če je \(R : A'×B' → Σ\) celovita, je potem tudi \(R∘\p{φ×ψ} : A×B → Σ\)
  celovita. Potem pa za to relacijo obstaja funkcija izbire, tako da je
  \(\for{a : A}{R∘\p{φ×ψ}(a, f(a))}\), kar je pa enako
  \(\for{a:A}{R(φ(a),ψ(f(a)))}\). Ker pa je \(φ\) izomorfizem, lahko
  preindeksiramo kvantifikator in dobimo \(\for{a:A'}{R(a',ψ∘f∘φ⁻¹(a'))}\),
  torej je \(ψ∘f∘φ⁻¹\) funkcija izbire za \(R\).
\end{dokaz}
V teoriji množic to pomeni, da je aksiom izbire določen do kardinalnosti
natančno. To nam tudi utemelji zakaj \(\CC\) lahko pravimo princip \emph{števne}
izbire, saj deluje za poljubno števno (neskončno) množico. Tako si lahko tudi
predstavljamo, kako to shemo imen razširiti na poljubno kardinalnost.
\begin{trditev}
  Če je \(Σ' ⊆ Σ\), \(\AC_Σ(A, B)\) implicira \(\AC_{Σ'}(A, B)\).
\end{trditev}
\begin{dokaz}
  Vsaka relacija nad \(Σ'\) je tudi relacija nad \(Σ\), tako da če imajo
  relacije nad \(Σ\) funkcije izbire jih imajo tudi relacije nad \(Σ'\).
\end{dokaz}
\begin{trditev}
  Princip končne izbire velja.
\end{trditev}
\begin{dokaz}
  TODO: sintaksa za standardne končne?
  Princip končne izbire je princip izbire, kjer domeno omejimo na končne
  množice. Brez škode za splošnost, naj bo domena kar enaka neki
  standardni končni množici \(\cli n\).

  Potem pa \(\AC(\cli n)\) pravi, da če
  obstajajo \(bₖ:B\), da velja \(R(1,b₁)∧\dots ∧R(n,bₙ)\), obstaja končno
  zaporedje (torej, obstajajo \(bₖ:B\)), da velja \(R(1,b₁)∧\dots ∧R(n,bₙ)\)… To
  sta pa isti stvari, tako da princip res drži.
\end{dokaz}

% \begin{definicija}
%   \emph{Princip \[Σ\]-izbire iz \(A\) v \(B\)} pravi, da za vsako relacijo
%   \(R : A×B → Σ\) za katero velja \(\for{a:A}{\exist{b:B}{R(a,b)}}\), obstaja
%   funkcija izbire \(f : A → B\), tako da velja \(\for{a:A}{R(a,f(a))}\).

%   To bomo označili z \(\AC_Σ(A,B)\). Z \(\AC_Σ(A)\) bomo označevali shemo
%   \(\for{B}{\AC_Σ(A,B)}\), z \(\AC_Σ\) pa shemo \(\for{A}{\AC_Σ(A)}\).
%   Tem pravimo \emph{princip \(Σ\)-izbire iz \(A\)} in \emph{princip \(Σ\)-izbire}.
%   Kadar je \(Σ = Ω\) bomo \(Σ\) v indeksu opuščali, in ustrezno to poimenovali
%   kar samo \emph{princip izbire (iz \(A\) v \(B\))}.

%   Standardno se principu \(\AC(ℕ)\) pravi \emph{princip števne izbire} in
%   označuje \(\CC\).
%   Posebno pomemben bo princip \(\AC(ℕ, 2)\), ki ga bomo označevali \(\CCv\) in
%   mu pravili \emph{princip števne dvojiške(disjunktivne?) izbire}.
% \end{definicija}



% \begin{definicija}
%   \emph{Princip števne dvojiške izbire} pravi, da če velja
%   \(\for{n : ℕ}{P(n, 0) ∨ P(n, 1)}\) (torej, \(P\) je celovita relacija na
%   \(ℕ×2\)) obstaja funkcija izbire \(f : ℕ → 2\), da velja
%   \(\for{n : ℕ}{P(n,f(n))}\).
% \end{definicija}


\subsection{Ostali principi}

Seveda pa se principi ne delijo popolnoma zgolj na principe izbire in odločitve.

\begin{definicija}
  \emph{Kripkejeva shema za \(Σ ⊆ Ω\)} pravi, da je \(Σ = Ω\). Formulo
  \(\for{p : Ω}{\exist{s : Σ}{s = p}}\) bomo označili z \(\ks*(Σ)\).

  Navadna \emph{kripkejeva shema} je kripkejeva shema za
  \(Σ₀¹ ≔ \set{α \apart 0}{α : 2^ℕ} ⊆ Ω\).

  TODO: dedekindova ali cauchyjeva ali kaj drugega.
  \emph{Analitična kripkejeva shema} je kripkejeva shema za
  \(Σ_ℝ ≔ \set{x > 0}{x : ℝ} ⊆ Ω\).
\end{definicija}

%%% Local Variables:
%%% mode: latex
%%% TeX-master: "main"
%%% End:
