\section{Nekonstruktivni principi}\label{sec:logika}

\section{Principi izbire}

\subsection{\(\cc\) in \(\ccv\) nad \(T₁\) prostori}


%%% Local Variables:
%%% mode: latex
%%% TeX-master: "main"
%%% End:

\section{Principi odločitve}\label{sec:odločitve}

Spomnimo se najprej karakterizacij \(\lem*\) in \(\wlem*\) iz
podrazdelka~\ref{sec:modeli-logika-odprtih}.

TODO: renumber
\begin{trditev}\label{th:lem-is-partition-second}
  Nad topološkim prostorom velja princip izključene tretje možnosti natanko
  tedaj, ko je prostor particijski.
\end{trditev}
\begin{trditev}\label{th:wlem-is-ext-disc-second}
  Nad topološkim prostorom velja DeMorganov zakon natanko tedaj, ko je prostor
  ekstremalno nepovezan.
\end{trditev}

Ker so naši prostori \(T₀\), so particijski prostori natanko diskretni. Res, če
so točke zaprte, in je vsaka zaprta množica odprta, so točke odprte, torej je
prostor diskreten.


Pokažimo sedaj, da implikacije iz~\ref{sec:logika-odločitve} niso obrnljive.

\begin{trditev}
  Nad \(2^ℕ\) ne velja \(\lpo*\).
\end{trditev}
\begin{dokaz}
  Naj bo \(α : 2^ℕ → 2^ℕ\) identiteta. Resničnostna vrednost \(\lpo(α)\) je
  \(2^ℕ⧵\{0\}\), saj je resničnostna vrednost \(α = 0\) enaka \(∅\).
\end{dokaz}

\begin{trditev}
  Nad \(ℝ\) velja \(\lpo*\) in ne velja \(\alpo*\).
\end{trditev}
\begin{dokaz}
  Naj bo \(x : ℝ\) zunaj identiteta. Potem je \(\i{\alpo(x)} = ℝ⧵\{0\}\).

  Preostanek tega dokaza izpeljemo malo kasneje, v izreku~\ref{th:lpov-lpo}
\end{dokaz}

\begin{trditev}
  Nad \(\Ncof\) velja \(\alpo*\) in ne velja \(\lem*\).
\end{trditev}
\begin{dokaz}
  Prostor očitno ni diskreten. Naj bo \(x : ℝ\), torej preslikava \(\e x → ℝ\).
  Naj bo \(t, t' ∈ \im x\). Potem sta \(x⁻¹(t)\) in \(x⁻¹(t')\) neprazni odprti
  množica. To pa pomeni, da sta obe kokončni, torej imata neprazen presek, od
  koder sledi, da sta si enaka. Torej je vsako notranje realno število
  konstantno. Sedaj pa lahko za to realno število \(\alpo(x)\) odločimo zunaj.
\end{dokaz}

\begin{trditev}
  Nad \(\Ncof\) velja \(\wlem*\) in ne velja \(\lem*\).
\end{trditev}
\begin{dokaz}
  Prostor še vedno ni diskreten. Naj bo sedaj \(U\) odprta množica.
  Njena zunanjost (torej negacija) je pa bodisi cel prostor, bodisi prazna. Za
  te odprte množice pa velja da so komplementirane, torej \(\wlem*\) drži.
\end{dokaz}

Obrate teh implikacij lahko torej jemljemo kot nekonstruktivni principi, a o teh
ni veliko znano. Vemo le, da očitno \(\Rd = \Rc\) implicira \(\alpo* ⇔ \lpo*\),
saj je \(\lpo*\) ekvivalenten \(\alpo*\) za Cauchyjeva realna števila, tako da
je to zelo šibek princip. Kasneje si bomo ogledali kdaj obrati veljajo na nek
strožji in bolj strukturiran način, ki ima povezavo s teorijo izračunljivosti,
in ta upamo, da nam da močnejše principe, ki bodo potem tudi topološko bolj
zanimivi.

TODO: premakni zgoraj?

Vemo pa vsaj nekaj.
\begin{izrek}\label{th:lpov-lpo}
  Če je \(X\) lokalno povezan, velja \(X ⊩ \lpo*\).
\end{izrek}
\begin{dokaz}
  Naj bo \(α : 2^ℕ\). Ker je \(X\) lokalno povezan, je množica \(2^ℕ\) kar
  množica preslikav iz \(ℕ\) v \(2\). To pa pomeni, da je lokalno \(α = f\), za
  nek \(f : ℕ → 2\). Zunaj pa imamo \(\lpo*\), tako da lahko odločimo \(f = 0 ∨
  f \apart 0\), kar pa pomeni, da to lahko odločimo tudi za \(α\), torej nad
  \(X\) velja \(\lpo*\).
\end{dokaz}
\begin{opomba}
  Ker je \(\lpo*\) natanko \(\alpo*\) za Cauchyjeva realna števila (TODO: daj to
  nekam na samo, zdej sem že petič napisala), in potrebujemo le, da so elementi
  lokalno konstantni, zadošča predpostaviti \(\Rc = \c ℝ\). To pomeni, da je to
  kar močen princip, saj je zadosten za \(\lpo*\).

  TODO: a je ekvivalenca? smz da, ker \(x : \Rc\) je zvezna \(\e x → ℝ\), če
  mamo \(\lpo*\) je \(\i{x = a} = \i{x \apart a}ᶜ\), tko da je \(x\) lokalno
  konstantna.
\end{opomba}

TODO: Elephant, D4.7: It can be shown that, for a locale \(X\), \(\Rc\) coincides
with \(\c ℝ\) iff for every open \(U\), the lattice of (relatively) clopen
sublocales of \(U\) is closed under countable unions. (For a second countable
locale this is equivalent to local connectedness)


%%% Local Variables:
%%% mode: latex
%%% TeX-master: "main"
%%% End:


\subsection{Realna števila}\label{sec:logika-reals}

Spomnimo se nekaj definicij realnih števil.
\begin{definicija}[Dedekindova realna števila]
  Par \(\p{L, U} ∈ 𝒫(ℚ)×𝒫(ℚ)\) je \emph{Dedekindov rez}, ko velja
  \begin{align}
    L \text{ je poseljen}\\
    U \text{ je poseljen}\\
    L \text{ je navzdol zaprt}\\
    U \text{ je navzgor zaprt}\\
    L \text{ je navzgor odprt}\\
    U \text{ je navzdol odprt}\\
    \text{za } a < b \text{ je } a ∈ L ∨ b ∈ U\label{real:located}\\
    \text{za } a ∈ L ∧ b ∈ U \text{ je } a < b
  \end{align}
  Množica Dedekindovih rezov tvori \emph{Dedekindova realna števila}, ki jih
  označimo z \(\Rd\).
\end{definicija}
\begin{opomba}
  Pogoju~\ref{real:located} pravimo \emph{lociranost} in če velja za katerikoli
  racionalni števili \(a\) in \(b\), bo veljala tudi za vse druge izbire.

  Res, relacija \(<\) je invariantna na afine preslikave realne premice, tako da
  lahko vsak interval \(\p{a,b}\) preslikamo (recimo) na \(\p{0,1}\) in nazaj.
\end{opomba}

\begin{definicija}[Cauchyjeva realna števila]
  Zaporedje racionalnih števil \((xₙ)ₙ\) je \emph{Cauchyjevo}, ko za vsaka
  \(i,j : ℕ\) velja \(|xᵢ - xⱼ| ≤ 2⁻ⁱ+2⁻ʲ\).

  % \emph{Modulus konvergence} je preslikava \(α : ℚ₊ → ℕ\), za katero za vsak
  % \(n : ℕ\) obstaja \(ε : ℚ₊\), da je \(α(ε) ≥ n\).

  % Zaporedje racionalnih števil \(\p{xₙ}\) z modulusom konvergence \(α\) je
  % \emph{Cauchyjevo}, ko za vsake \(ε : ℚ₊\) in \(i,j ≥ α(ε)\) velja
  % \(|xᵢ - xⱼ| < ε\).

  Dve Cauchyjevi zaporedji predstavljata isto realno število,
  ko za vse \(i : ℕ\) velja \(|xᵢ - yᵢ| ≤ 2⁻ⁱ⁺¹\).
  % ko obstaja modulus
  % konvergence \(α\), da za \(ε : ℚ₊\) in \(i ≥ α(ε)\) velja \(|xᵢ - yᵢ| ≤ ε\). 

  Množica Cauchyjevih zaporedij kvocientno z gornjo relacijo tvori
  \emph{Cauchyjeva realna števila}, ki jih označimo z \(\Rc\).
\end{definicija}

\begin{trditev}
  Vsako Cauchyjevo realno število je tudi Dedekindovo.
\end{trditev}
\begin{dokaz}
  Vzemimo \(L ≔ \set{q : ℚ}{q < x}\) in \(U ≔ \set{r : ℚ}{x < r}\).
  Očitno ta zadoščata prvim šestim pogojem zgoraj, in prav tako zadoščata
  zadnjemu pogoju, tako da se osredotočimo zgolj na zadnji pogoj.

  Pokazati moramo, da za vsako Cauchyjevo realno število \(x\) velja
  \(x > 0 ∨ x < 3\).

  %To pa pomeni, da mora obstajati nek indeks \(N : ℕ\), da so \(xᵢ > 0\) za vse
  %\(i ≥ N\), ali pa \(xᵢ < 3\) za vse \(i ≥ N\).
  Za racionalno število \(x₀\) velja \(x₀ > 1 ∨ x₀ < 2\), saj imajo racionalna
  števila odločljivo neenakost. Sedaj pa vemo, da za vse \(i : ℕ\) velja
  \(|x₀ - xᵢ| ≤ 1\), torej če je \(x₀ > 1\) bodo vsi \(xᵢ > 0\), po drugi strani
  pa če je \(x₀ < 2\) bodo vsi \(xᵢ < 3\), kar pa zaključi dokaz.
\end{dokaz}

Obrat pa ne velja konstruktivno. To pomeni, da lahko na ``\(\Rd = \Rc\)''
gledamo kot nekakšen nekonstruktiven princip. Iz konstruktivne matematike pa
poznamo tudi druge konstrukcije realnih števil, ki nam lahko dajo nove principe
kot ta zgoraj.

TODO: cite nlab? elephant?
\begin{definicija}[MacNeillova realna števila]
  Par \(\p{L, U} ∈ 𝒫(ℚ)×𝒫(ℚ)\) je \emph{Dedekind-MacNeilleov rez}, ko velja
  \begin{align}
    L \text{ je poseljen}\\
    U \text{ je poseljen}\\
    L \text{ je navzdol zaprt}\\
    U \text{ je navzgor zaprt}\\
    L \text{ je navzgor odprt}\\
    U \text{ je navzdol odprt}\\
    L = \int{\p{Uᶜ}}\\
    U = \int{\p{Lᶜ}}
  \end{align}
  Množica Dedekind-MacNeilleovih rezov tvori \emph{MacNeillova realna števila}, ki jih
  označimo z \(\Rm\).
\end{definicija}

\begin{trditev}
  Vsako Dedekindovo realno število je tudi MacNeillovo.
\end{trditev}
\begin{dokaz}
  Točke \(1\) skozi \(6\) so enake, tako da je treba pokazati zgolj zadnji dve
  lastnosti. Ker sta simetrični, pokažimo zgolj zadnjo.

  Če je \(a ∈ L\) potem ni v \(U\), torej je v \(Uᶜ\). Sedaj pa, ker je \(L\)
  navzgor odprt obstaja \(a' > a\), ki je tudi v \(L\) (torej v \(Uᶜ\) po enakem
  argumentu.) Sedaj pa vemo, da je \(Uᶜ\) navzdol zaprt, torej je
  \(a ∈ \p{-∞,a'} ⊆ Uᶜ\) in je \(a ∈ \int{\p{Uᶜ}}\).
\end{dokaz}

V obratno smer lahko pokažemo zgolj, da sta \(L\) in \(U\) ločena.
\begin{lema}
  Za MacNeillovo realno število \(\p{L,U}\) velja \(a∈L∧b∈U⇒a<b\).
\end{lema}
\begin{dokaz}
  Naj bo \(a∈L\) in \(b∈U\). Potem je \(b ∈ \int{\p{Lᶜ}} ⊆ Lᶜ\), torej ni v
  \(L\). Ker je \(L\) navzdol zaprt je torej \(b\) zgornja meja, in je večji od
  \(a\).
\end{dokaz}

Ostalo bi torej pokazati zgolj lociranost vsakega MacNeillovega realnega
števila, a to žal ni mogoče.
Temu principu bomo torej pravili ``\(\Rm = \Rd\)''.

Pokažimo še koristno lemo, ki pravi, da ima vsaka poseljena omejena množica
MacNeillovih realnih števil supremum.
\begin{lema}\label{th:Rm-sup}
  MacNeillova realna števila so polna.
\end{lema}
\begin{dokaz}
  TODO
\end{dokaz}

TODO: alpo itd


\subsection{Ostali principi}\label{sec:logika-ostalo}

Seveda pa se principi ne delijo popolnoma zgolj na principe izbire in odločitve.
Nekaj takih imamo zgoraj o realnih številih.

\begin{definicija}
  \emph{Kripkejeva shema za \(Σ ⊆ Ω\)} pravi, da je \(Σ = Ω\). Formulo
  \(\for{p : Ω}{\exist{s : Σ}{s = p}}\) bomo označili z \(\ks*(Σ)\).

  Navadna \emph{kripkejeva shema} je kripkejeva shema za
  \(Σ₀¹ ≔ \set{α \apart 0}{α : 2^ℕ} ⊆ Ω\).

  TODO: dedekindova ali cauchyjeva ali kaj drugega.
  \emph{Analitična kripkejeva shema za \(R\)} je kripkejeva shema za
  \(Σ_R ≔ \set{x > 0}{x : R} ⊆ Ω\), kjer je \(R\) nek objekt realnih števil.

  Posebej bomo \(\ks*(Σ_{\Rd})\) pravili \emph{analitična kripkejeva shema}.
\end{definicija}

\begin{trditev}
  TODO: če se ne motim je \(Σ_{\Rc} = Σ₀¹\). Mogoče.
\end{trditev}

TODO: global existence
\begin{trditev}\label{th:lT6-have-AKS}
  Nad lokalno \(T₆\) prostori velja analitična Kripkejeva shema.
\end{trditev}
\begin{proof}
  Brez škode za splošnost lahko predpostavimo, da je prostor \(X\) \(T₆\).
  Naj bo \(U ⊆ X\). Tedaj obstaja \(f : X → ℝ\), ki je \(0\) natanko na
  komplementu \(U\). To pa pomeni, da je natanko na \(U\) različen od \(0\)
  kar je pa točno to, kar zahtevamo za analitično Kripkejevo shemo.
\end{proof}
\begin{opomba}
  V dokazu smo zares pokazali \emph{globalen} obstoj za element \(x : ℝ\). To
  nam da slutiti, da je (lokalno) \(T₆\) lastnost močnejša od \(\aks*\). In res
  se izkaže, da je temu tako.
  TODO: a se zmislim primer?
\end{opomba}

Vseeno pa velja obrat, če analitično Kripkejevo shemo malo ojačamo. Specifično,
če predpostavimo obstoj \emph{funkcije izibre} za shemo \(\aks*\).
\begin{trditev}
  Če velja \(X ⊩ \exist{f : Ω → ℝ}{\for{p : Ω}{p ⇔ f(p) \apart 0}}\), je \(X\)
  lokalno \(T₆\).
\end{trditev}
\begin{dokaz}
  TODO: a je \(ℝ^Ω\) zuni funkcije? kaj je s tem?

  razponi \(f\) pokrijejo \(X\)

  za \(f\) velja da za vse \(U\) pod \(\e f\) velja \(U = \i{f(U) \apart 0}\).

  pika TODO: napiši zadevo normalno.
\end{dokaz}
\begin{opomba}
  Ponovno smo pokazali malo več kot zgolj lokalno \(T₆\) lastnost. Dobili smo
  \emph{funkcije izbire} za vsako od \(T₆\) komponent. Če v metateoriji
  predpostavimo princip izbire, potem je to ekvivalentno tej močnejši verziji
  \(\aks*\).
\end{opomba}

\section{Interpretacija pretvorbe primerkov v topoloških modelih}

\subsection{Pretvorba primerkov}

\subsection{Idempotenca \alpo* nad lokalno \(T₆\) prostori}

\begin{trditev}
  Nad vsakim okolišem \(X\) je \(\lem*\) idempotenten.
\end{trditev}
\begin{dokaz}
  Naj bosta \(p\) in \(q\) resničnostni vrednosti.
  Potem definiramo \(r ≔ \lem²{\p{p, q}}\).
  Ker \(\lem ⁿ\) slika v goste resničnostne vrednosti, je \(\lem{\p r} = r\), tako da
  je \[\lem{\p r} = r = \lem²{\p{p, q}}\text.\qedhere\]
\end{dokaz}

Spomnimo se na Trditev~\ref{th:lT6-have-AKS}:
% TODO: set number
\begin{trditev}\label{th:lT6-have-AKS}
  Nad lokalno \(T₆\) prostori velja analitična Kripkejeva shema.
\end{trditev}

% TODO: Faktoriziraj čez AKS ⇒ LEM ≤ ALPO
\begin{izrek}
  Naj bo \(X\) lokalno \(T₆\). Potem nad \(X\) velja redukcija \(\lem* ≤ \alpo*\).
\end{izrek}
\begin{proof}
  % TODO: reword kot 'ALPO je LEM za Ω_ℝ in AKS je Ω_ℝ = Ω'?
  Ker vemo, da nad \(X\) velja analitična Kripkejeva shema, zadošča
  konstruktivno pokazati, da velja \(\lem* ≤ \alpo*\).
  Po predpostavki, je vsaka resničnostna vrednost \(p\) oblike \(x > 0\) za nek
  \(x : ℝ\). Tedaj pa velja
  \[ \lem{\p p} = p ∨ ¬ p = x > 0 ∨ ¬\p{x > 0} = \alpo{\p x}\text, \]
  torej za vsak \(p : Ω\) obstaja \(x : ℝ\), da je \(\alpo{\p x} ⇒ \lem{\p p}\).
  % Ker dokazujemo resnico v topološkem modelu brez škode za splošnost
  % predpostavimo, da je prostor \(T₆\).
  % To pomeni da za vsako odprto množico \(U\) obstaja nenegativna funkcija
  % \(P_U : X → ℝ\), tako da velja \(P_U > 0 ⇔ U\).
  % Če želimo pretvoriti \(\lem*\) na \(\alpo*\), mora za vse
  % \(U ⊆ X\) veljati \(\eventually{x : ℝ}{x ≤ 0 ∨ x > 0 ⇒ U ∪ ¬U}\).
  % Uporabimo predpostavko na \(U ∪ ¬U\), torej nam ostane pokazati, da
  % velja \({P_U ≤ 0 ∨ P_U > 0 ⇒ P_U > 0}\).
  % Ker je \(⟦P_U ≤ 0⟧ = ¬{\p{U ∪ ¬ U}} = ∅\), velja \(¬\p{x ≤ 0}\), kar dokaže trditev.
\end{proof}
\begin{posledica}
  Če je prostor \(X\) lokalno \(T₆\), velja \(\alpo*×\alpo* ≤ \alpo*\).
\end{posledica}
\begin{posledica}
  Nad lokalno \(T₆\) prostorom je \(\alpo*\) idempotenten.
\end{posledica}


\subsection{Idempotenca \lpo* nad Cantorjevim prostorm}

\begin{trditev}
  Nad vsakim okolišem \(X\) velja \(\lpo* ≤ \alpo*\).
\end{trditev}
\begin{dokaz}
  Naj bo \(φ : 2^ℕ → ℝ\) definirana s predpisom
  \[α ↦ \lim_{n→∞}2^{-\min\set{k ∈ ℕ}{αₖ = 1 ∨ k = n}}\text.\]
  Denimo, da je \(α ∈ 2^ℕ\) in uporabimo \(\alpo*\) na \(φ(α)\).

  Če velja \(φ(α) ≤ 0\) vemo, da \(α\) ne more imeti enice na nobenem mestu.

  Alternativno, če je \(φ(α) > 0\), pa je tudi \(φ(α) > 2^{-n}\) za nek \(n\).
  Sledi, da mora imeti \(α\) na enem izmed prvih \(n\) mest enico.
\end{dokaz}

% Joint with prof. Bauer
\begin{lema}
  Za vsako funkcijo \(f : C → ℝ\) obstaja funkcija \(\hat f : C → C\), za katero
  velja \(C ⊩ \hat f \apart 0 ⇔ f > 0\).
\end{lema}
\begin{proof}
  Naj bo \(U := ⟦f > 0⟧\). Potem obstaja števna particija \(U = ⋃ₖVₖ\) z
  bazičnimi odprtimi. Ker so te kompaktne, za vsak \(k\) velja
  \(\exist{i ∈ ℕ}{f{\res{Vₖ}} > 2⁻ⁱ}\). Po aksiomu izbire torej obstaja funkcija
  izbire \(m : ℕ → ℕ\), da velja \(f{\res{Vₖ}} > 2⁻ᵐ⁽ᵏ⁾\). Potem pa definiramo
  \[ C ⊩ \hat fᵢ ≔
    \begin{cases}
      1;& \exist{k : ℕ}{Vₖ ∧ i = m(k)}\\
      0;& \text{sicer.}
    \end{cases}\]
  Potem pa velja
  \[ \hat f \apart 0 ⇔ \exist{i : ℕ}{\exist{k : ℕ}{Vₖ ∧ i = m(k)}} ⇔ \exist{k : ℕ}{Vₖ} ⇔ U\text, \]
  torej je \(C ⊩ \hat f \apart 0 ⇔ f > 0\).
\end{proof}
\begin{posledica}
  V zgornji lemi lahko domeno zamenjamo s poljubno odprto množico.
\end{posledica}

\begin{lema}
  Nad Cantorjevim prostorom je \(\lpo*\) ekvivalenten \(\alpo*\).
\end{lema}
\begin{dokaz}
  Vemo že, da velja \(\lpo* ≤ \alpo*\) tako da moramo pokazati zgolj obratno redukcijo.
  Naj bo \(U ⊩ x : ℝ\). Potem iz gornje leme dobimo \(U ⊩ \hat x : C\), za
  katerega velja \(\hat x \apart 0 ⇔ x > 0\).
  Kontrapizitivna oblika te ekvivalence pa pravi ravno, da je \(\hat x = 0 ⇔ x ≤ 0\),
  torej res velja \(\lpo{\p{\hat f}} ⇔ \alpo{\p{f}}\).
\end{dokaz}
\begin{posledica}
  Nad Cantorjevim prostorom je \(\lpo*\) idempotenten.
\end{posledica}
\begin{opomba}
  V tem podrazdelku bi lahko namesto Cantorjevega prostora zares vzeli
  katerikoli kompakten nič dimenzionalen prostor (z drugimi besedami, Stoneov prostor).
\end{opomba}



%%% Local Variables:
%%% mode: latex
%%% TeX-master: "main"
%%% End:



%%% Local Variables:
%%% mode: latex
%%% TeX-master: "main"
%%% End:
