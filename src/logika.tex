\section{Nekonstruktivni principi}

\subsection{Principi izbire}

Za princip izbire ste verjetno že slišali. Ta pravi, da za vsaki množici \(A\)
in \(B\), in vsako celovito relacijo \(R\) med njima, obstaja funkcija
\(f : A → B\), tako da velja \(\for{a : A}{R(a, f(a))}\).

Ta princip lahko ošibimo na tri načine: lahko omejimo množici \(A\) in \(B\),
lahko pa omejimo relacijo \(R\), tako da jemlje resničnostne vrednosti zgolj iz
določene množice \(Σ ⊆ Ω\).

\begin{definicija}
  \emph{Princip izbire nad \(Σ\)} je shema, ki za vsaka \(A\) in \(B\) pravi,
  da za vsako relacijo \(R : A×B → Σ\) za katero velja
  \(\for{a:A}{\exist{b:B}{R(a,b)}}\), obstaja funkcija izbire \(f : A → B\),
  tako da velja \(\for{a:A}{R(a,f(a))}\). Objektu \(A\) tako pravimo
  \emph{domena}, objektu \(B\) \emph{kodomena}, objektu \(Σ\) pa
  \emph{Sierpinskijev objekt}.
  Pogoju na \(R\) pravimo \emph{celovitost}.

  To označimo z \(\AC_Σ\). Če je \(Σ = Ω\) indeks opustimo in temu pravimo
  \emph{princip izbire}.
\end{definicija}
\begin{definicija}
  Če v principu izbire fiksiramo domeno na nek \(A\), temu pravimo
  \emph{princip izbire nad \(Σ\) iz \(A\)}, in če fiksiramo še kodomeno na nek
  \(B\) temu pravimo \emph{princip izbire nad \(Σ\) iz \(A\) v \(B\)}. Ta potem
  označujemo \(\AC_Σ(A)\) in \(\AC_Σ(A, B)\). Podobno kot zgoraj opuščamo \(Σ\)
  ko je ta enaka \(Ω\).
\end{definicija}

TODO: enolična izbira

Hitro lahko opazimo, da so principi izbire neobčutljivi za izomorfizme.
\begin{trditev}
  Naj bo \(φ : A ≅ A'\) in \(ψ : B ≅ B'\). Potem je \(\AC_Σ(A, B)\) ekvivalenten
  \(\AC_Σ(A', B')\).
\end{trditev}
\begin{dokaz}
  Situacija je očitno simetrična, tako da pokažimo zgolj eno implikacijo.

  Če je \(R : A'×B' → Σ\) celovita, je potem tudi \(R∘\p{φ×ψ} : A×B → Σ\)
  celovita. Potem pa za to relacijo obstaja funkcija izbire, tako da je
  \(\for{a : A}{R∘\p{φ×ψ}{(a, f(a))}}\), kar je pa enako
  \(\for{a:A}{R(φ(a),ψ(f(a)))}\). Ker pa je \(φ\) izomorfizem, lahko
  preindeksiramo kvantifikator in dobimo \(\for{a:A'}{R(a',ψ∘f∘φ⁻¹(a'))}\),
  torej je \(ψ∘f∘φ⁻¹\) funkcija izbire za \(R\).
\end{dokaz}
V teoriji množic to pomeni, da je aksiom izbire določen do kardinalnosti
natančno. To nam tudi utemelji zakaj lahko \(\AC(ℕ)\) pravimo princip
\emph{števne} izbire, saj deluje za poljubno števno (neskončno) množico. Tako si
lahko tudi predstavljamo, kako to shemo imen razširiti na poljubno kardinalnost.
Označimo ga torej \(\CC\). Pomembno vlogo bo igral tudi \emph{princip
  števne disjunktivne izbire} \(\AC(ℕ, 2)\), ki ga označimo z \(\CCv\). 

TODO: vložitve domene in kodomene tud
\begin{trditev}
  Če je \(Σ' ⊆ Σ\), \(\AC_Σ(A, B)\) implicira \(\AC_{Σ'}(A, B)\).
\end{trditev}
\begin{dokaz}
  Vsaka relacija nad \(Σ'\) je tudi relacija nad \(Σ\), tako da če imajo
  relacije nad \(Σ\) funkcije izbire jih imajo tudi relacije nad \(Σ'\).
\end{dokaz}

\begin{trditev}
  Princip končne izbire velja.
\end{trditev}
\begin{dokaz}
  TODO: sintaksa za standardne končne?
  Princip končne izbire je princip izbire, kjer domeno omejimo na končne
  množice. Brez škode za splošnost, naj bo domena kar enaka neki
  standardni končni množici \(\cli n\).

  Potem pa \(\AC(\cli n)\) pravi, da če
  obstajajo \(bₖ:B\), da velja \(R(1,b₁)∧\dots ∧R(n,bₙ)\), obstaja končno
  zaporedje (torej, obstajajo \(bₖ:B\)), da velja \(R(1,b₁)∧\dots ∧R(n,bₙ)\)… To
  sta pa isti stvari, tako da princip res drži.
\end{dokaz}

% \begin{definicija}
%   \emph{Princip \[Σ\]-izbire iz \(A\) v \(B\)} pravi, da za vsako relacijo
%   \(R : A×B → Σ\) za katero velja \(\for{a:A}{\exist{b:B}{R(a,b)}}\), obstaja
%   funkcija izbire \(f : A → B\), tako da velja \(\for{a:A}{R(a,f(a))}\).

%   To bomo označili z \(\AC_Σ(A,B)\). Z \(\AC_Σ(A)\) bomo označevali shemo
%   \(\for{B}{\AC_Σ(A,B)}\), z \(\AC_Σ\) pa shemo \(\for{A}{\AC_Σ(A)}\).
%   Tem pravimo \emph{princip \(Σ\)-izbire iz \(A\)} in \emph{princip \(Σ\)-izbire}.
%   Kadar je \(Σ = Ω\) bomo \(Σ\) v indeksu opuščali, in ustrezno to poimenovali
%   kar samo \emph{princip izbire (iz \(A\) v \(B\))}.

%   Standardno se principu \(\AC(ℕ)\) pravi \emph{princip števne izbire} in
%   označuje \(\CC\).
%   Posebno pomemben bo princip \(\AC(ℕ, 2)\), ki ga bomo označevali \(\CCv\) in
%   mu pravili \emph{princip števne dvojiške(disjunktivne?) izbire}.
% \end{definicija}



% \begin{definicija}
%   \emph{Princip števne dvojiške izbire} pravi, da če velja
%   \(\for{n : ℕ}{P(n, 0) ∨ P(n, 1)}\) (torej, \(P\) je celovita relacija na
%   \(ℕ×2\)) obstaja funkcija izbire \(f : ℕ → 2\), da velja
%   \(\for{n : ℕ}{P(n,f(n))}\).
% \end{definicija}


\subsection{Principi odločitve}

\begin{definicija}[Izključena tretja možnost]\label{pr:lem}
  \emph{Princip izključene tretje možnosti} pravi, da za vsako resničnostno
  vrednost \(p\) velja \(p∨¬p\). Formulo \(p∨¬p\) označimo \(\lem(p)\), formulo
  \(\for{p:Ω}{p∨¬p}\) pa z \(\lem*\).
\end{definicija}

\begin{definicija}[Šibka izključena tretja možnost]\label{pr:wlem}
  \emph{Šibak princip izključene tretje možnosti} pravi, da za vsako
  resničnostno vrednost \(p\) velja \(¬p∨¬¬p\). Formulo \(¬p∨¬¬p\) označimo
  \(\wlem(p)\), formulo \(\for{p:Ω}{¬p∨¬¬p}\) pa z \(\wlem*\).
\end{definicija}
\begin{trditev}
  Velja implikacija \(\lem* ⇒ \wlem*\).
\end{trditev}
\begin{dokaz}
  Formula \(\wlem(p)\) je natanko \(\lem(¬p)\), tako da trditev očitno velja.
\end{dokaz}

\begin{definicija}\label{pr:lpo}
  \emph{Princip števne odločitve} pravi, da za vsako števno zaporedje ničel in enic
  lahko odločimo, ali je celo nič, ali pa obstaja mesto, na katerem je enica.
  Formulo \(α = 0∨α\apart 0\) označimo \(\lpo(α)\), formulo
  \(\for{α : 2^ℕ}{\lpo(α)}\) pa z \(\lpo*\).
\end{definicija}

\begin{definicija}\label{pr:alpo}
  \emph{Analitični princip števne odločitve} pravi, da za vsako realno število
  lahko odločimo, ali je pozitivno ali nenegativno. Formulo \(x > 0 ∨ x ≤ 0\)
  označimo \(\alpo(x)\), formulo \(\for{x : ℝ}{\alpo(x)}\) pa z \(\alpo*\).
\end{definicija}

\begin{trditev}
  Velja veriga implikacij \(\lem* ⇒ \alpo* ⇒ \lpo*\).
\end{trditev}
\begin{dokaz}
  Ker je \(¬(x > 0)\) natanko \(x ≤ 0\), je \(\alpo*\) očitno posledica
  \(\lem*\).

  Za drugo implikacijo pa potrebujemo malo dela. Naj bo \(α\) zaporedje, za
  katerega odločamo, ali ima na kakem mestu enico. Definirajmo realno število
  \[ x ≔ \lim_{n → ∞}2^{-\min\set{k : ℕ}{α(k) = 1 ∨ k ≥ n}}\text. \]

  To je po definiciji Cauchyjevo realno število. Uporabimo sedaj predpostavko
  \(\alpo(x)\). Če velja \(x ≤ 0\), mora zaporedje limitirati proti \(0\), kar
  pa pomeni, da \(α(k) = 1\) ni nikoli zadoščeno, torej je \(α = 0\).
  Po drugi strani, če je \(x > 0\) pa obstaja nek \(k : ℕ\), tako da je
  \(x > 2⁻ᵏ\). To pa pomeni, da ima \(α\) enico na enem izmed prvih \(k\)
  mestih, kar konstruktivno pomeni, da želeni indeks obstaja.
\end{dokaz}

Pokazati moramo zgolj, da gornje implikacije niso obrnljive.
\begin{trditev}
  Nad \(2^ℕ\) ne velja \(\lpo*\).
\end{trditev}
\begin{dokaz}
  Naj bo \(α : 2^ℕ → 2^ℕ\) identiteta. Resničnostna vrednost \(\lpo(α)\) je
  \(2^ℕ⧵\{0\}\), saj je resničnostna vrednost \(α = 0\) enaka \(∅\).
\end{dokaz}

\begin{trditev}
  Nad \(ℝ\) velja \(\lpo*\) in ne velja \(\alpo*\).
\end{trditev}
\begin{dokaz}
  Naj bo \(x : ℝ\) zunaj identiteta. Potem je \(\i{\alpo(x)} = ℝ⧵\{0\}\).

  Naj bo sedaj \(α : 2^ℕ\). TODO: a res še enkat…
\end{dokaz}

\begin{trditev}
  Nad \(\Ncof\) velja \(\alpo*\) in ne velja \(\lem*\).
\end{trditev}
\begin{dokaz}
  Prostor očitno ni diskreten. Naj bo \(x : ℝ\), torej preslikava \(\e x → ℝ\).
  Naj bo \(t, t' ∈ \im x\). Potem sta \(x⁻¹(t)\) in \(x⁻¹(t')\) neprazni odprti
  množica. To pa pomeni, da sta obe kokončni, torej imata neprazen presek, od
  koder sledi, da sta si enaka. Torej je vsako notranje realno število
  konstantno. Sedaj pa lahko za to realno število \(\alpo(x)\) odločimo zunaj.
\end{dokaz}

\begin{trditev}
  Nad \(\Ncof\) velja \(\wlem*\) in ne velja \(\lem*\).
\end{trditev}
\begin{dokaz}
  Prostor še vedno ni diskreten. Naj bo sedaj \(U\) odprta množica.
  Njena zunanjost (torej negacija) je pa bodisi cel prostor, bodisi prazna. Za
  te odprte množice pa velja da so komplementirane, torej \(\wlem*\) drži.
\end{dokaz}


\subsection{Realna števila}

Spomnimo se nekaj definicij realnih števil.
\begin{definicija}[Dedekindova realna števila]
  Par \(\p{L, U} ∈ 𝒫(ℚ)×𝒫(ℚ)\) je \emph{Dedekindov rez}, ko velja
  \begin{align}
    L \text{ je poseljen}\\
    U \text{ je poseljen}\\
    L \text{ je navzdol zaprt}\\
    U \text{ je navzgor zaprt}\\
    L \text{ je navzgor odprt}\\
    U \text{ je navzdol odprt}\\
    \text{za } a < b \text{ je } a ∈ L ∨ b ∈ U\label{real:located}\\
    \text{za } a ∈ L ∧ b ∈ U \text{ je } a < b
  \end{align}
  Množica Dedekindovih rezov tvori \emph{Dedekindova realna števila}, ki jih
  označimo z \(\Rd\).
\end{definicija}
\begin{opomba}
  Pogoju~\ref{real:located} pravimo \emph{lociranost} in če velja za katerikoli
  racionalni števili \(a\) in \(b\), bo veljala tudi za vse druge izbire.

  Res, relacija \(<\) je invariantna na afine preslikave realne premice, tako da
  lahko vsak interval \(\p{a,b}\) preslikamo (recimo) na \(\p{0,1}\) in nazaj.
\end{opomba}

\begin{definicija}[Cauchyjeva realna števila]
  Zaporedje racionalnih števil \((xₙ)ₙ\) je \emph{Cauchyjevo}, ko za vsaka
  \(i,j : ℕ\) velja \(|xᵢ - xⱼ| ≤ 2⁻ⁱ+2⁻ʲ\).

  % \emph{Modulus konvergence} je preslikava \(α : ℚ₊ → ℕ\), za katero za vsak
  % \(n : ℕ\) obstaja \(ε : ℚ₊\), da je \(α(ε) ≥ n\).

  % Zaporedje racionalnih števil \(\p{xₙ}\) z modulusom konvergence \(α\) je
  % \emph{Cauchyjevo}, ko za vsake \(ε : ℚ₊\) in \(i,j ≥ α(ε)\) velja
  % \(|xᵢ - xⱼ| < ε\).

  Dve Cauchyjevi zaporedji predstavljata isto realno število,
  ko za vse \(i : ℕ\) velja \(|xᵢ - yᵢ| ≤ 2⁻ⁱ⁺¹\).
  % ko obstaja modulus
  % konvergence \(α\), da za \(ε : ℚ₊\) in \(i ≥ α(ε)\) velja \(|xᵢ - yᵢ| ≤ ε\). 

  Množica Cauchyjevih zaporedij kvocientno z gornjo relacijo tvori
  \emph{Cauchyjeva realna števila}, ki jih označimo z \(\Rc\).
\end{definicija}

\begin{trditev}
  Vsako Cauchyjevo realno število je tudi Dedekindovo.
\end{trditev}
\begin{dokaz}
  Vzemimo \(L ≔ \set{q : ℚ}{q < x}\) in \(U ≔ \set{r : ℚ}{x < r}\).
  Očitno ta zadoščata prvim šestim pogojem zgoraj, in prav tako zadoščata
  zadnjemu pogoju, tako da se osredotočimo zgolj na zadnji pogoj.

  Pokazati moramo, da za vsako Cauchyjevo realno število \(x\) velja
  \(x > 0 ∨ x < 3\).

  %To pa pomeni, da mora obstajati nek indeks \(N : ℕ\), da so \(xᵢ > 0\) za vse
  %\(i ≥ N\), ali pa \(xᵢ < 3\) za vse \(i ≥ N\).
  Za racionalno število \(x₀\) velja \(x₀ > 1 ∨ x₀ < 2\), saj imajo racionalna
  števila odločljivo neenakost. Sedaj pa vemo, da za vse \(i : ℕ\) velja
  \(|x₀ - xᵢ| ≤ 1\), torej če je \(x₀ > 1\) bodo vsi \(xᵢ > 0\), po drugi strani
  pa če je \(x₀ < 2\) bodo vsi \(xᵢ < 3\), kar pa zaključi dokaz.
\end{dokaz}

Obrat pa ne velja konstruktivno. Izračunajmo najprej, kaj so realna števila v
topoloških modelih.
\begin{trditev}
  Nad \(X\) je objekt Dedekindovih realnih števil \(𝒪X\)-množica iz
  primera~\ref{ex:reals}.
  % Nad \(X\) je objekt Dedekindovih realnih števil \(𝒪X\)-množica
  % \(\set{f : U → ℝ}{U ∈ 𝒪X\text{, }f\text{ zvezna}}\), z enakostjo definirano
  % kot \(\i{f = g} ≔ \int{\set{t ∈ X}{f(t) = g(t)}}\).
\end{trditev}
\begin{dokaz}
  Naj bo \(x = \p{L, U}\) dedekindov rez v interni logiki.
  Sedaj lahko zunaj za \(t ∈ \e x\) definiramo
  \[ Lₜ ≔ \set{q : ℚ}{t ∈ \i{q ∈ L}}\text{ in} \]
  \[ Uₜ ≔ \set{r : ℚ}{t ∈ \i{r ∈ U}}\text. \]
  Ta tvorita Dedekindov rez, tako da nam \(t ↦ \p{Lₜ, Uₜ}\) definira preslikavo
  iz \(\e x → ℝ\). TODO: Zveznost.

  Obratno, če je \(x : V → ℝ\) zvezna preslikava, lahko tvorimo preslikavi
  \[ L(t) ≔ \set{q : ℚ}{q < x(t)}\text{ in} \]
  \[ U(t) ≔ \set{r : ℚ}{x(t) < r}\text. \]
  Ti znotraj tvorita Dedekindov rez \(\p{L, U}\).
\end{dokaz}

\begin{trditev}
  Naj bo prostor \(X\) lokalno povezan. Tedaj je nad \(X\) objekt Cauchyjevih
  realnih števil \(𝒪X\)-množica \(\c ℝ\).
\end{trditev}
\begin{dokaz}
  Pokazati je zares treba, da je objekt Cauchyjevih realnih števil \(\g{\c ℝ}\),
  torej da so lokalno konstantne preslikave v \(ℝ\).

  Podobno kot zgoraj nad \(t ∈ \e x\) dobimo Cauchyjevo zaporedje \((xₙ(t))ₙ\),
  torej preslikavo \(\e x → ℝ\). Ker je \(X\) lokalno povezan ima pokritje iz
  povezanih množic, tako da brez škode za splošnost predpostavimo, da je
  \(\e x\) povezan. Ker pa je povezan, pa vemo, da so vsi \(xᵢ\) konstantne
  preslikave, torej morajo biti vsi \(xₙ(t)\) enaki (za vse \(t ∈ \e x\)), tako
  da je \(t ↦ (xₙ(t))ₙ\) konstantna.

  Obratno pa, če je \(f : U → ℝ\) konstantna preslikava, recimo konstantno
  \({x ∈ ℝ}\), ima potem pripadajoče Cauchyjevo zaporedje \((xₙ)ₙ\). Sedaj pa
  lahko tvorimo konstantne preslikave \(xₙ : U → ℚ\), ki torej znotraj tvorijo
  Cauchyjevo zaporedje.
\end{dokaz}

\begin{trditev}
  Nad \(ℝ\) sta objekta Dedekindovih in Cauchyjevih realnih števil različna.
\end{trditev}
\begin{dokaz}
  Preslikava \(t ↦ t\) je Dedekindovo realno število, ki pa ni lokalno
  konstantno, torej ni Cauchyjevo.
\end{dokaz}

To pomeni, da lahko na ``\(\Rd = \Rc\)'' in ``\(\Rc = \c ℝ\)'' gledamo kot
nekakšna nekonstruktivna principa. Iz konstruktivne matematike pa poznamo tudi
druge konstrukcije realnih števil, ki nam lahko dajo nove principe kot ta
zgoraj.

TODO: cite nlab? elephant?
\begin{definicija}[MacNeillova realna števila]
  Par \(\p{L, U} ∈ 𝒫(ℚ)×𝒫(ℚ)\) je \emph{Dedekind-MacNeilleov rez}, ko velja
  \begin{align}
    L \text{ je poseljen}\\
    U \text{ je poseljen}\\
    L \text{ je navzdol zaprt}\\
    U \text{ je navzgor zaprt}\\
    L \text{ je navzgor odprt}\\
    U \text{ je navzdol odprt}\\
    L = \int{\p{Uᶜ}}\\
    U = \int{\p{Lᶜ}}
  \end{align}
  Množica Dedekind-MacNeilleovih rezov tvori \emph{MacNeillova realna števila}, ki jih
  označimo z \(\Rm\).
\end{definicija}

\begin{trditev}
  Vsako Dedekindovo realno število je tudi MacNeillovo.
\end{trditev}
\begin{dokaz}
  Točke \(1\) skozi \(6\) so enake, tako da je treba pokazati zgolj zadnji dve
  lastnosti. Ker sta simetrični, pokažimo zgolj zadnjo.

  Če je \(a ∈ L\) potem ni v \(U\), torej je v \(Uᶜ\). Sedaj pa, ker je \(L\)
  navzgor odprt obstaja \(a' > a\), ki je tudi v \(L\) (torej v \(Uᶜ\) po enakem
  argumentu.) Sedaj pa vemo, da je \(Uᶜ\) navzdol zaprt, torej je
  \(a ∈ \p{-∞,a'} ⊆ Uᶜ\) in je \(a ∈ \int{\p{Uᶜ}}\).
\end{dokaz}

V obratno smer lahko pokažemo zgolj, da sta \(L\) in \(U\) disjunktna.
\begin{lema}
  Za MacNeillovo realno število \(\p{L,U}\) velja \(a∈L∧b∈U⇒a<b\).
\end{lema}
\begin{dokaz}
  Naj bo \(a∈L\) in \(b∈U\). Potem je \(b ∈ \int{\p{Lᶜ}} ⊆ Lᶜ\), torej ni v
  \(L\). Ker je \(L\) navzdol zaprt je torej \(b\) zgornja meja, in je večji od
  \(a\).
\end{dokaz}

Ostalo bi torej pokazati zgolj lociranost vsakega MacNeillovega realnega
števila, a to žal ni mogoče.
\begin{lema}\label{th:Rm-sup}
  MacNeillova realna števila so polna.
\end{lema}
\begin{dokaz}
  TODO
\end{dokaz}

TODO: cite elephant
\begin{trditev}\label{real:Rm-maps}
  Objekt MacNeillovih realnih števil je natanko \(ℒ\)-množica parov funkcij
  \(\p{\uline f, \bar f}\) tipa \(U → ℝ\) za \(U ∈ 𝒪X\), za katere velja
  \begin{align*}
    \bar f(x)   &= \inf\set{\sup \uline f(V)}{V\nbd x}\text{ in}\\
    \uline f(x) &= \sup\set{\inf \bar   f(V)}{V\nbd x}\text.
  \end{align*}
\end{trditev}
\begin{dokaz}
  TODO: izumi?

  Če je \(x = \p{L,U}\) Dedekind-MacNeilleov rez, lahko za \(L\) in \(U\)
  konstruiramo funkciji \(\uline f\) in \(\bar f\) kot pri Dedekindovih rezih,
  s tem da slikamo v enostranske reze zunaj. Klasično so te ekvivalentni
  Dedekindovim rezom, tako da dobimo preslikavi \(\e x → ℝ\).

  Naj bo \(t ∈ \e x\). Ker je \(L = \int{Uᶜ}\), je
  \(\uline f(t) = \int{\bar f(t)ᶜ} = \int{\set{r:ℚ}{t∈\i{r∈U}}ᶜ}\).
  Komplement notranje množice je pa kar \(\set{r:ℚ}{t∉\i{r∈U}}\), čigar
  notranjost je \(\set{r:ℚ}{t∈\i{r∉U}}\).

  TODO: končaj to

  % % i'm oopid, to ni treba…
  % Pokažimo zgolj, da je \(\uline f\) navzdol polzvenzna, saj je dokaz
  % simetričen. Naj bo \(a ∈ ℝ\). Potem je
  % \(\uline f⁻¹(a,∞) = \set{t ∈ \e x}{f(t) > a}\).
  % Po definiciji je \(\uline f(t) = \set{q : ℚ}{t ∈ \i{q < x}}\), in ta je večji
  % od \(a\) ko je \(a\) element \(f(t)\), torej ko velja \(t ∈ \i{a < x}\). Sledi,
  % da je \(\uline f⁻¹(a,∞) = \i{a < x}\), torej je navzdol polzvezna.

  Obratno, če je pa \(\p{\uline f, \bar f}\) tak par,
  TODO: Končaj to
\end{dokaz}

\begin{trditev}
  Nad \(\p{3,\{∅, \{1\}, \{2\}, \{1,2\}, \{0,1,2\}\}}\) ne velja \(\Rm = \Rd\).
\end{trditev}
\begin{dokaz}
  Naj bo \(x ≔ \sup\set{x : \Rm}{x = 0 ∨ x = 1∧\{1\}}\).
  To je po lemi~\ref{th:Rm-sup} MacNeillovo realno število, saj je poseljeno z
  \(0\) in omejeno z \(1\).

  Poglejmo sedaj, če velja \(x < 1 ∨ x > 0\). Če velja \(x < 1\), je potem
  \(x = 0\), torej \(\{1\}\) ne drži. Sledi, da je \(\i{x < 1} = \{2\}\).
  Podobno lahko sklepamo, da je \(\i{x > 1} = \{1\}\). To pa pomeni, da \(x\) ni
  locirano pri \(0\), torej ni Dedekindovo.
\end{dokaz}
\begin{dokaz}[Alternativni dokaz]
  Definirajmo MacNeillovo realno število s preslikavama
  \begin{align*}
    \uline f(0) = 0 && \bar f(0) = 1\\
    \uline f(1) = 1 && \bar f(1) = 1\\
    \uline f(2) = 0 && \bar f(2) = 0
  \end{align*}

  Ti očitno zadoščata pogojem zgoraj, in se na \(0\) ne ujemata, tako da ne
  definirata Dedekindovega realnega števila.
\end{dokaz}

Temu principu bomo torej pravili ``\(\Rm = \Rd\)''.


\subsection{Ostali principi}

Seveda pa se principi ne delijo popolnoma zgolj na principe izbire in odločitve.

\begin{definicija}
  \emph{Kripkejeva shema za \(Σ ⊆ Ω\)} pravi, da je \(Σ = Ω\). Formulo
  \(\for{p : Ω}{\exist{s : Σ}{s = p}}\) bomo označili z \(\ks*(Σ)\).

  Navadna \emph{kripkejeva shema} je kripkejeva shema za
  \(Σ₀¹ ≔ \set{α \apart 0}{α : 2^ℕ} ⊆ Ω\).

  TODO: dedekindova ali cauchyjeva ali kaj drugega.
  \emph{Analitična kripkejeva shema} je kripkejeva shema za
  \(Σ_ℝ ≔ \set{x > 0}{x : ℝ} ⊆ Ω\).
\end{definicija}

%%% Local Variables:
%%% mode: latex
%%% TeX-master: "main"
%%% End:
