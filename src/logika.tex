\section{Nekonstruktivni principi}\label{sec:logika}

\section{Principi izbire v topoloških modelih}\label{sec:izbire}

Od tu naprej bomo ponovno delali v klasični metateoriji. Čeprav bi bilo lepo
razviti ekvivalence konstruktivno, nas zanimajo topološke lastnosti, te se pa
običajno razvija v klasični metateorji.

Spodnji rezultat je v malo drugačni obliki pokazal Alex Simpson~\cite{Simpson24},
dokaz v primeru topoloških modelov je pa originalen.
\begin{izrek}\label{th:ac-and-conn-is-pgt}
  Nad \(X\) velja \(\AC(\c A, \c B)\) in \(\c A ↬ \c B ≅ \c{B^A}\) natanko
  tedaj, ko ima vsaka \(A\)-indeksirana množica \(B\)-indeksiranih
  pokritij skupno pofinitev.
  %Nad \(X\) velja \(\AC_Σ{\p{\c A, \c B}}\) in \({\c B}^{\c A} ≅ \c{B^A}\)
  %natanko tedaj, ko je Grothendieckova topologija \(X\) zaprta za \(A\)-mnoge
  %preseke \(B\)-indeksiranih \(Σ\)-krovov.
\end{izrek}
\begin{dokaz}
  \begin{itemize}
  \item[\(\p ⇐\)]
    Pokažimo najprej, da velja gornja enakost.
    Tip \(\c A ↬ \c B\) bo enak tipu \(\c{A → B}\) natanko tedaj, ko je vsak
    morfizem operacija. To velja, saj je vsaka funkcija množic operacija med
    konstantnimi tipi. Naj bo \(f : \c A ↬ \c B\). Potem
    \(Cₐ ≔ \set{f(a,b)}{b∈B}\) tvorijo pokritje \(\e a = X\). Po predpostavki
    imajo skupno pofinitev \(C\), tako da za vsak \(U ∈ C\) obstaja \(b_{a,U}\),
    tako da velja \(U ⊩ f(a,b_{a,U})\). Potem pa lahko na \(U\) definiramo
    preslikavo \(a ↦ b_{a,U}\), ki je enaka \(f\), torej izomorfizem
    \(\c A ↬ \c B ≅ \c{B^A}\) velja.

    Naj bo sedaj \(R\) celovita relacija med \(\c A\) in \(\c B\).
    To pomeni, da za vsak \(a ∈ A\) obstaja \(B\)-indeksirano pokritje \(Cₐ\),
    tako da za vsak element \(U ∈ Cₐ\) obstaja tak \(b_{a, U} ∈ B\), da velja
    \(U ⊩ R(a, b_{a, U})\).

    Po predpostavki vemo, da imajo te krovi skupno pofinitev \(C\), torej
    za vsak \(U ∈ C\) in \(a ∈ A\) lahko izmed \(b_{a,Uₐ}\), kjer je \(Uₐ\)
    nadmnožica \(U\) v \(Cₐ\), izberemo \(b_{a, U}'\). Potem pa lahko na \(U\)
    definiramo funkcijo tipa \(A → B\), ki slika \(a\) v \(b_{a,U}'\).
  \item[\(\p ⇒\)]
    Naj bodo \(Cₐ = \{U_{a,b}\}_{b ∈ B}\) pokritja.
    Ta definirajo relacijo \(R(a, b) ≔ U_{a,b}\), ki je celovita, saj so \(Cₐ\)
    pokritja. Tedaj po principu izbire in predpostavki o morfizmih obstaja
    pokritje \(C\), da na vsakem \(V ∈ C\) obstaja \emph{funkcija} \(f : A → B\),
    za katero velja \(V ⊩ \for{a : \c A}{U_{a,f(a)}}\). To pa pomeni, da je
    \(C\) skupna pofinitev pokritij \(Cₐ\), kar zaključi dokaz.
  \end{itemize}
\end{dokaz}
%NOTE: brez meta-ac je leva stran lahko za vse celovite \(R\) velja
%\[ \eventually{V⊆U}{\for{a∈A}{\exist{b∈B}{V⊩R(a,b)}}} \]
%NOTE: Mogoče tu rabimo \(\g B\) namesto \(B\).
%NOTE: interno je to \(\for{a:A}{\globalen{b:B}{R(a,b)}}\)
\begin{posledica}
  Nad \(X\) velja \(\AC(\c A)\) in \(\for{B}{\c A ↬ B ≅ \c{A ↝ B}}\) natanko
  tedaj, ko ima vsaka \(A\)-indeksirana množica pokritij skupno pofinitev.
\end{posledica}
To zares ni posledica, a je dokaz enak, tako da ga ne ponovimo.

\begin{opomba}
  Gornje lahko relativiziramo še glede na Sierpinskijev objekt \(Σ\), a se malo
  zaplete pri pogoju \(\c A ↬ \c B ≅ \c{B^A}\), tako da to pozabimo.
\end{opomba}

Ta lastnost o pofinitvah pa ni zares tipična topološka lastost, saj ponavadi
govorimo o pofinitvi specifičnega tipa \emph{enega} pokritja. Definicija izhaja
iz teorije Grothendieckovih topologij, kjer definiramo \emph{krov} kot navzdol
zaprto pokrtije, nato je pa Grothendieckova topologija na \(U\) enaka množici
krovov \(U\). Potem se lastnost zgoraj glasi ``Grothendieckova topologija na
\(X\) je zaprta za \(A\)-mnoge preseke''. To pa že zgleda bolj podobno čemu
znanemu, namreč prostori, katerih topologije so zaprte za števne preseke imajo
ime.

\begin{definicija}\label{def:psp}
  Prostor je \emph{P-prostor}, ko je števen presek odprtih množic odprt.
\end{definicija}

Ta lastnost prostora potem implicira tisto o Grothendieckovi topologiji.

\begin{trditev}\label{th:psp-is-pgt}
  Če je \(X\) P-prostor, ima vsaka množica števno mnogo pokritij \(X\) skupno
  pofinitev.
\end{trditev}
\begin{dokaz}
  Naj bodo \(Cₙ = \{U_{n,i}\}ᵢ\) pokritja. Brez škode za splošnost
  predpostavimo, da so vsa indeksirana z nekim \(I\), saj lahko pokrijim dodamo
  prazno množico.

  Potem naj bo \(φ : ℕ → I\) funkcija. Za to lahko tvorimo
  \(U_φ ≔ ⋂_nU_{n,φ(n)}\). To je števen presek odprtih množic, torej je po
  predpostavki odprta množica. Prav tako je pa vsak \(x ∈ X\) vsebovan v nekem
  \(U_{n,i}\) za vsak \(n ∈ ℕ\), torej po principu izbire v metateoriji obstaja
  preslikava \(φ : ℕ → I\), da je \(x ∈ U_φ\). To pomeni, da \(U_φ\) tvorijo
  pokritje, ki je pofinitev vsakega \(Cₙ\).
\end{dokaz}
\begin{posledica}\label{th:psp-has-cc}
  Nad P-prosotri velja \(\CC\).
\end{posledica}
Seveda se vse to posploši na poljbne kardinalnosti, a v literaturi razen za
števni primer nimamo imena. Poznamo le še prostore Aleksandrova, ki so natanko
prostori, kjer so \emph{vsi} preseki odprtih množic odprti. Ta nam da idejo, da
potem nad temi prostori velja polnomočen \(\AC\), a žal temu ni tako.

Očitno imajo diskretni prostori to lastnost, in kot smo že povedali, so to
natanko prostori, nad katerimi velja \(\lem*\). Ker pa velja aksiom izbire
implicira princip izključene tretje možnosti (ta izrek je znan kot izrek
Diaconescuja~\cite{Bauer16}), mora vsak prostor, nad katerim velja \(\AC\) biti
diskreten.

Očitno torej \(\AC\) ne more veljati nad \emph{vsemi} prostori Aleksandrova, saj
obstajajo taki, ki niso diskretni. In res, gornje nam da zgolj, da za vsako
množico \(A\) velja \(\AC(\c A)\), ne da velja \(\AC(A)\) za vse \emph{tipe}
\(A\). To je torej pomembna razlika, kar je do neke mere tudi za pričakovati.
Zgoraj smo bistveno uporabljali, da so elementi \(\c A\) definirani na istem
nivoju \(X\), torej so tudi pokritja \(Cₐ\) pokritja \(X\). Za splošen tip \(A\)
bodo pa pokritja \(Cₐ\) pokrivala \(\e a\), torej ne bomo mogli niti govoriti o
skupnih pofinitvah. Avtorici ni znana nobena karakterizacija \(\AC\), razen, da
ta ni topološka.

Torej gornje pokaže, da ima P-prostor gornjo lastnost na Grothendieckovi
topologiji. Obrat pa ne velja.
\begin{trditev}\label{th:psp-is-not-pgt}
  Nad prostorom \(\cli{0,1}\) s topologijo \(\set{[0,a)}{a ∈ \cli{0,1}}\) ima
  poljubno mnogo pokritij skupno pofinitev, a ta ni P-prostor.
\end{trditev}
\begin{dokaz}
  Množice \(Uₙ ≔ [0,2⁻ⁿ)\) so odprte, a njihov presek je \(\{0\}\), ki ni
  odprta množica, torej prostor ni P-prostor.

  Sedaj pa, če je \(C\) pokritje \(X\) more pokriti \(1\). Ampak edina okolica
  \(1\) je \(\cli{0,1}\), torej je \(\{\cli{0,1}\}\) pofinitev vsakega pokritja
  \(X\).
\end{dokaz}
To pravzaprav pomeni, da nad tem prostorom velja \(\for{A}{\AC(\c A)}\), 
vseeno ni {P-prostor}. Izkaže pa se, da pod predpostavko \(T₁\) lahko pokažemo
obrat.

\begin{trditev}\label{th:t1-pgt-is-psp}
  Če je prostor \(X\) \(T₁\) in so njegove Grothendieckove topologije zaprte za
  števne preseke, je P-prostor.
\end{trditev}
\begin{dokaz}
  Naj bo \(\{Uₙ\}ₙ\) števna družina odprtih množic in za \(t ∈ ⋂ₙUₙ\) tvorimo
  pokritja \(\{Uₙ, X⧵{\{t\}}\}\). Ta so po predpostavki \(T₁\) odprta pokritja,
  torej imajo skupno pofinitev \(C\). Naj \(U ∈ C\) pokriva točko \(t\). Tedaj
  očitno ne velja \(U ⊆ X⧵{\{t\}}\), torej je \(U ⊆ Uₙ\). Potem je pa \(U\)
  odprta okolica \(t\), ki je v celoti vsebovana v \(⋂ₙUₙ\), torej je ta presek
  odprt.
\end{dokaz}
\begin{opomba}
  Vseeno pa to ne pomeni, da so \(T₁\) P-prostori natanko \(T₁\) prostori, nad
  katerimi velja \(\CC\). Namreč, prostor lahko validira \(\CC\), a ne validira
  enakosti iz~\ref{th:ac-and-conn-is-pgt}, torej ne bo P-prostor.
\end{opomba}

Znana je pa delna karakterizacija \(\DC\)~\cite[lema~D4.5.16]{Johnstone02}\cite[trd.~2.2]{HL16}.
\begin{definicija}
  Prostor \(X\) je \emph{ultraparakompakten}, ko ima vsako pokritje prostora
  pofinitev s particijo.
\end{definicija}
\begin{trditev}
  Nad ultraparakompaktnimi prostori velja \(\DC\).
\end{trditev}
\begin{dokaz}
  Denimo, da smo že konstruirali \(xₙ\). Potem po celovitosti množice
  \(\i{R(xₙ,x)}\) tvorijo pokritje, torej po predpostavki obstaja pofinitev s
  particijo \(\{P_α\}_α\). Potem lahko uporabimo aksiom odvisne izbire v
  metateoriji, da za vsak \(α\) izberemo tak \(x_α\), da je \(P_α ⊆ \i{R(xₙ,x_α)}\).
  Potem pa lahko definiramo \(xₙ\) tako, da je na \(P_α\) enak \(x_α\). Ker
  \(P_α\) tvorijo pokritje, in so paroma disjunktni, to dobro definra term tipa
  \(X\).

  Ker smo začeli z globalnim \(x₀\), to definira globalno zaporedje \((xₙ)ₙ\).
\end{dokaz}
\begin{opomba}
  Gornjo trditev lahko preprosto posplošimo na \(\DC_Σ\), tako da od prostora
  zahtevamo, imajo \(Σ\)-pokrtija pofinitve s particijo.
\end{opomba}


% \begin{trditev}
%   Če nad \(X\) velja \(\AC_Σ(\c A, \c B)\) za vse \(B\), velja tudi
%   \(\AC_Σ(A, ℱ)\) za vse \(ℒ\)-množice \(ℱ\).
% \end{trditev}
% \begin{dokaz}
%   Naj bo \(F\) podležna množica \(ℱ\) in \(R : A×F → Σ\) celovita. Potem je tudi
%   celovita kot relacija med \(A\) in \(\c F\). Uporabimo princip izbire, da
%   dobimo preslikavo \(f: A → \c{\g F}\), za katero velja \(\for{a : A}{R(a, f(a))}\).

%   Pokažimo sedaj, da zunaj \(f\) definira \(ℒ\)-morfizem \(A ↬ ℱ\).
%   Če je \(b = b'\) in \(b' = f(a)\),
% \end{dokaz}


%\subsection{Števna izbira in P-prostori}

% \begin{lema}
%   Nad P-prostori velja \(\CC\).
% \end{lema}
% \begin{dokaz}
%   Trditev \(X ⊩ \AC{\p{ℕ, ℱ}}\) pravi, da če je \(P\) celovita relacija med \(ℕ\) in
%   \(ℱ\), potem obstaja funkcija \(f : ℕ → ℱ\), ki je podrelacija \(P\).

%   Naj bo torej \(P\) taka celovita relacija v \(\sh{X}\).
%   Relacija \(P\) je navzven zaporedje globalnih prerezov potenčnega snopa \(ℱ\).
%   To da je celovita pa pomeni, da imamo za vsak \(n ∈ ℕ\) indeksno množico
%   \(Iₙ\) in pokritje \(\{U_{n,i}\}_{i ∈ Iₙ}\), skupaj z lokalnimi prerezi
%   \({f_{n,i} ∈ ℱ{\p{U_{n,i}}}}\), tako da velja \( U_{n,i} ⊩ Pₙ{\p{f_{n,i}}}\).

%   Obstoj funkcije izbire pa pomeni, da mora obstajati pokritje \(\{Vⱼ\}ⱼ\) in
%   zaporedje prerezov \(f_{n,j} ∈ ℱ{\p{Vⱼ}}\), tako da velja
%   \(Vⱼ ⊩ \for{n : ℕ}{Pₙ{\p{f_{n,j}}}}\).

%   Za odvisno funkcijo \(φ : ∏_{n ∈ ℕ} Iₙ\) definirajmo množico
%   \(U_φ ≔ ⋂_{n ∈ ℕ} U_{n,φ(n)}\), ki je števen presek odprtih množic torej po
%   predpostavki odprt. Množice \(U_φ\) pokrijejo prostor, saj je vsak \(x ∈ X\)
%   vsebovan v nekem \(U_{n, i}\) za vse \(n ∈ ℕ\) torej po aksiomu števne izbire
%   (v metateoriji) obstaja tudi funkcija \(φ\), tako da bo \(x ∈ U_{n, φ{n}}\),
%   torej v \(U_φ\).

%   Še več, za vse \(n\) in \(φ\) velja
%   \[ U_φ ⊆ U_{n,φ(n)} ⊩ Pₙ{\p{f_{n,φ(n)}}}\text. \]

%   To pa pomeni, da za obstoj funkcije izbire vzamemo krov \(\{U_φ\}_φ\) in
%   preslikavo \(φ ↦ \p{f_{n, φ(n)} ∈ ℱ{p{U_φ}}}ₙ\), ki zadošča želenemu pogoju,
%   kot smo pokazali zgoraj.
% \end{dokaz}
% \begin{opomba}
%   Gornji dokaz ključno uporabi askiom števne izbire v metateoriji.
% \end{opomba}


% \begin{lema}\label{th:t1-ccv-is-psp}
%   Če nad \(T₁\) prostorom velja \(\CCv\), je prostor P-prostor.
% \end{lema}
% \begin{dokaz}
%   Naj torej velja \(X ⊩ \CCv\) in naj bo \(\{Uₙ\} ⊆ 𝒪X\) števna družina ter
%   \(a ∈ ⋂ₙ Uₙ\) poljubna točka. Dokazati želimo, da ta točka leži v notranjosti
%   preseka, kar bo pokazalo, da je odprt. Definirajmo interno relacijo \(R\) med
%   \(ℕ\) in \(2\) s predpisom
%   %\(X ⊩ R(n, b) ⇔ (b = 1 ∧ Uₙ) ∨ (b = 0 ∧ !a)\).
%   \[
%   R(n, b) ⇔
%   \begin{cases}
%     Uₙ &; b = 1\\
%     !a &; b = 0\text.
%   \end{cases}
%   \]
%   Ker so v \(T₁\) prostorih točke zaprte, je \(!a = X⧵\{a\}\).
%   Ker je \(Uₙ∪{!a} = Uₙ∪(X⧵\{a\}) = X\), je ta relacija celovita, torej lahko na
%   njej uporabimo dani aksiom izbire. Tako v okolici \(W\nbd a\) dobimo funkcijo
%   \({W ⊩ f : ℕ → 2}\), za katero velja \(W ⊩ \for{n : ℕ}{R(n, f(n))}\).

%   Ker je pa to geometrijska implikacija, pa velja tudi \(a ⊩ \for{n : ℕ}{R(n, f(n))}\).
%   Vendar pa za noben \(n\) ne velja \(a ⊩{!a} (= R(n, 0))\), torej je \(f = 1\).
%   Potem pa velja \(a ∈ W ⊆ ⋂ₙ ⟦R(n, f(n))⟧ = ⋂ₙ ⟦R(n, 1)⟧ = ⋂ₙ Uₙ\).
% \end{dokaz}
% % \begin{dokaz}
% %   Naj je \(Uₙ\) števno pokritje \(U\) in \(a ∈ ⋂ₙ Uₙ\). Dokazujemo, da obstaja
% %   odprta okolica \(a\), ki je vsebovana v preseku.
% %   Konstruirajmo množice \(Vₙ ≔ ⋃_{k ≠ n} Uₖ ⧵ \{a\}\) in pokritja \(Cₙ ≔ ↓{\{Uₙ, Vₙ\}}\).
% %   Po predpostavki je tudi presek \(C ≔ ⋂ₙ Cₙ\) pokritje, torej za nek \(W ∈ C\)
% %   velja \(a ∈ W\). Potem pa za vse \(n\) velja \(W ∈ Cₙ\), torej imamo \(W ⊆ Uₙ\)
% %   ali \(W ⊆ Vₙ\).

% %   Ker je pa \(a ∈ W\) in \(a ∉ Vₙ\), je potem nujno \(W ⊆ Uₙ\), torej dobimo
% %   \(a ∈ W ⊆ ⋂ₙ Uₙ\), kar je natanko kar smo želeli.
% % \end{dokaz}
% %\begin{opomba}
% %  Gornji dokaz deluje tudi za \(R₀\) prostore, saj lahko \(Vₙ\) definiramo z
% %  zaprtjem točke \(a\), in potrebujemo zgolj, da je to vsebovano v vsaki okolici
% %  točke \(a\), kar je pa natanko ekvivalentno pogoju \(R₀\). Za občutek, \(T₁\)
% %  prostori so natanko \(T₀\) in \(R₀\) prostori.
% %\end{opomba}




% Gornji lemi lahko združimo, da dobimo
% \begin{trditev}
%   Nad \(T₁\) prostori je števna izbira ekvivalentna disjunktivni števni izbiri.
% \end{trditev}


%%% Local Variables:
%%% mode: latex
%%% TeX-master: "main"
%%% End:

%\section{Principi odločitve v topoloških modelih}\label{sec:odločitve}

V podrazdelku~\ref{sec:modeli-logika-odprtih} smo karakterizirali \(\lem*\) in
\(\wlem*\) z logiko odprtih množic. V tej logiki igra množica \(𝒪X\) vlogo
``tipa resničnostnih vrednosti'', v topoloških modelih smo pa v
primeru~\ref{ex:omega} tip resničnostnih vrednosti definirali kot tip \(Ω\).
Izkaže se, da se interpretacije formul kvantificiranih po \(Ω\) ne spremenijo,
če \(Ω\) zamenjamo z \(\c{𝒪X}\), saj so vsi elementi \(Ω\) globalni, formuli pa
ne vsebujeta enakosti. Tako trditvi~\ref{th:lem-is-discrete}
in~\ref{th:wlem-is-ext-disc} veljata kot navedeni tudi v topoloških modelih.
\begin{retrditev}{th:lem-is-discrete}
  Nad topološkim prostorom velja princip izključene tretje možnosti natanko
  tedaj, ko je prostor diskreten.
\end{retrditev}
\begin{retrditev}{th:wlem-is-ext-disc}
  Nad topološkim prostorom velja DeMorganov zakon natanko tedaj, ko je prostor
  ekstremalno nepovezan.
\end{retrditev}
Ker so dokazi enaki, jih tu ne ponovimo.

S pomočjo tipov pa lahko povemo tudi kaj o principih števne odločitve.

\begin{trditev}\label{th:lpov-lpo}
  Nad lokalno povezanimi prostori velja \(\lpo*\).
\end{trditev}
\begin{dokaz}
  Naj bo \(α : 2^ℕ\). Po trditvi~\ref{th:lpov-exponentiable} je \(α\) lokalno
  funkcija \(ℕ → 2\). Ker imamo zunaj \(\lpo*\), lahko odločimo
  \(α = 0 ∨ α \apart 0\) tam, kar pa pomeni, da to lahko odločimo tudi znotraj.
\end{dokaz}
To je zgolj delna karakterizacija, tako da se k tej še vrnemo.

\begin{izrek}\label{th:alpo-is-zerosets-open}
  Nad \(X\) velja \(\alpo*\) natanko tedaj, ko je vsaka ničelna množica odprta.
\end{izrek}
\begin{dokaz}
  Če je vsaka ničelna množica odprta je \(\i{x = 0}\) enak ničelni množici,
  torej skupaj z \(\i{x\apart 0}\) pokrijeta prostor.

  Obratno, če velja \(\alpo*\), morata \(\i{x=0}\) in \(\i{x\apart 0}\) pokriti
  cel prostor. Ker je ničelna množica disjunktna množici \(\i{x\apart 0}\), je
  enaka \(\i{x=0}\), torej je odprta.
\end{dokaz}

Sedaj lahko pokažemo implikacije iz podrazdelka~\ref{sec:logika-odločitve} v
splošnem niso obrnljive.
\begin{trditev}
  Nad Cantorjevim prostorom ne velja \(\lpo*\).
\end{trditev}
\begin{dokaz}
  Identiteta \(α : 2^ℕ → 2^ℕ\) je znotraj element tipa \(2^ℕ\) in zanjo je
  \(\i{\lpo(α)} = 2^ℕ⧵\{0\}\), torej \(\lpo*\) ne velja povsod.
\end{dokaz}

\begin{trditev}
  Nad \(ℝ\) velja \(\lpo*\) in ne velja \(\alpo*\).
\end{trditev}
\begin{dokaz}
  Ker je prostor \(ℝ\) lokalno povezan, po trditvi~\ref{th:lpov-lpo} nad njim
  velja \(\lpo*\). Poglejmo si sedaj identiteto \(\id : ℝ → ℝ\). Ta je znotraj
  Dedekindovo realno število in zanjo je \(\i{\alpo(\id)} = ℝ⧵\{0\}\), torej
  \(\alpo*\) ne velja nad \(ℝ\).
\end{dokaz}

Ker prostor \(\Ncof\) naravnih števil s kokončno topologijo ni diskreten, nad
njim ne velja \(\lem*\). To lahko izkoristimo, za ločevanje \(\lem*\) od
\(\alpo*\) in \(\wlem*\) v naslednjih trditvah. Ta prostor sem odkrila s pomočjo
matematične podatkovne baze~\cite{pibase}, in ga bomo še srečali.

\begin{trditev}
  Nad \(\Ncof\) velja \(\alpo*\) in ne velja \(\lem*\).
\end{trditev}
\begin{dokaz}
  Naj bo \(f : U → ℝ\) zvezna preslikava in \(x, x' ∈ \im f\). Potem sta
  \(f⁻¹(x)\) in \(f⁻¹(x')\) neprazni odprti množici, torej imata neprazen
  presek. Sledi, da je \(x = x'\), torej je funkcija \(f\) konstantna. Ker lahko
  potem v metateoriji odločimo \(\alpo(f(t))\) (za nek \(t ∈ U\)), lahko
  odločimo tudi \(\alpo(f)\) znotraj.
\end{dokaz}

\begin{trditev}
  Nad \(\Ncof\) velja \(\wlem*\) in ne velja \(\lem*\).
\end{trditev}
\begin{dokaz}
  Naj bo \(U\) odprta množica. Njena zunanjost (torej negacija) je pa bodisi cel
  prostor, bodisi prazna. Za te odprte množice pa velja, da so komplementirane,
  torej \(\wlem*\) drži.
\end{dokaz}

% Obrate teh implikacij lahko torej jemljemo kot nekonstruktivni principi, a o teh
% ni veliko znano. Vemo le, da očitno \(\Rd = \Rc\) implicira \(\lpo* ⇒ \alpo*\),
% saj je \(\lpo*\) ekvivalenten \(\alpo*\) za Cauchyjeva realna števila, tako da
% je to zelo šibek princip. Kasneje si bomo ogledali kdaj obrati veljajo na nek
% strožji in bolj strukturiran način, ki ima povezavo s teorijo izračunljivosti,
% in ta upamo, da nam da močnejše principe, ki bodo potem tudi topološko bolj
% zanimivi.

Oglejmo si še \(\awlpo*\). Ta pravi, da za vsak \(x : ℝ\) velja \(x≤0 ∨¬(x≤0)\).
Ker je \(x≤0\) natanko \(¬(x>0)\) je \(\awlpo*\) ekvivalenten \(\wlem*_{Σ_{\Rd}}\).
Če si pogledamo definicijo ekstremalno nepovezanih prostorov, in v njej zamenjamo
odprte množice z realnimi, dobimo sledečo lastnost.
\begin{definicija}
  Prostor je \emph{realno nepovezan}, ko je za vsako funkcijo \(f : X → ℝ\)
  množica \(\cl\set{t∈X}{f(t)>0}\) odprta.
\end{definicija}

\begin{trditev}\label{th:awlpo-is-basically-disconnected}
  Nad \(X\) velja \(\awlpo*\) natanko tedaj, ko je vsaka odprta podmnožica \(X\)
  realno nepovezana.
\end{trditev}
Dokaz trditve je enak kot~\ref{th:wlem-is-ext-disc}, tako da ga ne ponovimo.
Velja pa zanimiva posledica, namreč, da nad realno nepovezanimi prostori
obstaja funkcija "predznak" v naslednjem smislu.
\begin{trditev}
  Če nad \(X\) velja \(\awlpo*\), potem za vsak \(x : ℝ\) obstaja
  \(u : \e x → ℝ\), tako da je \(u\) nad \(\e x\) obrnljiv, in velja
  \(\e x ⊩ x = u\abs x\).
\end{trditev}
\begin{dokaz}
  Brez škode za splošnost naj bo \(\e x = X\).

  Princip \(\awlpo*\) potem pravi, da je množica \(U ≔ \cl{\i{x>0}}\) odprta,
  torej tvori particijo in lahko definiramo
  \[ u ≔
    \begin{cases}
       1&; U\\
      -1&; Uᶜ\text.
    \end{cases}
  \]
  %\(U ⊩ u = 1\) in \(Uᶜ ⊩ u = -1\).
  Ta \(u\) je obrnljiv, saj je \(u⋅u = 1\). Prav tako velja želena enačba, saj
  je \(U\) vsebovan v \(x ≥ 0\), in velja \(f ≤ 0\) na \(Uᶜ\).
\end{dokaz}

Če želimo zares definirati predznak, moramo uporabiti \(\alpo\). Ta
pravi, da je ničelna množica \(x\) odprta, torej lahko na njej definiramo
\[ u ≔
  \begin{cases}
     1&; x < 0\\
     0&; x = 0\\
    -1&; x > 0\text.
  \end{cases}
\]

\begin{definicija}
  Prostor \(X\) je \emph{skoraj P-prostor}, ko je za vsak \(f : X → ℝ\) množica
  \(\i{f > 0}\) regularna~\cite{Levy77}.
\end{definicija}
Seveda je vsak P-prostor tudi skoraj P-prostor.
Res, naj bodo \(Zᵢ ≔ f⁻¹[2⁻ⁱ,∞)\). Njihova unija \(Z = \i{f > 0}\) je zaprta,
saj je \(X\) P-prostor. Potem je pa \(\int{\p{\cl{Z}}} = \int Z = Z\).

\begin{trditev}\label{th:amp-is-almost-psp}
  Nad \(X\) velja \(\amp*\) natanko tedaj, ko je vsaka odprta podmnožica \(X\)
  skoraj P-prostor.
\end{trditev}
\begin{dokaz}
  Princip pravi, da za vsak tak \(f\) velja
  \(\int{\p{\cl{\i{f > 0}}}} ⊆ \i{f > 0}\). Ker obratna enakost očitno velja, je
  to ravno definicija regularnosti odprte množice.
\end{dokaz}
Analogno velja tudi \(\mp*\) natanko tedaj, ko je vsaka semiodločljiva odprta
množica regularna.
\begin{opomba}
  V~\cite[2.1]{Levy77} piše, da je vsaka odprta podmnožica skoraj P-prostora
  ``očitno'' skoraj P-prostor. Jaz tega ne vidim, mogoče je res samo za
  \(T_{3.5}\) prostore, na katere se omeji članek.
\end{opomba}

Konstruktivno velja \(\awlpo*∧\amp* ⇔ \alpo*\).
V trditvah~\ref{th:awlpo-is-basically-disconnected},~\ref{th:amp-is-almost-psp},
in~\ref{th:alpo-is-zerosets-open}, smo karakterizirali vse prostore, ki so v
igri v tej ekvivalenci, tako da bi morala veljati ekvivalenca med njimi.
\begin{izrek}
  Ničelne množice \(f : X → ℝ\) so odprte natanko tedaj, ko je \(X\) skoraj
  P-prostor in realno nepovezan.
\end{izrek}
\begin{dokaz}
  Če je vsaka množica \(\i{f > 0}\) zaprta, je tudi regularna (vse odprto zaprte
  množice so regularne). Prav tako je njeno zaprtje odprto.

  Obratno, če je \(\i{f > 0} = \int{\p{\cl{\i{f>0}}}}\), in je zaprtje
  \(\i{f>0}\) odprto, je potem \(\i{f>0} = \cl{\i{f>0}}\), torej je zaprta.
\end{dokaz}
Avtorici ni znano, če se ta izrek pojavi kje v literaturi, predvsem ker je
večina literature o teh prostorih objavljene pod predpostavko \(T_{3.5}\).
Pod predpostavko \(T_{3.5}\) torej najdemo izrek
"P-prostori so natanko realno nepovezani skoraj P-prostori" v~\cite{Levy77}
in~\cite[4J(3)]{GJ60}.
To porodi dve zanimivi vprašanji. Prvič, kako nujna je predpostavka \(T_{3.5}\)
za razvoj teorije kolobarjev realnih funkcij, in drugič, kaj lastnost
\(T_{3.5}\) pomeni v interni logiki topološkega modela. Avtorica žal nima
odgovora na nobeno od teh vprašanj, saj je prvo preobsežno za to delo, za
drugega pa ni našla pravega navdiha.


%%% Local Variables:
%%% mode: latex
%%% TeX-master: "main"
%%% End:


\subsection{Realna števila}\label{sec:logika-reals}

Spomnimo se nekaj definicij realnih števil.
\begin{definicija}[Dedekindova realna števila]
  Par \(\p{L, U} ∈ 𝒫(ℚ)×𝒫(ℚ)\) je \emph{Dedekindov rez}, ko velja
  \begin{align}
    L \text{ je poseljen}\\
    U \text{ je poseljen}\\
    L \text{ je navzdol zaprt}\\
    U \text{ je navzgor zaprt}\\
    L \text{ je navzgor odprt}\\
    U \text{ je navzdol odprt}\\
    \text{za } a < b \text{ je } a ∈ L ∨ b ∈ U\label{real:located}\\
    \text{za } a ∈ L ∧ b ∈ U \text{ je } a < b
  \end{align}
  Množica Dedekindovih rezov tvori \emph{Dedekindova realna števila}, ki jih
  označimo z \(\Rd\).
\end{definicija}
\begin{opomba}
  Pogoju~\ref{real:located} pravimo \emph{lociranost} in če velja za katerikoli
  racionalni števili \(a\) in \(b\), bo veljala tudi za vse druge izbire.

  Res, relacija \(<\) je invariantna na afine preslikave realne premice, tako da
  lahko vsak interval \(\p{a,b}\) preslikamo (recimo) na \(\p{0,1}\) in nazaj.
\end{opomba}

\begin{definicija}[Cauchyjeva realna števila]
  Zaporedje racionalnih števil \((xₙ)ₙ\) je \emph{Cauchyjevo}, ko za vsaka
  \(i,j : ℕ\) velja \(|xᵢ - xⱼ| ≤ 2⁻ⁱ+2⁻ʲ\).

  % \emph{Modulus konvergence} je preslikava \(α : ℚ₊ → ℕ\), za katero za vsak
  % \(n : ℕ\) obstaja \(ε : ℚ₊\), da je \(α(ε) ≥ n\).

  % Zaporedje racionalnih števil \(\p{xₙ}\) z modulusom konvergence \(α\) je
  % \emph{Cauchyjevo}, ko za vsake \(ε : ℚ₊\) in \(i,j ≥ α(ε)\) velja
  % \(|xᵢ - xⱼ| < ε\).

  Dve Cauchyjevi zaporedji predstavljata isto realno število,
  ko za vse \(i : ℕ\) velja \(|xᵢ - yᵢ| ≤ 2⁻ⁱ⁺¹\).
  % ko obstaja modulus
  % konvergence \(α\), da za \(ε : ℚ₊\) in \(i ≥ α(ε)\) velja \(|xᵢ - yᵢ| ≤ ε\). 

  Množica Cauchyjevih zaporedij kvocientno z gornjo relacijo tvori
  \emph{Cauchyjeva realna števila}, ki jih označimo z \(\Rc\).
\end{definicija}

\begin{trditev}
  Vsako Cauchyjevo realno število je tudi Dedekindovo.
\end{trditev}
\begin{dokaz}
  Vzemimo \(L ≔ \set{q : ℚ}{q < x}\) in \(U ≔ \set{r : ℚ}{x < r}\).
  Očitno ta zadoščata prvim šestim pogojem zgoraj, in prav tako zadoščata
  zadnjemu pogoju, tako da se osredotočimo zgolj na zadnji pogoj.

  Pokazati moramo, da za vsako Cauchyjevo realno število \(x\) velja
  \(x > 0 ∨ x < 3\).

  %To pa pomeni, da mora obstajati nek indeks \(N : ℕ\), da so \(xᵢ > 0\) za vse
  %\(i ≥ N\), ali pa \(xᵢ < 3\) za vse \(i ≥ N\).
  Za racionalno število \(x₀\) velja \(x₀ > 1 ∨ x₀ < 2\), saj imajo racionalna
  števila odločljivo neenakost. Sedaj pa vemo, da za vse \(i : ℕ\) velja
  \(|x₀ - xᵢ| ≤ 1\), torej če je \(x₀ > 1\) bodo vsi \(xᵢ > 0\), po drugi strani
  pa če je \(x₀ < 2\) bodo vsi \(xᵢ < 3\), kar pa zaključi dokaz.
\end{dokaz}

Obrat pa ne velja konstruktivno. To pomeni, da lahko na ``\(\Rd = \Rc\)''
gledamo kot nekakšen nekonstruktiven princip. Iz konstruktivne matematike pa
poznamo tudi druge konstrukcije realnih števil, ki nam lahko dajo nove principe
kot ta zgoraj.

TODO: cite nlab? elephant?
\begin{definicija}[MacNeillova realna števila]
  Par \(\p{L, U} ∈ 𝒫(ℚ)×𝒫(ℚ)\) je \emph{Dedekind-MacNeilleov rez}, ko velja
  \begin{align}
    L \text{ je poseljen}\\
    U \text{ je poseljen}\\
    L \text{ je navzdol zaprt}\\
    U \text{ je navzgor zaprt}\\
    L \text{ je navzgor odprt}\\
    U \text{ je navzdol odprt}\\
    L = \int{\p{Uᶜ}}\\
    U = \int{\p{Lᶜ}}
  \end{align}
  Množica Dedekind-MacNeilleovih rezov tvori \emph{MacNeillova realna števila}, ki jih
  označimo z \(\Rm\).
\end{definicija}

\begin{trditev}
  Vsako Dedekindovo realno število je tudi MacNeillovo.
\end{trditev}
\begin{dokaz}
  Točke \(1\) skozi \(6\) so enake, tako da je treba pokazati zgolj zadnji dve
  lastnosti. Ker sta simetrični, pokažimo zgolj zadnjo.

  Če je \(a ∈ L\) potem ni v \(U\), torej je v \(Uᶜ\). Sedaj pa, ker je \(L\)
  navzgor odprt obstaja \(a' > a\), ki je tudi v \(L\) (torej v \(Uᶜ\) po enakem
  argumentu.) Sedaj pa vemo, da je \(Uᶜ\) navzdol zaprt, torej je
  \(a ∈ \p{-∞,a'} ⊆ Uᶜ\) in je \(a ∈ \int{\p{Uᶜ}}\).
\end{dokaz}

V obratno smer lahko pokažemo zgolj, da sta \(L\) in \(U\) ločena.
\begin{lema}
  Za MacNeillovo realno število \(\p{L,U}\) velja \(a∈L∧b∈U⇒a<b\).
\end{lema}
\begin{dokaz}
  Naj bo \(a∈L\) in \(b∈U\). Potem je \(b ∈ \int{\p{Lᶜ}} ⊆ Lᶜ\), torej ni v
  \(L\). Ker je \(L\) navzdol zaprt je torej \(b\) zgornja meja, in je večji od
  \(a\).
\end{dokaz}

Ostalo bi torej pokazati zgolj lociranost vsakega MacNeillovega realnega
števila, a to žal ni mogoče.
Temu principu bomo torej pravili ``\(\Rm = \Rd\)''.

Pokažimo še koristno lemo, ki pravi, da ima vsaka poseljena omejena množica
MacNeillovih realnih števil supremum.
\begin{lema}\label{th:Rm-sup}
  MacNeillova realna števila so polna.
\end{lema}
\begin{dokaz}
  TODO
\end{dokaz}

TODO: alpo itd


\subsection{Ostali principi}\label{sec:logika-ostalo}

Seveda pa se principi ne delijo popolnoma zgolj na principe izbire in odločitve.
Nekaj takih imamo zgoraj o realnih številih.

\begin{definicija}
  \emph{Kripkejeva shema za \(Σ ⊆ Ω\)} pravi, da je \(Σ = Ω\). Formulo
  \(\for{p : Ω}{\exist{s : Σ}{s = p}}\) bomo označili z \(\ks*(Σ)\).

  Navadna \emph{kripkejeva shema} je kripkejeva shema za
  \(Σ₀¹ ≔ \set{α \apart 0}{α : 2^ℕ} ⊆ Ω\).

  TODO: dedekindova ali cauchyjeva ali kaj drugega.
  \emph{Analitična kripkejeva shema za \(R\)} je kripkejeva shema za
  \(Σ_R ≔ \set{x > 0}{x : R} ⊆ Ω\), kjer je \(R\) nek objekt realnih števil.

  Posebej bomo \(\ks*(Σ_{\Rd})\) pravili \emph{analitična kripkejeva shema}.
\end{definicija}

\begin{trditev}
  TODO: če se ne motim je \(Σ_{\Rc} = Σ₀¹\). Mogoče.
\end{trditev}

TODO: global existence
\begin{trditev}\label{th:lT6-have-AKS}
  Nad lokalno \(T₆\) prostori velja analitična Kripkejeva shema.
\end{trditev}
\begin{proof}
  Brez škode za splošnost lahko predpostavimo, da je prostor \(X\) \(T₆\).
  Naj bo \(U ⊆ X\). Tedaj obstaja \(f : X → ℝ\), ki je \(0\) natanko na
  komplementu \(U\). To pa pomeni, da je natanko na \(U\) različen od \(0\)
  kar je pa točno to, kar zahtevamo za analitično Kripkejevo shemo.
\end{proof}
\begin{opomba}
  V dokazu smo zares pokazali \emph{globalen} obstoj za element \(x : ℝ\). To
  nam da slutiti, da je (lokalno) \(T₆\) lastnost močnejša od \(\aks*\). In res
  se izkaže, da je temu tako.
  TODO: a se zmislim primer?
\end{opomba}

Vseeno pa velja obrat, če analitično Kripkejevo shemo malo ojačamo. Specifično,
če predpostavimo obstoj \emph{funkcije izibre} za shemo \(\aks*\).
\begin{trditev}
  Če velja \(X ⊩ \exist{f : Ω → ℝ}{\for{p : Ω}{p ⇔ f(p) \apart 0}}\), je \(X\)
  lokalno \(T₆\).
\end{trditev}
\begin{dokaz}
  TODO: a je \(ℝ^Ω\) zuni funkcije? kaj je s tem?

  razponi \(f\) pokrijejo \(X\)

  za \(f\) velja da za vse \(U\) pod \(\e f\) velja \(U = \i{f(U) \apart 0}\).

  pika TODO: napiši zadevo normalno.
\end{dokaz}
\begin{opomba}
  Ponovno smo pokazali malo več kot zgolj lokalno \(T₆\) lastnost. Dobili smo
  \emph{funkcije izbire} za vsako od \(T₆\) komponent. Če v metateoriji
  predpostavimo princip izbire, potem je to ekvivalentno tej močnejši verziji
  \(\aks*\).
\end{opomba}

\section{Pretvorba primerkov}

\subsection{Kripkejeve sheme}

TODO: global existence
\begin{trditev}\label{th:lT6-have-AKS}
  Nad lokalno \(T₆\) prostori velja analitična Kripkejeva shema.
\end{trditev}
\begin{proof}
  Brez škode za splošnost lahko predpostavimo, da je prostor \(X\) \(T₆\).
  Naj bo \(U ⊆ X\). Tedaj obstaja \(f : X → ℝ\), ki je \(0\) natanko na
  komplementu \(U\). To pa pomeni, da je natanko na \(U\) različen od \(0\)
  kar je pa točno to, kar zahtevamo za analitično Kripkejevo shemo.
\end{proof}
\begin{opomba}
  V dokazu smo zares pokazali \emph{globalen} obstoj za element \(x : ℝ\). To
  nam da slutiti, da je (lokalno) \(T₆\) lastnost močnejša od \(\aks*\). In res
  se izkaže, da je temu tako.
  TODO: a se zmislim primer?
\end{opomba}

Vseeno pa velja obrat, če analitično Kripkejevo shemo malo ojačamo. Specifično,
če predpostavimo obstoj \emph{funkcije izibre} za shemo \(\aks*\).
\begin{trditev}
  Če velja \(X ⊩ \exist{f : Ω → ℝ}{\for{p : Ω}{p ⇔ f(p) \apart 0}}\), je \(X\)
  lokalno \(T₆\).
\end{trditev}
\begin{dokaz}
  TODO: a je \(ℝ^Ω\) zuni funkcije? kaj je s tem?

  razponi \(f\) pokrijejo \(X\)

  za \(f\) velja da za vse \(U\) pod \(\e f\) velja \(U = \i{f(U) \apart 0}\).

  pika TODO: napiši zadevo normalno.
\end{dokaz}
\begin{opomba}
  Ponovno smo pokazali malo več kot zgolj lokalno \(T₆\) lastnost. Dobili smo
  \emph{funkcije izbire} za vsako od \(T₆\) komponent. Če v metateoriji
  predpostavimo princip izbire, potem je to ekvivalentno tej močnejši verziji
  \(\aks*\).
\end{opomba}


\subsection{Idempotenca \alpo* nad lokalno \(T₆\) prostori}

Spomnimo se na trditev~\ref{th:lT6-have-AKS}:
% TODO: set number
\begin{trditev}\label{th:lT6-have-AKS-second}
  Nad lokalno \(T₆\) prostori velja analitična Kripkejeva shema.
\end{trditev}

% TODO: Faktoriziraj čez AKS ⇒ LEM ≤ ALPO
\begin{izrek}
  Naj bo \(X\) lokalno \(T₆\). Potem nad \(X\) velja redukcija \(\lem* ≤ \alpo*\).
\end{izrek}
\begin{proof}
  % TODO: reword kot 'ALPO je LEM za Ω_ℝ in AKS je Ω_ℝ = Ω'?
  Ker vemo, da nad \(X\) velja analitična Kripkejeva shema, zadošča
  konstruktivno pokazati, da velja \(\lem* ≤ \alpo*\).
  Po predpostavki, je vsaka resničnostna vrednost \(p\) oblike \(x > 0\) za nek
  \(x : ℝ\). Tedaj pa velja
  \[ \lem{\p p} = p ∨ ¬ p = x > 0 ∨ ¬\p{x > 0} = \alpo{\p x}\text, \]
  torej za vsak \(p : Ω\) obstaja \(x : ℝ\), da je \(\alpo{\p x} ⇒ \lem{\p p}\).
  % Ker dokazujemo resnico v topološkem modelu brez škode za splošnost
  % predpostavimo, da je prostor \(T₆\).
  % To pomeni da za vsako odprto množico \(U\) obstaja nenegativna funkcija
  % \(P_U : X → ℝ\), tako da velja \(P_U > 0 ⇔ U\).
  % Če želimo pretvoriti \(\lem*\) na \(\alpo*\), mora za vse
  % \(U ⊆ X\) veljati \(\eventually{x : ℝ}{x ≤ 0 ∨ x > 0 ⇒ U ∪ ¬U}\).
  % Uporabimo predpostavko na \(U ∪ ¬U\), torej nam ostane pokazati, da
  % velja \({P_U ≤ 0 ∨ P_U > 0 ⇒ P_U > 0}\).
  % Ker je \(⟦P_U ≤ 0⟧ = ¬{\p{U ∪ ¬ U}} = ∅\), velja \(¬\p{x ≤ 0}\), kar dokaže trditev.
\end{proof}
\begin{posledica}
  Če je prostor \(X\) lokalno \(T₆\), velja \(\alpo*×\alpo* ≤ \alpo*\).
\end{posledica}
\begin{posledica}
  Nad lokalno \(T₆\) prostorom je \(\alpo*\) idempotenten.
\end{posledica}


\subsection{Idempotenca \lpo* nad Cantorjevim prostorm}


% Joint with prof. Bauer
\begin{lema}
  Za vsako funkcijo \(f : C → ℝ\) obstaja funkcija \(\hat f : C → C\), za katero
  velja \(C ⊩ \hat f \apart 0 ⇔ f > 0\).
\end{lema}
\begin{proof}
  Naj bo \(U := ⟦f > 0⟧\). Potem obstaja števna particija \(U = ⋃ₖVₖ\) z
  bazičnimi odprtimi. Ker so te kompaktne, za vsak \(k\) velja
  \(\exist{i ∈ ℕ}{f{\res{Vₖ}} > 2⁻ⁱ}\). Po aksiomu izbire torej obstaja funkcija
  izbire \(m : ℕ → ℕ\), da velja \(f{\res{Vₖ}} > 2⁻ᵐ⁽ᵏ⁾\). Potem pa definiramo
  \[ C ⊩ \hat fᵢ ≔
    \begin{cases}
      1;& \exist{k : ℕ}{Vₖ ∧ i = m(k)}\\
      0;& \text{sicer.}
    \end{cases}\]
  Potem pa velja
  \[ \hat f \apart 0 ⇔ \exist{i : ℕ}{\exist{k : ℕ}{Vₖ ∧ i = m(k)}} ⇔ \exist{k : ℕ}{Vₖ} ⇔ U\text, \]
  torej je \(C ⊩ \hat f \apart 0 ⇔ f > 0\).
\end{proof}
\begin{posledica}
  V zgornji lemi lahko domeno zamenjamo s poljubno odprto množico.
\end{posledica}

\begin{lema}
  Nad Cantorjevim prostorom je \(\lpo*\) ekvivalenten \(\alpo*\).
\end{lema}
\begin{dokaz}
  Vemo že, da velja \(\lpo* ≤ \alpo*\) tako da moramo pokazati zgolj obratno redukcijo.
  Naj bo \(U ⊩ x : ℝ\). Potem iz gornje leme dobimo \(U ⊩ \hat x : C\), za
  katerega velja \(\hat x \apart 0 ⇔ x > 0\).
  Kontrapizitivna oblika te ekvivalence pa pravi ravno, da je \(\hat x = 0 ⇔ x ≤ 0\),
  torej res velja \(\lpo{\p{\hat f}} ⇔ \alpo{\p{f}}\).
\end{dokaz}
\begin{posledica}
  Nad Cantorjevim prostorom je \(\lpo*\) idempotenten.
\end{posledica}
\begin{opomba}
  V tem podrazdelku bi lahko namesto Cantorjevega prostora zares vzeli
  katerikoli kompakten nič dimenzionalen prostor (z drugimi besedami, Stoneov prostor).
\end{opomba}



%%% Local Variables:
%%% mode: latex
%%% TeX-master: "main"
%%% End:



%%% Local Variables:
%%% mode: latex
%%% TeX-master: "main"
%%% End:
