\section{Nekonstruktivni principi}\label{sec:logika}

\subsection{Realna števila}\label{sec:logika-reals}

Spomnimo se nekaj definicij realnih števil.
\begin{definicija}[Dedekindova realna števila]
  Par \(\p{L, U} ∈ 𝒫(ℚ)×𝒫(ℚ)\) je \emph{Dedekindov rez}, ko velja
  \begin{align}
    L \text{ je poseljen}\\
    U \text{ je poseljen}\\
    L \text{ je navzdol zaprt}\\
    U \text{ je navzgor zaprt}\\
    L \text{ je navzgor odprt}\\
    U \text{ je navzdol odprt}\\
    \text{za } a < b \text{ je } a ∈ L ∨ b ∈ U\label{real:located}\\
    \text{za } a ∈ L ∧ b ∈ U \text{ je } a < b
  \end{align}
  Množica Dedekindovih rezov tvori \emph{Dedekindova realna števila}, ki jih
  označimo z \(\Rd\).
\end{definicija}
\begin{opomba}
  Pogoju~\ref{real:located} pravimo \emph{lociranost} in če velja za katerikoli
  racionalni števili \(a\) in \(b\), bo veljala tudi za vse druge izbire.

  Res, relacija \(<\) je invariantna na afine preslikave realne premice, tako da
  lahko vsak interval \(\p{a,b}\) preslikamo (recimo) na \(\p{0,1}\) in nazaj.
\end{opomba}

\begin{definicija}[Cauchyjeva realna števila]
  Zaporedje racionalnih števil \((xₙ)ₙ\) je \emph{Cauchyjevo}, ko za vsaka
  \(i,j : ℕ\) velja \(|xᵢ - xⱼ| ≤ 2⁻ⁱ+2⁻ʲ\).

  % \emph{Modulus konvergence} je preslikava \(α : ℚ₊ → ℕ\), za katero za vsak
  % \(n : ℕ\) obstaja \(ε : ℚ₊\), da je \(α(ε) ≥ n\).

  % Zaporedje racionalnih števil \(\p{xₙ}\) z modulusom konvergence \(α\) je
  % \emph{Cauchyjevo}, ko za vsake \(ε : ℚ₊\) in \(i,j ≥ α(ε)\) velja
  % \(|xᵢ - xⱼ| < ε\).

  Dve Cauchyjevi zaporedji predstavljata isto realno število,
  ko za vse \(i : ℕ\) velja \(|xᵢ - yᵢ| ≤ 2⁻ⁱ⁺¹\).
  % ko obstaja modulus
  % konvergence \(α\), da za \(ε : ℚ₊\) in \(i ≥ α(ε)\) velja \(|xᵢ - yᵢ| ≤ ε\). 

  Množica Cauchyjevih zaporedij kvocientno z gornjo relacijo tvori
  \emph{Cauchyjeva realna števila}, ki jih označimo z \(\Rc\).
\end{definicija}

\begin{trditev}
  Vsako Cauchyjevo realno število je tudi Dedekindovo.
\end{trditev}
\begin{dokaz}
  Vzemimo \(L ≔ \set{q : ℚ}{q < x}\) in \(U ≔ \set{r : ℚ}{x < r}\).
  Očitno ta zadoščata prvim šestim pogojem zgoraj, in prav tako zadoščata
  zadnjemu pogoju, tako da se osredotočimo zgolj na zadnji pogoj.

  Pokazati moramo, da za vsako Cauchyjevo realno število \(x\) velja
  \(x > 0 ∨ x < 3\).

  %To pa pomeni, da mora obstajati nek indeks \(N : ℕ\), da so \(xᵢ > 0\) za vse
  %\(i ≥ N\), ali pa \(xᵢ < 3\) za vse \(i ≥ N\).
  Za racionalno število \(x₀\) velja \(x₀ > 1 ∨ x₀ < 2\), saj imajo racionalna
  števila odločljivo neenakost. Sedaj pa vemo, da za vse \(i : ℕ\) velja
  \(|x₀ - xᵢ| ≤ 1\), torej če je \(x₀ > 1\) bodo vsi \(xᵢ > 0\), po drugi strani
  pa če je \(x₀ < 2\) bodo vsi \(xᵢ < 3\), kar pa zaključi dokaz.
\end{dokaz}

Obrat pa ne velja konstruktivno. To pomeni, da lahko na ``\(\Rd = \Rc\)''
gledamo kot nekakšen nekonstruktiven princip. Iz konstruktivne matematike pa
poznamo tudi druge konstrukcije realnih števil, ki nam lahko dajo nove principe
kot ta zgoraj.

TODO: cite nlab? elephant?
\begin{definicija}[MacNeillova realna števila]
  Par \(\p{L, U} ∈ 𝒫(ℚ)×𝒫(ℚ)\) je \emph{Dedekind-MacNeilleov rez}, ko velja
  \begin{align}
    L \text{ je poseljen}\\
    U \text{ je poseljen}\\
    L \text{ je navzdol zaprt}\\
    U \text{ je navzgor zaprt}\\
    L \text{ je navzgor odprt}\\
    U \text{ je navzdol odprt}\\
    L = \int{\p{Uᶜ}}\\
    U = \int{\p{Lᶜ}}
  \end{align}
  Množica Dedekind-MacNeilleovih rezov tvori \emph{MacNeillova realna števila}, ki jih
  označimo z \(\Rm\).
\end{definicija}

\begin{trditev}
  Vsako Dedekindovo realno število je tudi MacNeillovo.
\end{trditev}
\begin{dokaz}
  Točke \(1\) skozi \(6\) so enake, tako da je treba pokazati zgolj zadnji dve
  lastnosti. Ker sta simetrični, pokažimo zgolj zadnjo.

  Če je \(a ∈ L\) potem ni v \(U\), torej je v \(Uᶜ\). Sedaj pa, ker je \(L\)
  navzgor odprt obstaja \(a' > a\), ki je tudi v \(L\) (torej v \(Uᶜ\) po enakem
  argumentu.) Sedaj pa vemo, da je \(Uᶜ\) navzdol zaprt, torej je
  \(a ∈ \p{-∞,a'} ⊆ Uᶜ\) in je \(a ∈ \int{\p{Uᶜ}}\).
\end{dokaz}

V obratno smer lahko pokažemo zgolj, da sta \(L\) in \(U\) ločena.
\begin{lema}
  Za MacNeillovo realno število \(\p{L,U}\) velja \(a∈L∧b∈U⇒a<b\).
\end{lema}
\begin{dokaz}
  Naj bo \(a∈L\) in \(b∈U\). Potem je \(b ∈ \int{\p{Lᶜ}} ⊆ Lᶜ\), torej ni v
  \(L\). Ker je \(L\) navzdol zaprt je torej \(b\) zgornja meja, in je večji od
  \(a\).
\end{dokaz}

Ostalo bi torej pokazati zgolj lociranost vsakega MacNeillovega realnega
števila, a to žal ni mogoče.
Temu principu bomo torej pravili ``\(\Rm = \Rd\)''.

Pokažimo še koristno lemo, ki pravi, da ima vsaka poseljena omejena množica
MacNeillovih realnih števil supremum.
\begin{lema}\label{th:Rm-sup}
  MacNeillova realna števila so polna.
\end{lema}
\begin{dokaz}
  TODO
\end{dokaz}


\subsection{Principi izbire}\label{sec:logika-izbire}

Za princip izbire ste verjetno že slišali. Ta pravi, da za vsaki množici \(A\)
in \(B\), in vsako celovito relacijo \(R\) med njima, obstaja funkcija
\(f : A → B\), tako da velja \(\for{a : A}{R(a, f(a))}\).

Ta princip lahko ošibimo na tri načine: lahko omejimo množici \(A\) in \(B\),
lahko pa omejimo relacijo \(R\), tako da jemlje resničnostne vrednosti zgolj iz
določene množice \(Σ ⊆ Ω\).

\begin{definicija}
  \emph{Princip izbire nad \(Σ\)} je shema, ki za vsaka \(A\) in \(B\) pravi,
  da za vsako relacijo \(R : A×B → Σ\) za katero velja
  \(\for{a:A}{\exist{b:B}{R(a,b)}}\), obstaja funkcija izbire \(f : A → B\),
  tako da velja \(\for{a:A}{R(a,f(a))}\). Objektu \(A\) tako pravimo
  \emph{domena}, objektu \(B\) \emph{kodomena}, objektu \(Σ\) pa
  \emph{Sierpinskijev objekt}.
  Pogoju na \(R\) pravimo \emph{celovitost}.
  To označimo z \(\AC_Σ\). Če je \(Σ = Ω\) indeks opustimo in temu pravimo
  \emph{princip izbire}.
\end{definicija}
\begin{definicija}
  Če v principu izbire fiksiramo domeno na nek \(A\), temu pravimo
  \emph{princip izbire nad \(Σ\) iz \(A\)}, in če fiksiramo še kodomeno na nek
  \(B\) temu pravimo \emph{princip izbire nad \(Σ\) iz \(A\) v \(B\)}. Ta potem
  označujemo \(\AC_Σ(A)\) in \(\AC_Σ(A, B)\). Podobno kot zgoraj opuščamo \(Σ\)
  ko je ta enaka \(Ω\).
\end{definicija}

TODO: enolična izbira

Hitro lahko opazimo, da so principi izbire neobčutljivi za izomorfizme.
\begin{trditev}
  Naj bo \(φ : A ≅ A'\) in \(ψ : B ≅ B'\). Potem je \(\AC_Σ(A, B)\) ekvivalenten
  \(\AC_Σ(A', B')\).
\end{trditev}
\begin{dokaz}
  Situacija je očitno simetrična, tako da pokažimo zgolj eno implikacijo.

  Če je \(R : A'×B' → Σ\) celovita, je potem tudi \({R∘\p{φ×ψ} : A×B → Σ\)
  celovita. Potem pa za to relacijo obstaja funkcija izbire, tako da je
  \(\for{a : A}{R∘\p{φ×ψ}{(a, f(a))}}\), kar je pa enako
  \(\for{a:A}{R(φ(a),ψ(f(a)))}\). Ker pa je \(φ\) izomorfizem, lahko
  preindeksiramo kvantifikator in dobimo \(\for{a:A'}{R(a',ψ∘f∘φ⁻¹(a'))}\),
  torej je \(ψ∘f∘φ⁻¹\) funkcija izbire za \(R\).
\end{dokaz}
V teoriji množic to pomeni, da je aksiom izbire določen do kardinalnosti
natančno. To nam tudi utemelji zakaj lahko \(\AC(ℕ)\) pravimo princip
\emph{števne} izbire, saj deluje za poljubno števno (neskončno) množico. Tako si
lahko tudi predstavljamo, kako to shemo imen razširiti na poljubno kardinalnost.
Označimo ga torej \(\CC\). Pomembno vlogo bo igral tudi \emph{princip
  števne disjunktivne izbire} \(\AC(ℕ, 2)\), ki ga označimo z \(\CCv\). 

TODO: vložitve domene in kodomene tud
\begin{trditev}
  Če je \(Σ' ⊆ Σ\), \(\AC_Σ(A, B)\) implicira \(\AC_{Σ'}(A, B)\).
\end{trditev}
\begin{dokaz}
  Vsaka relacija nad \(Σ'\) je tudi relacija nad \(Σ\), tako da če imajo
  relacije nad \(Σ\) funkcije izbire jih imajo tudi relacije nad \(Σ'\).
\end{dokaz}

\begin{trditev}
  Princip končne izbire velja.
\end{trditev}
\begin{dokaz}
  TODO: sintaksa za standardne končne?
  Princip končne izbire je princip izbire, kjer domeno omejimo na končne
  množice. Brez škode za splošnost, naj bo domena kar enaka neki
  standardni končni množici \(\cli n\).

  Potem pa \(\AC(\cli n)\) pravi, da če
  obstajajo \(bₖ:B\), da velja \(R(1,b₁)∧\dots ∧R(n,bₙ)\), obstaja končno
  zaporedje (torej, obstajajo \(bₖ:B\)), da velja \(R(1,b₁)∧\dots ∧R(n,bₙ)\)… To
  sta pa isti stvari, tako da princip res drži.
\end{dokaz}

% \begin{definicija}
%   \emph{Princip \[Σ\]-izbire iz \(A\) v \(B\)} pravi, da za vsako relacijo
%   \(R : A×B → Σ\) za katero velja \(\for{a:A}{\exist{b:B}{R(a,b)}}\), obstaja
%   funkcija izbire \(f : A → B\), tako da velja \(\for{a:A}{R(a,f(a))}\).

%   To bomo označili z \(\AC_Σ(A,B)\). Z \(\AC_Σ(A)\) bomo označevali shemo
%   \(\for{B}{\AC_Σ(A,B)}\), z \(\AC_Σ\) pa shemo \(\for{A}{\AC_Σ(A)}\).
%   Tem pravimo \emph{princip \(Σ\)-izbire iz \(A\)} in \emph{princip \(Σ\)-izbire}.
%   Kadar je \(Σ = Ω\) bomo \(Σ\) v indeksu opuščali, in ustrezno to poimenovali
%   kar samo \emph{princip izbire (iz \(A\) v \(B\))}.

%   Standardno se principu \(\AC(ℕ)\) pravi \emph{princip števne izbire} in
%   označuje \(\CC\).
%   Posebno pomemben bo princip \(\AC(ℕ, 2)\), ki ga bomo označevali \(\CCv\) in
%   mu pravili \emph{princip števne dvojiške(disjunktivne?) izbire}.
% \end{definicija}



% \begin{definicija}
%   \emph{Princip števne dvojiške izbire} pravi, da če velja
%   \(\for{n : ℕ}{P(n, 0) ∨ P(n, 1)}\) (torej, \(P\) je celovita relacija na
%   \(ℕ×2\)) obstaja funkcija izbire \(f : ℕ → 2\), da velja
%   \(\for{n : ℕ}{P(n,f(n))}\).
% \end{definicija}

Poznamo pa tudi še princip odvisne izbire.
\begin{definicija}
  \emph{Princip odvisne izbire nad \(Σ\) za \(A\)} pravi, da za vsako celovito
  relacijo \(R : A×A → Σ\) in \(a₀ : A\) obstaja zaporedje \((aᵢ)ᵢ\) začenši z
  \(a₀\), tako da za vsak \(n : ℕ\) velja \(R(aₙ, aₙ₊₁)\). Tega označimo z
  \(\DC_Σ(A)\). Če princip velja za vse \(A\) ga označimo \(\DC_Σ\).
\end{definicija}

\begin{trditev}
  Če je \(Σ⊆Ω\) zaprt za končne konjunkcije in števne disjunkcije, velja
  implikacija \(\DC_Σ ⇒ \CC_Σ\).
\end{trditev}
\begin{dokaz}
  Naj bo \(R : ℕ×B → Σ\) celovita.
  Definirajmo \(X ≔ \set{b:B}{\exist{n:ℕ}{R(n,b)}}\) in na \(X×X\) relacijo
  \(Q(x,y) ≔ \exist{n:ℕ}{R(n,x)∧R(n+1,y)}\).

  Ta je celovita, tako da lahko na njej uporabimo \(\DC_Σ\), torej dobimo
  zaporedje \(x : ℕ → X\). Ker je \(X⊑B\) je torej to preslikava \(ℕ → B\), ki
  ima želeno lastnost.
\end{dokaz}

V dokazu je pogoj na Sierpinskijevem objektu res pomemben, sicer \(Q\) ni
relacija nad \(Σ\). 

\subsection{Principi odločitve}\label{sec:logika-odločitve}

\begin{definicija}[Izključena tretja možnost]\label{pr:lem}
  \emph{Princip izključene tretje možnosti} pravi, da za vsako resničnostno
  vrednost \(p\) velja \(p∨¬p\). Formulo \(p∨¬p\) označimo \(\lem(p)\), formulo
  \(\for{p:Ω}{p∨¬p}\) pa z \(\lem*\).
\end{definicija}

\begin{definicija}[Šibka izključena tretja možnost]\label{pr:wlem}
  \emph{Šibak princip izključene tretje možnosti} pravi, da za vsako
  resničnostno vrednost \(p\) velja \(¬p∨¬¬p\). Formulo \(¬p∨¬¬p\) označimo
  \(\wlem(p)\), formulo \(\for{p:Ω}{¬p∨¬¬p}\) pa z \(\wlem*\).
\end{definicija}
\begin{trditev}
  Velja implikacija \(\lem* ⇒ \wlem*\).
\end{trditev}
\begin{dokaz}
  Formula \(\wlem(p)\) je natanko \(\lem(¬p)\), tako da trditev očitno velja.
\end{dokaz}

\begin{definicija}\label{pr:lpo}
  \emph{Princip števne odločitve} pravi, da za vsako števno zaporedje ničel in enic
  lahko odločimo, ali je celo nič, ali pa obstaja mesto, na katerem je enica.
  Formulo \(α = 0∨α\apart 0\) označimo \(\lpo(α)\), formulo
  \(\for{α : 2^ℕ}{\lpo(α)}\) pa z \(\lpo*\).
\end{definicija}

\begin{definicija}\label{pr:alpo}
  Naj bo \(ℝ\) nek objekt realnih števil (bodisi Dedekindova, Cauchyjeva, itd.).
  \emph{Analitični princip števne odločitve za \(ℝ\)} pravi, da za vsako realno
  število lahko odločimo, ali je pozitivno ali nenegativno. Formulo
  \(x > 0 ∨ x ≤ 0\) označimo \(\alpo_ℝ(x)\), formulo \(\for{x : ℝ}{\alpo_ℝ(x)}\) pa
  z \(\alpo*_ℝ\).

  Če je \(ℝ = \Rd\), indeks opustimo in pravimo le \emph{Analitični princip
    števne odločitve}.
\end{definicija}

\begin{trditev}
  Ekvivalentno lahko vzamemo \(\alpo_ℝ(x) = x = 0 ∨ x \apart 0\).
\end{trditev}
\begin{dokaz}
  Če uporabimo \(\alpo*_ℝ\) na \(x\) in \(-x\) dobimo želeno formulo. Obratno pa
  \({x = 0 ∨ x < 0}\) očitno implicira \(x ≤ 0\).
  TODO: a je to res za MacNeillova?
\end{dokaz}

\begin{trditev}\label{th:alpoc-is-lpo}
  Velja veriga implikacij \(\lem* ⇒ \alpo* ⇒ \alpo*_{\Rc} ⇔ \lpo*\).
\end{trditev}
\begin{dokaz}
  Ker je \(¬(x > 0)\) natanko \(x ≤ 0\), je \(\alpo*\) očitno posledica
  \(\lem*\). Prav tako je vsako Cauchyjevo realno število tudi Dedekindovo, tako
  da druga implikacija tudi velja.

  Za zadnjo ekvivalenco pa potrebujemo malo dela. Naj bo \(α\) zaporedje, za
  katerega odločamo, ali ima na kakem mestu enico. Definirajmo realno število
  \[ x ≔ \lim_{n → ∞}2^{-\min\set{k : ℕ}{α(k) = 1 ∨ k ≥ n}}\text. \]

  To je po definiciji Cauchyjevo realno število. Uporabimo sedaj predpostavko
  \(\alpo(x)\). Če velja \(x ≤ 0\), mora zaporedje limitirati proti \(0\), kar
  pa pomeni, da \(α(k) = 1\) ni nikoli zadoščeno, torej je \(α = 0\).
  Po drugi strani, če je \(x > 0\) pa obstaja nek \(k : ℕ\), tako da je
  \(x > 2⁻ᵏ\). To pa pomeni, da ima \(α\) enico na enem izmed prvih \(k\)
  mestih, kar pomeni, da želeni indeks obstaja.

  Obratno, naj bo \(x = (xᵢ)ᵢ\) Cauchyjevo zaporedje.
  Definiramo lahko zaporedje
  \[ β(k,n) ≔
    \begin{cases}
      1 &; \for  {m≥n}{xₘ ≤ 2⁻ᵏ}\\
      0 &; \exist{m≥n}{xₘ > 2⁻ⁿ}\text.
    \end{cases} \]

  Vse te odločitve lahko naredimo, saj velja \(\lpo*\) in je neenakost med
  racionalnimi števili odločljiva. Sedaj uporabimo \(\lpo*\) na \(n↦β(k,n)\).

  Tako definiramo
  \[ α(k) ≔
    \begin{cases}
      1 &; \for  {n:ℕ}{α(k,n) = 0}\\
      0 &; \exist{n:ℕ}{α(k,n) = 1}\text.
    \end{cases} \]
  Nazadnje sedaj uporabimo \(\lpo*\) na \(α\). Če je \(α = 0\), imamo za vsak
  \(k\) nek \(n\), da je za vsak \(m≥n\) \(xₘ ≤ 2⁻ᵏ\). To pa pomeni, da je
  \(x = 0\). Po drugi strani pa, če je \(α(k) = 1\) za nek \(k\) pa velja, da za
  vsak \(n\) obstaja \(m≥n\), da je \(xₘ > 2⁻ᵏ\), torej je \(x > 2⁻ᵏ\) in je
  \(x \apart 0\).
\end{dokaz}

To pomeni, da je analitični princip števne odločitve res smiselen zgolj za
Dedekindova realna števila. Od tu naprej prosto menjamo med \(\alpo*_{\Rc}\) in
\(\lpo*\) po potrebi, brez posebne omembe. Prav tako bomo raje pisali \(\lpo*\)
in \(\lpo\), kjer ni dvoumno.
TODO: odstrani posebne omembe.

\subsection{Ostali principi}\label{sec:logika-ostalo}

Seveda pa se principi ne delijo popolnoma zgolj na principe izbire in odločitve.
Nekaj takih imamo zgoraj o realnih številih.


\subsubsection{Pretvorba primerkov}

Oglejmo si najprej razne Sierpinskijeve objekte, ki smo jih do sedaj definirali.
\begin{align*}
  % TODO: a mamo še kake?
  Ω      &= \set{p}{p : Ω}\\
  %Σ_{\Rm} &= \set{x > 0}{x : \Rm}\\% TODO: a je ta smiselen
  Σ_{\Rd} &= \set{x > 0}{x : \Rd}\\
  Σ_{\Rc} &= \set{x > 0}{x : \Rc}\\
  Σ₀¹    &= \set{α \apart 0}{α : 2^ℕ}
\end{align*}

Izkaže se, da lahko principe odločitve izrazimo kot princip izključene tretje
možnosti, omejene na te Sierpinskijeve objekte.
\begin{trditev}
  Imamo \(\alpo* = \lem*(Σ_{\Rd})\), \(\alpo*_{\Rc} = \lem*(Σ_{\Rc})\), in
  \(\lpo* = \lem*(Σ₀¹)\).
\end{trditev}
\begin{dokaz}
  Ker je \(¬(x > 0) ⇔ x ≤ 0\), in \(¬(α\apart 0) ⇔ α = 0\), enakosti očitno sledijo.
\end{dokaz}

Ampak mi smo pa tudi pokazali, da je \(\alpo*_{\Rc} ⇔ \lpo*\). Ali to pomeni, da
je \(Σ_{\Rc} = Σ₀¹\)? Namreč, v dokazu~\ref{th:alpoc-is-lpo} skonsturiramo taka
\(x : \Rc\) in \(α : 2^ℕ\), da je \(x > 0\) natanko tedaj, ko je \(α \apart 0\).

A vendar se izkaže, da velja zgolj \(Σ₀¹ ⊆ Σ_{\Rc}\). Če pogledamo dokaz, v to
smer res na prazno skonstruiramo \(x : \Rc\), da velja gornje, ampak v obratno
smer pa bistveno uporabimo več (celo neskončno \emph{instanc}) \(\lpo*\) v
konstrukciji zaporedja \(α\).

Izkaže se, da lahko matematično natančno opišemo gornji pojav. Ideja pride iz
teorije izračunljivosti, specifično iz Weihrauchovih redukcij. Tam skrbno
pazimo, kolikokrat se uporabi posamezne predpostavke.

TODO: definicije in vse

\begin{trditev}
  Princip \(\lem*\) je idempotenten.
\end{trditev}
\begin{dokaz}
  Naj bosta \(p\) in \(q\) resničnostni vrednosti.
  Potem definiramo \(r ≔ \lem²{\p{p, q}}\).
  Ker \(\lem ⁿ\) slika v goste resničnostne vrednosti, je \(\lem{\p r} = r\), tako da
  je \[\lem{\p r} = r = \lem²{\p{p, q}}\text.\qedhere\]
\end{dokaz}

\begin{trditev}
  Velja redukcija \(\lpo* ≤ \alpo*_{\Rc} ≤ \alpo*\).
\end{trditev}
\begin{dokaz}
  TODO: to smo dokazali že zgoraj
  Naj bo \(φ : 2^ℕ → ℝ\) definirana s predpisom
  \[α ↦ \lim_{n→∞}2^{-\min\set{k ∈ ℕ}{αₖ = 1 ∨ k = n}}\text.\]
  Denimo, da je \(α ∈ 2^ℕ\) in uporabimo \(\alpo*\) na \(φ(α)\).

  Če velja \(φ(α) ≤ 0\) vemo, da \(α\) ne more imeti enice na nobenem mestu.

  Alternativno, če je \(φ(α) > 0\), pa je tudi \(φ(α) > 2^{-n}\) za nek \(n\).
  Sledi, da mora imeti \(α\) na enem izmed prvih \(n\) mest enico.
\end{dokaz}

\begin{trditev}
  Velja redukcija \(\alpo*_{\Rc} ≤ \lpo*^ω\).
\end{trditev}
\begin{dokaz}
  To smo zares pokazali že zgoraj. Uporabimo \(ω⋅ω + ω + 1\) instanc \(\lpo*\),
  kar je števno mnogo, torej je števno mnogo instanc \(\lpo*\) dovolj za gornjo
  redukcijo.
\end{dokaz}

\subsubsection{Kripkejeve sheme}

Oglejmo si sedaj malce strožjo reducibilnost, namreč, ko v~\ref{eq:inst-red} zamenjamo
\(⇒\) z \(⇔\). Specifično nas to zanima, saj si želimo ogledati principe
\(Σ = Ω\), za razne Sierpinskijeve objekte, kajti to velja natanko tedaj, ko
velja \(\for{p:Ω}{\exist{s∈Σ}{s ⇔ p}}\). Kot vidimo, je to enaka formula
kot~\ref{eq:inst-red}, le da smo \(⇒\) zamenjali z \(⇔\).

TODO: a niso use te \(Σ\) zaprte navzdol? ker mamo zožitve? mogoče to samo za
\(Σ_{\Rd}\).
\begin{definicija}
  \emph{Kripkejeva shema za \(Σ ⊆ Ω\)} pravi, da je \(Σ = Ω\). Formulo
  \(\for{p : Ω}{\exist{s : Σ}{s = p}}\) bomo označili z \(\ks*(Σ)\).

  Navadna \emph{kripkejeva shema} je kripkejeva shema za
  \(Σ₀¹ ≔ \set{α \apart 0}{α : 2^ℕ} ⊆ Ω\).

  TODO: dedekindova ali cauchyjeva ali kaj drugega.
  \emph{Analitična kripkejeva shema za \(ℝ\)} je kripkejeva shema za
  \(Σ_ℝ ≔ \set{x > 0}{x : ℝ} ⊆ Ω\), kjer je \(ℝ\) nek objekt realnih števil.
  Označili jo bomo z \(\aks*_ℝ\)

  Posebej bomo \(\ks*(Σ_{\Rd})\) pravili \emph{analitična kripkejeva shema}.
\end{definicija}

\begin{trditev}
  Sierpinski objekt \(Σ_ℝ\) je enak \(\set{x \apart 0}{x : ℝ}\).
  %Princip \(\aks*_ℝ\) je ekvivalenten \(\ks*(\set{x \apart 0}{x : ℝ})\).
\end{trditev}
\begin{dokaz}
  Če je \(x : ℝ\), je potem \(x \apart 0 ⇔ \abs x > 0\).
  Obratno pa, je \(x > 0 ⇔ \max\{x,0\} \apart 0\).  
\end{dokaz}
Prav tako bomo potem v dokazih prosto menjavali med \(x > 0\) in \(x \apart 0\).

\begin{trditev}
  TODO: če se ne motim je \(Σ_{\Rc} = Σ₀¹\). Zih je \(⊇\). aha to ni res, k
  rabmo like neskončno instanc...
\end{trditev}


%%% Local Variables:
%%% mode: latex
%%% TeX-master: "main"
%%% End:
