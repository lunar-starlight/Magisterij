%\section{Realna števila v topoloških modelih}\label{sec:reals}


\subsection{\(\Rc{} = \c ℝ\)}\label{sec:reals-Rc=R}

Poglejmo si tu še enkrat trditev~\ref{th:lpov-lpo}. Ta pravi, da nad lokalno
povezanimi prostori velja \(\lpo*\). Zares pa potrebujemo zgolj \(\Rc = \c ℝ\),
saj smo potrebovali le, da so elementi \(\Rc\) lokalno konstantni. Izkaže se, da
velja tudi obrat.

\begin{trditev}
  Če nad \(X\) velja \(\lpo*\), velja \(\Rc = \c ℝ\).
\end{trditev}
\begin{dokaz}
  Naj bo \(x : \Rc\). To je zunaj preslikava \(\hat x : \e x → ℝ\), saj je vsako
  Cauchyjevo realno število tudi Dedekindovo.

  Ker velja \(\lpo*\), imamo za vsak \(a ∈ ℝ\) \(\lpo(x-a)\), torej
  \(x = a ∨ x \apart a\). Med drugim to pomeni, da so množice
  \(\set{t∈\e x}{\hat x(t) = a}\) odprte, torej je \(\hat x\) lokalno konstantna
  preslikava.
\end{dokaz}
\begin{opomba}
  Tu smo uporabili \(ℝ\) mnogo instanc \(\lpo*\).
\end{opomba}

Skupaj s trditvijo~\ref{th:lpov-lpo} lahko zatrdimo ekvivalenco.
\begin{izrek}\label{th:Rc=R-is-lpo}
  Nad \(X\) velja \(\lpo*\) natanko tedaj, ko je \(\Rc = \c ℝ\).
\end{izrek}
Tako lahko namesto \(\Rc = \c ℝ\) od tu naprej pišemo kar \(\lpo*\)

Izkaže pa se, da začetna formulacija trditve~\ref{th:lpov-lpo} ni tako napačna.
Nad \(2\)-števnimi prostori je namreč princip \(\lpo*\) ekvivalenten lokalni
povezanosti~\cite[1026]{Johnstone02}.

% Karakterizirajmo najprej \(\lpo*\) brez realnih števil.
% \begin{lema}
%   Nad \(X\) velja \(\lpo*\) natanko tedaj, ko za vsak \(U\) in števno družino
%   \(\{Vₙ\}ₙ\) odprtih množic, ki so relativno zaprte v \(U\), je njihov presek
%   odprt.
% \end{lema}
% \begin{dokaz}
%   Označimo \(V ≔ ⋂ₙVₙ\) in definirajmo relacijo
%   \[ α(n, b) ≔ \i{b=1∧(U⧵Vₙ) ∨ b=0∧Vₙ} \text. \]
%   Ker so \(Vₙ\) odprte in zaprte, je ta celovita in enolična, torej definira
%   \(α : 2^ℕ\) z razponom \(U\). Uporabimo sedaj \(\lpo*\), in dobimo, da je
%   \(U = \i{α = 0} ∪ \i{α\apart 0}\). Vrednost \(α\apart 0\) je natanko unija
%   vseh \(U⧵Vₙ\), torej natanko \(U⧵V\). Sledi, da je \(\i{α=0} = V\), torej je
%   presek odprt.

%   Obratno, naj bo \(α : 2^ℕ\) in definirajmo \(Uₙ ≔ \i{α(n) = 0}\) in
%   \(Vₙ ≔ \i{α(n) = 1}\). Velja \(\e α = Uₙ ∪ Vₙ\), in \(Uₙ ⊥ Vₙ\), torej so
%   \(Uₙ\) in \(Vₙ\) relativno zaprte v \(\e α\).

%   Sledi, da je presek \(Uₙ\) odprt, torej je odprta tudi množica \(t ∈ \e α\),
%   kjer je \(α = 0\). To pa pomeni, da velja \(α = 0 ∨ α \apart 0\), torej velja
%   \(\lpo*\).
% \end{dokaz}

\begin{trditev}
  Če nad \(2\)-števnim prostorom velja \(\lpo*\), je lokalno povezan.
\end{trditev}
Dokaz za vajo prepuščen bralki (avtorica ne zna izpeljat dokaza, literatura pa
pravi, da je očiten).


\subsection{\(\Rd{} = \Rc\)}\label{sec:reals-Rd=Rc}

Najprej si oglejmo klasičen dokaz ekvivalence Dedekindovih in Cauchyjevih
realnih števil.
\begin{izrek}[Klasični]
  Pokazati je zgolj potrebno, da ima vsak Dedekindov rez pripadajoče Cauchyjevo
  zaporedje.
  Naj bo \(\p{L, U}\) obojestranski dedekindov rez. Cauchyjevo zaporedje lahko
  podamo kot zaporedje hitro padajočih racionalnih intervalov.

  Naj bosta \(p₀ ∈ L\) in \(q₀ ∈ U\) racionalni števili.
  Potem pa na \(n\)-tem koraku definiramo
  \[ a ≔ \frac{2pₙ + qₙ}{3}\text,\quad b ≔ \frac{pₙ + 2qₙ}{3}\text{, in}\quad
     \p{pₙ₊₁, qₙ₊₁} ≔ \begin{cases}
       \p{a, qₙ} ;& a ∈ L\\
       \p{pₙ, b} ;& b ∈ U\text.
     \end{cases}
  \]
  Te intervali hitro konvergirajo proti \(\p{L, U}\), torej je to želeno
  Cauchyjevo število.
\end{izrek}

Gornji izrek naredi števno mnogo odločitev, ko se odločamo, če velja \(a ∈ L\)
ali \(b ∈ U\) (oziroma ali velja \(a < x\) ali \(x < b\)), torej dokaz ni
konstruktiven.
Znana sta dva nekonstruktivna principa, ki sta zadostna za gornji odkaz in sta
šibkejša od izključene tretje možnosti. To sta \(\alpo*\) ter \(\CCv\).
Če velja \(\alpo*\) je potem \(a < x\) odločljivo, torej lahko vnaprej popravimo
drugi primer na \(x < b ∧ ¬\p{a < x}\).

Če pa imamo na voljo \(\CCv\) pa preprosto lahko naredimo števno mnogo
odločitev. Zares potrebujemo tu zgolj \(\CCv_{Σ_ℝ}\), kar je
pa tudi posledica \(\alpo*\).

Čeprav je iz tega očitno, da niti \(\alpo*\) niti \(\CCv\) nista potrebna za
\(\Rd = \Rc\), jih vseeno želimo strogo ločiti.

\begin{konstrukcija}
  Nad \(\Ncof\) velja \(\Rd = \Rc\), a ne velja princip števne
  disjunktivne izbire.
\end{konstrukcija}
\begin{dokaz}
  Prostor naravnih števil s kokončno topologijo ima lastnost, da je vsaka
  funkcija \(ℕ → ℝ\) konstantna. To velja tudi za (neprazne) odprte podmnožice,
  ker so števno neskončne s kokončno topologijo, kar pa pomeni, da se, v toposu
  snopov nad \(ℕ\) s to topologijo, Dedekindova in Cauchyjeva realna števila
  ujemajo.

  Naj bo \(R(n, b) ≔ ℕ⧵\{2n+b\}\). Ta relacija je celovita, saj lahko za vsak
  \(n : ℕ\) \(ℕ\) pokrijemo z \(R(n,0)\) in \(R(n,1)\). Pokažimo, da za to
  relacijo ne obstaja funkcija izbire.

  Denimo, da je \(f : ℕ ↬ 2\) njena funkcija izbire, torej da velja
  \(\for{n:ℕ}{R(n,f(n))}\). Če bi bila \(f\) konstantno \(b\), bi potem moralo
  veljati \(ℕ ⊩ R(n, b)\) za vse \(n\), kar pa ni res. To pomeni, da sta množici
  \(f(n,0)\) in \(f(n,1)\) obe neprazni, torej je njun presek neprazna odprta
  množica. Na tej množici po enoličnosti \(f\) velja \(0 = 1\), kar pa očitno ne
  drži.\contradiction

  Sledi, da funkcija izbire za ta \(R\) ne more obstajati, torej \(\CCv\) ne
  drži.
\end{dokaz}
\begin{dokaz}
  Naj bodo \(Cₙ ≔ \{ℕ⧵\{2n\}, ℕ⧵\{2n+1\}\}\) pokritja \(\Ncof\) in \(C\) njihova
  skupna pofinitev. Potem mora vsak \(U ∈ C\) biti podmnožica enega od elementov
  vsakega od \(Cₙ\). To pa pomeni, da ima \(U\) neskončen komplement, torej je
  prazna množica. Sledi, da \(C\) pokrije zgolj prazno množico, torej \(\CCv\)
  ne drži.
\end{dokaz}

Primer podan zgoraj pa vseeno zadošča principu \(\alpo*\), ki konstruktivno
implicira ujemanje Cantorjevih in Dedekindovih realnih
števil~\cite{Birchfield24}.

Izkaže se, da za lokalno povezane prostore to tudi pričakujemo.
\begin{trditev}
  Če je \(X\) lokalno povezan, velja \(X ⊩ \Rd = \Rc ⇒ \alpo*\).
\end{trditev}
Ta izrek je zares kar posledica izreka~\ref{th:lpov-lpo}, saj je \(\lpo*\)
natanko \(\alpo*\) za Cauchyjeva realna števila.
\begin{opomba}
  Zares namesto lokalne povezanosti zadošča \(X ⊩ \Rc = \c ℝ\). To je zato, ker
  je \(\alpo*\) za Cauchyjeva realna števila ekvivalenten \(\lpo*\).

  V splošnem bi se izrek torej lahko glasil \(X ⊩ R = \c ℝ ⇒ \alpo*_R\), kjer je
  \(R\) nek objekt realnih števil (Dedekindova, Cauchyjeva, Escardo-Simpsonova,
  MacNeillova, itd.), kar smo pa za primer Dedekindovih realnih števil zares
  pokazali že v izreku~\ref{th:alpo-is-zerosets-open}.
\end{opomba}

To pomeni, da če želimo ločiti \(\Rd = \Rc\) in \(\alpo*\) potrebujemo prostor,
ki ni lokalno povezan. Seveda, \(\Ncof + (ω+1)\) dela. Na prvi komponenti ne
velja \(\CCv\), na drugi pa \(\alpo*\). A vseeno pa velja \(\CCv ∨ \alpo*\),
torej klasični dokaz enakosti \(\Rd = \Rc\) še vedno deluje.

\emph{Modificiran Fortov prostor na \(ℝ\)}, je prostor \(X=ℝ∪\{∞₀,∞₁\}\), kjer so
odprte množice podmnožice \(X\) bodisi točke \(ℝ\), bodisi kokončne množice, ki
vsebujejo kako od točk v neskončnosti.

Nad tem prostorom \(\alpo*\) in \(\CCv\) ne veljata. Prvi spodleti zaradi
funkcije, ki je na naravnih številih \(2⁻ⁿ\) in \(0\) sicer, drugi pa zaradi
relacije \(R(n,i) = ℝ⧵\{2n+i\}∪\{∞ᵢ\}\). Vseeno pa velja \(\CCv_{Σ_{\Rd}}\).
Argument je malo zakompliciran, in ni zares pomemben. Ključno je, da zvezne
funkcije iz \(X\) vedno slikajo obe točki v neskončnosti v isto vrednost, torej
se celovite relacije z vrednostmi v \(Σ_{\Rd}\) obnašajo ravno tako kot v
navadnem Fortovem prostoru na \(ℝ\), tam pa velja odvisna izbira.

Več od tega avtorici ni znano, saj je zelo težko preverjati, če nad prostorom
velja \(\Rd = \Rc\), kajti nimamo nobene karakterizacije Cauchyjevih realnih
števil v topoloških modelih. Kandidata za preveriti bi bila prebita
Knaster-Kuratowskijeva pahljača in Erdősev prostor.

% Kar je pa tudi za pričakovati (namreč, da je \(\Rd = \Rc\) šibkejši), saj če
% podrobno pogledamo klasičen dokaz enakosti, se potrebujemo števno mnogokrat
% odločiti za neenakosti z \emph{istim} realnim številom, \(\CCv\) pa govori o
% števno mnogo različnih realnih številih. Avtorica sumi, da je \(\Rd = \Rc\)
% ekvivalentno temu:
% \begin{quotation}
%   Naj bo \(p\) naraščajoče in \(q\) padajoče zaporedje, in naj bo \(\sup p ≤ \inf q\)
%   (tu morda \(=\)?), in naj bo \(x : ℝ\). Potem velja \(\for{n : ℕ}{pₙ < x ∨ x < qₙ}\),
% \end{quotation}
% kar je pa zelo šibek princip števne odločitve.

\subsection{\(\Rm{} = \Rd\)}\label{sec:reals-Rm=Rd}

% https://gist.github.com/andrejbauer/689b17b10a4e80ea409d03ec030c98b3
Andrej Bauer je 2023 za prvoaprilsko šalo objavil, kar zgleda kot konstruktiven
dokaz \(\wlem*\). V njem začne z ``znanimi dejstvi'' o MacNeilleovih realnih
številih, zraven pa podtakne še lociranost. Vemo že, da je vsako MacNeillovo
realno število locirano natanko tedaj, ko se ujemajo z Dedekindovimi realnimi
števili.
Skratka, pokazal je sledečo trditev.
\begin{trditev}
  Nad vsakim \(X\) velja \(\Rm = \Rd ⇒ \wlem*\).
\end{trditev}
\begin{dokaz}
  Dokaz je povsem konstruktiven, ampak ga lahko za topološke modele malo
  poenostavimo.

  Naj bo \(U ∈ 𝒪X\), \(\uline f ≔ χ_{¬¬U}\), in \(\bar f ≔ χ_{\cl U}\). Ti
  preslikavi sta tesni, torej definirata MacNeillovo realno število. Ta so po
  predpostavki enaka Dedekindovim, torej sta preslikavi \(\uline f\) in
  \(\bar f\) enaki, kar pomeni, da sta množici \(¬¬U\) in \(\cl U\) enaki.
  Sledi, da je \(X = \cl U ∪ ¬U = ¬¬U ∪ ¬U\), torej nad \(X\) velja \(\wlem*\).
\end{dokaz}

Izkaže pa se, da velja tudi obrat! Inspiracija za to dejstvo, je prišla iz
predmeta ``Banachove mreže'', kjer se obravnava slednji izrek:
\begin{izrek}
  Če je \(X\) ekstremalno nepovezan je \(𝒞(X,ℝ)\) polna mreža.
\end{izrek}

To pa zgleda zelo sumljivo! Če se spomnimo, to da je \(X\) ekstremalno povezan
ravno pomeni, da velja \(\wlem*\). Prav tako je \(𝒞(X,ℝ)\) znotraj množica
globalnih Dedekindovih realnih števil. Konstruktivno ta niso polna, so pa
MacNeillova realna števila, tako da bo supremum Dedekindovih realnih v
MacNeillovih realnih številih obstajal, le da bo to MacNeillovo realno število.
Potem pa lahko polnost Dedekindovih realnih števil izrazimo kot ``vsako
MacNeillovo realno število je Dedekindovo''.

Vredno je še omeniti, da je ekstremalna nepovezanost dedna lastnost na odprte
podmnožice, tako da lahko konsekvent napišemo tudi kot ``\(𝒞(U,ℝ)\) je polna
mreža za vse \(U\)''. To je pa že bistveno bližje temu, da bi rekli ``realna
števila so polna'' v interni logiki. Ampak, preveriti moramo še, da je polnost
znotraj enaka stvar kot polnost zunaj. 

Izkaže se, da nista zares ista stvar, polnost znotraj pravi da ima vsaka
poseljena \emph{odsekoma} omejena množica supremum, medtem ko polnost zunaj
zahteva globalno omejenost. Ampak to ni problem, saj je vsaka omejena množica
očitno tudi odsekoma omejena, torej je notranja trditev močnejša.

Sledi, da gornji izrek sledi iz interpretacije spodnje trditve v interni
logiki.
\begin{trditev}
  Nad vsakim \(X\) velja \(⊩ \wlem* ⇒ \Rm = \Rd\).
\end{trditev}
\begin{dokaz}
  Spomnimo se, da za MacNeillova realna števila \(x : \Rm\) velja
  \(¬(x < q) ⇒ \for{s<q}{s < x}\).

  Uporabimo \(\wlem*\) na \(x < 1\) in \(2 < x\) in ločimo na štiri primere.
  Ker hkrati oba očitno ne moreta veljati, lahko primer, ko sta oba dvojno
  negirana zanemarimo. Ostanejo le še primeri, ko vsaj ena od neenakosti ne
  velja.
  \begin{itemize}
  \item Če ne velja \(x < 1\), potem gotovo velja \(0 < x\).
  \item Če ne velja \(2 < x\), potem gotovo velja \(x < 3\).
  \end{itemize}
  V vseh primerih torej dobimo \(0 < x ∨ x < 3\), torej je \(x\) lociran.
\end{dokaz}
\begin{dokaz}[Topološki dokaz]
  Obstaja tudi topološki dokaz, a je bolj zapleten kot gornji.

  Iz dejstva, da je \(X\) ekstremalno nepovezan sledi, da je ekstremalno
  nepovezana tudi vsaka njegova odprta podmnožica. Naj bosta potem \(\uline f\)
  in \(\bar f\) par tesnih preslikav. Potem obstaja zvezna preslikava \(f\), za
  katero velja \(\uline f ≤ f ≤ \bar f\). Ker sta pa polzvezni preslikavi tesni,
  je pa \(f ≤ \uline f\) in \(\bar f ≤ f\), torej so preslikave enake in so zvezne.
\end{dokaz}

Gornji dokaz sem tudi formalizirala v dokazovalnem pomočniku Agda, kar nam da
formaliziran, popolnoma konstruktiven dokaz sledečega izreka.
\begin{izrek}\label{th:Rm=Rd-wlem}
  MacNeillova realna števila se ujemajo z Dedekindovimi natanko tedaj, ko velja
  princip šibke izključene tretje možnosti.
\end{izrek}

V literaturi se to pojavi v recimo~\cite[trd.~D4.7.11]{Johnstone02}, sem pa to
tudi formalizirala v dokazovalnem pomočniku Agda~\cite{BS25}.

Kaj pa, če bi zahtevali šibkejši pogoj, da ima vsaka števna omejena množica
supremum? Literatura kaže, da bi to moralo biti natanko ekvivalentno
\(\awlpo*\)~\cite[vaja~3N.5]{GJ60}, a avtorici ni uspelo najti dokaza tega
dejstva.

% \begin{trditev}
%   Nad \(X\) velja \(\awlpo*\) natanko tedaj, ko ima vsaka omejena, poseljena,
%   števna družina realnih števil supremum.
% \end{trditev}


%%% Local Variables:
%%% mode: latex
%%% TeX-master: "main"
%%% End:
