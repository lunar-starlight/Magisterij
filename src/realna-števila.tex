\section{Realna števila v topoloških modelih}\label{sec:reals}

\subsection{Objekti realnih števil}

V~\ref{sec:modeli-heyting} smo videli dva ``topološka'' objekta realnih števil,
objekt \(ℛ\) iz primera~\ref{ex:reals} in objekt \(\c ℝ\), definiran
v~\ref{def:constant-hvs}. Vprašanje sedaj je, ali sta ta dva objekta kako
povezana z internimi konstrukcijami realnih števil definiranih
v~\ref{sec:logika-reals}.

Izkaže se, da je odvisno. Objekt \(ℛ\) je vedno enak objektu Dedekindovih
realnih števil, objekt \(\c ℝ\) je pa malo poseben.

\begin{trditev}\label{th:Rd-maps}
  Nad \(X\) je objekt Dedekindovih realnih števil \(𝒪X\)-množica iz
  primera~\ref{ex:reals}.
  % Nad \(X\) je objekt Dedekindovih realnih števil \(𝒪X\)-množica
  % \(\set{f : U → ℝ}{U ∈ 𝒪X\text{, }f\text{ zvezna}}\), z enakostjo definirano
  % kot \(\i{f = g} ≔ \int{\set{t ∈ X}{f(t) = g(t)}}\).
\end{trditev}
\begin{dokaz}
  Naj bo \(x = \p{L, U}\) dedekindov rez v interni logiki.
  Sedaj lahko zunaj za \(t ∈ \e x\) definiramo
  \begin{align*}
    Lₜ &≔ \set{q : ℚ}{t ∈ \i{q ∈ L}}\text{ in}\\
    Uₜ &≔ \set{r : ℚ}{t ∈ \i{r ∈ U}}\text.
  \end{align*}
  Ta tvorita Dedekindov rez, tako da nam \(t ↦ \p{Lₜ, Uₜ}\) definira preslikavo
  iz \(\e x → ℝ\).
  Pokazati ostane, da je ta preslikava (označimo jo \(\hat x\)) zvezna.
  Naj bo \(\p{a,b}\) racionalen interval v \(ℝ\). Praslika tega intervala z
  \(\hat x\) je množica \(\set{t ∈ \e x}{a < \hat x(t) < b}\).
  Ta pogoj je ekvivalenten \(t ∈ \i{a ∈ L ∧ b ∈ U}\), torej je praslika kar
  enaka \(\i{a∈L∧b∈U}\), torej je odprta in je \(\hat x\) zvezna.

  Obratno, če je \(x : \e x → ℝ\) zvezna preslikava, lahko tvorimo preslikavi
  \begin{align*}
    L(t) &≔ \set{q : ℚ}{q < x(t)}\text{ in}\\
    U(t) &≔ \set{r : ℚ}{x(t) < r}\text.
  \end{align*}
  Ti znotraj tvorita Dedekindov rez \(\p{L, U}\).
\end{dokaz}

Med drugim to pomeni, da so Cauchyjeva realna števila podmnožica zveznih
preslikav v \(ℝ\). Ampak katerih?

\begin{trditev}
  Vsaka lokalno konstantna preslikava definira Cauchyjevo realno število.
\end{trditev}
\begin{dokaz}
  Brez škode za splošnost naj bo \(f : U → ℝ\) konstantna preslikava, recimo
  konstantno \({x ∈ ℝ}\). Ta ima potem pripadajoče Cauchyjevo zaporedje
  \((xₙ)ₙ\). Sedaj pa lahko tvorimo konstantne preslikave \(\hat xₙ : U → ℚ\), ki
  znotraj tvorijo Cauchyjevo zaporedje.
\end{dokaz}

\begin{trditev}\label{th:Rc-maps}
  Naj bo prostor \(X\) lokalno povezan. Tedaj je nad \(X\) objekt Cauchyjevih
  realnih števil \(𝒪X\)-množica \(\c ℝ\).
\end{trditev}
\begin{dokaz}
  Pokazati moramo zgolj, da je vsako Cauchyjevo realno število zunaj lokalno
  konstantna preslikava.

  TODO: tle mi neki ni čist všeč, glede povezanosti \(\e x\). a kej elaboriram?
  
  Nad \(t ∈ \e x\) dobimo Cauchyjevo zaporedje \((xₙ(t))ₙ\). Ker je \(X\)
  lokalno povezan ima pokritje iz povezanih množic, tako da brez škode za
  splošnost predpostavimo, da je \(\e x\) povezan. Ker pa je povezan, pa vemo,
  da so vsi \(xᵢ\) konstantne preslikave, torej morajo biti vsi \(xₙ(t)\) enaki
  (za vse \(t ∈ \e x\)), tako da je \(t ↦ (xₙ(t))ₙ\) konstantna.
\end{dokaz}

Dokaj očitno to pomeni, da noben od teh objektov v splošnem ni enak drugim.

\begin{trditev}
  Nad \(ℝ\) sta objekta Dedekindovih in Cauchyjevih realnih števil različna.
\end{trditev}
\begin{dokaz}
  Identiteta na \(ℝ\) je Dedekindovo realno število, ki pa ni lokalno
  konstantno, torej ni Cauchyjevo.
\end{dokaz}

\begin{trditev}
  Nad \(2^ℕ\) sta objekta Cauchyjevih in lokalno konstantnih realnih števil
  različna.
\end{trditev}
\begin{dokaz}
  Definirajmo \(x : 2^ℕ → ℝ\) kot \(x(α) ≔ \sum_{n=0}^∞ \frac{2α(n)}{3ⁿ}\).
  To je znotraj Cauchyjevo, saj je limita racionalnih števil. Ampak ni lokalno
  konstantno, saj je \(x⁻¹(0) = \{0\}\), ki pa ni odprta množica.
\end{dokaz}

Kaj pa MacNeillova realna števila? Teh je več kot Dedekindovih, tako da če jih
želimo karakterizirati kot nekakšne funkcije iz odprtih podmnožic \(X\), bodo te
morale biti nezvezne. Izkaže se, da so odgovor polzvezne funkcije. Seveda, če je
funkcija polzvezna navzdol in navzgor, je zvezna. Ampak, lahko vzamemo par
navzdol in navzgor polzveznih preslikav \(\uline f\) in \(\bar f\), ki sta si
nekako ``čimbolj blizu''. To pomeni, da želimo, da je \(\uline f\) največja
navzdol polzvezna preslikava manjša ali enaka \(\bar f\), in obratno.

TODO: cite elephant
\begin{trditev}\label{real:Rm-maps}
  Objekt MacNeillovih realnih števil je natanko \(ℒ\)-množica parov funkcij
  \(\p{\uline f, \bar f}\) tipa \(U → ℝ\) za \(U ∈ 𝒪X\), za katere velja
  \begin{align*}
    \bar   f(t) &=
    \limsup_{y → t}\uline f(y)\text{ in}\\
    %\inf\set{\sup \uline f(V)}{V\nbd t}\text{ in}\\
    \uline f(t) &=
    \liminf_{y → t}\bar   f(y)\text.
    % \sup\set{\inf \bar   f(V)}{V\nbd t}\text.
  \end{align*}
\end{trditev}
\begin{dokaz}
  TODO: reword all this. kaj je s temi liminfi? a to dela sploh?

  Če je \(x = \p{L,U}\) Dedekind-MacNeilleov rez, lahko za \(L\) in \(U\)
  konstruiramo funkciji \(\uline f\) in \(\bar f\) kot pri Dedekindovih rezih,
  s tem da slikamo v enostranske reze zunaj. Klasično so te ekvivalentni
  Dedekindovim rezom, tako da dobimo preslikavi \(\e x → ℝ\).

  Limes inferior definira navzdol polzvezno preslikavo, namreč supremum vseh
  navzgor polzveznih preslikav manjših ali enakih \(\bar f\).
  To pomeni, da moramo pokazati, da je vsaka navzdol polzvezna preslikava manjša
  ali enaka \(\uline f\).

  Vemo, da so navzdol polzvezne preslikave natanko enostranski rezi v interni
  logiki \(X\), tako da naj bo \(L'\) tak rez. Pogoj, da je preslikava manjša al
  enaka \(\bar f\) pa pravi, da je \(L'\) vsebovan v komplementu \(U\).

  Sedaj pa naj bo \(q ∈ L'\). Ker je \(L'\) navzgor odprt obstaja nek \(s > q\),
  ki je še vedno v \(L'\), torej je \(s ∈ Uᶜ\). To pa pomeni, da je
  \(q ∈ \int{\p{Uᶜ}} = L\), in je \(L' ⊆ L\). To pa očitno pomeni, da je
  \(f ≤ \uline f\) in je \(\uline f = \liminf_{y → t}\bar f(y)\).

  % Naj bo \(t ∈ \e x\). Ker je \(L = \int{Uᶜ}\), je
  % \(\uline f(t) = \int{\bar f(t)ᶜ} = \int{\set{r:ℚ}{t∈\i{r∈U}}ᶜ}\).
  % Komplement notranje množice je pa kar \(\set{r:ℚ}{t∉\i{r∈U}}\), čigar
  % notranjost je \(\set{r:ℚ}{t∈\i{r∉U}}\).

  % % i'm oopid, to ni treba…

  Obratno, če je pa \(\p{\uline f, \bar f}\) tak par, preslikavi definirata
  enostranska reza \(L\) in \(U\). Očitno je \(L ⊆ \int{\p{Uᶜ}}\). Ampak
  \(\int{\p{Uᶜ}}\) je tudi enostranski rez, torej definira navzdol polzvezno
  preslikavo, ki je manjša od \(\bar f\), torej je tudi \(\int{\p{Uᶜ}} ⊆ L\).
\end{dokaz}

Pogoja zgoraj implicirata, da sta preslikavi polzvezni, tako da tega posebej ne
omenjamo.
Pokažimo, da je \(\uline f\) navzdol polzvenzna, saj je dokaz za \(\bar f\)
simetričen. Naj bo \(a ∈ ℝ\). Potem je
\(\uline f⁻¹(a,∞) = \set{t ∈ \e x}{f(t) > a}\).
Po definiciji je \(\uline f(t) = \set{q : ℚ}{t ∈ \i{q < x}}\), in ta je večji
od \(a\) ko je \(a\) element \(f(t)\), torej ko velja \(t ∈ \i{a < x}\). Sledi,
da je \(\uline f⁻¹(a,∞) = \i{a < x}\), torej je navzdol polzvezna.

Spet očitno vidimo, da sta objekta Dedekindovih in MacNeillovih realnih števil
različna. MacNeillov rez definira Dedekindovo realno število, natanko tedaj, ko
sta preslikavi \(\uline f\) in \(\bar f\) enaki.

\begin{trditev}
  Nad \(\p{3,\{∅, \{1\}, \{2\}, \{1,2\}, \{0,1,2\}\}}\) ne velja \(\Rm = \Rd\).
\end{trditev}
\begin{dokaz}
  Naj bo \(x ≔ \sup\set{x : \Rm}{x = 0 ∨ x = 1∧\{1\}}\).
  To je po lemi~\ref{th:Rm-sup} MacNeillovo realno število, saj je poseljeno z
  \(0\) in omejeno z \(1\).

  Poglejmo sedaj, če velja \(x < 1 ∨ x > 0\). Če velja \(x < 1\), je potem
  \(x = 0\), torej \(\{1\}\) ne drži. Sledi, da je \(\i{x < 1} = \{2\}\).
  Podobno lahko sklepamo, da je \(\i{x > 1} = \{1\}\). To pa pomeni, da \(x\) ni
  locirano pri \(0\), torej ni Dedekindovo.
\end{dokaz}
\begin{dokaz}[Alternativni dokaz]
  Definirajmo MacNeillovo realno število s preslikavama
  \begin{align*}
    \uline f(0) = 0 && \bar f(0) = 1\\
    \uline f(1) = 1 && \bar f(1) = 1\\
    \uline f(2) = 0 && \bar f(2) = 0
  \end{align*}

  Ti očitno zadoščata pogojem zgoraj, in se na \(0\) ne ujemata, tako da ne
  definirata Dedekindovega realnega števila.
\end{dokaz}

\subsection{\(\Rc{} = \c ℝ\)}\label{sec:reals-Rc=ℝ}

Poglejmo si tu še enkrat trditev~\ref{th:lpov-lpo}. Ta pravi, da nad lokalno
povezanimi prostori velja \(\lpo*\). Zares pa potrebujemo zgolj \(\Rc = \c ℝ\).
Izkaže se, da velja tudi obrat.

\begin{trditev}
  Če nad \(X\) velja \(\lpo*\), velja \(\Rc = \c ℝ\).
\end{trditev}
\begin{dokaz}
  Naj bo \(x : \Rc\). To je zunaj preslikava \(\hat x : \e x → ℝ\), saj je vsako
  Cauchyjevo realno število tudi Dedekindovo.

  Ker velja \(\lpo*\), imamo za vsak \(a ∈ ℝ\) \(\lpo(x-a)\), torej
  \(x = a ∨ x \apart a\). Med drugim to pomeni, da so množice
  \(\set{t∈\e x}{\hat x(t) = a}\) odprte, torej je \(\hat x\) lokalno konstantna
  preslikava.
\end{dokaz}
\begin{opomba}
  Tu smo uporabili \(ℝ\) mnogo instanc \(\lpo*\).
\end{opomba}

\subsection{\(\Rd{} = \Rc\)}\label{sec:reals-Rd=Rc}

Najprej si oglejmo klasičen dokaz ekvivalence Dedekindovih in Cauchyjevih
realnih števil.
\begin{izrek}[Klasični]
  Pokazati je zgolj potrebno, da ima vsak Dedekindov rez pripadajoče Cauchyjevo
  zaporedje.
  Naj bo \(\p{L, U}\) obojestranski dedekindov rez. Cauchyjevo zaporedje lahko
  podamo kot zaporedje hitro padajočih racionalnih intervalov.

  Naj bosta \(p₀ ∈ L\) in \(q₀ ∈ U\) racionalni števili.
  Potem pa na \(n\)-tem koraku definiramo
  \[ a ≔ \frac{2pₙ + qₙ}{3}\text,\quad b ≔ \frac{pₙ + 2qₙ}{3}\text{, in}\quad
     \p{pₙ₊₁, qₙ₊₁} ≔ \begin{cases}
       \p{a, qₙ} ;& a ∈ L\\
       \p{pₙ, b} ;& b ∈ U\text.
     \end{cases}
  \]
  Te intervali hitro konvergirajo proti \(\p{L, U}\), torej je to želeno
  Cauchyjevo število.
\end{izrek}

Gornji izrek naredi števno mnogo odločitev, ko se odločamo, če velja \(a ∈ L\)
ali \(b ∈ U\) (oziroma ali velja \(a < x\) ali \(x < b\)), torej dokaz ni
konstruktiven.
Znana sta dva nekonstruktivna principa, ki sta zadostna za gornji odkaz in sta
šibkejša od izključene tretje možnosti. To sta \(\alpo*\) ter \(\CCv\).
Če velja \(\alpo*\) je potem \(a < x\) odločljivo, torej lahko vnaprej popravimo
drugi primer na \(x < b ∧ ¬\p{a < x}\).

Če pa imamo na voljo \(\CCv\) pa preprosto lahko naredimo števno mnogo
odločitev. Zares potrebujemo tu zgolj \(\CCv_{Σ_ℝ}\), kar je
pa tudi posledica \(\alpo*\).

Čeprav je iz tega očitno, da niti \(\alpo*\) niti \(\CCv\) nista potrebna za
\(\Rd = \Rc\), jih vseeno želimo strogo ločiti.

\begin{konstrukcija}
  Nad \(\Ncof\) velja \(\Rd = \Rc\), a ne velja princip števne
  disjunktivne izbire.
\end{konstrukcija}
\begin{dokaz}
  Prostor naravnih števil s kokončno topologijo ima lastnost, da je vsaka
  funkcija \(ℕ → ℝ\) konstantna. To velja tudi za (neprazne) odprte podmnožice,
  ker so števno neskončne s kokončno topologijo, kar pa pomeni, da se, v toposu
  snopov nad \(ℕ\) s to topologijo, Dedekindova in Cauchyjeva realna števila
  ujemajo.

  Naj bo \(R(n, b) ≔ ℕ⧵\{2n+b\}\). Ta relacija je celovita, saj lahko za vsak
  \(n : ℕ\) \(ℕ\) pokrijemo z \(R(n,0)\) in \(R(n,1)\). Pokažimo, da za to
  relacijo ne obstaja funkcija izbire.

  Denimo, da je \(f : ℕ → 2\) njena funkcija izbire, torej da velja
  \(\for{n:ℕ}{R(n,f(n))}\). Če bi bila \(f\) konstantno \(b\), bi potem moralo
  veljati \(ℕ ⊩ R(n, b)\) za vse \(n\), kar pa ni res. To pomeni, da sta
  \(\i{0 = f(n)}\) in \(\i{1 = f(n)}\) obe neprazni, torej, ker sta odprti,
  imata neprazen presek (ki je tudi neskončna odprta množica). Potem pa na tej
  množici velja \(0 = 1\), kar pa očitno ne drži.\contradiction

  Sledi, da funkcija izbire za ta \(R\) ne more obstajati, torej \(\CCv\) ne
  drži.
\end{dokaz}
\begin{dokaz}
  Naj bodo \(Cₙ ≔ \{ℕ⧵\{2n\}, ℕ⧵\{2n+1\}\}\) pokritja \(\Ncof\) in \(C\) njihova
  skupna pofinitev. Potem mora vsak \(U ∈ C\) biti podmnožica enega od elementov
  vsakega od \(Cₙ\). To pa pomeni, da ima \(U\) neskončen komplement, torej je
  prazna množica. Sledi, da \(C\) pokrije zgolj prazno množico, torej \(\CCv\)
  ne drži.
\end{dokaz}

Primer podan zgoraj pa vseeno zadošča principu \(\alpo*\), ki konstruktivno
implicira ujemanje Cantorjevih in Dedekindovih realnih
števil~\cite{Birchfield24}.

TODO: restructure?
Izkaže se, da za lokalno povezane prostore to tudi pričakujemo.
\begin{izrek}
  Če je \(X\) lokalno povezan in velja \(X ⊩ \Rd = \Rc\), velja \(X ⊩ \alpo*\).
\end{izrek}
Ta izrek je zares kar posledica izreka~\ref{th:lpov-lpo}, saj je \(\lpo*\)
natanko \(\alpo*\) za Cauchyjeva realna števila.
\begin{opomba}
  Zares namesto lokalne povezanosti zadošča \(X ⊩ \Rc = \c ℝ\). To je zato, ker
  je \(\alpo*\) za Cauchyjeva realna števila ekvivalenten \(\lpo*\).
  TODO: a velja \(\lpo* ⇒ \Rc = \c ℝ\)?

  V splošnem bi se izrek torej lahko glasil \(X ⊩ R = \c ℝ ⇒ \alpo*_R\), kjer je
  \(R\) nek objekt realnih števil (Dedekindova, Cauchyjeva, Escardo-Simpsonova,
  MacNeillova, itd.).
\end{opomba}

TODO: ta prostor ne dela, oziroma loči ta dva, ampak \(\CC\) vela
To pomeni, da če želimo ločiti \(\Rd = \Rc\) in \(\alpo*\) potrebujemo prostor,
ki ni lokalno povezan. Avtorica meni, da je Fortov prostor na števno neskončni
množici dober kandidat, a ji ni uspelo preveriti detajlov. Vseeno, se je pa
zadostno prepričala, da niti \(\alpo*\) niti \(\CCv\) nad tem prostorom ne
držita. Še več, ker je ta prostor \(T₆\) validira \(\aks*\), torej ne velja niti
\(\CCv\) za ``realne'' resničnostne vrednosti. 

NOTE: technically to pomen da \(\Ncof + (ω+1)\) dela, ampak ta validira
\(\CCv_{Σ_{\Rd}}\).

% TODO: res rabim prevod za to
Fortov prostor zgoraj je \(T₆\), tako da v njemu ne velja niti šibkejša oblika
\(\CCv\), ki je omejena na ``realne'' resničnostne vrednosti.
Res, če še enkrat podrobno pogledamo dokaz enakosti, se potrebujemo števno
mnogokrat odločiti za neenakosti z \emph{istim} realnim številom, \(\CCv\) pa
govori o števno mnogo različnih realnih številih.
Avtorica sumi, da je \(\Rd = \Rc\) ekvivalentno temu:
\begin{quotation}
  Naj bo \(p\) naraščajoče in \(q\) padajoče zaporedje, in naj bo \(\sup p ≤ \inf q\)
  (tu morda \(=\)?), in naj bo \(x : ℝ\). Potem velja \(\for{n : ℕ}{pₙ < x ∨ x < qₙ}\),
\end{quotation}
kar je pa zelo šibek princip števne odločitve.

\subsection{\(\Rm{} = \Rd\)}\label{sec:reals-Rm=Rd}

% https://gist.github.com/andrejbauer/689b17b10a4e80ea409d03ec030c98b3
Andrej Bauer je 2023 za prvoaprilsko šalo objavil, kar zgleda kot konstruktiven
dokaz \(\wlem*\). V njem začne z ``znanimi dejstvi'' o MacNeilleovih realnih
številih, zraven pa podtakne še lociranost. Vemo že, da je vsako MacNeillovo
realno število locirano natanko tedaj, ko se ujemajo z Dedekindovimi realnimi
števili.
Skratka, pokazal je sledečo trditev.
\begin{trditev}
  Če velja \(\Rm = \Rd\) velja \(\wlem*\).
\end{trditev}
\begin{dokaz}
  Dokaz je povsem konstruktiven, ampak ga lahko za topološke modele malo
  poenostavimo.

  Naj bo \(U ∈ 𝒪X\). Definirajmo preslikavi \(\uline f\) in \(\bar f\) tako, da
  sta na \(U\) obe \(1\), na \(¬U\) obe \(0\), izven tega naj je pa \(\uline f\)
  enaka \(0\), \(\bar f\) pa enaka \(1\).

  Na \(¬U\) torej velja \(x ≤ 0\), na \(U\) pa \(x ≥ 1\).

  To zadošča pogojema~\ref{real:Rm-maps}, torej definira MacNeillovo realno
  število.
  Sedaj pa uporabimo predpostavko, da je locirano, torej \(x < 1 ∨ x > 0\).
  Vrednost \(x < 1\) je \(¬U\), saj je \(\bar f < 1\) natanko na tej množici,
  vrednost \(x > 0\) je pa vsebovana v \(¬¬U\), saj na \(¬U\) velja \(x ≤ 0\),
  torej \(\i{x>0}\) ne more sekati \(¬U\). 
\end{dokaz}

Izkaže pa se, da velja tudi obrat! Inspiracija za to dejstvo, je prišla iz
predmeta ``Banachove mreže'', kjer se obravnava slednji izrek:
\begin{izrek}
  Če je \(X\) ekstremalno nepovezan je \(𝒞(X,ℝ)\) polna mreža.
\end{izrek}

To pa izgleda zelo sumljivo! Če se spomnimo, to da je \(X\) ekstremalno povezan
ravno pomeni, da velja \(\wlem*\). Prav tako je \(𝒞(X,ℝ)\) znotraj množica
globalnih Dedekindovih realnih števil. Konstruktivno ta niso polna, so pa
MacNeillova realna števila, tako da bo supremum Dedekindovih realnih v
MacNeillovih realnih številih obstajal, le da bo to MacNeillovo realno število.
Potem pa lahko polnost Dedekindovih realnih števil izrazimo kot ``vsako
MacNeillovo realno število je Dedekindovo''.

Vredno je še omeniti, da je ekstremalna nepovezanost dedna lastnost na odprte
podmnožice, tako da lahko konsekvent napišemo tudi kot ``\(𝒞(U,ℝ)\) je polna
mreža za vse \(U\)''. To je pa že bistveno bližje temu, da bi rekli ``realna
števila so polna'' v interni logiki. Ampak, preveriti moramo še, da je polnost
znotraj enaka stvar kot polnost zunaj. 

Izkaže se, da nista zares ista stvar, polnost znotraj pravi da ima vsaka
poseljena \emph{odsekoma} omejena množica supremum, medtem ko polnost zunaj
zahteva globalno omejenost. Ampak to ni problem, saj je vsaka omejena množica
očitno tudi odsekoma omejena, torej je notranja trditev močnejša.

TODO: a dobim tapravi obstoj tu?

Sledi, da lahko gornji izrek sledi iz interpretacije spodnje trditve v interni
logiki.
\begin{trditev}
  Velja implikacija \(\wlem* ⇒ \Rm = \Rd\).
\end{trditev}
\begin{dokaz}
  Za \(x : \Rm\) velja \(¬(x < q) ⇒ \for{s<q}{s < x}\).

  Uporabimo \(\wlem*\) na \(x < 1\) in \(2 < x\).
  Ker hkrati oba očitno ne moreta veljati, lahko primer, ko sta oba \(¬¬\)
  veljavna zanemarimo.
  \begin{itemize}
  \item Če velja \(¬(x < 1)\) potem gotovo velja \(0 < x\).
  \item Če velja \(¬(2 < x)\) potem gotovo velja \(x < 3\).
  \end{itemize}
  V vseh primerih torej dobimo \(0 < x ∨ x < 3\), torej je \(x\) lociran.
\end{dokaz}
\begin{dokaz}[Topološki dokaz]
  TODO: a v topoloških rečemo, da če je \(X\) ekstremalno nepovezan morta bit
  \(\uline f\) in \(\bar f\) zvezni, in a to pomen da sta enaki?

  NOTE: Najdla sm da pomen da je vmes med njima en zvezen… nevem zakaj to pomeni
  da sta enaka… Aaha, sem ugotovila, ker sta ostra dobimo tud da sta enaka.
\end{dokaz}

Gornji dokaz sem tudi formalizirala v dokazovalnem pomočniku Agda, kar nam da
formaliziran, popolnoma konstruktiven dokaz sledečega izreka.
\begin{izrek}\label{th:Rm=Rd-wlem}
  MacNeillova realna števila se ujemajo z Dedekindovimi natanko tedaj, ko velja
  šibka izključena tretja možnost.
\end{izrek}
TODO: objavi kodo in jo citiraj tu.

TODO: elephant D4.7.11 je to


%%% Local Variables:
%%% mode: latex
%%% TeX-master: "main"
%%% End:
