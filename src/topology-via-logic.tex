\section{Topološka interpretacija neklasičnih logičnih principov}
\label{sec:tvl}

\subsection{Znane karakterizacije \lem* in \lpo*}
\subsection{Karakterizacija (analitične) Kripkejeve sheme bis}
TODO: global existence
\begin{trditev}\label{th:lT6-have-AKS}
  Nad lokalno \(T₆\) prostori velja analitična Kripkejeva shema.
\end{trditev}
\begin{proof}
  Brez škode za splošnost lahko predpostavimo, da je prostor \(X\) \(T₆\).
  Naj bo \(U ⊆ X\). Tedaj obstaja \(f : X → ℝ\), ki je \(0\) natanko na
  komplementu \(U\). To pa pomeni, da je natanko na \(U\) različen od \(0\)
  kar je pa točno to, kar zahtevamo za analitično Kripkejevo shemo.
\end{proof}
\begin{opomba}
  V dokazu smo zares pokazali \emph{globalen} obstoj za element \(x : ℝ\). To
  nam da slutiti, da je (lokalno) \(T₆\) lastnost močnejša od \(\aks*\). In res
  se izkaže, da je temu tako.
  TODO: a se zmislim primer?
\end{opomba}

Vseeno pa velja obrat, če analitično Kripkejevo shemo malo ojačamo. Specifično,
če predpostavimo obstoj \emph{funkcije izibre} za shemo \(\aks*\).
\begin{trditev}
  Če velja \(X ⊩ \exist{f : Ω → ℝ}{\for{p : Ω}{p ⇔ f(p) \apart 0}}\), je \(X\)
  lokalno \(T₆\).
\end{trditev}
\begin{dokaz}
  TODO: a je \(ℝ^Ω\) zuni funkcije? kaj je s tem?

  razponi \(f\) pokrijejo \(X\)

  za \(f\) velja da za vse \(U\) pod \(\e f\) velja \(U = \i{f(U) \apart 0}\).

  pika TODO: napiši zadevo normalno.
\end{dokaz}
\begin{opomba}
  Ponovno smo pokazali malo več kot zgolj lokalno \(T₆\) lastnost. Dobili smo
  \emph{funkcije izbire} za vsako od \(T₆\) komponent. Če v metateoriji
  predpostavimo princip izbire, potem je to ekvivalentno tej močnejši verziji
  \(\aks*\).
\end{opomba}

% TODO: res rabim prevod za to
Fortov prostor zgoraj je \(T₆\), tako da v njemu ne velja niti šibkejša oblika
\(\CCv\), ki je omejena na ``realne'' resničnostne vrednosti.
Res, če še enkrat podrobno pogledamo dokaz enakosti, se potrebujemo števno
mnogokrat odločiti za neenakosti z \emph{istim} realnim številom, \(\CCv\) pa
govori o števno mnogo različnih realnih številih.
Avtorica sumi, da je \(\Rd = \Rc\) ekvivalentno temu:
\begin{quotation}
  Naj bo \(p\) naraščajoče in \(q\) padajoče zaporedje, in naj bo \(\sup p ≤ \inf q\)
  (tu morda \(=\)?), in naj bo \(x : ℝ\). Potem velja \(\for{n : ℕ}{pₙ < x ∨ x < qₙ}\),
\end{quotation}
kar je pa zelo šibek princip števne odločitve.


\subsection{Idempotenca \lpo* bis}
\subsection{Posplošitev funkcijske \(Tₙ\) hirearhije}

%%% Local Variables:
%%% mode: latex
%%% TeX-master: "main"
%%% End:
