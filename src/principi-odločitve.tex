\subsection{Principi odločitve v topoloških modelih}\label{sec:odločitve-odločitve}

V podrazdelku~\ref{sec:modeli-logika-odprtih} smo karakterizirali \(\lem*\) in
\(\wlem*\) z logiko odprtih množic. V tej logiki igra množica \(𝒪X\) vlogo
``tipa resničnostnih vrednosti'', v topoloških modelih smo pa v
primeru~\ref{ex:omega} tip resničnostnih vrednosti definirali kot tip \(Ω\).
Izkaže se, da se interpretacije formul kvantificiranih po \(Ω\) ne spremenijo,
če \(Ω\) zamenjamo z \(\c{𝒪X}\), saj so vsi elementi \(Ω\) globalni, formuli pa
ne vsebujeta enakosti. Tako trditvi~\ref{th:lem-is-discrete}
in~\ref{th:wlem-is-ext-disc} veljata kot navedeni tudi v topoloških modelih.
\begin{retrditev}{th:lem-is-discrete}
  Nad topološkim prostorom velja princip izključene tretje možnosti natanko
  tedaj, ko je prostor diskreten.
\end{retrditev}
\begin{retrditev}{th:wlem-is-ext-disc}
  Nad topološkim prostorom velja DeMorganov zakon natanko tedaj, ko je prostor
  ekstremalno nepovezan.
\end{retrditev}
Ker so dokazi enaki, jih tu ne ponovimo.

S pomočjo tipov pa lahko povemo tudi kaj o principih števne odločitve.

\begin{trditev}\label{th:lpov-lpo}
  Nad lokalno povezanimi prostori velja \(\lpo*\).
\end{trditev}
\begin{dokaz}
  Naj bo \(α : 2^ℕ\). Po trditvi~\ref{th:lpov-exponentiable} je \(α\) lokalno
  funkcija \(ℕ → 2\). Ker imamo zunaj \(\lpo*\), lahko odločimo
  \(α = 0 ∨ α \apart 0\) tam, kar pa pomeni, da to lahko odločimo tudi znotraj.
\end{dokaz}
To je zgolj delna karakterizacija, tako da se k tej še vrnemo.

\begin{izrek}\label{th:alpo-is-zerosets-open}
  Nad \(X\) velja \(\alpo*\) natanko tedaj, ko je vsaka ničelna množica odprta.
\end{izrek}
\begin{dokaz}
  Če je vsaka ničelna množica odprta je \(\i{x = 0}\) enak ničelni množici,
  torej skupaj z \(\i{x\apart 0}\) pokrijeta prostor.

  Obratno, če velja \(\alpo*\), morata \(\i{x=0}\) in \(\i{x\apart 0}\) pokriti
  cel prostor. Ker je ničelna množica disjunktna množici \(\i{x\apart 0}\), je
  enaka \(\i{x=0}\), torej je odprta.
\end{dokaz}

Sedaj lahko pokažemo, da implikacije iz podrazdelka~\ref{sec:logika-odločitve} v
splošnem niso obrnljive.
\begin{trditev}
  Nad Cantorjevim prostorom ne velja \(\lpo*\).
\end{trditev}
\begin{dokaz}
  Identiteta \(\id : 2^ℕ → 2^ℕ\) je zvezna, torej je znotraj element tipa \(2^ℕ\)
  in je zanjo \(\i{\lpo(\id)} = 2^ℕ⧵\{0\}\). Sledi, da \(\lpo*\) ne velja.
\end{dokaz}

\begin{trditev}
  Nad \(ℝ\) velja \(\lpo*\) in ne velja \(\alpo*\).
\end{trditev}
\begin{dokaz}
  Ker je prostor \(ℝ\) lokalno povezan, po trditvi~\ref{th:lpov-lpo} nad njim
  velja \(\lpo*\). Poglejmo si sedaj identiteto \(\id : ℝ → ℝ\). Ta je znotraj
  Dedekindovo realno število in zanjo je \(\i{\alpo(\id)} = ℝ⧵\{0\}\), torej
  \(\alpo*\) ne velja nad \(ℝ\).
\end{dokaz}

Ker prostor \(\Ncof\) naravnih števil s kokončno topologijo ni diskreten, nad
njim ne velja \(\lem*\). To lahko izkoristimo, za ločevanje \(\lem*\) od
\(\alpo*\) in \(\wlem*\) v naslednjih trditvah.

\begin{lema}\label{th:Ncof-str-conn}
  Za neprazno \(U ∈ 𝒪X\) je vsaka zvezna preslikava \(f : U → ℝ\) konstantna.
\end{lema}
\begin{dokaz}
  Naj bo \(f : U → ℝ\) zvezna preslikava in \(x, x' ∈ \im f\). Denimo, da
  velja \({x < x'}\). Potem sta množici \(f⁻¹(\frac{x+x'}2,∞)\) in
  \(f⁻¹(-∞,\frac{x+x'}2)\) neprazni odprti podmnožici \(\Ncof\), torej imata
  neprazen presek. Ampak ti množici sta očitno disjunktni, torej je \(x = x'\)
  in je funkcija \(f\) konstantna.
\end{dokaz}

\begin{trditev}
  Nad \(\Ncof\) velja \(\alpo*\) in ne velja \(\lem*\).
\end{trditev}
\begin{dokaz}
  Naj bo \(x : \Rd\). Če je \(\e x = ∅\), je \(\alpo(x)\) na prazno izpolnjen.
  Sicer, je \(x : \e x → ℝ\) zvezna preslikava, torej je po
  lemi~\ref{th:Ncof-str-conn} konstantna. Ker lahko zunaj, za nek \(t ∈ \e x\),
  odločimo \(\alpo(x(t))\), lahko posledično odločimo \(\alpo(x)\) znotraj.
\end{dokaz}

\begin{trditev}
  Nad \(\Ncof\) velja \(\wlem*\) in ne velja \(\lem*\).
\end{trditev}
\begin{dokaz}
  Naj bo \(U\) odprta množica. Njena zunanjost (torej negacija) je bodisi cel
  prostor bodisi prazna. Za ti odprti množici pa velja, da sta komplementirani
  ena drugi, torej \(\wlem*\) drži.
\end{dokaz}

% Obrate teh implikacij lahko torej jemljemo kot nekonstruktivni principi, a o teh
% ni veliko znano. Vemo le, da očitno \(\Rd = \Rc\) implicira \(\lpo* ⇒ \alpo*\),
% saj je \(\lpo*\) ekvivalenten \(\alpo*\) za Cauchyjeva realna števila, tako da
% je to zelo šibek princip. Kasneje si bomo ogledali kdaj obrati veljajo na nek
% strožji in bolj strukturiran način, ki ima povezavo s teorijo izračunljivosti,
% in ta upamo, da nam da močnejše principe, ki bodo potem tudi topološko bolj
% zanimivi.

Oglejmo si še \(\awlpo*\). Ta pravi, da za vsak \(x : ℝ\) velja \(x≤0 ∨¬(x≤0)\).
Ker je \(x≤0\) natanko \(¬(x>0)\) je \(\awlpo*\) ekvivalenten \(\wlem*_{Σ_{\Rd}}\).
Če si pogledamo definicijo ekstremalno nepovezanih prostorov, in v njej zamenjamo
odprte množice z realno odprtimi, torej množicam \(\i{x > 0}\) za \(x : \Rd\),
dobimo sledečo lastnost.
\begin{definicija}
  Prostor je \emph{realno nepovezan}, ko je za vsako funkcijo \(f : X → ℝ\)
  množica \(\cl\set{t∈X}{f(t)>0}\) odprta.
\end{definicija}

\begin{trditev}\label{th:awlpo-is-basically-disconnected}
  Nad \(X\) velja \(\awlpo*\) natanko tedaj, ko je vsak odprt podprostor \(X\)
  realno nepovezan.
\end{trditev}
Dokaz trditve je enak kot~\ref{th:wlem-is-ext-disc}, tako da ga ne ponovimo.
Velja pa zanimiva posledica, namreč, da nad realno nepovezanimi prostori
obstaja funkcija "predznak" v naslednjem smislu.
\begin{trditev}
  Če nad \(X\) velja \(\awlpo*\), potem za vsak \(x : ℝ\) obstaja
  \(u : \e x → ℝ\), tako da je \(u\) nad \(\e x\) obrnljiv in
  \(\e x ⊩ x = u⋅\abs x\).
\end{trditev}
\begin{dokaz}
  Brez škode za splošnost naj bo \(\e x = X\).

  Princip \(\awlpo*\) pravi, da je množica \(U ≔ \cl{\i{x>0}}\) odprta, zato je
  preslikava
  \[ u ≔
    \begin{cases}
       1&; U\\
      -1&; Uᶜ
    \end{cases}
  \]
  zvezna.
  %\(U ⊩ u = 1\) in \(Uᶜ ⊩ u = -1\).
  Poleg tega je \(u\) obrnljiv, saj je \(u⋅u = 1\). Prav tako velja želena
  enačba, saj je \(U\) vsebovan v \(x ≥ 0\), in velja \(f ≤ 0\) na \(Uᶜ\).
\end{dokaz}

Če želimo zares definirati predznak, moramo uporabiti \(\alpo*\). Ta
pravi, da je ničelna množica \(x\) odprta, torej lahko na njej definiramo
\[ u ≔
  \begin{cases}
     1&; x < 0\\
     0&; x = 0\\
    -1&; x > 0\text.
  \end{cases}
\]

\begin{definicija}
  Odprta množica \(U ∈ 𝒪X\) je \emph{regularna}, ko je \(U = \int{\p{\cl U}}\).
\end{definicija}
\begin{definicija}[\cite{Levy77}]
  Prostor \(X\) je \emph{skoraj P-prostor}, ko je za vsak \(f : X → ℝ\) množica
  \(\i{f > 0}\) regularna.
\end{definicija}
Seveda je vsak P-prostor tudi skoraj P-prostor.
Res, naj bodo \(Zᵢ ≔ f⁻¹[2⁻ⁱ,∞)\). Njihova unija \(Z = \i{f > 0}\) je zaprta,
saj je \(X\) P-prostor in velja \(\int{\p{\cl{Z}}} = \int Z = Z\).

\begin{trditev}\label{th:amp-is-almost-psp}
  Nad \(X\) velja \(\amp*\) natanko tedaj, ko je vsak odprt podprostor \(X\)
  skoraj P-prostor.
\end{trditev}
\begin{dokaz}
  Princip pravi, da za vsak tak \(f\) velja
  \(\int{\p{\cl{\i{f > 0}}}} ⊆ \i{f > 0}\). Ker obratna enakost očitno velja, je
  to ravno definicija regularnosti odprte množice.
\end{dokaz}
\begin{opomba}
  V~\cite[2.1]{Levy77} piše, da je vsak odprt podprostor skoraj P-prostora
  ``očitno'' skoraj P-prostor. Jaz tega ne vidim, mogoče je res samo za
  \(T_{3.5}\) prostore, na katere se omeji članek.
\end{opomba}

Konstruktivno velja \(\awlpo*∧\amp* ⇔ \alpo*\).
V trditvah~\ref{th:awlpo-is-basically-disconnected},~\ref{th:amp-is-almost-psp},
in~\ref{th:alpo-is-zerosets-open}, smo karakterizirali vse prostore, ki so v
igri v tej ekvivalenci, tako da bi morala veljati ekvivalenca med njimi.
\begin{izrek}\label{th:alpo-is-awlpo-and-amp}
  Ničelne množice zveznih preslikav \(X → ℝ\) so odprte natanko tedaj, ko je
  \(X\) skoraj P-prostor in realno nepovezan.
\end{izrek}
\begin{dokaz}
  Naj bo \(f : X → ℝ\) zvezna.
  Če je vsaka množica \(\i{f > 0}\) zaprta, je tudi regularna (vse odprto zaprte
  množice so regularne). Prav tako je njeno zaprtje odprto.

  Obratno, če je \(\i{f > 0} = \int{\p{\cl{\i{f>0}}}}\), in je zaprtje
  \(\i{f>0}\) odprto, je potem \(\i{f>0} = \cl{\i{f>0}}\), torej je zaprta.
\end{dokaz}
Avtorici ni znano, če se ta izrek pojavi kje v literaturi, predvsem ker je
večina literature o teh prostorih objavljene pod predpostavko \(T_{3.5}\).
Pod predpostavko \(T_{3.5}\) torej najdemo izrek
"P-prostori so natanko realno nepovezani skoraj P-prostori" v~\cite{Levy77}
in~\cite[4J(3)]{GJ60}.
To porodi dve zanimivi vprašanji. Prvič, kako nujna je predpostavka \(T_{3.5}\)
za razvoj teorije kolobarjev realnih funkcij, in drugič, kaj lastnost
\(T_{3.5}\) pomeni v interni logiki topološkega modela. Avtorica žal nima
odgovora na nobeno od teh vprašanj, saj je prvo preobsežno za to delo, za
drugega pa ni našla pravega navdiha.


%%% Local Variables:
%%% mode: latex
%%% TeX-master: "main"
%%% End:
