\section{Principi odločitve v topoloških modelih}\label{sec:odločitve}

Spomnimo se najprej karakterizacij \(\lem*\) in \(\wlem*\) iz
podrazdelka~\ref{sec:modeli-logika-odprtih}.

\begin{retrditev}{th:lem-is-discrete}
  Nad topološkim prostorom velja princip izključene tretje možnosti natanko
  tedaj, ko je prostor diskreten.
\end{retrditev}
\begin{retrditev}{th:wlem-is-ext-disc}
  Nad topološkim prostorom velja DeMorganov zakon natanko tedaj, ko je prostor
  ekstremalno nepovezan.
\end{retrditev}

Pokažimo sedaj, da implikacije iz~\ref{sec:logika-odločitve} niso obrnljive.

\begin{trditev}
  Nad \(2^ℕ\) ne velja \(\lpo*\).
\end{trditev}
\begin{dokaz}
  Naj bo \(α : 2^ℕ → 2^ℕ\) identiteta. Resničnostna vrednost \(\lpo(α)\) je
  \(2^ℕ⧵\{0\}\), saj je resničnostna vrednost \(α = 0\) enaka \(∅\).
\end{dokaz}

\begin{trditev}
  Nad \(ℝ\) velja \(\lpo*\) in ne velja \(\alpo*\).
\end{trditev}
\begin{dokaz}
  Naj bo \(x : ℝ\) zunaj identiteta. Potem je \(\i{\alpo(x)} = ℝ⧵\{0\}\).

  Preostanek tega dokaza izpeljemo malo kasneje, v trditvi~\ref{th:lpov-lpo},
  saj je \(ℝ\) lokalno povezana.
\end{dokaz}

\begin{trditev}
  Nad \(\Ncof\) velja \(\alpo*\) in ne velja \(\lem*\).
\end{trditev}
\begin{dokaz}
  Prostor očitno ni diskreten. Naj bo \(x : ℝ\), torej preslikava \(\e x → ℝ\).
  Naj bo \(t, t' ∈ \im x\). Potem sta \(x⁻¹(t)\) in \(x⁻¹(t')\) neprazni odprti
  množica. To pa pomeni, da sta obe kokončni, torej imata neprazen presek, od
  koder sledi, da sta si enaka. Torej je vsako notranje realno število
  konstantno. Sedaj pa lahko za to realno število \(\alpo(x)\) odločimo zunaj.
\end{dokaz}

\begin{trditev}
  Nad \(\Ncof\) velja \(\wlem*\) in ne velja \(\lem*\).
\end{trditev}
\begin{dokaz}
  Prostor še vedno ni diskreten. Naj bo sedaj \(U\) odprta množica.
  Njena zunanjost (torej negacija) je pa bodisi cel prostor, bodisi prazna. Za
  te odprte množice pa velja da so komplementirane, torej \(\wlem*\) drži.
\end{dokaz}

Obrate teh implikacij lahko torej jemljemo kot nekonstruktivni principi, a o teh
ni veliko znano. Vemo le, da očitno \(\Rd = \Rc\) implicira \(\lpo* ⇒ \alpo*\),
saj je \(\lpo*\) ekvivalenten \(\alpo*\) za Cauchyjeva realna števila, tako da
je to zelo šibek princip. Kasneje si bomo ogledali kdaj obrati veljajo na nek
strožji in bolj strukturiran način, ki ima povezavo s teorijo izračunljivosti,
in ta upamo, da nam da močnejše principe, ki bodo potem tudi topološko bolj
zanimivi.

TODO: vezno besedilo

\begin{trditev}\label{th:lpov-lpo}
  Če je \(X\) lokalno povezan, velja \(X ⊩ \lpo*\).
\end{trditev}
\begin{dokaz}
  Naj bo \(α : 2^ℕ\). Po trdtivi~\ref{th:lpov-exponentiable} je torej \(α\)
  lokalno funkcija \(ℕ → 2\). Ker imamo zunaj \(\lpo*\), lahko odločimo
  \(α = 0 ∨ α \apart 0\) tam, kar pa pomeni, da to lahko odločimo tudi znotraj.
\end{dokaz}


\begin{izrek}\label{th:alpo-is-zerosets-open}
  Nad \(X\) velja \(\alpo*\) natanko tedaj, ko je vsaka ničelna množica odprta.
\end{izrek}
\begin{dokaz}
  Če je vsaka ničelna množica odprta je \(\i{x = 0}\) enak ničelni množici,
  torej skupaj z \(\i{x\apart 0}\) pokrijeta prostor.

  Obratno, če velja \(\alpo*\), morata \(\i{x=0}\) in \(\i{x\apart 0}\) pokriti
  cel prostor. Ker je ničelna množica disjunktna z \(\i{x\apart 0}\), mora biti
  potem enaka \(\i{x=0}\), torej je odprta.
\end{dokaz}

Oglejmo si še \(\awlpo*\). Ta pravi, da za vsak \(x : ℝ\) velja \(x≤0 ∨¬(x≤0)\).
Ker je \(x≤0\) natanko \(¬(x>0)\) je to natanko \(\wlem*_{Σ_{\Rd}}\).
Če si pogledamo definicijo ekstremalno nepovezanih prostorov, in v nej zamenjamo
odprte množice z realnimi, dobimo sledečo lastnost.
\begin{definicija}
  Prostor je \emph{TODO: basically disconnected}, ko je za vsako funkcijo
  \(f : X → ℝ\) množica \(\cl\set{t∈X}{f(t)>0}\) odprta.
\end{definicija}


\begin{trditev}\label{th:awlpo-is-basically-disconnected}
  Nad \(X\) velja \(\awlpo*\) natanko tedaj, ko je vsaka odpra podmnožica \(X\)
  basically disconnected.
\end{trditev}
Dokaz trditve je enak kot~\ref{th:wlem-is-ext-disc}, tako da ga ne ponovimo.
Velja pa zanimiva posledica, namreč, da nad basically disconnected prostori
obstaja funkcija \(\mathrm{sgn}\) (oziroma nekaj dovolj podobnega).
\begin{trditev}
  Če nad \(X\) velja \(\awlpo*\), potem za vsak \(x : ℝ\) obstaja
  \(u : \e x → ℝ\), tako da je \(u\) nad \(\e x\) obrnljiv, in velja
  \(\e x ⊩ x = u\abs x\).
\end{trditev}
\begin{dokaz}
  Brez škode za splošnost naj bo \(\e x = X\).

  Princip \(\awlpo*\) potem pravi, da je množica \(U ≔ \cl{\i{x>0}}\) odprta,
  torej tvori particijo in lahko definiramo
  \[ u ≔
    \begin{cases}
       1&; U\\
      -1&; Uᶜ\text.
    \end{cases}
  \]
  %\(U ⊩ u = 1\) in \(Uᶜ ⊩ u = -1\).
  Ta \(u\) je obrnljiv, saj je \(u⋅u = 1\). Prav tako velja želena enačba, saj
  je \(U\) vsebovan v \(x ≥ 0\), in velja \(f ≤ 0\) na \(Uᶜ\).
\end{dokaz}

Če pa želimo zares definirati \(\mathrm{sgn}\), moramo uporabiti \(\alpo\). Ta
pravi, da je ničelna množica \(x\) odprta, torej lahko tam definiramo
\[ u ≔
  \begin{cases}
     1&; x < 0\\
     0&; x = 0\\
    -1&; x > 0\text.
  \end{cases}
\]

\begin{definicija}
  Prostor \(X\) je \emph{skoraj P-prostor}, ko je za vsak \(f : X → ℝ\) množica
  \(\i{f > 0}\) regularna~\cite{Levy77}.
\end{definicija}
Kot bi sklepali iz imena je vsak P-prostor tudi skoraj P-prostor.
\begin{dokaz}
  Naj bodo \(Zᵢ ≔ f⁻¹[2⁻ⁱ,∞)\). Njihova unija \(Z = \i{f > 0}\) je zaprta, saj
  je \(X\) P-prostor. Potem je pa \(\int{\cl{Z}} = \int Z = Z\).
\end{dokaz}

\begin{trditev}\label{th:amp-is-almost-psp}
  Nad \(X\) velja \(\amp*\) natanko tedaj, ko je vsaka odprta podmnožica \(X\)
  skoraj P-prostor.
\end{trditev}
\begin{dokaz}
  Princip pravi, da za vsak tak \(f\) velja
  \(\int{\p{\cl{\i{f > 0}}}} ⊆ \i{f > 0}\). Ker obratna enakost očitno velja, je
  to ravno definicija regularnosti odprte množice.
\end{dokaz}
Analogno velja tudi \(\mp*\) natanko tedaj, ko je vsaka semiodločljiva odprta
množica regularna.
\begin{opomba}
  V~\cite[2.1]{Levy77} piše, da je vsaka odprta podmnožica skoraj P-prostora
  ``očitno'' skoraj P-prostor. Jaz tega ne vidim, mogoče je res samo za
  \(T_{3.5}\) prostore, kot imajo v članku.
\end{opomba}

Konstruktivno velja \(\awlpo*∧\amp* ⇔ \alpo*\).
V trditvah~\ref{th:awlpo-is-basically-disconnected},~\ref{th:amp-is-almost-psp},
in~\ref{th:alpo-is-zerosets-open}, smo karakterizirali vse prostore, ki so v
igri v tej ekvivalenci, tako da bi morala veljati ekvivalenca med njimi.
\begin{izrek}
  Ničelne množice \(f : X → ℝ\) so odprte natanko tedaj, ko je \(X\) skoraj
  P-prostor in basically disconnected.
\end{izrek}
\begin{dokaz}
  Če je vsaka množica \(\i{f > 0}\) zaprta, je tudi regularna (vse odprto zaprte
  množice so regularne). Prav tako je njeno zaprtje odprto.

  Obratno, če je \(\i{f > 0} = \int{\p{\cl{\i{f>0}}}}\), in je zaprtje
  \(\i{f>0}\) odprto, je potem \(\i{f>0} = \cl{\i{f>0}}\), torej je zaprta.
\end{dokaz}
Avtorici ni znano, če se ta izrek pojavi kje v literaturi, predvsem ker je
večina literature o teh prostorih objavljene pod predpostavko \(T_{3.5}\).
Če to predpostavimo dobimo, da je \(X\) P-prostor natanko tedaj, ko je skoraj
P-prostor in basically disconnected~\cite{Levy77}\cite[4J(3)]{GJ60}.


%%% Local Variables:
%%% mode: latex
%%% TeX-master: "main"
%%% End:
