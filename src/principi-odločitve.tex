\subsection{Principi odločitve}\label{sec:odločitve}

\begin{definicija}[Izključena tretja možnost]\label{pr:lem}
  \emph{Princip izključene tretje možnosti} pravi, da za vsako resničnostno
  vrednost \(p\) velja \(p∨¬p\). Formulo \(p∨¬p\) označimo \(\lem(p)\), formulo
  \(\for{p:Ω}{p∨¬p}\) pa z \(\lem*\).
\end{definicija}

\begin{definicija}[Šibka izključena tretja možnost]\label{pr:wlem}
  \emph{Šibak princip izključene tretje možnosti} pravi, da za vsako
  resničnostno vrednost \(p\) velja \(¬p∨¬¬p\). Formulo \(¬p∨¬¬p\) označimo
  \(\wlem(p)\), formulo \(\for{p:Ω}{¬p∨¬¬p}\) pa z \(\wlem*\).
\end{definicija}
\begin{trditev}
  Velja implikacija \(\lem* ⇒ \wlem*\).
\end{trditev}
\begin{dokaz}
  Formula \(\wlem(p)\) je natanko \(\lem(¬p)\), tako da trditev očitno velja.
\end{dokaz}

Spomnimo se najprej karakterizacij \(\lem*\) in \(\wlem*\) iz
podrazdelka~\ref{sec:modeli-logika-odprtih}.

TODO: renumber
\begin{trditev}\label{th:lem-is-partition-second}
  Nad topološkim prostorom velja princip izključene tretje možnosti natanko
  tedaj, ko je prostor particijski.
\end{trditev}
\begin{trditev}\label{th:wlem-is-ext-disc-second}
  Nad topološkim prostorom velja DeMorganov zakon natanko tedaj, ko je prostor
  ekstremalno nepovezan.
\end{trditev}

Ker so naši prostori \(T₀\), so particijski prostori natanko diskretni. Res, če
so točke zaprte, in je vsaka zaprta množica odprta, so točke odprte, torej je
prostor diskreten.

\begin{definicija}\label{pr:lpo}
  \emph{Princip števne odločitve} pravi, da za vsako števno zaporedje ničel in enic
  lahko odločimo, ali je celo nič, ali pa obstaja mesto, na katerem je enica.
  Formulo \(α = 0∨α\apart 0\) označimo \(\lpo(α)\), formulo
  \(\for{α : 2^ℕ}{\lpo(α)}\) pa z \(\lpo*\).
\end{definicija}

\begin{definicija}\label{pr:alpo}
  \emph{Analitični princip števne odločitve} pravi, da za vsako realno število
  lahko odločimo, ali je pozitivno ali nenegativno. Formulo \(x > 0 ∨ x ≤ 0\)
  označimo \(\alpo(x)\), formulo \(\for{x : ℝ}{\alpo(x)}\) pa z \(\alpo*\).
\end{definicija}

\begin{trditev}
  Velja veriga implikacij \(\lem* ⇒ \alpo* ⇒ \lpo*\).
\end{trditev}
\begin{dokaz}
  Ker je \(¬(x > 0)\) natanko \(x ≤ 0\), je \(\alpo*\) očitno posledica
  \(\lem*\).

  Za drugo implikacijo pa potrebujemo malo dela. Naj bo \(α\) zaporedje, za
  katerega odločamo, ali ima na kakem mestu enico. Definirajmo realno število
  \[ x ≔ \lim_{n → ∞}2^{-\min\set{k : ℕ}{α(k) = 1 ∨ k ≥ n}}\text. \]

  To je po definiciji Cauchyjevo realno število. Uporabimo sedaj predpostavko
  \(\alpo(x)\). Če velja \(x ≤ 0\), mora zaporedje limitirati proti \(0\), kar
  pa pomeni, da \(α(k) = 1\) ni nikoli zadoščeno, torej je \(α = 0\).
  Po drugi strani, če je \(x > 0\) pa obstaja nek \(k : ℕ\), tako da je
  \(x > 2⁻ᵏ\). To pa pomeni, da ima \(α\) enico na enem izmed prvih \(k\)
  mestih, kar konstruktivno pomeni, da želeni indeks obstaja.
\end{dokaz}

Pokažimo sedaj, da gornje implikacije niso obrnljive.

\begin{trditev}
  Nad \(2^ℕ\) ne velja \(\lpo*\).
\end{trditev}
\begin{dokaz}
  Naj bo \(α : 2^ℕ → 2^ℕ\) identiteta. Resničnostna vrednost \(\lpo(α)\) je
  \(2^ℕ⧵\{0\}\), saj je resničnostna vrednost \(α = 0\) enaka \(∅\).
\end{dokaz}

\begin{trditev}
  Nad \(ℝ\) velja \(\lpo*\) in ne velja \(\alpo*\).
\end{trditev}
\begin{dokaz}
  Naj bo \(x : ℝ\) zunaj identiteta. Potem je \(\i{\alpo(x)} = ℝ⧵\{0\}\).

  Preostanek tega dokaza izpeljemo malo kasneje, v izreku~\ref{th:lpov-lpo}
\end{dokaz}

\begin{trditev}
  Nad \(\Ncof\) velja \(\alpo*\) in ne velja \(\lem*\).
\end{trditev}
\begin{dokaz}
  Prostor očitno ni diskreten. Naj bo \(x : ℝ\), torej preslikava \(\e x → ℝ\).
  Naj bo \(t, t' ∈ \im x\). Potem sta \(x⁻¹(t)\) in \(x⁻¹(t')\) neprazni odprti
  množica. To pa pomeni, da sta obe kokončni, torej imata neprazen presek, od
  koder sledi, da sta si enaka. Torej je vsako notranje realno število
  konstantno. Sedaj pa lahko za to realno število \(\alpo(x)\) odločimo zunaj.
\end{dokaz}

\begin{trditev}
  Nad \(\Ncof\) velja \(\wlem*\) in ne velja \(\lem*\).
\end{trditev}
\begin{dokaz}
  Prostor še vedno ni diskreten. Naj bo sedaj \(U\) odprta množica.
  Njena zunanjost (torej negacija) je pa bodisi cel prostor, bodisi prazna. Za
  te odprte množice pa velja da so komplementirane, torej \(\wlem*\) drži.
\end{dokaz}

Obrate teh implikacij lahko torej jemljemo kot nekonstruktivni principi, a o teh
ni veliko znano. Vemo le, da očitno \(\Rd = \Rc\) implicira \(\alpo* ⇔ \lpo*\),
saj je \(\lpo*\) ekvivalenten \(\alpo*\) za Cauchyjeva realna števila, tako da
je to zelo šibek princip. Kasneje si bomo ogledali kdaj obrati veljajo na nek
strožji in bolj strukturiran način, ki ima povezavo s teorijo izračunljivosti,
in ta upamo, da nam da močnejše principe, ki bodo potem tudi topološko bolj
zanimivi.

Vemo pa vsaj nekaj.
\begin{izrek}\label{th:lpov-lpo}
  Če je \(X\) lokalno povezan, velja \(X ⊩ \lpo*\).
\end{izrek}
\begin{dokaz}
  Naj bo \(α : 2^ℕ\). Ker je \(X\) lokalno povezan, je množica \(2^ℕ\) kar
  množica preslikav iz \(ℕ\) v \(2\). To pa pomeni, da je lokalno \(α = f\), za
  nek \(f : ℕ → 2\). Zunaj pa imamo \(\lpo*\), tako da lahko odločimo \(f = 0 ∨
  f \apart 0\), kar pa pomeni, da to lahko odločimo tudi za \(α\), torej nad
  \(X\) velja \(\lpo*\).
\end{dokaz}
\begin{opomba}
  Ker je \(\lpo*\) natanko \(\alpo*\) za Cauchyjeva realna števila (TODO: daj to
  nekam na samo, zdej sem že petič napisala), in potrebujemo le, da so elementi
  lokalno konstantni, zadošča predpostaviti \(\Rc = \c ℝ\). To pomeni, da je to
  kar močen princip, saj je zadosten za \(\lpo*\).

  TODO: a je ekvivalenca? smz da, ker \(x : \Rc\) je zvezna \(\e x → ℝ\), če
  mamo \(\lpo*\) je \(\i{x = a} = \i{x \apart a}ᶜ\), tko da je \(x\) lokalno
  konstantna.
\end{opomba}

TODO: Elephant, D4.7: It can be shown that, for a locale \(X\), \(\Rc\) coincides
with \(\c ℝ\) iff for every open \(U\), the lattice of (relatively) clopen
sublocales of \(U\) is closed under countable unions. (For a second countable
locale this is equivalent to local connectedness)


%%% Local Variables:
%%% mode: latex
%%% TeX-master: "main"
%%% End:
