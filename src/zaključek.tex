\section{Zaključek}

Videli smo, da lahko topološke modele uporabimo, da študiramo topologoijo samo.
Zgradili smo slovar, ki nekatere topološke lastnosti identificira z
nekonstruktivnimi principi. Za lastnosti, ki se ne izrazijo tako lepo, smo pa
našli čim boljše približke, ali pa vsaj implikacije, v jeziku topoloških
modelov. Videli smo, da topološke lastnosti, ki se preverjajo z realnimi
funkcijami, lahko vidimo kot izreke o realnih številih v našem modelu. To ni
najbolj naravno v konstruktivni logiki, kjer bolj pogosto govorimo o zaporedjih
ničel in enk, kar pa tudi ponudi očitno posplošitev \(Tₙ\) lastnosti
(za \(n = 2.5,~3.5,4,6\)), kjer obravnavamo funkcije z vrednostmi v Kantorjevem
prostoru, ali pa katerem drugem topološkem prostoru. Študirali smo tudi, kaj je
izračunljivo glede topoloških lastnosti in spoznali, da Kantorjev prostor nekako
podpira neskončno sočasno izvajanje procesov.

Nadaljno delo bi lahko bolje raziskalo povezavo med izračunljivostjo in \(Tₙ\)
lastnostmi za druge prostore (Kantorjev prostor zgleda na prvi pogled najbolj
primeren za začetek take raziskave). Poleg tega nam je zmanjkalo časa globje
preučiti pretvorbe primerkov, da bi našli \emph{topološko} interpretacija zanje.
V slovarju je veliko lukenj, predvsem bi bilo pa zanimivo preučiti kaj pravi
% TODO: kaj je to dejansko? IVT po angl.
recimo izrek o vmesni vrednosti, ali kak drug izrek iz analize, ki ne drži
konstruktivno. Prav tako se očitno da program razširiti na okoliše, kar nam v
tem delu ni uspelo, predvsem zaradi časovnih omejitev, pa tudi zaradi pedagoških
razlogov to za to delo raje izpustimo.


%%% Local Variables:
%%% mode: latex
%%% TeX-master: "main"
%%% End:

