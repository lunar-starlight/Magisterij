\documentclass[handout]{beamer}
%\documentclass{beamer}

% \usetheme{boxes} % see http://www.deic.uab.es/~iblanes/beamer_gallery/ for lots of examples
\usetheme{metropolis}
\usecolortheme{rose}
% \useinnertheme{circles}
% \useoutertheme{split}
% \setbeamertemplate{blocks}[rounded][shadow=true]

\setbeamertemplate{navigation symbols}{} % remove navigation symbols
\setbeamertemplate{footline}{} % remove title, too long

% %next set colors - not needed
% \setbeamercolor{title}{fg=black!70!black}
% \setbeamercolor{frametitle}{fg=blue!70!black}
% \setbeamercolor{framesubtitle}{fg=green!30!black}
% \setbeamercolor{author}{fg=red!70!black}
% \setbeamercolor{institute}{fg=green!30!black}
% \setbeamercolor{date}{fg=blue!50!black}

% \usepackage[T1]{fontenc}        % kodiranje znakov v .pdf
% \usepackage[utf8]{inputenc}     % kodiranje znakov v .tex
% \usepackage[slovene]{babel}     % nastavimo slovenščino
% \usepackage{stmaryrd}

\usepackage{fontspec}
\usepackage{mathtools}
\usepackage{unicode-math}
\usepackage{enumerate}

\usepackage{graphicx}


\setmainfont{Latin Modern Roman}
% \setmainfont{TeX Gyre Pagella}
% \setmathfont{TeX Gyre Pagella Math}
\setmathfont{Latin Modern Math}
\setmathfont{Asana Math}[range={scr}]
\setmathfont{STIX Two Math-Regular}[range={bb,"1D538-"1D56B,"0211D}]
\setmathfont{STIX Two Math-Regular}[range={"2102,"210D,"2115,"2119,"211A,"211D,"2124}]
% \setmathfont{TeX Gyre Pagella Math}[range={8730}] %sqrt
% \setmathfont{Asana Math}[range={"007B,"007D}]  % {}
\setmathfont{Asana Math}[range={8709,\setminus,"2216,"29F5,"219D}]  % U+2205, emptyset
\setmathfont{Asana Math}[range={10631, 10632}]  % U+2987, U+2988, bb parenthesis

\usepackage[normalem]{ulem}
\renewcommand{\ULdepth}{1.8pt}
\newcommand{\ul}[1]{\uline{#1}}


\usepackage{amsfonts,amsmath,amssymb,amsthm}
\theoremstyle{plain}
\newtheorem{izrek}{Izrek}
\newtheorem{trditev}{Trditev}
\newtheorem{lema}{Lema}
\newtheorem*{posledica}{Posledica}

\theoremstyle{definition}
\newtheorem{definicija}{Definicija}

\newtheorem*{primer}{Primer}
\newtheorem*{primer*}{Primer}
\newtheorem*{primeri}{Primeri}

\theoremstyle{remark}
\newtheorem*{opomba}{Opomba}

% \beamertemplatetransparentcovereddynamic

\title{Topološke lastnosti so logični principi v topoloških modelih}
\author{Luna Strah}
\institute{Univerza v Ljubljani, Fakulteta za matematiko in fiziko}
\date{19.~5.~2025}

\newcommand{\eps}{\varepsilon}
\renewcommand{\hat}{\widehat}
\renewcommand{\tilde}{\widetilde}


\def\bN{\mb{N}}
\def\bR{\mb{R}}
\def\subq{\subseteq}
\def\phi{\varphi}


\DeclarePairedDelimiterX{\parens}[1]{(}{)}{#1}
\DeclarePairedDelimiterX{\absolute}[1]{\lvert}{\rvert}{#1}
\def\abs{\absolute*}
\newcommand{\cli}[1]{\left[ {#1} \right]}
\newcommand{\floor}[1]{\left\lfloor {#1} \right\rfloor}
\newcommand{\set}[2]{\left\{ #1 \mid #2 \right\}}
\newcommand{\apart}{\ensuremath{\mathrel{\#}}}
\newcommand{\nbd}{\oset{\text{\clap{\tiny nbd}}}{∋}}
\newcommand{\quot}[2]{{#1}/_{\!#2}}
\newcommand{\mb}[1]{\mathbf{#1}}
\newcommand{\bb}[1]{\mathbb{#1}}
\newcommand{\mf}[1]{\mathfrak{#1}}
\newcommand{\mc}[1]{\mathcal{#1}}
\newcommand{\cat}[1]{\mathbf{#1}}
\newcommand{\opcat}[1]{\left(\mathbf{#1}\right)^{op}}
%\newcommand{\op}[1]{{#1}^{op}}
\newcommand{\op}{\oset{\text{op}}{⊆}}
\newcommand{\res}[1]{↾_{\hspace{-0.15em}#1}}
% \newcommand{\res}[2]{#1{↾_{\hspace{-0.15em}#2}}}
\newcommand{\sh}[1]{\textrm{Sh}{\left( #1 \right)}}
%\newcommand{\psh}[1]{\textrm{PSh}{\left( #1 \right)}}
\newcommand{\psh}[1]{\hat{#1}}
\DeclareMathOperator{\im}{im}
\DeclareMathOperator{\coker}{coker}
\DeclareMathOperator{\coim}{coim}
\DeclareMathOperator{\id}{id}
\DeclareMathOperator{\codim}{codim}
\DeclareMathOperator{\cl}{cl}
\DeclareMathOperator{\ext}{ext}
\DeclareMathOperator{\dom}{dom}
\DeclareMathOperator{\ev}{ev}
\DeclareMathOperator{\dm}{DM}
%\DeclareMathOperator{\cov}{Cov}
\AtBeginDocument{
  %\def\c#1{\left( {#1} \right)^c}
  %\def\c#1{{#1}^c}
  \def\c{\uline}
  \newcommand{\g}[1]{\left\langle {#1} \right\rangle}
  \renewcommand{\b}[1]{\left\{ {#1} \right\}}
  \renewcommand{\i}[1]{\left⟦ {#1} \right⟧}
  \newcommand{\e}[1]{\left‖ {#1} \right‖}
  \newcommand{\s}[1]{\left\{ {#1} \right\}}
  \renewcommand{\O}[1]{\mathcal{O}{#1}}
  \renewcommand{\int}{\textrm{int}}
  \def\p{\parens*}
}


\newcommand{\defquantifier}[2]{%
  \expandafter\undef\csname #1\endcsname%
  \expandafter\newcommand\csname #1\endcsname[2]{{#2 ##1.}\;##2}%
}
\defquantifier{for}{\forall}
\defquantifier{exist}{\exists}
\defquantifier{unique}{\exists!}
%\defquantifier{globalen}{\exists ᵍ}
\defquantifier{exact}{ι}
\defquantifier{eventually}{\nabla}
\AtBeginDocument{
\defquantifier{sigma}{\Sigma}
\defquantifier{pi}{\Pi}
}
\renewcommand{\check}{ \(\checkmark\)}

\newcommand{\germ}[2]{\textrm{germ}_{#2}#1}
%\usepackage{xparse}
\makeatletter
\NewDocumentCommand{\@defprinciple}{mmm}{%
  \ExpandArgs{c}\NewDocumentCommand{#1}{s}{%
    \IfBooleanTF##1%
    {\textnormal{\sffamily #2}}%
    {\textnormal{\sffamily #3}}%
  }%
  \AtEndPreamble{%
    \pdfstringdefDisableCommands{%
      \expandafter\def\csname #1\endcsname*{#2}%
    }%
  }%
}
\newcommand{\defprinciple}[1]{\@defprinciple{#1}{\MakeUppercase{#1}}{\MakeLowercase{#1}}}
\makeatother
\newcommand{\principle}[1]{\textnormal{\sffamily #1}} % TODO: rename
\defprinciple{lem}
\defprinciple{wlem}
\defprinciple{lpo}
\defprinciple{wlpo}
\AtBeginDocument{  
\undef\mp
\defprinciple{mp}
}
\defprinciple{ks}
\defprinciple{alpo}
\defprinciple{awlpo}
%\defprinciple{allpo}
\defprinciple{amp}
\defprinciple{aks}
\newcommand{\AC}{\principle{AC}}
\newcommand{\IAC}{\principle{IAC}}
\newcommand{\CC}{\principle{CC}}
\newcommand{\CCv}{\principle{CC}^∨}
\newcommand{\DC}{\principle{DC}}
\def\Rm{ℝ_{\text{M}}}
\def\Rd{ℝ_{\text{D}}}
\def\Rc{ℝ_{\text{C}}}
\def\Ncof{ℕ^{\text{cof}}}

\setbeameroption{hide notes} % Only slides
% \setbeameroption{show only notes} % Only notes
% \setbeameroption{show notes} % Both
% \setbeameroption{show notes on second screen=right} % Both

\usepackage{hyperref}
\hypersetup{pdfpagemode=FullScreen}

\usepackage[style=alphabetic,
            hyperref=auto,
            isbn=false,
            doi=true,
            url=false,
            date=year,
            giveninits=true,
            maxnames=3,
            useprefix=true,
            backend=biber]{biblatex}
\bibliography{../src/magisterij.bib}

\begin{document}
%%%%%
\frame{
  \titlepage
  
  \note{
    Dober dan, ime mi je Luna, dans vam bom predstavljala mojo magistrsko
    nalogo, ki ma zaenkrat naslov ``Nekonstruktivni principi v topoloških
    modelih'', ampak mi ni čist všeč, tko da.

    Zdej, kaj je zgodba tuki, mal mi boste mogl na besedo verjet, ampak glavno
    je, da za vsak topološki prostor lahko skonstruiramo matematični svet (tem
    se potem reče topološki modeli), kjer so resničnostne vrednosti natanko
    odprte množice prostora. To je recimo praktično, če želimo gledati recimo
    pozitivnost realne funkcije. Namesto, da se vprašamo, če je funkcija
    pozitivna, se vprašamo \emph{kje} je funkcija pozitivna, kar je mogoče
    zanimivo, oziroma js trdim, da je.
  }
}
\frame{
  \titlepage
  
  \note{
    Zdej mamo pa na eni strani prostore, na drugi pa topološke modele. In
    nekak za prostore kot celota mamo neke lastnosti, recimo \(T₁\), \(T₂\),
    \(T₆\), lokalna povezanost, kompaktnost, svašta.
    Na drugi strani mamo pa modele logike, ki se izkaže, da ponavadi ni
    klasična. To pomeni, da recimo izključena tretja možnost ali pa aksiom
    izbire ne nujno držjo. Ampak tako kot za prostore, eni bojo mel, drugi pa
    ne. In zdaj se lahko vprašamo, a se katere topološke lastnosti da smiselno
    izražat z nekonstruktivnimi principi v konstruktivni logiki, preko te
    konstrukcije topoloških modelov.
  }
}
%%%%%
\begin{frame}
  \frametitle{Heytingovo vrednotene množice}

  \begin{definicija}
    \emph{Heytingovo vrednotena množica} ali \emph{tip} \(A\) je množica \(A\)
    skupaj s preslikavo \(⟦- =_A -⟧ : A×A → 𝒪X\), tako da za vse \(a\), \(b\),
    in \(c ∈ A\) velja
    \begin{gather*}
      ⟦ a =_A b ⟧ ⊆ ⟦ b =_A a ⟧\text{ in}\\
      ⟦ a =_A b ⟧ ∩ ⟦ b =_A c ⟧ ⊆ ⟦ a =_A c ⟧\text.
    \end{gather*}

    \emph{Razpon} \(\e a\) je \(\i{a = a}\).
  \end{definicija}
\end{frame}

\begin{frame}
  \frametitle{Relacije in operacije}

  \begin{definicija}
    Naj bodo \(A₁,\dots,Aₙ,B\) tipi.

    \emph{Relacija} \(R\) na \(A₁×\dots×Aₙ\) je preslikava \(A₁×\dots×Aₙ → 𝒪X\), ki
    za vse \(aᵢ ∈ Aᵢ\) zadošča pogojema
    \begin{gather*}
      \i{a₁ = a₁'} ∩ \dots ∩ \i{aₙ = aₙ'} ∩ R(a₁,…,aₙ) ⊆ R(a₁',…,aₙ')\text{ in}\\
      R(a₁,…,aₙ) ⊆ \e{a₁}∩\dots∩\e{aₙ}\text.
    \end{gather*}

  \end{definicija}
\end{frame}
\begin{frame}
  \frametitle{Relacije in operacije}

  \begin{definicija}
    Naj bodo \(A₁,\dots,Aₙ,B\) tipi.

    \emph{Operacija} \(f : A₁×\dots×Aₙ ↝ B\) je preslikava \(A₁×\dots×Aₙ → B\),
    ki za vse \(aᵢ ∈ Aᵢ\) zadošča pogojema
    \begin{gather*}
      \i{a₁ = a₁'} ∩ \dots ∩ \i{aₙ = aₙ'} ⊆ \i{f(a₁,…,aₙ) = f(a₁',…,aₙ')}\text{ in}\\
      f(a₁,…,aₙ) ⊆ \e{a₁}∩\dots∩\e{aₙ}\text.
    \end{gather*}
    Dve operaciji \(f,g : A₁×\dots×Aₙ ↝ B\) sta enaki, ko za vse \(aᵢ ∈ Aᵢ\)
    velja \[ \i{f(a₁,…,aₙ) = g(a₁,…,aₙ)} = \e{a₁}∩\dots∩\e{aₙ}\text. \]

  \end{definicija}
\end{frame}

\begin{frame}
  \frametitle{Morfizmi}

  \begin{definicija}\label{def:ℒmap}
    \emph{Morfizem} \(f : A ↬ B\) med tipoma \(A\) in \(B\) je relacija med \(A\)
    in \(B\), za katero za vse \(a ∈ A\) in \(b\), \(b' ∈ B\) velja
    \begin{gather*}
      f(a,b) ∩ f(a,b') ⊆ \i{b =_B b'}\\
      \e a ⊆ ⋃_{b ∈ B} f(a,b)
    \end{gather*}
  \end{definicija}

\end{frame}

\begin{frame}
  \frametitle{Logika odprtih množic}

  \begin{align*}
    &\i ⊤                    ≔ X\\
    &\i ⊥                    ≔ ∅\\
    &\i{φ ∧ ψ}               ≔ \i φ ∩ \i ψ\\
    &\i{φ ∨ ψ}               ≔ \i φ ∪ \i ψ\\
    &\i{φ ⇒ ψ}               ≔ \int{\p{\i φ ∪ \i φᶜ}}\\
    &\i{\for{x : A}{φ(x)}}   ≔ \int{⋂_{a ∈ A}\i{\e a ⇒ φ(a)}}\\
    &\i{\exist{x : A}{φ(x)}} ≔ ⋃_{a ∈ A}\i{\e a ∧ φ(a)}\\
    &\i{τ = σ}               ≔ \i{\i τ_A = \i σ_A}\\
    &\i{R(τ)}                ≔ R{\p{\i τ_A}}\text{, za relacijo \(R\) na \(A\) in \(τ : A\)}
  \end{align*}
  \note{
    Formula \(φ\) je veljavna na \(U\), ko je \(U ⊆ \i φ\).
  }
\end{frame}

\begin{frame}
  \frametitle{Objekti}

  \(A\) množica, \(T\) topološki prostor
  \[ T_X ≔ \set{f : 𝒞(U, T)}{U ∈ 𝒪(X)} \]
  Nad realnimi števili je torej \(\id : ℝ → ℝ\) Dedekindovo realno število.

  \note{
    \(\Rc, \Rd, \Rm, \c ℝ, Ω\), Cantor
  }
\end{frame}

\begin{frame}
  \frametitle{Izbira}

  \begin{definicija}
    Princip \(\AC_Σ(A,B)\) pravi, da za vsako celovito relacijo \(R : A×B→Σ\)
    obstaja funkcija izbire.

    \begin{align*}
      &\AC_Σ(A) ≔ \for{B}{\AC_Σ(A,B)}\\
      &\AC_Σ ≔ \for{A}{\AC_Σ(A)}\\
      &\CC_Σ ≔ \for{B}{\AC_Σ(ℕ,B)}\\
      &\CCv_Σ ≔ \AC_Σ(ℕ,2)
    \end{align*}
  \end{definicija}
  
\end{frame}
\begin{frame}

  \begin{izrek}[\cite{Simpson24}]
    Nad \(X\) velja \(\AC(\c A)\) in \(\for{B}{\c A ↬ B ≅ \c{A ↝ B}}\) natanko
    tedaj, ko ima vsaka \(A\)-indeksirana družina pokritij \(X\) skupno
    pofinitev.
  \end{izrek}

  \pause

  \begin{definicija}\label{def:psp}
    Prostor je \emph{P-prostor}, ko je števen presek odprtih množic odprt.
  \end{definicija}
  \begin{trditev}\label{th:psp-is-pgt}
    Če je \(X\) P-prostor, ima vsaka števna družina pokritij \(X\) skupno
    pofinitev.
  \end{trditev}
  \begin{posledica}\label{th:psp-has-cc}
    Nad P-prostori velja \(\CC\).
  \end{posledica}

\end{frame}

\begin{frame}

  \begin{trditev}\label{th:psp-is-not-pgt}
    Nad prostorom \(\cli{0,1}\) s topologijo \(\set{[0,a)}{a ∈ \cli{0,1}} ∪ \cli{0,1}\)
    ima vsaka družina pokritij skupno pofinitev, a ta ni P-prostor.
  \end{trditev}

  \pause

  \begin{trditev}\label{th:t1-pgt-is-psp}
    Če je prostor \(X\) \(T₁\) in ima vsaka števna družina pokritij \(X\) skupno
    pofinitev, je P-prostor.
  \end{trditev}

  \begin{posledica}
    Nad lokalno povezanimi \(T₁\) prostori je \(\CC\) ekvivalentna \(\CCv\).
  \end{posledica}

\end{frame}

\begin{frame}

  \begin{trditev}
    Nad prostori Aleksandova velja \(\AC(\c A)\) za vse množice \(A\).
  \end{trditev}

  \pause
  
  \begin{trditev}\label{th:disc-has-ac}
    Če je \(X\) diskreten prostor, nad njem velja \(\AC\).
  \end{trditev}

  \begin{izrek}\label{th:ac-is-lem}
    V topoloških modelih je \(\AC\) ekvivalenten \(\lem*\).
  \end{izrek}

\end{frame}

\begin{frame}
  \frametitle{Dedekindova konstrukcija}

  \begin{definicija}[Dedekindova konstrukcija]\label{def:Rd}
    Par \(\p{L, U} ∈ 𝒫(ℚ)×𝒫(ℚ)\) je \emph{Dedekindov rez}, ko velja
    \begin{enumerate}
    \item \(L\) je naseljena, dolnja, in navzgor odprta
    \item \(U\) je naseljena, gornja, in navzdol odprta
    \item za vsaka \(p\), \(q : ℚ\) velja \(p ∈ L ∧ q ∈ U ⇒ p < q\)
    \item za vsaka \(p < q\) velja \(p ∈ L ∨ q ∈ U\)
    \end{enumerate}
  \end{definicija}

\end{frame}

% \begin{frame}
%   \frametitle{MacNeillova konstrukcija}

%   \begin{definicija}[MacNeilleova konstrukcija]\label{def:Rm}
%     Par \(\p{L, U} ∈ 𝒫(ℚ)×𝒫(ℚ)\) je \emph{Dedekind-MacNeilleov rez}, ko velja
%     \begin{enumerate}[(a)]
%     \item \(L\) je naseljena, dolnja, in navzgor odprta
%     \item \(U\) je naseljena, gornja, in navzdol odprta
%     \item \(p ∈ L ⇔ \exist{q>p}{q ∉ U}\)
%     \item \(q ∈ U ⇔ \exist{p<q}{p ∉ L}\)
%     \end{enumerate}
%   \end{definicija}

% \end{frame}

% \begin{frame}
%   \frametitle{MacNeillova konstrukcija}

%   \begin{trditev}[\cite{Johnstone02}]\label{th:Rm-maps}
%     Objekt MacNeilleovih realnih števil je natanko tip parov funkcij
%     \(\p{\uline f, \bar f}\) tipa \(U → ℝ\) za \(U ∈ 𝒪X\), za kateri velja
%     pogoj \emph{tesnosti}, ki se glasi:
%     \[
%       \uline f(t) = \liminf_{y → t}\bar   f(y)\qquad\text{ in}\qquad
%       \bar   f(t) = \limsup_{y → t}\uline f(y)\text.
%     \]
%   \end{trditev}

% \end{frame}

\begin{frame}
  \frametitle{Principi}

  \begin{align*}
    &\lpo* ≔ \for{α : 2^ℕ}{α = 0 ∨ α \apart 0}\\
    &\wlpo* ≔ \for{α : 2^ℕ}{α = 0 ∨ α ≠ 0}\\
    &\alpo* ≔ \for{x : \Rd}{x = 0 ∨ x \apart 0}\\ 
    &\awlpo* ≔ \for{x : \Rd}{x = 0 ∨ x ≠ 0}\\
    &\ks* ≔ \for{p : Ω}{\exist{α : 2^ℕ}{p ⇔ α \apart 0}}\\
    &\aks* ≔ \for{p : Ω}{\exist{x : \Rd}{p ⇔ x \apart 0}}\\
    &\mp* ≔ \for{α : 2^ℕ}{¬(α = 0) ⇒ α \apart 0}\\
    &\amp* ≔ \for{x : \Rd}{¬(x = 0) ⇒ x \apart 0}
  \end{align*}
\end{frame}

\begin{frame}

  \begin{trditev}
    Nad \(X\) drži \(\alpo*\) natanko tedaj, ko so ničelne množice funkcij
    \(U → ℝ\) odprte.
  \end{trditev}

  \pause
  
  \begin{trditev}
    Če je \(X\) lokalno \(T₆\), nad njem drži \(\aks*\).
  \end{trditev}

  \begin{trditev}
    Če nad \(X\) obstaja funkcija izbire za \(\aks*\), je lokalno \(T₆\).
  \end{trditev}
  \note{
  }
\end{frame}

\begin{frame}

  \begin{definicija}
    Prostor je \emph{realno nepovezan}, ko je za vsako funkcijo \(f : X → ℝ\)
    množica \(\cl\set{t∈X}{f(t)>0}\) odprta.
  \end{definicija}
  \begin{trditev}\label{th:awlpo-is-basically-disconnected}
    Nad \(X\) velja \(\awlpo*\) natanko tedaj, ko je vsak odprt podprostor \(X\)
    realno nepovezana.
  \end{trditev}
  
\end{frame}

\begin{frame}

  \begin{definicija}
    Prostor \(X\) je \emph{skoraj P-prostor}, ko je za vsak \(f : X → ℝ\) množica
    \(\i{f > 0}\) regularna.
  \end{definicija}
  \begin{trditev}\label{th:amp-is-almost-psp}
    Nad \(X\) velja \(\amp*\) natanko tedaj, ko je vsaka odprta podmnožica \(X\)
    skoraj P-prostor.
  \end{trditev}

\end{frame}

\begin{frame}

  \begin{izrek}
    Velja \(\amp* ∧ \awlpo* ⇒ \alpo*\).
  \end{izrek}
  
  \begin{izrek}\label{th:alpo-is-awlpo-and-amp}
    Ničelne množice \(f : X → ℝ\) so odprte natanko tedaj, ko je \(X\) skoraj
    P-prostor in realno nepovezan.
  \end{izrek}

\end{frame}

\begin{frame}

  \begin{trditev}
    Nad lokalno povezanim prostorom velja \(\lpo*\).
  \end{trditev}

  \begin{trditev}[??]
    Če nad \(2\)-števnim prostorom velja \(\lpo*\), je lokalno povezan.
  \end{trditev}

\end{frame}


\begin{frame}
  \frametitle{}

  Vprašanja?

  \note{
  }
\end{frame}


\end{document}
